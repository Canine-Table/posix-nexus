\newpage
\section{Int}
\label{Int}
\begin{NexMainBox}[light, hdrA, sdwA, crnA, grwB, secA]
	\begin{NexMainBox}[dark, crnA]
		The following functions provide powerful computational tools for performing advanced numerical operations, including base-specific arithmetic, distribution, range adjustment, and mathematical constants handling.
	\end{NexMainBox}
	\begin{NexMainBox}[dark, crnA]
		\begin{NexListDark}
			\NexItemDark{\NexFunction{get\_int\_conv()}: Converts a number \NexFunction{num} from its original base \NexFunction{from} to another base \NexFunction{to}, supporting optional signed number handling.}
			\NexItemDark{\NexFunction{get\_int\_bsubt()}: Computes the difference of two numbers, \NexFunction{minuend} and \NexFunction{subtrahend}, in base \NexFunction{from} with a specified precision \NexFunction{prec}, supporting signed numbers.}
			\NexItemDark{\NexFunction{get\_int\_badd()}: Computes the sum of two numbers, \NexFunction{addend1} and \NexFunction{addend2}, in base \NexFunction{from} with a specified precision \NexFunction{prec}, supporting signed numbers.}
			\NexItemDark{\NexFunction{get\_int\_comp()}: Computes the complement of a number \NexFunction{num} in the specified \NexFunction{base}, leveraging AWK utility functions for base-specific computations.}
			\NexItemDark{\NexFunction{get\_int\_abs()}: Calculates the absolute value of \NexFunction{num}, ensuring the result is always a positive number, using AWK's utility functions.}
			\NexItemDark{\NexFunction{get\_int\_fact()}: Computes the factorial of \NexFunction{num}, with an option to print intermediate steps if \NexFunction{prnt} is set to true.}
			\NexItemDark{\NexFunction{get\_int\_fib()}: Computes the \NexFunction{num}-th Fibonacci number, optionally printing intermediate sums if \NexFunction{prnt} is set to true.}
			\NexItemDark{\NexFunction{get\_int\_round()}: Rounds \NexFunction{num} according to the specified method \NexFunction{rnd} (e.g., \texttt{ceiling} or \texttt{round}), defaulting to truncation if no method is provided.}
			\NexItemDark{\NexFunction{get\_int\_gcd()}: Computes the greatest common divisor (GCD) of two numbers, \NexFunction{num1} and \NexFunction{num2}, using the Euclidean algorithm.}
			\NexItemDark{\NexFunction{get\_int\_remainder()}: Computes the remainder of dividing \NexFunction{num1} by \NexFunction{num2}, ensuring both inputs are valid digits.}
			\NexItemDark{\NexFunction{get\_int\_lcd()}: Calculates the least common denominator (LCD) of \NexFunction{num1} and \NexFunction{num2} using AWK's mathematical utilities.}
			\NexItemDark{\NexFunction{get\_int\_tau()}: Returns the value of \(\tau\) (the circle constant, \(\tau = 2\pi\)), optionally based on the input \NexFunction{num} for calculations or prints a default \(\tau\) if no input is provided.}
			\NexItemDark{\NexFunction{get\_int\_pi()}: Returns the value of \(\pi\) (pi constant), optionally using the input \NexFunction{num} for calculations or defaults to a general \(\pi\) value when no input is specified.}
		\end{NexListDark}
	\end{NexMainBox}
\end{NexMainBox}
\begin{NexMainBox}[light, hdrA, sdwA, crnA, grwB, secB]
	\begin{NexMainBox}[dark, crnA]
		\begin{NexListDark}
			\NexItemDark{\NexFunction{get\_int\_distribute()}: Distributes \NexFunction{num1} evenly across the range defined by \NexFunction{num2} and \NexFunction{num3}, ensuring all inputs are valid digits.}
			\NexItemDark{\NexFunction{get\_int\_range()}: Adjusts \NexFunction{num1} to fit within the range defined by \NexFunction{num2} and \NexFunction{num3}, using modulus operations for precise computation.}
		\end{NexListDark}
	\end{NexMainBox}
\end{NexMainBox}

