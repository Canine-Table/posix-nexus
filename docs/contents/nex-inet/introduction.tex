\newpage
\section{Introduction to Networking}
\label{Introduction to Networking}
\begin{NexMainBox}[light, hdrA, sdwA, crnA, grwB, secA]
	\begin{NexMainBox}[dark, crnA]
	\end{NexMainBox}
	\begin{NexMainBox}[dark, crnA]
		\begin{NexListDark}
			\NexItemDark{\NexLink{Network Types}{Network Types}: Categorizes networks based on their scope, size, and purpose. Common types include LAN, WAN, MAN, and PAN, each with unique characteristics and use cases.}
		\end{NexListDark}
	\end{NexMainBox}
\end{NexMainBox}

\newpage
\subsection{Network Types}
\label{Network Types}
\begin{NexMainBox}[light, hdrA, sdwA, crnA, grwB, ssecA]
	\begin{NexMainBox}[dark, crnA]
		Networks are broadly categorized based on their geographical scope, size, and purpose. This categorization helps determine the appropriate technologies, devices, and configurations for specific networking needs.
	\end{NexMainBox}
	\begin{NexMainBox}[dark, crnA]
		\begin{NexListDark}
			\NexItemDark{\NexLink{PAN (Personal Area Network)}{PAN (Personal Area Network)}: Designed for individual use within a limited range, including Bluetooth or USB connections.}
			\NexItemDark{\NexLink{LAN (Local Area Network)}{LAN (Local Area Network)}: Covers small, localized areas such as homes, schools, or offices. Typically relies on Ethernet or Wi-Fi.}
			\NexItemDark{\NexLink{MAN (Metropolitan Area Network)}{MAN (Metropolitan Area Network)}: Covers regions like cities or campuses, bridging multiple LANs within a metropolitan area.}
			\NexItemDark{\NexLink{WAN (Wide Area Network)}{WAN (Wide Area Network)}: Spans large geographical areas and connects multiple LANs. The internet is the largest example.}
			\NexItemDark{\NexLink{GAN (Global Area Network)}{GAN (Global Area Network)}: A type of network that spans across the globe, connecting multiple WANs and providing international connectivity. Commonly used by multinational organizations and global services.}
		\end{NexListDark}
	\end{NexMainBox}
\end{NexMainBox}

\newpage
\subsubsection{PAN (Personal Area Network)}
\label{PAN (Personal Area Network)}
\begin{NexMainBox}[light, hdrA, sdwA, crnA, grwB, sssecA]
	\begin{NexMainBox}[dark, crnA]
		A Personal Area Network (PAN) is a short-range network designed for individual use, often connecting personal devices such as smartphones, laptops, and wearable devices.
	\end{NexMainBox}
	\begin{NexMainBox}[dark, crnA]
		\begin{NexListDark}
			\NexItemDark{\NexOption{Scope}: Limited to an individual's workspace, typically within a 10-meter range.}
			\NexItemDark{\NexOption{Applications}:
				\begin{NexListLight}
					\NexItemLight{Facilitates Bluetooth connections for device pairing.}
					\NexItemLight{Used for connecting peripherals like keyboards, mice, and printers.}
					\NexItemLight{Supports data transfer between personal devices.}
				\end{NexListLight}
			}
			\NexItemDark{\NexOption{Technology}: Commonly uses Bluetooth, NFC, or USB for connectivity.}
		\end{NexListDark}
	\end{NexMainBox}
\end{NexMainBox}

\newpage
\subsubsection{LAN (Local Area Network)}
\label{LAN (Local Area Network)}
\begin{NexMainBox}[light, hdrA, sdwA, crnA, grwB, sssecA]
		\begin{NexMainBox}[dark, crnA]
			A Local Area Network (LAN) is a network that spans a small geographical area, such as a home, office, or school. It is ideal for connecting devices within close proximity, typically using Ethernet or Wi-Fi.
		\end{NexMainBox}
		\begin{NexMainBox}[dark, crnA]
			\begin{NexListDark}
			\NexItemDark{\NexOption{Scope}: Covers limited areas such as buildings or small campuses.}
			\NexItemDark{\NexOption{Applications}:
				\begin{NexListLight}
					\NexItemLight{Used in offices for sharing printers, files, and internet access.}
					\NexItemLight{Commonly set up in homes for device connectivity.}
					\NexItemLight{Facilitates gaming networks for multiplayer gaming.}
				\end{NexListLight}
			}
			\NexItemDark{\NexOption{Technology}: Implements Ethernet (wired) or Wi-Fi (wireless) protocols for connectivity.}
		\end{NexListDark}
	\end{NexMainBox}
\end{NexMainBox}

\newpage
\subsubsection{MAN (Metropolitan Area Network)}
\label{MAN (Metropolitan Area Network)}
\begin{NexMainBox}[light, hdrA, sdwA, crnA, grwB, sssecA]
	\begin{NexMainBox}[dark, crnA]
		A Metropolitan Area Network (MAN) connects multiple LANs within a metropolitan area, such as a city or a large campus. MANs bridge the gap between LANs and WANs for broader regional connectivity.
	\end{NexMainBox}
	\begin{NexMainBox}[dark, crnA]
		\begin{NexListDark}
			\NexItemDark{\NexOption{Scope}: Covers metropolitan regions, such as cities or extensive campuses.}
			\NexItemDark{\NexOption{Applications}:
				\begin{NexListLight}
					\NexItemLight{Used by city governments for public services like surveillance and internet access.}
					\NexItemLight{Facilitates inter-campus connectivity for universities.}
					\NexItemLight{Enables metro Ethernet services for business districts.}
				\end{NexListLight}
			}
			\NexItemDark{\NexOption{Technology}: Employs fiber optic networks and metro Ethernet for high-speed connectivity.}
		\end{NexListDark}
	\end{NexMainBox}
\end{NexMainBox}

\newpage
\subsubsection{WAN (Wide Area Network)}
\label{WAN (Wide Area Network)}
\begin{NexMainBox}[light, hdrA, sdwA, crnA, grwB, sssecA]
	\begin{NexMainBox}[dark, crnA]
		A Wide Area Network (WAN) spans large geographical areas, connecting multiple LANs. WANs are essential for communication between regions and are the foundation for the global internet.
	\end{NexMainBox}
	\begin{NexMainBox}[dark, crnA]
		\begin{NexListDark}
			\NexItemDark{\NexOption{Scope}: Extends across cities, countries, or continents.}
			\NexItemDark{\NexOption{Applications}:
				\begin{NexListLight}
					\NexItemLight{Used by enterprises to connect offices across regions.}
					\NexItemLight{Supports internet service providers (ISPs) to deliver connectivity.}
					\NexItemLight{Provides connectivity for large-scale communication systems.}
				\end{NexListLight}
			}
			\NexItemDark{\NexOption{Technology}: Utilizes fiber optic cables, satellite links, and dedicated leased lines.}
		\end{NexListDark}
	\end{NexMainBox}
\end{NexMainBox}

\newpage
\subsubsection{GAN (Global Area Network)}
\label{GAN (Global Area Network)}
\begin{NexMainBox}[light, hdrA, sdwA, crnA, grwB, sssecA]
	\begin{NexMainBox}[dark, crnA]
	A Global Area Network (GAN) connects networks worldwide, offering seamless communication and data sharing on an international scale. It combines several WANs into a unified network infrastructure, enabling global connectivity.
	\end{NexMainBox}
	\begin{NexMainBox}[dark, crnA]
		\begin{NexListDark}
			\NexItemDark{\NexOption{Scope}: A GAN spans multiple continents, connecting regions and countries worldwide.}
			\NexItemDark{\NexOption{Applications}:
				\begin{NexListLight}
					\NexItemLight{Used by multinational corporations to synchronize operations across global offices.}
					\NexItemLight{Enables real-time communication and collaboration for global teams.}
					\NexItemLight{Supports international services like content delivery networks (CDNs) and global cloud platforms.}
				\end{NexListLight}
			}
			\NexItemDark{\NexOption{Technology}: Relies on satellite links, undersea cables, and high-speed WAN technologies to achieve global reach.}
		\end{NexListDark}
	\end{NexMainBox}
\end{NexMainBox}

