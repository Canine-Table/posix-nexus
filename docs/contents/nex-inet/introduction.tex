\newpage
\section{Introduction}
\label{Introduction}
\begin{NexMainBox}[light, hdrA, sdwA, crnA, grwB, secA]
	\begin{NexMainBox}[dark, crnA]
	\end{NexMainBox}
	\begin{NexMainBox}[dark, crnA]
		\begin{NexListDark}
			\NexItemDark{\NexLink{Network Types}{Network Types}: Categorizes networks based on their scope, size, and purpose. Common types include LAN, WAN, MAN, and PAN, each with unique characteristics and use cases.}
			\NexItemDark{\NexLink{Broadband Technologies}{Broadband Technologies}: Encompasses various technologies such as DSL, Cable, Fiber-optic, Satellite, and Mobile broadband, each offering unique advantages based on speed, availability, and cost.}
			\NexItemDark{\NexLink{IP Addressing: IPv4 and IPv6}{IP Addressing: IPv4 and IPv6}: Explores the structure and functionality of IPv4 and IPv6, highlighting their applications, advantages, and challenges during the transition from IPv4 to IPv6.}
		\end{NexListDark}
	\end{NexMainBox}
\end{NexMainBox}

\newpage
\subsection{Network Types}
\label{Network Types}
\begin{NexMainBox}[light, hdrA, sdwA, crnA, grwB, ssecA]
	\begin{NexMainBox}[dark, crnA]
		Networks are broadly categorized based on their geographical scope, size, and purpose. This categorization helps determine the appropriate technologies, devices, and configurations for specific networking needs.
	\end{NexMainBox}
	\begin{NexMainBox}[dark, crnA]
		\begin{NexListDark}
			\NexItemDark{\NexLink{PAN (Personal Area Network)}{PAN (Personal Area Network)}: Designed for individual use within a limited range, including Bluetooth or USB connections.}
			\NexItemDark{\NexLink{LAN (Local Area Network)}{LAN (Local Area Network)}: Covers small, localized areas such as homes, schools, or offices. Typically relies on Ethernet or Wi-Fi.}
			\NexItemDark{\NexLink{MAN (Metropolitan Area Network)}{MAN (Metropolitan Area Network)}: Covers regions like cities or campuses, bridging multiple LANs within a metropolitan area.}
			\NexItemDark{\NexLink{WAN (Wide Area Network)}{WAN (Wide Area Network)}: Spans large geographical areas and connects multiple LANs. The internet is the largest example.}
			\NexItemDark{\NexLink{GAN (Global Area Network)}{GAN (Global Area Network)}: A type of network that spans across the globe, connecting multiple WANs and providing international connectivity. Commonly used by multinational organizations and global services.}
		\end{NexListDark}
	\end{NexMainBox}
\end{NexMainBox}

\newpage
\subsubsection{PAN (Personal Area Network)}
\label{PAN (Personal Area Network)}
\begin{NexMainBox}[light, hdrA, sdwA, crnA, grwB, sssecA]
	\begin{NexMainBox}[dark, crnA]
		A Personal Area Network (PAN) is a short-range network designed for individual use, often connecting personal devices such as smartphones, laptops, and wearable devices.
	\end{NexMainBox}
	\begin{NexMainBox}[dark, crnA]
		\begin{NexListDark}
			\NexItemDark{\NexOption{Scope}: Limited to an individual's workspace, typically within a 10-meter range.}
			\NexItemDark{\NexOption{Applications}:
				\begin{NexListLight}
					\NexItemLight{Facilitates Bluetooth connections for device pairing.}
					\NexItemLight{Used for connecting peripherals like keyboards, mice, and printers.}
					\NexItemLight{Supports data transfer between personal devices.}
				\end{NexListLight}
			}
			\NexItemDark{\NexOption{Technology}: Commonly uses Bluetooth, NFC, or USB for connectivity.}
		\end{NexListDark}
	\end{NexMainBox}
\end{NexMainBox}

\newpage
\subsubsection{LAN (Local Area Network)}
\label{LAN (Local Area Network)}
\begin{NexMainBox}[light, hdrA, sdwA, crnA, grwB, sssecA]
		\begin{NexMainBox}[dark, crnA]
			A Local Area Network (LAN) is a network that spans a small geographical area, such as a home, office, or school. It is ideal for connecting devices within close proximity, typically using Ethernet or Wi-Fi.
		\end{NexMainBox}
		\begin{NexMainBox}[dark, crnA]
			\begin{NexListDark}
			\NexItemDark{\NexOption{Scope}: Covers limited areas such as buildings or small campuses.}
			\NexItemDark{\NexOption{Applications}:
				\begin{NexListLight}
					\NexItemLight{Used in offices for sharing printers, files, and internet access.}
					\NexItemLight{Commonly set up in homes for device connectivity.}
					\NexItemLight{Facilitates gaming networks for multiplayer gaming.}
				\end{NexListLight}
			}
			\NexItemDark{\NexOption{Technology}: Implements Ethernet (wired) or Wi-Fi (wireless) protocols for connectivity.}
		\end{NexListDark}
	\end{NexMainBox}
\end{NexMainBox}

\newpage
\subsubsection{MAN (Metropolitan Area Network)}
\label{MAN (Metropolitan Area Network)}
\begin{NexMainBox}[light, hdrA, sdwA, crnA, grwB, sssecA]
	\begin{NexMainBox}[dark, crnA]
		A Metropolitan Area Network (MAN) connects multiple LANs within a metropolitan area, such as a city or a large campus. MANs bridge the gap between LANs and WANs for broader regional connectivity.
	\end{NexMainBox}
	\begin{NexMainBox}[dark, crnA]
		\begin{NexListDark}
			\NexItemDark{\NexOption{Scope}: Covers metropolitan regions, such as cities or extensive campuses.}
			\NexItemDark{\NexOption{Applications}:
				\begin{NexListLight}
					\NexItemLight{Used by city governments for public services like surveillance and internet access.}
					\NexItemLight{Facilitates inter-campus connectivity for universities.}
					\NexItemLight{Enables metro Ethernet services for business districts.}
				\end{NexListLight}
			}
			\NexItemDark{\NexOption{Technology}: Employs fiber optic networks and metro Ethernet for high-speed connectivity.}
		\end{NexListDark}
	\end{NexMainBox}
\end{NexMainBox}

\newpage
\subsubsection{WAN (Wide Area Network)}
\label{WAN (Wide Area Network)}
\begin{NexMainBox}[light, hdrA, sdwA, crnA, grwB, sssecA]
	\begin{NexMainBox}[dark, crnA]
		A Wide Area Network (WAN) spans large geographical areas, connecting multiple LANs. WANs are essential for communication between regions and are the foundation for the global internet.
	\end{NexMainBox}
	\begin{NexMainBox}[dark, crnA]
		\begin{NexListDark}
			\NexItemDark{\NexOption{Scope}: Extends across cities, countries, or continents.}
			\NexItemDark{\NexOption{Applications}:
				\begin{NexListLight}
					\NexItemLight{Used by enterprises to connect offices across regions.}
					\NexItemLight{Supports internet service providers (ISPs) to deliver connectivity.}
					\NexItemLight{Provides connectivity for large-scale communication systems.}
				\end{NexListLight}
			}
			\NexItemDark{\NexOption{Technology}: Utilizes fiber optic cables, satellite links, and dedicated leased lines.}
		\end{NexListDark}
	\end{NexMainBox}
\end{NexMainBox}

\newpage
\subsubsection{GAN (Global Area Network)}
\label{GAN (Global Area Network)}
\begin{NexMainBox}[light, hdrA, sdwA, crnA, grwB, sssecA]
	\begin{NexMainBox}[dark, crnA]
	A Global Area Network (GAN) connects networks worldwide, offering seamless communication and data sharing on an international scale. It combines several WANs into a unified network infrastructure, enabling global connectivity.
	\end{NexMainBox}
	\begin{NexMainBox}[dark, crnA]
		\begin{NexListDark}
			\NexItemDark{\NexOption{Scope}: A GAN spans multiple continents, connecting regions and countries worldwide.}
			\NexItemDark{\NexOption{Applications}:
				\begin{NexListLight}
					\NexItemLight{Used by multinational corporations to synchronize operations across global offices.}
					\NexItemLight{Enables real-time communication and collaboration for global teams.}
					\NexItemLight{Supports international services like content delivery networks (CDNs) and global cloud platforms.}
				\end{NexListLight}
			}
			\NexItemDark{\NexOption{Technology}: Relies on satellite links, undersea cables, and high-speed WAN technologies to achieve global reach.}
		\end{NexListDark}
	\end{NexMainBox}
\end{NexMainBox}

\newpage
\subsection{Broadband Technologies}
\label{Broadband Technologies}
\begin{NexMainBox}[light, hdrA, sdwA, crnA, grwB, ssecA]
	\begin{NexMainBox}[dark, crnA]
		Broadband can be delivered through various technologies, each suited to specific user needs and geographic areas. Choosing the right technology depends on availability, cost, and required speed.
	\end{NexMainBox}
	\begin{NexMainBox}[dark, crnA]
		\begin{NexListDark}
			\NexItemDark{\NexLink{DSL (Digital Subscriber Line)}{DSL (Digital Subscriber Line)}: Provides internet through existing telephone lines, offering speeds up to 100 Mbps depending on distance and infrastructure.}
			\NexItemDark{\NexLink{Cable Broadband}{Cable Broadband}: Uses coaxial cables to deliver higher speeds than DSL, commonly ranging from 100 Mbps to 1 Gbps.}
			\NexItemDark{\NexLink{Fiber-Optic Broadband}{Fiber-Optic Broadband}: Utilizes light signals over fiber-optic cables for ultra-high speeds up to several Gbps, ensuring reliability and minimal latency.}
			\NexItemDark{\NexLink{Satellite Broadband}{Satellite Broadband}: Ideal for rural or remote areas, offering speeds up to 100 Mbps via satellite connections but with higher latency.}
			\NexItemDark{\NexLink{Mobile Broadband}{Mobile Broadband}: Delivers internet via cellular networks (3G, 4G, or 5G) with varying speeds and mobility advantages.}
		\end{NexListDark}
	\end{NexMainBox}
\end{NexMainBox}

\newpage
\subsubsection{DSL (Digital Subscriber Line)}
\label{DSL (Digital Subscriber Line)}
\begin{NexMainBox}[light, hdrA, sdwA, crnA, grwB, sssecA]
	\begin{NexMainBox}[dark, crnA]
		Digital Subscriber Line (DSL) is a broadband technology that delivers high-speed internet using existing copper telephone lines. It offers cost-effective connectivity and is widely available in urban and suburban areas.
	\end{NexMainBox}
	\begin{NexMainBox}[dark, crnA]
		\begin{NexListDark}
			\NexItemDark{\NexOption{Scope}: Limited by the distance between the user’s location and the telephone exchange, with speeds decreasing as the distance increases.}
			\NexItemDark{\NexOption{Applications}:
				\begin{NexListLight}
					\NexItemLight{Enables everyday browsing, streaming, and email communication.}
					\NexItemLight{Supports small business operations requiring stable internet.}
					\NexItemLight{Provides a solution for residential internet connectivity.}
				\end{NexListLight}
			}
			\NexItemDark{\NexOption{Technology}: Uses frequency division multiplexing (FDM) to separate voice calls and internet data over the same line.}
			\NexItemDark{\NexOption{Variants}:
				\begin{NexListLight}
					\NexItemLight{ADSL (Asymmetric DSL): Offers faster download speeds compared to upload speeds, ideal for web browsing and streaming.}
					\NexItemLight{VDSL (Very-high-bit-rate DSL): Provides higher speeds than ADSL, supporting activities like HD streaming and online gaming.}
					\NexItemLight{SDSL (Symmetric DSL): Ensures equal download and upload speeds, often used for business applications.}
				\end{NexListLight}
			}
		\end{NexListDark}
	\end{NexMainBox}
\end{NexMainBox}

\newpage
\subsubsection{Cable Broadband}
\label{Cable Broadband}
\begin{NexMainBox}[light, hdrA, sdwA, crnA, grwB, sssecA]
	\begin{NexMainBox}[dark, crnA]
		Cable Broadband is a popular high-speed internet technology that uses coaxial cables to transmit data. It offers faster speeds compared to DSL and is widely used in residential and commercial settings.
	\end{NexMainBox}
	\begin{NexMainBox}[dark, crnA]
		\begin{NexListDark}
			\NexItemDark{\NexOption{Scope}: Primarily serves homes and businesses in areas where cable television infrastructure is available.}
			\NexItemDark{\NexOption{Applications}:
				\begin{NexListLight}
					\NexItemLight{Enables streaming of high-definition videos and multimedia content.}
					\NexItemLight{Supports online gaming and real-time communication.}
					\NexItemLight{Facilitates faster downloads and uploads for personal and professional use.}
				\end{NexListLight}
			}
			\NexItemDark{\NexOption{Technology}: Utilizes DOCSIS (Data Over Cable Service Interface Specification) standards to deliver broadband services.}
			\NexItemDark{\NexOption{Advantages}:
				\begin{NexListLight}
					\NexItemLight{Provides higher speeds and reliability compared to DSL.}
					\NexItemLight{Supports multiple users and devices simultaneously.}
					\NexItemLight{Leverages existing cable TV infrastructure for widespread availability.}
				\end{NexListLight}
			}
			\NexItemDark{\NexOption{Limitations}:
				\begin{NexListLight}
					\NexItemLight{Shared bandwidth can cause speed fluctuations during peak usage.}
					\NexItemLight{Availability may be limited in rural or remote areas.}
					\NexItemLight{Requires a cable modem for connectivity.}
				\end{NexListLight}
			}
		\end{NexListDark}
	\end{NexMainBox}
\end{NexMainBox}

\newpage
\subsubsection{Fiber-Optic Broadband}
\label{Fiber-Optic Broadband}
\begin{NexMainBox}[light, hdrA, sdwA, crnA, grwB, sssecA]
	\begin{NexMainBox}[dark, crnA]
		Fiber-Optic Broadband is a cutting-edge internet technology that uses light signals transmitted through thin strands of glass or plastic (fiber-optic cables). It delivers ultra-fast and highly reliable internet speeds, making it ideal for modern digital demands.
	\end{NexMainBox}
	\begin{NexMainBox}[dark, crnA]
		\begin{NexListDark}
			\NexItemDark{\NexOption{Scope}: Designed to provide high-speed internet over large distances without signal degradation, suitable for residential, commercial, and industrial applications.}
			\NexItemDark{\NexOption{Applications}:
				\begin{NexListLight}
					\NexItemLight{Supports streaming ultra-high-definition (4K or 8K) videos seamlessly.}
					\NexItemLight{Enables real-time online gaming with minimal latency.}
					\NexItemLight{Facilitates large-scale data transfers for businesses and data centers.}
				\end{NexListLight}
			}
			\NexItemDark{\NexOption{Technology}: Utilizes pulses of light through fiber-optic cables to encode and transmit data at speeds up to several Gbps.}
			\NexItemDark{\NexOption{Advantages}:
				\begin{NexListLight}
					\NexItemLight{Provides unparalleled internet speeds and reliability.}
					\NexItemLight{Minimizes signal loss over long distances compared to copper cables.}
					\NexItemLight{Offers resistance to electromagnetic interference (EMI), ensuring stable connections.}
				\end{NexListLight}
			}
			\NexItemDark{\NexOption{Limitations}:
				\begin{NexListLight}
					\NexItemLight{Higher installation costs compared to other broadband technologies.}
					\NexItemLight{Limited availability in rural or underserved areas.}
					\NexItemLight{Requires professional installation due to the complexity of fiber-optic infrastructure.}
				\end{NexListLight}
			}
		\end{NexListDark}
	\end{NexMainBox}
\end{NexMainBox}

\newpage
\subsubsection{Satellite Broadband}
\label{Satellite Broadband}
\begin{NexMainBox}[light, hdrA, sdwA, crnA, grwB, sssecA]
	\begin{NexMainBox}[dark, crnA]
		Satellite Broadband provides internet access via communication satellites. It is especially beneficial for rural and remote areas where other broadband technologies are unavailable.
	\end{NexMainBox}
	\begin{NexMainBox}[dark, crnA]
		\begin{NexListDark}
			\NexItemDark{\NexOption{Scope}: Designed to cover areas where terrestrial broadband infrastructure is absent, including isolated regions and mobile platforms.}
			\NexItemDark{\NexOption{Applications}:
				\begin{NexListLight}
					\NexItemLight{Provides internet connectivity to remote homes, offices, and farms.}
					\NexItemLight{Supports communication for ships, aircraft, and other mobile platforms.}
					\NexItemLight{Enables emergency response services in disaster-affected regions.}
				\end{NexListLight}
			}
			\NexItemDark{\NexOption{Technology}: Relies on geostationary and low Earth orbit (LEO) satellites to transmit and receive signals. The user requires a satellite dish and modem for connectivity.}
			\NexItemDark{\NexOption{Advantages}:
				\begin{NexListLight}
					\NexItemLight{Provides internet access where other broadband technologies are unavailable.}
					\NexItemLight{Covers large geographical areas, including sparsely populated regions.}
					\NexItemLight{Supports continuous connectivity for mobile platforms.}
				\end{NexListLight}
			}
			\NexItemDark{\NexOption{Limitations}:
				\begin{NexListLight}
					\NexItemLight{Higher latency compared to terrestrial broadband due to signal travel distance.}
					\NexItemLight{Weather conditions can affect signal quality and reliability.}
					\NexItemLight{Typically more expensive than other broadband options due to equipment and service costs.}
				\end{NexListLight}
			}
		\end{NexListDark}
	\end{NexMainBox}
\end{NexMainBox}

\newpage
\subsubsection{Mobile Broadband}
\label{Mobile Broadband}
\begin{NexMainBox}[light, hdrA, sdwA, crnA, grwB, sssecA]
	\begin{NexMainBox}[dark, crnA]
		Mobile Broadband delivers internet connectivity through cellular networks such as 3G, 4G, and 5G. It enables on-the-go access for smartphones, tablets, and mobile hotspots.
	\end{NexMainBox}
	\begin{NexMainBox}[dark, crnA]
		\begin{NexListDark}
			\NexItemDark{\NexOption{Scope}: Ideal for personal or business use in urban, suburban, and even some rural areas with cellular coverage.}
			\NexItemDark{\NexOption{Applications}:
				\begin{NexListLight}
					\NexItemLight{Provides internet access for mobile devices like smartphones and tablets.}
					\NexItemLight{Facilitates remote work with portable hotspots.}
					\NexItemLight{Supports IoT devices and applications requiring constant connectivity.}
				\end{NexListLight}
			}
			\NexItemDark{\NexOption{Technology}: Operates on cellular networks such as LTE (4G) and 5G, offering varying speeds and coverage.}
			\NexItemDark{\NexOption{Advantages}:
				\begin{NexListLight}
					\NexItemLight{Provides mobility and flexibility, allowing users to stay connected anywhere with coverage.}
					\NexItemLight{Supports a wide range of devices beyond traditional computers.}
					\NexItemLight{Expands broadband access to underserved areas with cellular networks.}
				\end{NexListLight}
			}			\NexItemDark{\NexOption{Limitations}:
				\begin{NexListLight}
					\NexItemLight{Data caps and speed throttling may apply based on service plans.}
					\NexItemLight{Signal strength can vary depending on location and network congestion.}
					\NexItemLight{Latency may be higher compared to fiber or cable broadband.}
				\end{NexListLight}
			}
		\end{NexListDark}
	\end{NexMainBox}
\end{NexMainBox}

\newpage
\subsection{IP Addressing: IPv4 and IPv6}
\label{IP Addressing: IPv4 and IPv6}
\begin{NexMainBox}[light, hdrA, sdwA, crnA, grwB, ssecB]
	\begin{NexMainBox}[dark, crnA]
		IP Addressing is a critical component of networking, enabling devices to identify and communicate over a network. Two primary versions of IP addressing are in use: IPv4 and IPv6. Both serve the same fundamental purpose but differ significantly in structure and capacity.
	\end{NexMainBox}
	\begin{NexMainBox}[dark, crnA]
		\begin{NexListDark}
			\NexItemDark{\NexOption{IPv4 Addressing}: Uses 32-bit addresses, represented in decimal format (e.g., 192.168.1.1). IPv4 supports approximately \(4.3 \, \text{billion}\) unique addresses, which has led to depletion due to the exponential growth in connected devices.}
			\NexItemDark{\NexOption{IPv6 Addressing}: Employs 128-bit addresses, expressed in hexadecimal notation (e.g., 2001:0db8:85a3:0000:0000:8a2e:0370:7334). IPv6 vastly expands the address space, accommodating \(3.4 \times 10^{38}\) unique addresses. It also introduces features like simplified routing and better security mechanisms.}
			\NexItemDark{\NexOption{Applications}: Enables communication for:
				\begin{NexListLight}
					\NexItemLight{Internet-connected devices (e.g., computers, smartphones, IoT sensors).}
					\NexItemLight{Hosting websites and servers.}
					\NexItemLight{Enabling peer-to-peer communication (e.g., file sharing).}
				\end{NexListLight}
			}
			\NexItemDark{\NexOption{Transition Challenges}: Highlights the issues faced during the shift from IPv4 to IPv6, such as compatibility with older systems and the need for updated networking equipment.}
		\end{NexListDark}
	\end{NexMainBox}
\end{NexMainBox}

