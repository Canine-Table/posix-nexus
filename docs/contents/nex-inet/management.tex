\newpage
\section{Configuration and Management}
\label{Configuration and Management}
\begin{NexMainBox}[light, hdrA, sdwA, crnA, grwB, secA]
	\begin{NexMainBox}[dark, crnA]
	\end{NexMainBox}
	\begin{NexMainBox}[dark, crnA]
		\begin{NexListDark}
			\NexItemDark{\NexLink{Network Topologies}{Network Topologies}: Defines the arrangement of devices within a network, including Star, Ring, Mesh, Bus, and Hybrid layouts.}
			%\NexItemDark{\NexLink{Design Considerations}{Design Considerations}: Covers essential factors for scalability, fault tolerance, redundancy, and performance.}
			%\NexItemDark{\NexLink{Network Planning Tools}{Network Planning Tools}: Introduces tools for visualization, configuration, and management of networks.}
			%\NexItemDark{\NexLink{Best Practices for Configuration}{Best Practices for Configuration}: Provides guidelines for efficient and secure network setups, such as VLANs and access control.}
		\end{NexListDark}
	\end{NexMainBox}
\end{NexMainBox}

\newpage
\subsection{Network Topologies}
\label{Network Topologies}
\begin{NexMainBox}[light, hdrA, sdwA, crnA, grwB, ssecA]
	\begin{NexMainBox}[dark, crnA]
		Network topologies define the physical or logical arrangement of devices within a network, directly impacting its performance, scalability, and fault tolerance.
	\end{NexMainBox}
	\begin{NexMainBox}[dark, crnA]
		\begin{NexListDark}
			\NexItemDark{\NexLink{Star Topology}{Star Topology}: All devices connect to a central hub or switch.}
			\NexItemDark{\NexLink{Ring Topology}{Ring Topology}: Devices form a circular structure where data travels unidirectionally.}
			\NexItemDark{\NexLink{Bus Topology}{Bus Topology}: Devices share a single communication line or backbone.}
			\NexItemDark{\NexLink{Mesh Topology}{Mesh Topology}: Devices are interconnected for redundancy and fault tolerance.}
			\NexItemDark{\NexLink{Hybrid Topology}{Hybrid Topology}: Combines multiple topologies to fit complex network needs.}
		\end{NexListDark}
	\end{NexMainBox}
\end{NexMainBox}

\newpage
\subsubsection{Star Topology}
\label{Star Topology}
\begin{NexMainBox}[light, hdrA, sdwA, crnA, grwB, sssecA]
	\begin{NexMainBox}[dark, crnA]
		Star topology connects all devices to a central hub or switch. This layout is easy to manage but has a single point of failure in the hub.
	\end{NexMainBox}
	\begin{NexMainBox}[dark, crnA]
		\begin{NexListDark}
			\NexItemDark{\NexOption{Scope}: Ideal for small to medium networks such as offices or homes.}
			\NexItemDark{\NexOption{Applications}:
				\begin{NexListLight}
					\NexItemLight{Used in Ethernet networks for centralized communication.}
					\NexItemLight{Popular for Wi-Fi setups connecting multiple devices.}
					\NexItemLight{Enables easy troubleshooting and management.}
				\end{NexListLight}
			}
			\NexItemDark{\NexOption{Technology}: Relies on switches, hubs, and wireless access points (WAPs).}
		\end{NexListDark}
	\end{NexMainBox}
\end{NexMainBox}

\newpage
\subsubsection{Ring Topology}
\label{Ring Topology}
\begin{NexMainBox}[light, hdrA, sdwA, crnA, grwB, sssecA]
	\begin{NexMainBox}[dark, crnA]
		Ring topology connects devices in a circular structure, with data traveling in one direction (or bidirectional in some cases). This topology ensures equal access but can be susceptible to a single point of failure.
	\end{NexMainBox}
	\begin{NexMainBox}[dark, crnA]
		\begin{NexListDark}
			\NexItemDark{\NexOption{Scope}: Suitable for small networks or systems requiring orderly data flow.}
			\NexItemDark{\NexOption{Applications}:
				\begin{NexListLight}
					\NexItemLight{Commonly used in token ring networks for shared bandwidth control.}
					\NexItemLight{Used in certain industrial control systems for predictable communication.}
					\NexItemLight{Facilitates orderly message passing in academic or collaborative networks.}
				\end{NexListLight}
			}
			\NexItemDark{\NexOption{Technology}: Implements network cables (e.g., coaxial or fiber optic) to connect devices in a ring-like structure.}
		\end{NexListDark}
	\end{NexMainBox}
\end{NexMainBox}

\newpage
\subsubsection{Bus Topology}
\label{Bus Topology}
\begin{NexMainBox}[light, hdrA, sdwA, crnA, grwB, sssecA]
	\begin{NexMainBox}[dark, crnA]
		Bus topology connects all devices to a single communication line or backbone, allowing data to be transmitted to all devices simultaneously. This topology is cost-effective but prone to collision issues.
	\end{NexMainBox}
	\begin{NexMainBox}[dark, crnA]
		\begin{NexListDark}
			\NexItemDark{\NexOption{Scope}: Ideal for small-scale networks or temporary setups.}
			\NexItemDark{\NexOption{Applications}:
				\begin{NexListLight}
					\NexItemLight{Commonly used in early Ethernet implementations and small networks.}
					\NexItemLight{Useful in labs or testing environments with limited devices.}
					\NexItemLight{Can be applied to situations requiring simple, linear connections.}
				\end{NexListLight}
			}
			\NexItemDark{\NexOption{Technology}: Relies on coaxial cables, terminators at each end of the backbone, and passive devices.}
		\end{NexListDark}
	\end{NexMainBox}
\end{NexMainBox}

\newpage
\subsubsection{Mesh Topology}
\label{Mesh Topology}
\begin{NexMainBox}[light, hdrA, sdwA, crnA, grwB, sssecA]
	\begin{NexMainBox}[dark, crnA]
		Mesh topology connects every device in the network to every other device, creating a highly redundant and fault-tolerant structure. This topology excels in reliability and performance but may require significant resources for setup and maintenance.
	\end{NexMainBox}
	\begin{NexMainBox}[dark, crnA]
		\begin{NexListDark}
			\NexItemDark{\NexOption{Scope}: Well-suited for high-reliability networks such as data centers or critical systems.}
			\NexItemDark{\NexOption{Applications}:
				\begin{NexListLight}
					\NexItemLight{Commonly used in military communication networks for robustness.}
					\NexItemLight{Ideal for decentralized networks requiring minimal downtime.}
					\NexItemLight{Facilitates high-performance systems like IoT networks or enterprise setups.}
				\end{NexListLight}
			}
			\NexItemDark{\NexOption{Technology}: Utilizes extensive cabling or wireless connections to link all nodes, often with advanced routing protocols for optimal efficiency.}
		\end{NexListDark}
	\end{NexMainBox}
\end{NexMainBox}

\newpage
\subsubsection{Hybrid Topology}
\label{Hybrid Topology}
\begin{NexMainBox}[light, hdrA, sdwA, crnA, grwB, sssecA]
	\begin{NexMainBox}[dark, crnA]
		Hybrid topology combines two or more different types of network topologies (e.g., Star, Mesh, and Bus) to meet complex networking requirements. This topology provides flexibility and scalability but may involve higher costs and complexity.
	\end{NexMainBox}
	\begin{NexMainBox}[dark, crnA]
		\begin{NexListDark}
			\NexItemDark{\NexOption{Scope}: Suitable for large-scale enterprise networks or multifaceted systems requiring diverse configurations.}
			\NexItemDark{\NexOption{Applications}:
				\begin{NexListLight}
					\NexItemLight{Used in data centers combining star and mesh elements for redundancy and performance.}
					\NexItemLight{Ideal for university campuses with different zones, each requiring unique topologies.}
					\NexItemLight{Adopted in corporate environments for segmenting departments with varied network demands.}
				\end{NexListLight}
			}
			\NexItemDark{\NexOption{Technology}: Utilizes multiple networking devices, such as routers, switches, and wireless access points, to implement a blend of topologies.}
		\end{NexListDark}
	\end{NexMainBox}
\end{NexMainBox}

