\newpage
\section{Networking Hardware}
\label{Networking Hardware}
\begin{NexMainBox}[light, hdrA, sdwA, crnA, grwB, secA]
	\begin{NexMainBox}[dark, crnA]
	\end{NexMainBox}
	\begin{NexMainBox}[dark, crnA]
		\begin{NexListDark}
			\NexItemDark{\NexLink{Types of Network Devices}{Types of Network Devices}: Categorizes key devices used in networks, such as Layer 2 and Layer 3 switches, routers, hubs, nodes, access points, firewalls, repeaters, and extenders, emphasizing their unique functions and applications.}
			\NexItemDark{\NexLink{Residential Subscriber Units (RSUs)}{Residential Subscriber Units (RSUs)}: Hardware devices forming the interface between residential users and networks, including modems, routers, antennas, and set-top boxes for broadband, satellite, and wireless services.}
			\NexItemDark{\NexLink{Customer-Premises Equipment (CPE)}{Customer-Premises Equipment (CPE)}: Covers hardware installed at subscriber locations, such as modems, routers, residential subscriber units (RSUs), and set-top boxes, providing connectivity for broadband, multimedia, and voice services.}
		\end{NexListDark}
	\end{NexMainBox}
\end{NexMainBox}

\newpage
\subsection{Types of Network Devices}
\label{Types of Network Devices}
\begin{NexMainBox}[light, hdrA, sdwA, crnA, grwB, ssecB]
	\begin{NexMainBox}[dark, crnA]
		Network devices form the backbone of communication systems, enabling data transmission, routing, and management within and between networks. These devices vary widely in functionality, scale, and application.
	\end{NexMainBox}
	\begin{NexMainBox}[dark, crnA]
		\begin{NexListDark}
			\NexItemDark{\NexLink{Layer 2 Switches}{Layer 2 Switches}: Operate at the Data Link Layer, forwarding traffic based on MAC addresses. Commonly used in local area networks (LANs) for efficient packet management.}
			\NexItemDark{\NexLink{Layer 3 Switches}{Layer 3 Switches}: Combine Layer 2 switching with Layer 3 routing capabilities, enabling IP-based data forwarding in larger, segmented networks.}
			\NexItemDark{\NexLink{Home Routers}{Home Routers}: Typically part of RSU or CPE setups. May support wireless standards like 802.11n (Wi-Fi 4), 802.11ac (Wi-Fi 5), and 802.11ax (Wi-Fi 6), operating on 2.4 GHz, 5 GHz, or 6 GHz frequencies.}
			\NexItemDark{\NexLink{Enterprise Routers}{Enterprise Routers}: High-performance devices with advanced routing protocols (e.g., BGP, OSPF), featuring multiple network interfaces to support complex topologies.}
			\NexItemDark{\NexLink{Standalone Access Points}{Standalone Access Points}: Extend wireless connectivity by allowing devices to connect to wired networks. Operate on multiple frequency bands depending on the Wi-Fi standard.}
			\NexItemDark{\NexLink{Mesh Access Points}{Mesh Access Points}: Enable seamless coverage in large areas, using dynamic connections between devices to eliminate dead zones.}
			\NexItemDark{\NexLink{Hardware Firewalls}{Hardware Firewalls}: Dedicated devices for traffic monitoring and security enforcement. Protect networks by blocking unauthorized access and mitigating threats.}
			\NexItemDark{\NexLink{Hubs}{Hubs}: Basic Layer 1 devices that connect multiple systems, broadcasting all incoming data to every connected port. Rarely used in modern networks due to inefficiency.}
			\NexItemDark{\NexLink{Repeaters and Extenders}{Repeaters and Extenders}: Amplify signals to improve range in wired and wireless networks. Ideal for eliminating weak signal areas.}
		\end{NexListDark}
	\end{NexMainBox}
\end{NexMainBox}

\subsubsection{Layer 2 Switches}
\label{Layer 2 Switches}
\begin{NexMainBox}[light, hdrA, sdwA, crnA, grwB, sssecA]
	\begin{NexMainBox}[dark, crnA, title=Overview]
		Layer 2 Switches operate at the Data Link Layer (Layer 2) of the OSI model. These devices forward data packets based on MAC (Media Access Control) addresses, ensuring efficient communication within a local area network (LAN). Their ability to segment networks into VLANs (Virtual LANs) enhances traffic management and security.
	\end{NexMainBox}
	\begin{NexMainBox}[dark, crnA, title=Key Features]
		\begin{NexListDark}
			\NexItemDark{Forwarding Decisions: Use MAC addresses to decide the destination port for data packets within a LAN.}
			\NexItemDark{Supports VLANs: Allow network segmentation for improved efficiency and security.}
			\NexItemDark{High Port Density: Provide multiple ports for connecting devices, making them suitable for large-scale LANs.}
			\NexItemDark{Full Duplex Operation: Enable simultaneous transmission and reception of data, optimizing performance.}
		\end{NexListDark}
	\end{NexMainBox}
	\begin{NexMainBox}[dark, crnA, title=Applications]
		\begin{NexListDark}
			\NexItemDark{Enterprise LANs: Commonly deployed in offices for connecting computers, printers, and servers.}
			\NexItemDark{Campus Networks: Facilitate communication between devices across multiple buildings.}
			\NexItemDark{Access Layer: Act as the first layer in network hierarchies, connecting end-user devices.}
		\end{NexListDark}
	\end{NexMainBox}
\end{NexMainBox}

\subsubsection{Layer 3 Switches}
\label{Layer 3 Switches}
\begin{NexMainBox}[light, hdrA, sdwA, crnA, grwB, sssecA]
	\begin{NexMainBox}[dark, crnA, title=Overview]
		Layer 3 Switches combine the functionalities of Layer 2 switches and routers by operating at both the Data Link Layer (Layer 2) and the Network Layer (Layer 3) of the OSI model. These devices can route data packets based on IP addresses while also enabling high-speed switching for efficient traffic management within a local or wide area network (LAN/WAN).
	\end{NexMainBox}
	\begin{NexMainBox}[dark, crnA, title=Key Features]
		\begin{NexListDark}
			\NexItemDark{Inter-VLAN Routing: Enable communication between VLANs, eliminating the need for separate routers.}
			\NexItemDark{IP-Based Forwarding: Route traffic between networks using IP addresses for Layer 3 routing.}
			\NexItemDark{Integrated Routing Protocols: Support dynamic protocols like OSPF (Open Shortest Path First) and RIP (Routing Information Protocol).}
			\NexItemDark{High Port Density: Provide multiple interfaces for connecting devices and segments in large networks.}
			\NexItemDark{Wire-Speed Routing: Perform routing at the same speed as Layer 2 switching, minimizing latency.}
		\end{NexListDark}
	\end{NexMainBox}
	\begin{NexMainBox}[dark, crnA, title=Applications]
		\begin{NexListDark}
			\NexItemDark{Enterprise Networks: Facilitate routing between VLANs in large organizations.}
			\NexItemDark{Campus Networks: Integrate Layer 3 functionality to support complex hierarchical network designs.}
			\NexItemDark{Data Centers: Manage traffic between servers, storage systems, and external networks.}
			\NexItemDark{Remote Office Connectivity: Act as intermediate devices for connecting distributed locations to central networks.}
		\end{NexListDark}
	\end{NexMainBox}
\end{NexMainBox}

\begin{NexMainBox}[light, hdrA, sdwA, crnA, grwB, sssecB]
	\begin{NexMainBox}[dark, crnA, title=Advantages]
		\begin{NexListDark}
			\NexItemDark{Cost Efficiency: Combine switching and routing functionalities in a single device.}
			\NexItemDark{Enhanced Performance: Deliver wire-speed routing for high-throughput networks.}
			\NexItemDark{Simplified Network Design: Reduce hardware requirements by integrating routing and switching.}
			\NexItemDark{Flexibility: Adapt to various network architectures, from small businesses to large enterprises.}
		\end{NexListDark}
	\end{NexMainBox}
\end{NexMainBox}

\subsubsection{Home Routers}
\label{Home Routers}
\begin{NexMainBox}[light, hdrA, sdwA, crnA, grwB, sssecA]
	\begin{NexMainBox}[dark, crnA, title=Overview]
		Home Routers are integral devices within residential networks, providing internet access and connectivity for multiple devices. These routers often integrate routing, wireless access, and basic security features into a single unit, making them an essential part of customer-premises equipment (CPE).
	\end{NexMainBox}
	\begin{NexMainBox}[dark, crnA, title=Key Features]
		\begin{NexListDark}
			\NexItemDark{Wireless Standards: Support a range of Wi-Fi protocols, such as:
				\begin{NexListLight}
					\NexItemLight{\textbf{802.11n (Wi-Fi 4)}: Operates on the 2.4 GHz band, offering widespread compatibility.}
					\NexItemLight{\textbf{802.11ac (Wi-Fi 5)}: Supports both 2.4 GHz and 5 GHz bands for dual-band functionality.}
					\NexItemLight{\textbf{802.11ax (Wi-Fi 6)}: Enhances efficiency, speed, and capacity, with additional support for the 6 GHz band in Wi-Fi 6E.}
				\end{NexListLight}
			}
			\NexItemDark{Routing Capability: Use NAT (Network Address Translation) to connect private networks to the internet via a single public IP address.}
			\NexItemDark{Integrated Switch: Include a built-in Layer 2 switch with multiple Ethernet ports for wired connectivity.}
			\NexItemDark{Basic Security: Provide firewall capabilities, WPA/WPA2 wireless encryption, and parental controls.}
		\end{NexListDark}
	\end{NexMainBox}
	\begin{NexMainBox}[dark, crnA, title=Applications]
		\begin{NexListDark}
			\NexItemDark{Residential Networks: Connect devices such as computers, smartphones, smart TVs, and IoT devices to the internet.}
			\NexItemDark{Small Home Offices (SOHO): Provide reliable connectivity for work-from-home setups.}
			\NexItemDark{Media Streaming: Enable smooth video streaming and online gaming experiences.}
		\end{NexListDark}
	\end{NexMainBox}
\end{NexMainBox}

\begin{NexMainBox}[light, hdrA, sdwA, crnA, grwB, sssecB]
	\begin{NexMainBox}[dark, crnA, title=Advantages]
		\begin{NexListDark}
			\NexItemDark{Ease of Use: Simple installation and user-friendly interfaces, often supported by mobile apps.}
			\NexItemDark{Compact Design: Combines multiple functionalities in a single device, reducing clutter.}
			\NexItemDark{Cost-Effective: Affordable solutions for meeting residential networking needs.}
			\NexItemDark{Flexibility: Support for both wired and wireless connections, catering to diverse device types.}
		\end{NexListDark}
	\end{NexMainBox}
\end{NexMainBox}

\subsubsection{Enterprise Routers}
\label{Enterprise Routers}
\begin{NexMainBox}[light, hdrA, sdwA, crnA, grwB, sssecA]
	\begin{NexMainBox}[dark, crnA, title=Overview]
		Enterprise Routers are advanced devices designed to manage large-scale, complex networks. Operating at the Network Layer (Layer 3) of the OSI model, these routers enable the efficient routing of data packets across different networks and geographic locations, ensuring high performance, scalability, and reliability.
	\end{NexMainBox}
	\begin{NexMainBox}[dark, crnA, title=Key Features]
		\begin{NexListDark}
			\NexItemDark{High Throughput: Support large volumes of traffic, making them ideal for enterprise and data center environments.}
			\NexItemDark{Multiple Interfaces: Include a variety of network interfaces (e.g., Ethernet, fiber-optic, serial) to connect diverse network segments.}
			\NexItemDark{Dynamic Routing Protocols: Employ advanced protocols like BGP (Border Gateway Protocol), OSPF (Open Shortest Path First), and EIGRP (Enhanced Interior Gateway Routing Protocol) for efficient routing decisions.}
			\NexItemDark{QoS (Quality of Service): Prioritize traffic based on policies to ensure critical applications (e.g., VoIP, video conferencing) receive adequate bandwidth.}
			\NexItemDark{Redundancy and Failover: Provide features like dual power supplies and multiple routing paths to ensure network uptime.}
			\NexItemDark{Integrated Security: Include built-in firewalls, VPN (Virtual Private Network) support, and intrusion detection systems.}
		\end{NexListDark}
	\end{NexMainBox}
	\begin{NexMainBox}[dark, crnA, title=Applications]
		\begin{NexListDark}
			\NexItemDark{Corporate Networks: Connect multiple branches and offices, ensuring seamless communication and data exchange.}
			\NexItemDark{Data Centers: Act as core network devices, handling traffic between servers and external networks.}
			\NexItemDark{Cloud Connectivity: Facilitate secure and high-speed connections to cloud platforms and services.}
			\NexItemDark{WAN Optimization: Improve the performance of wide-area networks by reducing latency and enhancing data transfer speeds.}
		\end{NexListDark}
	\end{NexMainBox}
\end{NexMainBox}

\begin{NexMainBox}[light, hdrA, sdwA, crnA, grwB, sssecB]
	\begin{NexMainBox}[dark, crnA, title=Advantages]
		\begin{NexListDark}
			\NexItemDark{Scalability: Designed to handle the growth of enterprise networks with support for additional interfaces and users.}
			\NexItemDark{Reliability: Equipped with failover mechanisms to minimize downtime in case of hardware or software failures.}
			\NexItemDark{Performance: Capable of managing high-bandwidth traffic for mission-critical applications.}
			\NexItemDark{Flexibility: Support a wide range of configurations and protocols, making them adaptable to various network architectures.}
		\end{NexListDark}
	\end{NexMainBox}
\end{NexMainBox}

\newpage
\subsection{Residential Subscriber Units (RSUs)}
\label{Residential Subscriber Units (RSUs)}
\begin{NexMainBox}[light, hdrA, sdwA, crnA, grwB, ssecB]
	\begin{NexMainBox}[dark, crnA]
		Residential Subscriber Units (RSUs) serve as the interface between residential users and network providers, enabling connectivity for internet, voice, and multimedia services. These units are integral components of Customer-Premises Equipment (CPE) in telecommunication systems.
	\end{NexMainBox}
	\begin{NexMainBox}[dark, crnA]
		\begin{NexListDark}
			\NexItemDark{\NexOption{Components}: RSUs are composed of various hardware elements tailored to specific technologies and services, such as:
				\begin{NexListLight}
					\NexItemLight{Modems: Facilitate communication between the subscriber's home and the service provider's network.}
					\NexItemLight{Routers: Manage internal data distribution within homes, connecting multiple devices to the network.}
					\NexItemLight{Antennas: Enable fixed wireless or satellite communication by receiving and transmitting signals.}
					\NexItemLight{Set-Top Boxes: Provide broadcasting and multimedia services to residential users.}
				\end{NexListLight}
			}
			\NexItemDark{\NexOption{Applications}: RSUs serve various use cases, including:
				\begin{NexListLight}
					\NexItemLight{Fixed Wireless Access: RSUs with outdoor antennas enable connectivity in areas lacking wired infrastructure.}
					\NexItemLight{Broadband Delivery: RSUs like routers and modems offer internet access for homes.}
					\NexItemLight{Satellite Communication: RSUs include satellite dishes and receiver units for internet and broadcasting.}
					\NexItemLight{Smart Homes: RSUs integrate with IoT devices, allowing seamless home automation.}
				\end{NexListLight}
			}
			\NexItemDark{\NexOption{Advantages}: RSUs provide several benefits to residential users:
				\begin{NexListLight}
					\NexItemLight{Cost-Effective: Affordable designs make RSUs accessible for widespread use.}
					\NexItemLight{Ease of Use: Simple installation and user-friendly interfaces.}
					\NexItemLight{Enhanced Performance: Integration of technologies like MIMO and OFDM improves connectivity and reliability.}
				\end{NexListLight}
			}
		\end{NexListDark}
	\end{NexMainBox}
\end{NexMainBox}

\newpage
\subsection{Customer-Premises Equipment (CPE)}
\label{Customer-Premises Equipment (CPE)}
\begin{NexMainBox}[light, hdrA, sdwA, crnA, grwB, ssecB]
	\begin{NexMainBox}[dark, crnA]
		Customer-Premises Equipment (CPE) refers to the hardware devices installed at a subscriber's location to enable access to telecommunication services. These devices serve as the bridge between the end-user and the service provider network, ensuring reliable connectivity.
	\end{NexMainBox}
	\begin{NexMainBox}[dark, crnA]
		\begin{NexListDark}
			\NexItemDark{\NexOption{Examples of CPE Devices}: Includes:
				\begin{NexListLight}
					\NexItemLight{Modems: Facilitate communication between the user's premises and the broadband network.}
					\NexItemLight{Routers: Distribute internet connectivity within homes and offices.}
					\NexItemLight{Residential Subscriber Units (RSUs): Connect subscribers in fixed wireless and satellite systems.}
					\NexItemLight{Set-Top Boxes: Provide broadcasting and multimedia services to residential users.}
					\NexItemLight{VoIP Phones: Enable voice communication over internet protocols.}
				\end{NexListLight}
			}
			\NexItemDark{\NexOption{Applications of CPE}: Enables:
				\begin{NexListLight}
					\NexItemLight{Broadband internet access for residential and business users.}
					\NexItemLight{Television broadcasting and live streaming services.}
					\NexItemLight{Voice over IP (VoIP) communication for enhanced call quality.}
					\NexItemLight{Smart home connectivity through integration with IoT devices.}
					\NexItemLight{Wireless communication for remote locations.}
				\end{NexListLight}
			}
			\NexItemDark{\NexOption{Advantages of CPE}: Provides:
				\begin{NexListLight}
					\NexItemLight{Customizability: Tailored setups for different user needs and environments.}
					\NexItemLight{Ease of Use: Plug-and-play functionality for subscribers.}
					\NexItemLight{Scalability: Compatibility with advanced technologies like 5G and fiber-optics.}
					\NexItemLight{Affordability: Designed to balance cost-effectiveness with high performance.}
				\end{NexListLight}
			}
		\end{NexListDark}
	\end{NexMainBox}
\end{NexMainBox}

