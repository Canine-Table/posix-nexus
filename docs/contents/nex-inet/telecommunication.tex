\newpage
\section{Telecommunication}
\label{Telecommunication}
\begin{NexMainBox}[light, hdrA, sdwA, crnA, grwB, secA]
	\begin{NexMainBox}[dark, crnA]
	\end{NexMainBox}
	\begin{NexMainBox}[dark, crnA]
		\begin{NexListDark}
			\NexItemDark{\NexLink{Technologies}{Technologies}: Explores the use of geostationary, low Earth orbit (LEO), and medium Earth orbit (MEO) satellites in communication systems, enabling global coverage and bridging connectivity gaps.}
			%\NexItemDark{\NexLink{Satellite Telecommunications}{Satellite Telecommunications}: Describes the use of satellites for global communication, bridging connectivity gaps in remote areas and supporting applications like internet access, broadcasting, and disaster response.}
			%\NexItemDark{\NexLink{Mathematical Concepts in Satellite Telecommunications}{Mathematical Concepts in Satellite Telecommunications}: Highlights formulas for latency, signal strength, and channel capacity to optimize satellite-based communication systems.}
			%\NexItemDark{\NexLink{Limitations of Satellite Telecommunications}{Limitations of Satellite Telecommunications}: Discusses challenges like high latency, weather-related disruptions, and the significant setup costs of satellite systems.}
		\end{NexListDark}
	\end{NexMainBox}
\end{NexMainBox}

\newpage
\subsection{Technologies}
\label{Technologies}
\begin{NexMainBox}[light, hdrA, sdwA, crnA, grwB, ssecA]
	\begin{NexMainBox}[dark, crnA]
		Telecommunication Technologies encompass a wide range of systems and tools used to transmit information across distances. These technologies form the backbone of global communication, supporting internet, voice, and multimedia services.
	\end{NexMainBox}
	\begin{NexMainBox}[dark, crnA]
		\begin{NexListDark}
			\NexItemDark{\NexLink{Satellite Telecommunications}{Satellite Telecommunications}: Uses satellites in geostationary, low Earth orbit (LEO), and medium Earth orbit (MEO) to provide communication services, ensuring connectivity across vast geographical areas. Supports applications like internet access in remote areas, television broadcasting, and emergency communication.}
			\NexItemDark{\NexLink{Fiber-Optic Communication}{Fiber-Optic Communication}: Employs light waves to transmit data through optical fibers, enabling high-speed and low-latency connections. Fiber-optic technology is widely used for backbone networks due to its scalability and resistance to electromagnetic interference.}
			\NexItemDark{\NexLink{Mobile Telecommunications}{Mobile Telecommunications}: Facilitates wireless communication through cellular networks like 4G LTE, 5G, and beyond. Enables mobile connectivity for voice calls, video streaming, and IoT devices with increasing speeds and lower latencies.}
			\NexItemDark{\NexLink{Radio and Microwave Communication}{Radio and Microwave Communication}: Uses electromagnetic waves for point-to-point communication, especially in long-distance scenarios. Commonly applied in broadcasting, satellite links, and backhaul connections for mobile networks.}
		\end{NexListDark}
	\end{NexMainBox}
\end{NexMainBox}

\newpage
\subsubsection{Satellite Telecommunications}
\label{Satellite Telecommunications}
\begin{NexMainBox}[light, hdrA, sdwA, crnA, grwB, sssecA]
	\begin{NexMainBox}[dark, crnA]
		Satellite Telecommunications leverage orbiting satellites to facilitate communication across vast distances, providing vital connectivity in regions where terrestrial networks are unavailable. This technology plays a crucial role in bridging global communication gaps and supporting essential services.
	\end{NexMainBox}
	\begin{NexMainBox}[dark, crnA]
		\begin{NexListDark}
			\NexItemDark{\NexOption{Types of Satellites}:
				\begin{NexListLight}
					\NexItemLight{Geostationary Satellites: Positioned ~35,786 km above Earth, they remain fixed relative to the Earth's surface and are ideal for broadcasting and high-coverage communication.}
					\NexItemLight{Low Earth Orbit (LEO) Satellites: Orbit at altitudes between ~160-2,000 km, offering low latency and high data rates, commonly used in modern satellite internet systems.}
					\NexItemLight{Medium Earth Orbit (MEO) Satellites: Orbit at altitudes of ~2,000-35,786 km, suitable for GPS and navigation systems.}
				\end{NexListLight}
			}
			\NexItemDark{\NexOption{Applications}:
				\begin{NexListLight}
					\NexItemLight{Internet Access: Provides connectivity to remote and rural regions where terrestrial networks are absent.}
					\NexItemLight{Broadcasting: Supports satellite television, radio, and live event streaming globally.}
					\NexItemLight{Emergency Communication: Enables communication in disaster zones where conventional infrastructure is compromised.}
					\NexItemLight{Aviation and Maritime Communication: Facilitates real-time communication for aircraft and ships in transit.}
					\NexItemLight{Military Operations: Provides secure communication channels for strategic operations.}
				\end{NexListLight}
			}
			\NexItemDark{\NexOption{Advantages}:
				\begin{NexListLight}
					\NexItemLight{Global Coverage: Ensures connectivity in even the most remote locations.}
					\NexItemLight{Mobility: Supports communication for moving platforms such as ships, planes, and vehicles.}
					\NexItemLight{Redundancy: Serves as a backup during terrestrial network outages.}
				\end{NexListLight}
			}
		\end{NexListDark}
	\end{NexMainBox}
\end{NexMainBox}

\begin{NexMainBox}[light, hdrA, sdwA, crnA, grwB, sssecB]
	\begin{NexMainBox}[dark, crnA, title=Signal Travel Time]
		We calculate the round-trip latency for signals traveling between Earth and satellites using:
		\begin{NexMainBox}[light]
			$$\NexVariable{t} = \frac{2\NexVariable{d}}{\NexConstant{c}}$$
		\end{NexMainBox}
		\begin{NexListDark}
			\NexItemDark{\NexVariable{\LARGE $t$} $\to$ Total signal travel time ($seconds$)}
			\NexItemDark{\NexVariable{\LARGE $d$} $\to$ Distance from Earth to satellite ($meters$)}
			\NexItemDark{\NexConstant{\LARGE $c$} $\to$ Speed of light $(3 \times 10^8 m/s)$}
			\NexItemDark{\NexOption{LEO} ($\NexVariable{d} \approx \text{500 km}$):
				\begin{NexMainBox}[light]
 					$$\NexVariable{t} = \frac{2 \times (\NexVariable{d} \times 10^3)}{\NexConstant{c}} = 0.0033 \, \text{seconds (3.3 milliseconds)}$$
				\end{NexMainBox}
			}
			\NexItemDark{\NexOption{MEO} ($\NexVariable{d} \approx \text{20,000 km}$):
				\begin{NexMainBox}[light]
 					$$\NexVariable{t} = \frac{2 \times (\NexVariable{d} \times 10^3)}{\NexConstant{c}} = 0.133 \, \text{seconds (133 milliseconds)}$$
				\end{NexMainBox}
			}
			\NexItemDark{\NexOption{Geostationary} ($\NexVariable{d} \approx \text{35,786 km}$):
				\begin{NexMainBox}[light]
 					$$\NexVariable{t} = \frac{2 \times (\NexVariable{d} \times 10^3)}{\NexConstant{c}} = 0.239 \, \text{seconds (239 milliseconds)}$$
				\end{NexMainBox}
			}
		\end{NexListDark}
	\end{NexMainBox}
\end{NexMainBox}

\begin{NexMainBox}[light, hdrA, sdwA, crnA, grwB, sssecB]
	\begin{NexMainBox}[dark, crnA, title=Coverage Area]
		We calculate the coverage area for a satellite based on its altitude and the Earth's radius using:
		\begin{NexMainBox}[light]
			$$ \NexVariable{r} = \sqrt{(\NexVariable{h} + \NexConstant{R_{\text{Earth}}})^2 - \NexConstant{R_{\text{Earth}}}^2} $$
		\end{NexMainBox}
		\begin{NexListDark}
			\NexItemDark{\NexVariable{\LARGE $r$} $\to$ Maximum coverage radius ($km$).}
			\NexItemDark{\NexVariable{\LARGE $h$} $\to$ Satellite altitude ($km$).}
			\NexItemDark{\NexConstant{\LARGE $R_{\text{Earth}}$} $\to$ Earth's radius ($6371 \, \text{km}$).}
			\NexItemDark{\NexOption{LEO} ($\NexVariable{h} \approx 500 \, \text{km}$):
				\begin{NexMainBox}[light]
					$$ \NexVariable{r} = \sqrt{(500 + 6371)^2 - 6371^2} = 2560 \, \text{km} $$
				\end{NexMainBox}
			}
			\NexItemDark{\NexOption{MEO} ($\NexVariable{h} \approx 20,000 \, \text{km}$):
				\begin{NexMainBox}[light]
					$$ \NexVariable{r} = \sqrt{(20,000 + 6371)^2 - 6371^2} = 17,423 \, \text{km} $$
				\end{NexMainBox}
			}
			\NexItemDark{\NexOption{Geostationary} ($\NexVariable{h} \approx 35,786 \, \text{km}$):
				\begin{NexMainBox}[light]
					$$ \NexVariable{r} = \sqrt{(35,786 + 6371)^2 - 6371^2} = 35,618 \, \text{km} $$
				\end{NexMainBox}
			}
		\end{NexListDark}
	\end{NexMainBox}
\end{NexMainBox}
\begin{NexMainBox}[light, hdrA, sdwA, crnA, grwB, sssecB]
	\begin{NexMainBox}[dark, crnA, title=Orbital Period]
		We calculate the orbital period of a satellite using Kepler's Third Law:
		\begin{NexMainBox}[light]
			$$ \NexVariable{T} = 2\NexPI \sqrt{\frac{\NexVariable{h} + \NexROTE^3}{\NexGOTE} $$
		\end{NexMainBox}
		\begin{NexListDark}
			\NexItemDark{\NexVariable{\LARGE $T$} $\to$ Orbital period ($seconds$).}
			\NexItemDark{\NexVariable{\LARGE $h$} $\to$ Satellite altitude ($meters$).}
			\NexItemDark{\NexConstant{\LARGE $R_{\text{Earth}}$} $\to$ Earth's radius ($6371 \, \text{km}$).}
			\NexItemDark{\NexConstant{\LARGE $G_{\text{Earth}}$} $\to$ Earth's gravitational constant ($3.986 \times 10^{14} \, \text{m}^3/\text{s}^2$).}
			\NexItemDark{\NexOption{LEO} ($\NexVariable{h} \approx 500 \, \text{km}$):
				\begin{NexMainBox}[light]
					$$ \NexVariable{T} = 2\NexPI \sqrt{\frac{(500 + 6371)^3}{3.986 \times 10^{14}}} = \text{~90 minutes} $$
				\end{NexMainBox}
			}
			\NexItemDark{\NexOption{MEO} ($\NexVariable{h} \approx 20,000 \, \text{km}$):
				\begin{NexMainBox}[light]
					$$ \NexVariable{T} = 2\NexPI \sqrt{\frac{(20,000 + 6371)^3}{3.986 \times 10^{14}}} = \text{~12 hours} $$
				\end{NexMainBox}
			}
			\NexItemDark{\NexOption{Geostationary} ($\NexVariable{h} \approx 35,786 \, \text{km}$):
				\begin{NexMainBox}[light]
					$$ \NexVariable{T} = 2\NexPI \sqrt{\frac{(35,786 + 6371)^3}{3.986 \times 10^{14}}} = \text{~24 hours} $$
				\end{NexMainBox}
			}
		\end{NexListDark}
	\end{NexMainBox}
\end{NexMainBox}


\begin{comment}
\newpage
\subsection{Satellite}
\label{Satellite}
\begin{NexMainBox}[light, hdrA, sdwA, crnA, grwB, ssecA]
	\begin{NexMainBox}[dark, crnA]
		Satellite Telecommunications rely on orbiting satellites to facilitate communication over vast distances. This technology bridges connectivity gaps in remote and underserved areas, playing a critical role in global communication infrastructure.
	\end{NexMainBox}
	\begin{NexMainBox}[dark, crnA]
		\begin{NexListDark}
			\NexItemDark{\NexOption{Scope}: Enables global communication, including regions without terrestrial network coverage.}
			\NexItemDark{\NexOption{Applications}:
				\begin{NexListLight}
					\NexItemLight{Facilitates internet access in remote and rural areas.}
					\NexItemLight{Supports television broadcasting and live streaming.}
					\NexItemLight{Provides communication for aviation, maritime, and military operations.}
					\NexItemLight{Enables disaster recovery and emergency response services.}
				\end{NexListLight}
			}
			\NexItemDark{\NexOption{Technology}: Utilizes geostationary, low Earth orbit (LEO), and medium Earth orbit (MEO) satellites to achieve global coverage.}
			\NexItemDark{\NexOption{Mathematical Concepts}:
				\begin{NexListLight}
					\NexItemLight{Signal travel time: Explains latency calculations for signals traveling between Earth and satellites.}
					\NexItemLight{Link budget: Illustrates how signal strength varies based on distance, transmitter power, and antenna gains.}
					\NexItemLight{Channel capacity: Highlights bandwidth efficiency and SNR implications.}
				\end{NexListLight}
			}
		\end{NexListDark}
	\end{NexMainBox}
\end{NexMainBox}

\newpage
\subsection{Mathematical Concepts}
\label{Mathematical Concepts}
\begin{NexMainBox}[light, hdrA, sdwA, crnA, grwB, ssecA]
	\begin{NexMainBox}[dark, crnA]
		Mathematical calculations form the basis for understanding and optimizing satellite communication systems. These concepts help quantify latency, signal strength, and data transmission rates.
	\end{NexMainBox}
	\begin{NexMainBox}[dark, crnA]
		\begin{NexListDark}
			\NexItemDark{\NexOption{Signal Travel Time}: Calculates round-trip latency for signals between Earth and satellites, based on distance and the speed of light.}
			\NexItemDark{\NexOption{Link Budget}: Quantifies signal strength accounting for transmission power, antenna gain, and path loss.}
			\NexItemDark{\NexOption{Channel Capacity}: Determines the maximum data rate achievable given bandwidth and signal-to-noise ratio (SNR).}
		\end{NexListDark}
	\end{NexMainBox}
\end{NexMainBox}\newpage
\end{comment}
