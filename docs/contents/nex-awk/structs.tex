\newpage
\section{Structures}
\label{Structures}
\begin{NexMainBox}[light, hdrA, sdwA, crnA, grwB, secA]
	\begin{NexMainBox}[dark, crnA]
		The following functions provide comprehensive utilities for creating, managing, and manipulating structured data like arrays and hashmaps, enabling efficient operations across indexed elements.
	\end{NexMainBox}
	\begin{NexMainBox}[dark, crnA]
		\begin{NexListDark}
			\NexItemDark{\NexLink{nx_bijective}{nx_bijective(\NexOption{V}, \NexOption{D1}, \NexOption{B}, \NexOption{D2})}: Manages bijective mappings within an associative array (\NexOption{V}), allowing the creation, updating, or removal of one-to-one relationships between keys and values.}
			\NexItemDark{\NexLink{nx_find_index}{nx_find_index(\NexOption{D1}, \NexOption{S}, \NexOption{D2})}: Searches for the first occurrence of a pattern within a string, with additional constraints. Returns the index of the match or modifies behavior based on optional parameters.}
			\NexItemDark{\NexLink{nx_next_pair}{nx_next_pair(\NexOption{D1}, \NexOption{V1}, \NexOption{V2}, \NexOption{D2}, \NexOption{B1}, \NexOption{B2})}: Retrieves the next pair of start and end indices within a string (\NexOption{D1}), based on associative array delimiters (\NexOption{V1}). Outputs indices and their lengths to the result vector (\NexOption{V2}), while handling escape constraints (\NexOption{D2}). Logic flags (\NexOption{B1}, \NexOption{B2}) control fallback behavior and prioritization during evaluation.}
			\NexItemDark{\NexLink{nx_tokenize}{nx_tokenize(\NexOption{D1}, \NexOption{V1}, \NexOption{V2}, \NexOption{D2}, \NexOption{B1}, \NexOption{B2})}: Tokenizes an input string (\NexOption{D1}) based on start and end delimiters (\NexOption{V2}). Extracted tokens are stored in the output vector (\NexOption{V1}). Handles escape sequences (\NexOption{D2}) and allows prioritization or fallback using logical flags (\NexOption{B1}, \NexOption{B2}).}
			\NexItemDark{\NexLink{nx_vector}{nx_vector(\NexOption{D}, \NexOption{V1}, \NexOption{S}, \NexOption{V2})}: Processes a data string (\NexOption{D}) by tokenizing it into parts using a delimiter (\NexOption{S}) and associative array mappings (\NexOption{V2}). Uses \NexFunction{__nx_quote_map} for initialization and \NexFunction{nx_tokenize} for parsing.}
			\NexItemDark{\NexLink{nx_trim_vector}{nx_trim_vector(\NexOption{D}, \NexOption{V1}, \NexOption{S}, \NexOption{V2})}: Tokenizes a data string (\NexOption{D}) into parts using delimiters and mappings (\NexOption{V2}), then trims each token to remove unnecessary whitespace or characters.}
			\NexItemDark{\NexLink{nx_uniq_vector}{nx_uniq_vector(\NexOption{D}, \NexOption{V1}, \NexOption{S}, \NexOption{V2}, \NexOption{B1}, \NexOption{B2})}: Constructs a unique vector (\NexOption{V1}) from input data (\NexOption{D}), eliminating duplicates, optionally counting occurrences, and leveraging auxiliary vectors (\NexOption{V2}) for processing.}
			\NexItemDark{\NexLink{nx_length}{nx_length(\NexOption{V}, \NexOption{B})}: Calculates the length of elements within a vector (\NexOption{V}) and returns the largest or smallest length based on the logical condition (\NexOption{B}).}
			\NexItemDark{\NexLink{nx_boundary}{nx_boundary(\NexOption{D}, \NexOption{V1}, \NexOption{V2}, \NexOption{B1}, \NexOption{B2})}: Filters elements in a vector (\NexOption{V1}) that match boundary conditions with a given string (\NexOption{D}), and stores the results in another vector (\NexOption{V2}). Provides options to specify matching direction (\NexOption{B1}) and optionally delete the input vector (\NexOption{B2}).}
			\NexItemDark{\NexLink{nx_filter}{nx_filter(\NexOption{D1}, \NexOption{D2}, \NexOption{V1}, \NexOption{V2}, \NexOption{B})}: Filters elements from an input vector (\NexOption{V1}) based on a flexible equality condition (\NexOption{D1}, \NexOption{D2}) and stores the results in an output vector (\NexOption{V2}). Provides an option (\NexOption{B}) to delete the input vector after processing.}
			\NexItemDark{\NexLink{nx_option}{nx_option(\NexOption{D}, \NexOption{V1}, \NexOption{V2}, \NexOption{B1}, \NexOption{B2})}: Determines a selected index from an input vector (\NexOption{V1}) based on boundary conditions (\NexOption{D}) and advanced filtering logic. Stores intermediate results in a secondary vector (\NexOption{V2}), applying conditions (\NexOption{B1}, \NexOption{B2}) to refine the output.}
		\end{NexListDark}
	\end{NexMainBox}
\end{NexMainBox}

\newpage
\subsection{nx_bijective}
\label{nx_bijective}
\begin{NexCodeBox}{awk}{title={\NexFunction{nx_bijective(\NexOption{V}, \NexOption{D1}, \NexOption{B}, \NexOption{D2})}}}
function nx_bijective(V, D1, B, D2) {
	if (length(V) && D1 != "") {
		if (D2 != "") {
			if (length(B)) {
				V[D1] = B
				V[B] = D2
				if (B)
					V[D2] = D1
			} else {
				V[D1] = D2
				V[D2] = D1
			}
		} else {
			V[V[D1]] = D1
			if (B)
				delete V[D1]
		}
	}
}
\end{NexCodeBox}

\begin{NexMainBox}[light, hdrA, sdwA, crnA, grwB, ssecB]
	\begin{NexMainBox}[dark, crnA]
		The \NexFunction{nx_bijective} function manages bijective mappings within an associative array (\NexOption{V}). It supports adding, updating, and removing one-to-one relationships, ensuring both keys and values are interconnected consistently.
	\end{NexMainBox}
	\begin{NexMainBox}[dark, crnA]
		\begin{NexListDark}
			\NexItemDark{\NexOption{V}: The associative array storing mappings. Keys are linked to values and vice versa.}
			\NexItemDark{\NexOption{D1}: The primary key for establishing or modifying mappings.}
			\NexItemDark{\NexOption{B}: Optional intermediate key used for chaining multi-step mappings.}
			\NexItemDark{\NexOption{D2}: The value paired with \NexOption{D1}. An empty \NexOption{D2} triggers removal.}
		\end{NexListDark}
	\end{NexMainBox}
\end{NexMainBox}

\begin{NexCodeBox}{awk}{title={Adding a Multi-Step Mapping}}
	V["D1"] = ""
	V["B"] = ""
	V["D2"] = ""
	nx_bijective(V, "key1", "middle", "key2")
	# Result:
	# V["key1"] = "middle"
	# V["middle"] = "key2"
	# V["key2"] = "key1"
\end{NexCodeBox}

\begin{NexCodeBox}{awk}{title={Updating Reverse Mapping}}
	V["key1"] = "middle"
	V["middle"] = "key2"
	V["key2"] = "key1"
	nx_bijective(V, "key1", "", "")
	# Result:
	# V["key2"] = "key1"
	# V["key1"] is deleted if B is truthy.
\end{NexCodeBox}

\begin{NexCodeBox}{awk}{title={Removing a Mapping}}
	V["key1"] = "middle"
	V["middle"] = "key2"
	V["key2"] = "key1"
	nx_bijective(V, "key1", "", "")
	# Result:
	# Deletes V["key1"] and adjusts the reverse mapping.
\end{NexCodeBox}

\newpage
\subsection{nx_find_index}
\label{nx_find_index}
\begin{NexCodeBox}{bash}{title={\NexFunction{nx_find_index(\NexOption{D1}, \NexOption{S}, \NexOption{D2})}}}
function nx_find_index(D1, S, D2,	f, m)
{
	if (__nx_defined(D1, 1)) {
		f = 0
		S = __nx_else(S, " ")
		D2 = __nx_else(__nx_defined(D2, 1), "\\\\")
		while (match(D1, S)) {
			f = f + RSTART
			if (! (match(substr(D1, 1, RSTART - 1), D2 "+$") && D2) || int(RLENGTH % 2) == 0)
				break
			f = f + RLENGTH
			D1 = substr(D1, f + 1)
		}
		return f
	}
}
\end{NexCodeBox}

\begin{NexMainBox}[light, hdrA, sdwA, crnA, grwB, ssecB]
	\begin{NexMainBox}[dark, crnA]
		Searches for the first match of a given pattern (\NexOption{S}) within a string (\NexOption{D1}) while applying optional constraints (\NexOption{D2}). The function handles fallback conditions and uses nuanced logic to account for escape characters and repeated patterns.
	\end{NexMainBox}
	\begin{NexMainBox}[dark, crnA]
		\begin{NexListDark}
			\NexItemDark{\NexOption{D1}: The input string to search.}
			\NexItemDark{\NexOption{S}: The primary pattern to search for. Defaults to the space character (` `).}
			\NexItemDark{\NexOption{D2}: An optional secondary pattern used to constrain matches (e.g., escape sequences). Defaults to the backslash (`\backslash`).}
		\end{NexListDark}
	\end{NexMainBox}
\end{NexMainBox}

\begin{NexCodeBox}{bash}{title={Basic pattern matching}}
	D1 = "hell\o world"
	S = "o"
	result = nx_find_index(D1, S)
	# result is 8
	# Explanation: The first occurrence of "o" in "hello world" is excaped, the next occurence is at index 8.
\end{NexCodeBox}

\begin{NexCodeBox}{bash}{title={No match for the pattern}}
	D1 = "hello world"
	S = "z"
	result = nx_find_index(D1, S)
	# result is 0
	# Explanation: Since "z" doesn't exist in the string, the function returns 0.
\end{NexCodeBox}

\begin{NexCodeBox}{bash}{title={Default parameters}}
	D1 = "this is an example"
	result = nx_find_index(D1)
	# result is 5
	# Explanation: The default pattern `S` is a space character, and the first space is at index 5.
\end{NexCodeBox}

\begin{NexCodeBox}{bash}{title={Complex string with escape sequences}}
	D1 = "path\\to\\file"
	S = "\\\\"
	D2 = "\\\\"
	result = nx_find_index(D1, S, D2)
	# result is 5
	# Explanation: The function navigates the string while respecting escape constraints and finds the first valid match.
\end{NexCodeBox}

\newpage
\subsection{nx_next_pair}
\label{nx_next_pair}
\begin{NexCodeBox}{bash}{title={\NexFunction{nx_next_pair(\NexOption{D1}, \NexOption{V1}, \NexOption{V2}, \NexOption{D2}, \NexOption{B1}, \NexOption{B2})}}}
function nx_next_pair(D1, V1, V2, D2, B1, B2, s, s_l, e, e_l, f, i) {
	if (length(V1) && D1 != "") {
		for (i in V1) {
			if ((f = nx_find_index(D1, i, D2)) && (! s || __nx_if(B2, f > s, f < s))) {
				s = f
				s_l = length(i)
				if (length(V1[i]) && (f = nx_find_index(substr(D1, s + s_l + 1), V1[i], D2))) {
					e = f
					e_l = length(V1[i])
				} else {
					e = ""
					e_l = ""
				}
			}
		}
		if (! s && B1) {
			s = length(D1) + 1
		}
		V2[++V2[0]] = s
		V2[V2[0] "_" s] = s_l
		V2[++V2[0]] = e
		V2[V2[0] "_" e] = e_l
		return V2[0] - 1
	}
}
\end{NexCodeBox}

\begin{NexMainBox}[light, hdrA, sdwA, crnA, grwB, ssecB]
	\begin{NexMainBox}[dark, crnA]
		Retrieves the next pair of start and end indices from the input string (\NexOption{D1}) based on specified delimiters (\NexOption{V1}). Stores indices and their lengths in the output vector (\NexOption{V2}) for subsequent operations. Handles escape sequences (\NexOption{D2}) and prioritizes pairs based on logical conditions (\NexOption{B1}, \NexOption{B2}).
	\end{NexMainBox}
	\begin{NexMainBox}[dark, crnA]
		\begin{NexListDark}
			\NexItemDark{\NexOption{D1}: The input string to search for start and end pairs.}
			\NexItemDark{\NexOption{V1}: An associative array mapping start delimiters (keys) to end delimiters (values).}
			\NexItemDark{\NexOption{V2}: A vector to store indices and lengths of matched pairs.}
			\NexItemDark{\NexOption{D2}: Constraints for handling escape sequences or specific delimiters.}
			\NexItemDark{\NexOption{B1}: A flag to set a fallback start index if none is found.}
			\NexItemDark{\NexOption{B2}: A logical parameter to prioritize pairs based on index comparison (above or below the previous match).}
		\end{NexListDark}
	\end{NexMainBox}
\end{NexMainBox}

\begin{NexCodeBox}{bash}{title={Matching a single pair of delimiters}}
	D1 = "<pair start content end />"
	V1["<pair start"] = "end />"
	V2[0] = 0
	result = nx_next_pair(D1, V1, V2)
	# Result:
	# V2[1] = 1, V2[1_s] = 11
	# V2[2] = 23, V2[2_e] = 6
	# Explanation: Finds the start and end delimiters, capturing their indices and lengths.
\end{NexCodeBox}

\begin{NexCodeBox}{bash}{title={Handling fallback start index}}
	D1 = "no delimiters here"
	V1["<start"] = "end />"
	V2[0] = 0
	result = nx_next_pair(D1, V1, V2, "", 1, 0)
	# Result:
	# V2[1] = 21, V2[1_s] = ""
	# V2[2] = "", V2[2_e] = ""
	# Explanation: Sets the fallback start index as the length of D1 + 1 since no match was found.
\end{NexCodeBox}

\begin{NexCodeBox}{bash}{title={Multiple pairs with prioritization}}
	D1 = "<startA>contentA<endA><startB>contentB<endB>"
	V1["<startA>"] = "<endA>"
	V1["<startB>"] = "<endB>"
	result = nx_next_pair(D1, V1, V2, "", 0, 1)
	# Result:
	# V2[1] = 1, V2[1_s] = 8
	# V2[2] = 20, V2[2_e] = 7
	# Explanation: Prioritizes pairs based on index comparison and logical conditions.
\end{NexCodeBox}

\newpage
\subsection{nx_tokenize}
\label{nx_tokenize}
\begin{NexCodeBox}{bash}{title={\NexFunction{nx_tokenize(\NexOption{D1}, \NexOption{V1}, \NexOption{V2}, \NexOption{D2}, \NexOption{B1}, \NexOption{B2})}}}
function nx_tokenize(D1, V1, V2, D2, B1, B2, i, j, m, v, s) {
	if (length(V2) && D1 != "") {
		while (D1) {
			i = nx_next_pair(D1, V2, v, D2, 1, B1)
			m = substr(D1, v[i], v[i "_" v[i]])
			j = v[i] + v[i "_" v[i]]
			s = s substr(D1, 1, v[i] - 1)
			if (length(V2[m])) {
				s = s substr(D1, j, v[++i])
				j = j + v[i] + v[i "_" v[i]]
			} else {
				V1[++V1[0]] = s
				s = ""
			}
			D1 = substr(D1, j)
		}
		if (s != "")
			V1[++V1[0]] = s
		delete v
		return V1[0]
	}
}
\end{NexCodeBox}

\begin{NexMainBox}[light, hdrA, sdwA, crnA, grwB, ssecB]
	\begin{NexMainBox}[dark, crnA]
		Processes an input string (\NexOption{D1}) to extract and tokenize content into an output vector (\NexOption{V1}). Tokens are defined based on start and end pairs in (\NexOption{V2}). Handles constraints (\NexOption{D2}) and logical prioritization (\NexOption{B1}, \NexOption{B2}).
	\end{NexMainBox}
	\begin{NexMainBox}[dark, crnA]
		\begin{NexListDark}
			\NexItemDark{\NexOption{D1}: The input string to be tokenized.}
			\NexItemDark{\NexOption{V1}: A vector to store tokenized outputs. Each entry represents a complete token.}
			\NexItemDark{\NexOption{V2}: An associative array containing start and end delimiters for tokenization.}
			\NexItemDark{\NexOption{D2}: Constraints to handle escape sequences or special rules during tokenization.}
			\NexItemDark{\NexOption{B1}: A flag to determine logical prioritization for the next token (e.g., based on position).}
			\NexItemDark{\NexOption{B2}: A flag to control fallback behavior or logical prioritization when matching pairs.}
		\end{NexListDark}
	\end{NexMainBox}
\end{NexMainBox}

\begin{NexCodeBox}{bash}{title={Tokenizing with basic delimiters}}
	D1 = "token1, token2, token3"
	V2[""] = ","
	V1[0] = 0
	result = nx_tokenize(D1, V1, V2, "")
	# Result:
	# V1[1] = "token1"
	# V1[2] = "token2"
	# V1[3] = "token3"
	# Explanation: Tokenizes based on the delimiter "," and stores the tokens in V1.
\end{NexCodeBox}

\begin{NexCodeBox}{bash}{title={Handling nested delimiters}}
	D1 = "start1(content1)end1 start2(content2)end2"
	V2["start1"] = "end1"
	V2["start2"] = "end2"
	V1[0] = 0
	result = nx_tokenize(D1, V1, V2, "")
	# Result:
	# V1[1] = "content1"
	# V1[2] = "content2"
	# Explanation: Matches "start1" with "end1" and "start2" with "end2", extracting content between them.
\end{NexCodeBox}

\begin{NexCodeBox}{bash}{title={Tokenizing with escapes}}
	D1 = "token1\,token2, token3"
	V2[""] = ","
	V1[0] = 0
	result = nx_tokenize(D1, V1, V2, "\\")
	# Result:
	# V1[1] = "token1\,token2"
	# V1[2] = "token3"
	# Explanation: Accounts for escaped commas ("\,") and ensures they are not treated as delimiters.
\end{NexCodeBox}

\newpage
\subsection{nx_vector}
\label{nx_vector}
\begin{NexCodeBox}{bash}{title={\NexFunction{nx_vector(\NexOption{D}, \NexOption{V1}, \NexOption{S}, \NexOption{V2})}}}
function nx_vector(D, V1, S, V2) {
	__nx_quote_map(V2)
	V2[__nx_else(S, ",")] = ""
	return nx_tokenize(D, V1, V2)
}
\end{NexCodeBox}

\begin{NexMainBox}[light, hdrA, sdwA, crnA, grwB, ssecB]
	\begin{NexMainBox}[dark, crnA]
		The \NexFunction{nx_vector} function tokenizes an input string (\NexOption{D}) based on tokens (\NexOption{V1}), a delimiter (\NexOption{S}), and custom mappings (\NexOption{V2}). It uses \texttt{__nx_quote_map} to initialize mappings in \NexOption{V2} and \texttt{nx_tokenize} to perform the tokenization.
	\end{NexMainBox}
	\begin{NexMainBox}[dark, crnA]
		\begin{NexListDark}
			\NexItemDark{\NexOption{D}: The input data string to be tokenized.}
			\NexItemDark{\NexOption{V1}: A vector for storing the parsed tokens.}
			\NexItemDark{\NexOption{S}: The delimiter for separating tokens. Defaults to \texttt{','} if unspecified.}
			\NexItemDark{\NexOption{V2}: An associative array for custom mappings, initialized using \NexFunction{__nx_quote_map}.}
			\NexItemDark{\NexOption{Return Value}: The processed vector (\NexOption{V1}) containing the extracted tokens.}
		\end{NexListDark}
	\end{NexMainBox}
\end{NexMainBox}

\begin{NexCodeBox}{bash}{title={Tokenizing a simple list}}
	D = "item1, item2, item3"
	V1[0] = 0
	S = ","
	__nx_quote_map(V2)
	result = nx_vector(D, V1, S, V2)
	# Result:
	# V1[1] = "item1"
	# V1[2] = "item2"
	# V1[3] = "item3"
	# Explanation: Splits the input string by the delimiter "," and stores tokens in \NexOption{V1}.
\end{NexCodeBox}

\begin{NexCodeBox}{bash}{title={Using custom mappings}}
	D = "value1|value2,value3"
	V1[0] = 0
	S = "|"
	__nx_quote_map(V2)
	V2[","] = ""
	result = nx_vector(D, V1, S, V2)
	# Result:
	# V1[1] = "value1"
	# V1[2] = "value2,value3"
	# Explanation: Splits D by "|", but retains commas in tokens due to custom mapping in V2.
\end{NexCodeBox}

\newpage
\subsection{nx_trim_vector}
\label{nx_trim_vector}
\begin{NexCodeBox}{bash}{title={\NexFunction{nx_trim_vector(\NexOption{D}, \NexOption{V1}, \NexOption{S}, \NexOption{V2})}}}
function nx_trim_vector(D, V1, S, V2,   i)
{
	nx_vector(D, V1, S, V2)
	for (i = 1; i <= V1[0]; i++)
		V1[i] = nx_trim_str(V1[i])
	return V1[0]
}
\end{NexCodeBox}

\begin{NexMainBox}[light, hdrA, sdwA, crnA, grwB, ssecB]
	\begin{NexMainBox}[dark, crnA]
		Processes a data string (\NexOption{D}) by tokenizing it into parts using delimiters and associative array mappings (\NexOption{V2}), and then trims each token to remove any leading or trailing whitespace or unwanted characters.
	\end{NexMainBox}
	\begin{NexMainBox}[dark, crnA]
		\begin{NexListDark}
			\NexItemDark{\NexOption{D}: The input string to be tokenized and trimmed.}
			\NexItemDark{\NexOption{V1}: A vector to store the processed and trimmed tokens.}
			\NexItemDark{\NexOption{S}: The delimiter used to split the input string (\NexOption{D}) into tokens. Defaults to `,`.}
			\NexItemDark{\NexOption{V2}: An associative array mapping delimiters or escape sequences used during tokenization.}
		\end{NexListDark}
	\end{NexMainBox}
\end{NexMainBox}

\begin{NexCodeBox}{bash}{title={Trimming tokens from a string}}
	D = "  token1 ,  token2 ,  token3  "
	V1[0] = 0
	S = ","
	result = nx_trim_vector(D, V1, S)
	# Result:
	# V1[1] = "token1"
	# V1[2] = "token2"
	# V1[3] = "token3"
	# Explanation: Splits the input string on commas, trims whitespace from each token, and stores them in the vector V1.
\end{NexCodeBox}

\newpage
\subsection{nx_uniq_vector}
\label{nx_uniq_vector}
\begin{NexCodeBox}{awk}{title={\NexFunction{nx_uniq_vector(\NexOption{D}, \NexOption{V1}, \NexOption{S}, \NexOption{V2}, \NexOption{B1}, \NexOption{B2})}}}
function nx_uniq_vector(D, V1, S, V2, B1, B2, v, i) {
	if (B1)
		i = nx_vector(D, v, S, V2)
	else
		i = nx_trim_vector(D, v, S, V2)
	if (i) {
		for (i = 1; i <= v[0]; i++) {
			if (! (v[i] in V1)) {
				V1[v[i]] = __nx_if(B2, 0, ++V1[0])
				V1[V1[0]] = v[i]
			}
			if (B2)
				V1[v[i]]++
		}
		i = V1[0]
	}
	delete v
	return i
}
\end{NexCodeBox}

\begin{NexMainBox}[light, hdrA, sdwA, crnA, grwB, ssecB]
	\begin{NexMainBox}[dark, crnA]
		The \NexFunction{nx_uniq_vector} function creates a unique vector (\NexOption{V1}) from input data (\NexOption{D}). It ensures that only distinct elements are added, with optional occurrence counts, while leveraging auxiliary vectors for processing.
	\end{NexMainBox}
	\begin{NexMainBox}[dark, crnA]
		\begin{NexListDark}
			\NexItemDark{\NexOption{D}: Input data to be processed for unique elements.}
			\NexItemDark{\NexOption{V1}: Output vector storing unique elements and optional occurrence counts.}
			\NexItemDark{\NexOption{S}: Separator used for splitting or processing input data.}
			\NexItemDark{\NexOption{V2}: Auxiliary vector used during the processing of \NexOption{D}.}
			\NexItemDark{\NexOption{B1}: Flag indicating whether to use \NexFunction{nx_vector} or \NexFunction{nx_trim_vector} for processing.}
			\NexItemDark{\NexOption{B2}: Optional flag for maintaining occurrence counts of unique elements.}
		\end{NexListDark}
	\end{NexMainBox}
\end{NexMainBox}

\begin{NexCodeBox}{awk}{title={Processing with occurrence counts}}
	D = "apple,orange,apple,banana,orange"
	V1[0] = 0
	S = ","
	V2[0] = 0
	nx_uniq_vector(D, V1, S, V2, 1, 1)
	# Result:
	# V1[1] = "apple", V1["apple"] = 2
	# V1[2] = "orange", V1["orange"] = 2
	# V1[3] = "banana", V1["banana"] = 1
\end{NexCodeBox}

\begin{NexCodeBox}{awk}{title={Processing without occurrence counts}}
	D = "apple,orange,apple,banana,orange"
	V1[0] = 0
	S = ","
	V2[0] = 0
	nx_uniq_vector(D, V1, S, V2, 0, 0)
	# Result:
	# V1[1] = "apple"
	# V1[2] = "orange"
	# V1[3] = "banana"
\end{NexCodeBox}


\newpage
\subsection{nx_length}
\label{nx_length}
\begin{NexCodeBox}{bash}{title={\NexFunction{nx_length(\NexOption{V}, \NexOption{B})}}}
function nx_length(V, B,	i, j, k)
{
	if (length(V) && 0 in V) {
		for (i = 1; i <= V[0]; i++) {
			j = length(V[i])
			if (! k || __nx_if(B, k < j, k > j))
				k = j
		}
		return int(k)
	}
}
\end{NexCodeBox}

\begin{NexMainBox}[light, hdrA, sdwA, crnA, grwB, ssecB]
	\begin{NexMainBox}[dark, crnA]
		Calculates the length of elements within an input vector (\NexOption{V}) and determines the largest or smallest length based on a given logical condition (\NexOption{B}).
	\end{NexMainBox}
	\begin{NexMainBox}[dark, crnA]
		\begin{NexListDark}
			\NexItemDark{\NexOption{V}: The input vector containing elements whose lengths need to be evaluated.}
			\NexItemDark{\NexOption{B}: A logical flag that determines whether to return the smallest length (if `false`) or the largest length (if `true`).}
		\end{NexListDark}
	\end{NexMainBox}
\end{NexMainBox}

\begin{NexCodeBox}{bash}{title={Finding the largest length}}
	V[0] = 3
	V[1] = "short"
	V[2] = "longer"
	V[3] = "lengthiest"
	B = 1
	result = nx_length(V, B)
	# Result: 10
	# Explanation: Evaluates the lengths of elements in V and returns the largest value, which is 10 (from "lengthiest").
\end{NexCodeBox}

\begin{NexCodeBox}{bash}{title={Finding the smallest length}}
	V[0] = 3
	V[1] = "short"
	V[2] = "longer"
	V[3] = "lengthiest"
	B = 0
	result = nx_length(V, B)
	# Result: 5
	# Explanation: Evaluates the lengths of elements in V and returns the smallest value, which is 5 (from "short").
\end{NexCodeBox}

\newpage
\subsection{nx_boundary}
\label{nx_boundary}
\begin{NexCodeBox}{bash}{title={\NexFunction{nx_boundary(\NexOption{D}, \NexOption{V1}, \NexOption{V2}, \NexOption{B1}, \NexOption{B2})}}}
function nx_boundary(D, V1, V2, B1, B2,	 i)
{
	if (length(V) && 0 in V && D != "") {
		for (i = 1; i <= V1[0]; i++) {
			if (__nx_if(B1, V1[i] ~ D "$", V1[i] ~ "^" D))
				V2[++V2[0]] = V1[i]
		}
		if (B2)
			delete V1
		return V2[0]
	}
}
\end{NexCodeBox}

\begin{NexMainBox}[light, hdrA, sdwA, crnA, grwB, ssecB]
	\begin{NexMainBox}[dark, crnA]
		Filters elements from an input vector (\NexOption{V1}) that match the boundary conditions specified by a string (\NexOption{D}). Results are stored in an output vector (\NexOption{V2}), with the option to prioritize start or end boundary matching (\NexOption{B1}). The function can optionally delete the input vector (\NexOption{B2}) after processing.
	\end{NexMainBox}
	\begin{NexMainBox}[dark, crnA]
		\begin{NexListDark}
			\NexItemDark{\NexOption{D}: The string used to define boundary conditions for filtering elements.}
			\NexItemDark{\NexOption{V1}: The input vector containing elements to be filtered.}
			\NexItemDark{\NexOption{V2}: The output vector to store elements matching the boundary conditions.}
			\NexItemDark{\NexOption{B1}: A logical flag to specify boundary matching. If `true`, matches elements ending with (\NexOption{D}); if `false`, matches elements starting with (\NexOption{D}).}
			\NexItemDark{\NexOption{B2}: A flag to delete the input vector (\NexOption{V1}) after filtering, if set to `true`.}
		\end{NexListDark}
	\end{NexMainBox}
\end{NexMainBox}

\begin{NexCodeBox}{bash}{title={Filtering elements ending with a boundary}}
	V1[1] = "boundary_end"
	V1[2] = "no_match"
	V1[3] = "another_end"
	D = "end"
	B1 = 1
	nx_boundary(D, V1, V2, B1, 0)
	# Result:
	# V2[1] = "boundary_end"
	# V2[2] = "another_end"
	# Explanation: Filters elements in V1 that end with "end" and stores them in V2.
\end{NexCodeBox}

\begin{NexCodeBox}{bash}{title={Filtering elements starting with a boundary}}
	V1[1] = "start_boundary"
	V1[2] = "no_match"
	V1[3] = "start_another"
	D = "start"
	B1 = 0
	nx_boundary(D, V1, V2, B1, 0)
	# Result:
	# V2[1] = "start_boundary"
	# V2[2] = "start_another"
	# Explanation: Filters elements in V1 that start with "start" and stores them in V2.
\end{NexCodeBox}

\begin{NexCodeBox}{bash}{title={Deleting the input vector after filtering}}
	V1[1] = "boundary_delete"
	V1[2] = "no_match"
	V1[3] = "delete_end"
	D = "delete"
	B1 = 1
	B2 = 1
	nx_boundary(D, V1, V2, B1, B2)
	# Result:
	# V2[1] = "delete_end"
	# V1: Cleared
	# Explanation: Filters elements in V1 matching the boundary condition "delete" (ending with "delete"), stores "delete_end" in V2, and clears V1 after processing due to B2 being set to true.
\end{NexCodeBox}

\newpage
\subsection{nx_filter}
\label{nx_filter}
\begin{NexCodeBox}{bash}{title={\NexFunction{nx_filter(\NexOption{D1}, \NexOption{D2}, \NexOption{V1}, \NexOption{V2}, \NexOption{B})}}}
function nx_filter(D1, D2, V1, V2, B,   i, v1, v2)
{
	if (length(V1) && 0 in V1) {
		for (i = 1; i <= V1[0]; i++) {
			if (__nx_equality(D1, D2, V1[i]))
				V2[++V2[0]] = V1[i]
		}
		if (B)
			delete V1
		return V2[0]
	}
}
\end{NexCodeBox}

\begin{NexMainBox}[light, hdrA, sdwA, crnA, grwB, ssecB]
	\begin{NexMainBox}[dark, crnA]
		Filters elements from an input vector (\NexOption{V1}) based on a flexible equality condition defined by (\NexOption{D1}) and (\NexOption{D2}). Matching elements are appended to an output vector (\NexOption{V2}). The function supports optional deletion of the input vector (\NexOption{V1}) after filtering to optimize memory usage.
	\end{NexMainBox}
	\begin{NexMainBox}[dark, crnA]
		\begin{NexListDark}
			\NexItemDark{\NexOption{D1}: The primary value or pattern used for comparison.}
			\NexItemDark{\NexOption{D2}: The secondary value or pattern used to refine the comparison.}
			\NexItemDark{\NexOption{V1}: The input vector containing elements to be filtered.}
			\NexItemDark{\NexOption{V2}: The output vector to store elements that meet the equality condition.}
			\NexItemDark{\NexOption{B}: A flag to delete the input vector (\NexOption{V1}) after processing if set to `true`.}
		\end{NexListDark}
	\end{NexMainBox}
\end{NexMainBox}

\begin{NexCodeBox}{bash}{title={Filtering elements with a simple equality condition}}
	V1[1] = "apple"
	V1[2] = "orange"
	V1[3] = "apple"
	D1 = "apple"
	D2 = "="
	nx_filter(D1, D2, V1, V2, 0)
	# Result:
	# V2[1] = "apple"
	# V2[2] = "apple"
	# Explanation: Filters elements in V1 that are equal to "apple" and stores them in V2.
\end{NexCodeBox}

\begin{NexCodeBox}{bash}{title={Filtering elements with a numeric condition}}
	V1[1] = 10
	V1[2] = 20
	V1[3] = 30
	D1 = 15
	D2 = ">"
	nx_filter(D1, D2, V1, V2, 0)
	# Result:
	# V2[1] = 20
	# V2[2] = 30
	# Explanation: Filters elements in V1 greater than 15 and stores them in V2.
\end{NexCodeBox}

\begin{NexCodeBox}{bash}{title={Deleting the input vector after filtering}}
	V1[1] = "match"
	V1[2] = "no_match"
	V1[3] = "match"
	D1 = "match"
	D2 = "="
	B = 1
	nx_filter(D1, D2, V1, V2, B)
	# Result:
	# V2[1] = "match"
	# V2[2] = "match"
	# V1: Cleared
	# Explanation: Filters elements in V1 that match "match", stores them in V2, and clears V1 after processing.
\end{NexCodeBox}

\newpage
\subsection{nx_option}
\label{nx_option}
\begin{NexCodeBox}{bash}{title={\NexFunction{nx_option(\NexOption{D}, \NexOption{V1}, \NexOption{V2}, \NexOption{B1}, \NexOption{B2})}}}
function nx_option(D, V1, V2, B1, B2,   i, v1)
{
	if (length(V1) && 0 in V1) {
		if (nx_boundary(D, V1, v1, B1) > 1) {
			if (nx_filter(nx_append_str("0", nx_length(v1, B2)), "=_", v1, V2, 1) == 1) {
				i = V2[1]
				delete V2
				return i
			}
		} else {
			i = v1[1]
			delete v1
			return i
		}
	}
}
\end{NexCodeBox}

\begin{NexMainBox}[light, hdrA, sdwA, crnA, grwB, ssecB]
	\begin{NexMainBox}[dark, crnA]
		Selects a string from an input vector (\NexOption{V1}) that matches boundary conditions (\NexOption{D}) and additional logic refinements. The function processes intermediate results using a secondary vector and logical conditions, enabling flexible selection criteria.
	\end{NexMainBox}
	\begin{NexMainBox}[dark, crnA]
		\begin{NexListDark}
			\NexItemDark{\NexOption{D}: A string or pattern used as a boundary condition for selection.}
			\NexItemDark{\NexOption{V1}: The primary input vector containing elements to evaluate.}
			\NexItemDark{\NexOption{V2}: A secondary vector used for intermediate filtering results.}
			\NexItemDark{\NexOption{B1}: A logical flag controlling boundary matching (e.g., start or end).}
			\NexItemDark{\NexOption{B2}: A logical flag influencing the filtering length criteria during refinement.}
		\end{NexListDark}
	\end{NexMainBox}
\end{NexMainBox}

\begin{NexCodeBox}{bash}{title={Selecting a string based on boundary conditions}}
	V1[1] = "start_option"
	V1[2] = "middle_match"
	V1[3] = "end_option"
	V1[0] = 3 # This was populated dynamically earlier
	D = "option"
	B1 = 1
	B2 = 1
	result = nx_option(D, V1, V2, B1, B2)
	# Result: "end_option"
	# Explanation: Filters V1 for elements matching the boundary condition "option" at the end and returns the matching string ("end_option").
\end{NexCodeBox}

\begin{NexCodeBox}{bash}{title={Returning a single matching string}}
	V1[1] = "single_match"
	V1[2] = "no_match"
	V1[0] = 2 # Populated dynamically via earlier functions
	D = "single"
	B1 = 0
	B2 = 1
	result = nx_option(D, V1, V2, B1, B2)
	# Result: "single_match"
	# Explanation: Matches "single_match" based on the boundary condition "single" at the start and returns the matching string.
\end{NexCodeBox}

\begin{NexCodeBox}{bash}{title={No valid boundary match}}
	V1[1] = "no_boundary"
	V1[2] = "no_match"
	V1[0] = 2 # Managed dynamically by earlier functions
	D = "option"
	B1 = 0
	B2 = 0
	result = nx_option(D, V1, V2, B1, B2)
	# Result: None
	# Explanation: No elements in V1 match the boundary condition "option", so the function returns no result.
\end{NexCodeBox}

