\newpage
\section{Miscellaneous}
\label{Miscellaneous}
\begin{NexMainBox}[light, hdrA, sdwA, crnA, grwB, secA]
	\begin{NexMainBox}[dark, crnA]
		The following functions provide generic utilities for handling values and conditional operations, offering flexible solutions for various logical scenarios.
	\end{NexMainBox}
	\begin{NexMainBox}[dark, crnA]
		\begin{NexListDark}
			\NexItemDark{\NexLink{__nx_num_map}{__nx_num_map(\NexOption{V})}: Initializes an associative array (\NexOption{V}) with mappings for digits (\texttt{0-9}), where each digit maps to itself. Ideal for digit-based validations or transformations.}
			\NexItemDark{\NexLink{__nx_lower_map}{__nx_lower_map(\NexOption{V})}: Initializes an associative array (\NexOption{V}) with mappings for digits (\texttt{0-9}) and lowercase letters (\texttt{a-z}), leveraging \NexFunction{nx_bijective} for the letter mappings. Ideal for scenarios involving alphanumeric parsing or transformations.}
			\NexItemDark{\NexLink{__nx_upper_map}{__nx_upper_map(\NexOption{V})}: Extends \NexFunction{__nx_lower_map} by adding bijective mappings for uppercase letters (\texttt{A-Z}) to an associative array (\NexOption{V}), ensuring full alphanumeric coverage.}
			\NexItemDark{\NexLink{__nx_quote_map}{__nx_quote_map(\NexOption{V})}: Initializes an associative array (\NexOption{V}) with mappings for common quote characters (double quotes, single quotes, and backticks). Useful for handling quotes in parsing, escaping, or validation tasks.}
			\NexItemDark{\NexLink{__nx_bracket_map}{__nx_bracket_map(\NexOption{V})}: Initializes an associative array (\NexOption{V}) with mappings for common opening and closing brackets (e.g., \texttt{[} to \texttt{]}, \texttt{\{} to \texttt{\}}, and \texttt{(} to \texttt{)}). Useful for parsing, validation, and other operations involving balanced brackets.}
			\NexItemDark{\NexLink{__nx_str_map}{__nx_str_map(\NexOption{V})}: Initializes an associative array (\NexOption{V}) with mappings for string character classes. These include ranges for uppercase and lowercase letters, digits, hexadecimal characters, alphanumeric characters, printable ASCII characters, and punctuation. Useful for validation, parsing, and string processing.}
			\NexItemDark{\NexLink{__nx_escape_map}{__nx_escape_map(\NexOption{V})}: Initializes an associative array (\NexOption{V}) with mappings for common escape sequences (\texttt{\textbackslash x20}, \texttt{\textbackslash x09}, \texttt{\textbackslash x0a}, \texttt{\textbackslash x0b}, \texttt{\textbackslash x0c}), each assigned to an empty string. Useful for removing whitespace and control characters during string processing.}
			\NexItemDark{\NexLink{__nx_defined}{__nx_defined(\NexOption{D}, \NexOption{B})}: Validates whether \NexOption{D} is defined or evaluates as truthy under additional constraints provided by \NexOption{B}.}
			\NexItemDark{\NexLink{__nx_else}{__nx_else(\NexOption{D1}, \NexOption{D2}, \NexOption{B})}: Returns \NexOption{D1} if it is truthy or satisfies the condition set by \NexOption{B}, otherwise returns \NexOption{D2}.}
			\NexItemDark{\NexLink{__nx_if}{__nx_if(\NexOption{B1}, \NexOption{D1}, \NexOption{D2}, \NexOption{B2})}: Returns \NexOption{D1} if \NexOption{B1} is truthy or satisfies the condition set by \NexOption{B2}, otherwise returns \NexOption{D2}.}
			\NexItemDark{\NexLink{__nx_elif}{__nx_elif(\NexOption{B1}, \NexOption{B2}, \NexOption{B3}, \NexOption{B4}, \NexOption{B5}, \NexOption{B6})}: Evaluates multiple conditions and their relationships, returning a boolean value based on comparisons of the results of \NexFunction{__nx_defined} for the provided inputs.}
		\end{NexListDark}
	\end{NexMainBox}
\end{NexMainBox}

\begin{NexMainBox}[light, hdrA, sdwA, crnA, grwB, secB]
	\begin{NexMainBox}[dark, crnA]
		\begin{NexListDark}
			\NexItemDark{\NexLink{__nx_or}{__nx_or(\NexOption{B1}, \NexOption{B2}, \NexOption{B3}, \NexOption{B4}, \NexOption{B5}, \NexOption{B6})}: Evaluates logical OR conditions between multiple inputs, incorporating constraints applied via \NexFunction{__nx_defined} and fallback adjustments from \NexFunction{__nx_else}.}
			\NexItemDark{\NexLink{__nx_xor}{__nx_xor(\NexOption{B1}, \NexOption{B2}, \NexOption{B3}, \NexOption{B4})}: Evaluates exclusive OR (XOR) conditions between multiple inputs, incorporating constraints applied via \NexFunction{__nx_defined} and fallback adjustments from \NexFunction{__nx_else}.}
			\NexItemDark{\NexLink{__nx_compare}{__nx_compare(\NexOption{B1}, \NexOption{B2}, \NexOption{B3}, \NexOption{B4})}: Compares two inputs based on type, length, or specified comparison rules, leveraging \NexFunction{awk}'s dynamic behavior and optional constraints.}
			\NexItemDark{\NexLink{__nx_equality}{__nx_equality(\NexOption{B1}, \NexOption{B2}, \NexOption{B3})}: Evaluates the equality or relational conditions between \NexOption{B1} and \NexOption{B3} based on the operator specified in \NexOption{B2}, leveraging \NexFunction{awk} behavior for dynamic comparisons.}
			\NexItemDark{\NexLink{__nx_swap}{__nx_swap(\NexOption{V}, \NexOption{D1}, \NexOption{D2})}: Swaps the values of two specified indices in the provided array or associative array. Utilizes a temporary variable to ensure the operation is safe and lossless.}
		\end{NexListDark}
	\end{NexMainBox}
\end{NexMainBox}

\newpage
\subsection{__nx_upper_map}
\label{__nx_upper_map}
\begin{NexCodeBox}{bash}{title={\NexFunction{__nx_upper_map(\NexOption{V})}}}
function __nx_upper_map(V, i) {
	__nx_lower_map(V)
	for (i = 36; i < 62; i++)
		nx_bijective(V, i, "", sprintf("%c", i + 29))
}
\end{NexCodeBox}

\begin{NexMainBox}[light, hdrA, sdwA, crnA, grwB, ssecB]
	\begin{NexMainBox}[dark, crnA]
		The \NexFunction{__nx_upper_map} function extends the mappings established by \NexFunction{__nx_lower_map} by adding bijective mappings for uppercase English letters (\texttt{A-Z}). It combines numerical digit mappings (\texttt{0-9}), lowercase letters (\texttt{a-z}), and uppercase letters (\texttt{A-Z}) into the associative array (\NexOption{V}).
	\end{NexMainBox}
	\begin{NexMainBox}[dark, crnA]
		\begin{NexListDark}
			\NexItemDark{\NexOption{V}: An associative array passed by reference. After execution, it contains mappings for digits (\texttt{0-9}), lowercase letters (\texttt{a-z}), and uppercase letters (\texttt{A-Z}).}
		\end{NexListDark}
	\end{NexMainBox}
\end{NexMainBox}

\newpage
\subsection{__nx_num_map}
\label{__nx_num_map}
\begin{NexCodeBox}{bash}{title={\NexFunction{__nx_num_map(\NexOption{V})}}}
function __nx_num_map(V, i) {
	for (i = 0; i < 10; i++)
		V[i] = i
}
\end{NexCodeBox}

\begin{NexMainBox}[light, hdrA, sdwA, crnA, grwB, ssecB]
	\begin{NexMainBox}[dark, crnA]
		The \NexFunction{__nx_num_map} function initializes an associative array (\NexOption{V}) with mappings for numerical digits, where each digit maps to itself (e.g., \texttt{0} maps to \texttt{0}, \texttt{1} maps to \texttt{1}, etc.). This utility is useful for tasks involving digit-based validations or transformations.
	\end{NexMainBox}
	\begin{NexMainBox}[dark, crnA]
		\begin{NexListDark}
			\NexItemDark{\NexOption{V}: An associative array passed by reference, populated with digit mappings (\texttt{0-9}).}
		\end{NexListDark}
	\end{NexMainBox}
\end{NexMainBox}

\newpage
\subsection{__nx_lower_map}
\label{__nx_lower_map}
\begin{NexCodeBox}{bash}{title={\NexFunction{__nx_lower_map(\NexOption{V})}}}
function __nx_lower_map(V, i) {
	__nx_num_map(V)
	for (i = 10; i < 36; i++)
		nx_bijective(V, i, "", sprintf("%c", i + 87))
}
\end{NexCodeBox}

\begin{NexMainBox}[light, hdrA, sdwA, crnA, grwB, ssecB]
	\begin{NexMainBox}[dark, crnA]
		The \NexFunction{__nx_lower_map} function initializes an associative array (\NexOption{V}) with mappings for numerical digits (\texttt{0-9}) and lowercase English letters (\texttt{a-z}). Numerical digits map to themselves, while lowercase letters are bijectively mapped using their ASCII values.
	\end{NexMainBox}
	\begin{NexMainBox}[dark, crnA]
		\begin{NexListDark}
			\NexItemDark{\NexOption{V}: An associative array passed by reference. After execution, it includes digit-to-digit mappings (\texttt{0-9}) and bijective mappings for lowercase letters (\texttt{a-z}).}
		\end{NexListDark}
	\end{NexMainBox}
\end{NexMainBox}

\subsection{__nx_quote_map}
\label{__nx_quote_map}
\begin{NexCodeBox}{bash}{title={\NexFunction{__nx_quote_map(\NexOption{V})}}}
	function __nx_quote_map(V) {
		V["\""] = "\""
		V["'"] = "'"
		V["`"] = "`"
	}
\end{NexCodeBox}

\begin{NexMainBox}[light, hdrA, sdwA, crnA, grwB, ssecB]
	\begin{NexMainBox}[dark, crnA]
		Initializes an associative array (\NexOption{V}) with common quote characters (double quotes, single quotes, and backticks), mapping each to itself. This is useful for handling quotes in parsing, escaping, or validation tasks.
	\end{NexMainBox}
	\begin{NexMainBox}[dark, crnA]
		\begin{NexListDark}
			\NexItemDark{\NexOption{V}: An associative array passed by reference. After execution, it contains mappings for the following characters:
				\begin{NexListLight}
					\NexItemLight{\texttt{"} (double quote)}
					\NexItemLight{\texttt{' (single quote)}}
					\NexItemLight{\texttt{` (backtick)}}
				\end{NexListLight}
			}
		\end{NexListDark}
	\end{NexMainBox}
\end{NexMainBox}

\newpage
\subsection{__nx_bracket_map}
\label{__nx_bracket_map}
\begin{NexCodeBox}{bash}{title={\NexFunction{__nx_bracket_map(\NexOption{V})}}}
function __nx_bracket_map(V) {
	V["\x5b"] = "\x5d"
	V["\x7b"] = "\x7d"
	V["\x28"] = "\x29"
}
\end{NexCodeBox}

\begin{NexMainBox}[light, hdrA, sdwA, crnA, grwB, ssecB]
	\begin{NexMainBox}[dark, crnA]
		Initializes an associative array (\NexOption{V}) with mappings for common bracket characters. Each opening bracket is mapped to its corresponding closing bracket, using their ASCII hexadecimal representations.
	\end{NexMainBox}
	\begin{NexMainBox}[dark, crnA]
		\begin{NexListDark}
			\NexItemDark{\NexOption{V}: An associative array passed by reference. After execution, it contains the following mappings:
				\begin{NexListLight}
					\NexItemLight{\texttt{\textbackslash x5b} (ASCII for \texttt{[}) maps to \texttt{\textbackslash x5d} (ASCII for \texttt{]}).}
					\NexItemLight{\texttt{\textbackslash x7b} (ASCII for \texttt{\{}) maps to \texttt{\textbackslash x7d} (ASCII for \texttt{\}}).}
					\NexItemLight{\texttt{\textbackslash x28} (ASCII for \texttt{(}) maps to \texttt{\textbackslash x29} (ASCII for \texttt{)}).}
				\end{NexListLight}
			}
		\end{NexListDark}
	\end{NexMainBox}
\end{NexMainBox}

\subsection{__nx_str_map}
\label{__nx_str_map}
\begin{NexCodeBox}{bash}{title={\NexFunction{__nx_str_map(\NexOption{V})}}}
function __nx_str_map(V)
{
	nx_bijective(V, ++V[0], "upper", "A-Z")
	nx_bijective(V, ++V[0], "lower", "a-z")
	nx_bijective(V, ++V[0], "xupper", "A-F")
	nx_bijective(V, ++V[0], "xlower", "a-f")
	nx_bijective(V, ++V[0], "digit", "0-9")
	nx_bijective(V, ++V[0], "alpha", V["upper"] V["lower"])
	nx_bijective(V, ++V[0], "xdigit", V["digit"] V["xupper"] V["xlower"])
	nx_bijective(V, ++V[0], "alnum", V["digit"] V["alpha"])
	nx_bijective(V, ++V[0], "print", "\x20-\x7e")
	nx_bijective(V, ++V[0], "punct", "\x21-\x2f\x3a-\x40\x5b-\x60\x7b-\x7e")
}
\end{NexCodeBox}

\begin{NexMainBox}[light, hdrA, sdwA, crnA, grwB, ssecB]
	\begin{NexMainBox}[dark, crnA]
		Initializes an associative array (\NexOption{V}) with mappings for string character classes. These mappings include ranges for uppercase letters, lowercase letters, digits, hexadecimal characters, printable characters, and punctuation.
	\end{NexMainBox}
	\begin{NexMainBox}[dark, crnA]
		\begin{NexListDark}
			\NexItemDark{\NexOption{V}: An associative array passed by reference. After execution, it contains the following mappings:
				\begin{NexListLight}
					\NexItemLight{\NexOption{"upper"}: Maps to \texttt{"A-Z"} (uppercase English letters).}
					\NexItemLight{\NexOption{"lower"}: Maps to \texttt{"a-z"} (lowercase English letters).}
					\NexItemLight{\NexOption{"xupper"}: Maps to \texttt{"A-F"} (uppercase hexadecimal characters).}
					\NexItemLight{\NexOption{"xlower"}: Maps to \texttt{"a-f"} (lowercase hexadecimal characters).}
					\NexItemLight{\NexOption{"digit"}: Maps to \texttt{"0-9"} (numerical digits).}
					\NexItemLight{\NexOption{"alpha"}: Concatenation of \NexOption{"upper"} and \NexOption{"lower"}; maps to \texttt{"A-Za-z"} (alphabetical characters).}
					\NexItemLight{\NexOption{"xdigit"}: Concatenation of \NexOption{"digit"}, \NexOption{"xupper"}, and \NexOption{"xlower"}; maps to \texttt{"0-9A-Fa-f"} (hexadecimal digits).}
					\NexItemLight{\NexOption{"alnum"}: Concatenation of \NexOption{"digit"} and \NexOption{"alpha"}; maps to \texttt{"0-9A-Za-z"} (alphanumeric characters).}
					\NexItemLight{\NexOption{"print"}: Maps to \texttt{"\textbackslash x20-\textbackslash x7e"} (printable ASCII characters).}
					\NexItemLight{\NexOption{"punct"}: Maps to \texttt{"\textbackslash x21-\textbackslash x2f\textbackslash x3a-\textbackslash x40\textbackslash x5b-\textbackslash x60\textbackslash x7b-\textbackslash x7e"} (punctuation characters within printable ASCII).}
				\end{NexListLight}
			}
		\end{NexListDark}
	\end{NexMainBox}
\end{NexMainBox}

\subsection{__nx_escape_map}
\label{__nx_escape_map}
\begin{NexCodeBox}{bash}{title={\NexFunction{__nx_escape_map(\NexOption{V})}}}
function __nx_escape_map(V) {
	V["\x20"] = ""
	V["\x09"] = ""
	V["\x0a"] = ""
	V["\x0b"] = ""
	V["\x0c"] = ""
}
\end{NexCodeBox}

\begin{NexMainBox}[light, hdrA, sdwA, crnA, grwB, ssecB]
	\begin{NexMainBox}[dark, crnA]
		Initializes an associative array (\NexOption{V}) with mappings for common escape sequences, assigning each escape sequence to an empty string. This is useful for processing or sanitizing strings by removing whitespace and control characters.
	\end{NexMainBox}
	\begin{NexMainBox}[dark, crnA]
		\begin{NexListDark}
			\NexItemDark{\NexOption{V}: An associative array passed by reference. After execution, it contains the following mappings:
				\begin{NexListLight}
					\NexItemLight{\texttt{\textbackslash x20}: Maps to an empty string (ASCII for space).}
					\NexItemLight{\texttt{\textbackslash x09}: Maps to an empty string (ASCII for tab).}
					\NexItemLight{\texttt{\textbackslash x0a}: Maps to an empty string (ASCII for newline).}
					\NexItemLight{\texttt{\textbackslash x0b}: Maps to an empty string (ASCII for vertical tab).}
					\NexItemLight{\texttt{\textbackslash x0c}: Maps to an empty string (ASCII for form feed).}
				\end{NexListLight}
			}
		\end{NexListDark}
	\end{NexMainBox}
\end{NexMainBox}

\newpage
\subsection{__nx_defined}
\label{__nx_defined}
\begin{NexCodeBox}{bash}{title={\NexFunction{__nx_defined(\NexOption{D}, \NexOption{B})}}}
function __nx_defined(D, B) {
	return (D || (length(D) && B))
}
\end{NexCodeBox}

\begin{NexMainBox}[light, hdrA, sdwA, crnA, grwB, ssecB]
	\begin{NexMainBox}[dark, crnA]
		Determines whether \NexOption{D} is defined or evaluates to a truthy value, optionally constrained by \NexOption{B}. Returns a boolean value accordingly.
	\end{NexMainBox}
	\begin{NexMainBox}[dark, crnA]
		\begin{NexListDark}
			\NexItemDark{\NexOption{D}: The variable or value to check for definition or truthiness.}
			\NexItemDark{\NexOption{B}: An optional additional condition for validation when \NexOption{D} has length.}
		\end{NexListDark}
	\end{NexMainBox}
\end{NexMainBox}

\begin{NexCodeBox}{bash}{title={Basic truth check}}
	B1 = 1
	B2 = ""
	result = __nx_defined(B1, B2)
	# result is true, as B1 is true (1) regardless of B2
\end{NexCodeBox}

\begin{NexCodeBox}{bash}{title={Empty string check}}
	B1 = ""
	B2 = 1
	result = __nx_defined(B1, B2)
	# result is false, as B1 is an empty string, which fails the defined check
\end{NexCodeBox}

\begin{NexCodeBox}{bash}{title={String with length check}}
	B1 = "hello"
	B2 = 0
	result = __nx_defined(B1, B2)
	# result is true, as B1 ("hello") is non-empty and therefore defined
\end{NexCodeBox}

\begin{NexCodeBox}{bash}{title={Numeric length check}}
	B1 = 0
	B2 = 1
	result = __nx_defined(B1, B2)
	# result is true, as B1 (0) has a length, even though it evaluates as false in conditions
\end{NexCodeBox}

\begin{NexCodeBox}{bash}{title={Combined truth and fallback check}}
	B1 = ""
	B2 = "fallback"
	result = __nx_defined(B1, B2)
	# result is true, as B2 ("fallback") is defined and compensates for B1 being empty
\end{NexCodeBox}

\newpage
\subsection{__nx_else}
\label{__nx_else}
\begin{NexCodeBox}{bash}{title={\NexFunction{__nx_else(\NexOption{D1}, \NexOption{D2}, \NexOption{B})}}}
function __nx_else(D1, D2, B) {
	if (D1 || __nx_defined(D1, B))
		return D1
	return D2
}
\end{NexCodeBox}

\begin{NexMainBox}[light, hdrA, sdwA, crnA, grwB, ssecB]
	\begin{NexMainBox}[dark, crnA]
		Returns \NexOption{D1} if it is truthy or satisfies the condition set by \NexOption{B}. If neither condition is met, \NexOption{D2} is returned.
	\end{NexMainBox}
	\begin{NexMainBox}[dark, crnA]
		\begin{NexListDark}
			\NexItemDark{\NexOption{D1}: The primary value to evaluate and potentially return.}
			\NexItemDark{\NexOption{D2}: The fallback value returned if \NexOption{D1} does not meet the conditions.}
			\NexItemDark{\NexOption{B}: An optional constraint applied to \NexOption{D1} using \NexFunction{__nx_defined}.}
		\end{NexListDark}
	\end{NexMainBox}
\end{NexMainBox}

\begin{NexCodeBox}{bash}{title={Simple fallback adjustment}}
	B1 = 1
	B2 = 0
	result = __nx_else(B1, B2)
	# result is true, as B1 is true (1), overriding the fallback condition of B2 (0)
\end{NexCodeBox}

\begin{NexCodeBox}{bash}{title={Fallback with string input}}
	B1 = "abc"
	B2 = ""
	result = __nx_else(B1, B2)
	# result is true, as B1 ("abc") is valid and overrides the empty fallback (B2 = "")
\end{NexCodeBox}

\begin{NexCodeBox}{bash}{title={Numeric fallback adjustment}}
	B1 = ""
	B2 = 42
	result = __nx_else(B1, B2)
	# result is true, as B1 fails the condition, falling back to B2 (42)
\end{NexCodeBox}

\begin{NexCodeBox}{bash}{title={Fallback with pattern matching}}
	B1 = "[a-z]+"
	B2 = "hello"
	result = __nx_else(B2 ~ B1, 0)
	# result is true, as the pattern "[a-z]+" matches B2 ("hello"), overriding the fallback (0)
\end{NexCodeBox}

\newpage
\subsection{__nx_if}
\label{__nx_if}
\begin{NexCodeBox}{bash}{title={\NexFunction{__nx_if(\NexOption{B1}, \NexOption{D1}, \NexOption{D2}, \NexOption{B2})}}}
function __nx_if(B1, D1, D2, B2) {
	if (B1 || __nx_defined(B1, B2))
		return D1
	return D2
}
\end{NexCodeBox}

\begin{NexMainBox}[light, hdrA, sdwA, crnA, grwB, ssecB]
	\begin{NexMainBox}[dark, crnA]
		Returns \NexOption{D1} if \NexOption{B1} is truthy or satisfies the condition set by \NexOption{B2}. If neither condition is met, \NexOption{D2} is returned. This function extends conditional operations by integrating the \NexFunction{__nx_defined} utility.
	\end{NexMainBox}
	\begin{NexMainBox}[dark, crnA]
		\begin{NexListDark}
			\NexItemDark{\NexOption{B1}: The primary condition to evaluate for truthiness.}
			\NexItemDark{\NexOption{D1}: The value returned if \NexOption{B1} meets the conditions.}
			\NexItemDark{\NexOption{D2}: The fallback value returned if \NexOption{B1} does not satisfy the conditions.}
			\NexItemDark{\NexOption{B2}: An optional additional constraint applied to \NexOption{B1} using \NexFunction{__nx_defined}.}
		\end{NexListDark}
	\end{NexMainBox}
\end{NexMainBox}

\begin{NexCodeBox}{bash}{title={Basic conditional check}}
	B1 = 1
	B2 = "True Case"
	B3 = "False Case"
	result = __nx_if(B1, B2, B3)
	# result is "True Case", as B1 is true (1), returning the second argument
\end{NexCodeBox}

\begin{NexCodeBox}{bash}{title={Evaluating string-based condition}}
	B1 = "non-empty"
	B2 = "Condition Met"
	B3 = "Condition Not Met"
	result = __nx_if(B1, B2, B3)
	# result is "Condition Met", as B1 is non-empty and therefore true, returning the second argument
\end{NexCodeBox}

\begin{NexCodeBox}{bash}{title={Numeric comparison in conditional check}}
	B1 = (5 > 3)
	B2 = "Greater"
	B3 = "Lesser or Equal"
	result = __nx_if(B1, B2, B3)
	# result is "Greater", as B1 evaluates to true (5 > 3)
\end{NexCodeBox}

\begin{NexCodeBox}{bash}{title={Regex-based condition}}
	B1 = ("abc123" ~ /^[a-z]+[0-9]+$/)
	B2 = "Pattern Matches"
	B3 = "Pattern Doesn't Match"
	result = __nx_if(B1, B2, B3)
	# result is "Pattern Matches", as B1 evaluates to true due to the regex match
\end{NexCodeBox}

\begin{NexCodeBox}{bash}{title={Fallback when condition is false}}
	B1 = 0
	B2 = "Will Not Return"
	B3 = "Fallback Case"
	result = __nx_if(B1, B2, B3)
	# result is "Fallback Case", as B1 is false (0), returning the third argument
\end{NexCodeBox}

\newpage
\subsection{__nx_elif}
\label{__nx_elif}
\begin{NexCodeBox}{bash}{title={\NexFunction{__nx_elif(\NexOption{B1}, \NexOption{B2}, \NexOption{B3}, \NexOption{B4}, \NexOption{B5}, \NexOption{B6})}}}
function __nx_elif(B1, B2, B3, B4, B5, B6) {
	if (B4) {
		B5 = __nx_else(B5, B4)
		B6 = __nx_else(B6, B5)
	}
	return (__nx_defined(B1, B4) == __nx_defined(B2, B5) && __nx_defined(B3, B6) != __nx_defined(B1, B4))
}
\end{NexCodeBox}

\begin{NexMainBox}[light, hdrA, sdwA, crnA, grwB, ssecB]
	\begin{NexMainBox}[dark, crnA]
		Evaluates multiple conditions and their relationships. Returns a boolean value based on comparisons of the outputs from \NexFunction{__nx_defined} applied to the provided inputs.
	\end{NexMainBox}
	\begin{NexMainBox}[dark, crnA]
	\begin{NexListDark}
		\NexItemDark{\NexOption{B1}: The first condition to validate using \NexFunction{__nx_defined}.}
		\NexItemDark{\NexOption{B2}: The second condition to validate using \NexFunction{__nx_defined}.}
		\NexItemDark{\NexOption{B3}: The third condition to validate using \NexFunction{__nx_defined}.}
		\NexItemDark{\NexOption{B4}: Optional constraint applied to subsequent conditions \NexOption{B5} and \NexOption{B6}.}
		\NexItemDark{\NexOption{B5}: Adjusted condition based on \NexOption{B4} if provided, otherwise unchanged.}
		\NexItemDark{\NexOption{B6}: Adjusted condition based on \NexOption{B5} if provided, otherwise unchanged.}
	\end{NexListDark}
	\end{NexMainBox}
\end{NexMainBox}

\begin{NexCodeBox}{bash}{title={Simple relational checks}}
	B1 = 1
	B2 = 2
	B3 = 3
	B4 = 0
	B5 = 1
	B6 = 0
	result = __nx_elif(B1, B2, B3, B4, B5, B6)
	# result is false, as the comparisons of __nx_defined(B1, B4), __nx_defined(B2, B5), and __nx_defined(B3, B6)
	# do not satisfy the logic for XOR relationships
\end{NexCodeBox}

\begin{NexCodeBox}{bash}{title={Pattern matching logic}}
	B1 = "hello"
	B2 = "world"
	B3 = "[a-z]+"
	B4 = 0
	B5 = ""
	B6 = "_"
	result = __nx_elif(B1, B2, B3, B4, B5, B6)
	# result is false, as __nx_defined(B3, B6) evaluates to true with pattern "[a-z]+",
	# but the fallback adjustments for B1 and B2 overlap in truth, violating XOR logic
\end{NexCodeBox}

\begin{NexCodeBox}{bash}{title={Complex nested XOR conditions}}
	B1 = 10
	B2 = 20
	B3 = 30
	B4 = ""
	B5 = "a"
	B6 = "i"
	result = __nx_elif(B1, B2, B3, B4, B5, B6)
	# result is false, as none of the relationships between B1, B2, and B3 fulfill the XOR conditions
	# after fallback adjustments with __nx_else
\end{NexCodeBox}

\begin{NexCodeBox}{bash}{title={Nested condition adjustments}}
	B1 = "abc"
	B2 = "abc"
	B3 = ""
	B4 = "a"
	B5 = "def"
	B6 = ""
	result = __nx_elif(B1, B2, B3, B4, B5, B6)
	# result is true, as __nx_defined(B1, B4) and __nx_defined(B2, B5) are true,
	# and the adjusted relationships satisfy the condition logic
\end{NexCodeBox}

\newpage
\subsection{__nx_or}
\label{__nx_or}
\begin{NexCodeBox}{bash}{title={\NexFunction{__nx_or(\NexOption{B1}, \NexOption{B2}, \NexOption{B3}, \NexOption{B4}, \NexOption{B5}, \NexOption{B6})})}}
function __nx_or(B1, B2, B3, B4, B5, B6) {
	if (B4) {
		B5 = __nx_else(B5, B4)
		B6 = __nx_else(B6, B5)
	}
	return ((__nx_defined(B1, B4) && __nx_defined(B2, B5)) || (__nx_defined(B3, B6) && ! __nx_defined(B1, B4)))
}
\end{NexCodeBox}

\begin{NexMainBox}[light, hdrA, sdwA, crnA, grwB, ssecB]
	\begin{NexMainBox}[dark, crnA]
		Evaluates logical OR conditions between multiple inputs. Uses \NexFunction{__nx_defined} to validate conditions and applies fallback adjustments using \NexFunction{__nx_else} for specific inputs.
	\end{NexMainBox}
	\begin{NexMainBox}[dark, crnA]
		\begin{NexListDark}
			\NexItemDark{\NexOption{B1}: The first condition to evaluate using \NexFunction{__nx_defined}.}
			\NexItemDark{\NexOption{B2}: The second condition to evaluate using \NexFunction{__nx_defined}.}
			\NexItemDark{\NexOption{B3}: The third condition to evaluate using \NexFunction{__nx_defined}.}
			\NexItemDark{\NexOption{B4}: Optional constraint applied to conditions and used in fallback adjustments via \NexFunction{__nx_else}.}
			\NexItemDark{\NexOption{B5}: Adjusted condition based on \NexOption{B4} if provided, otherwise unchanged.}
			\NexItemDark{\NexOption{B6}: Adjusted condition based on \NexOption{B5} if provided, otherwise unchanged.}
		\end{NexListDark}
	\end{NexMainBox}
\end{NexMainBox}

\begin{NexCodeBox}{bash}{title={OR condition for integer validation}}
	B = 1
	N = "-123"
	result = __nx_or(B, N ~ /^([-]|[+])?[0-9]+$/, N ~ /^[0-9]+$/)
	# result is true, as N matches the first regex pattern for signed integers (-123)
\end{NexCodeBox}

\begin{NexCodeBox}{bash}{title={OR condition for decimal number validation}}
	B = 0
	N = "123.45"
	result = __nx_or(B, N ~ /^([-]|[+])?[0-9]+[.][0-9]+$/, N ~ /^[0-9]+[.][0-9]+$/)
	# result is true, as N matches the first regex pattern for signed decimals (123.45)
\end{NexCodeBox}

\newpage
\subsection{__nx_xor}
\label{__nx_xor}
\begin{NexCodeBox}{bash}{title={\NexFunction{__nx_xor(\NexOption{B1}, \NexOption{B2}, \NexOption{B3}, \NexOption{B4})}}}
function __nx_xor(B1, B2, B3, B4) {
	if (B3)
		B4 = __nx_else(B4, B3)
	return ((! __nx_defined(B2, B4) && __nx_defined(B1, B3)) || (__nx_defined(B2, B4) && ! __nx_defined(B1, B3)))
}
\end{NexCodeBox}

\begin{NexMainBox}[light, hdrA, sdwA, crnA, grwB, ssecB]
	\begin{NexMainBox}[dark, crnA]
		Evaluates exclusive OR (XOR) conditions between multiple inputs. Uses \NexFunction{__nx_defined} to validate conditions and applies fallback adjustments using \NexFunction{__nx_else} for specific inputs.
	\end{NexMainBox}
	\begin{NexMainBox}[dark, crnA]
		\begin{NexListDark}
			\NexItemDark{\NexOption{B1}: The first condition to evaluate using \NexFunction{__nx_defined}.}
			\NexItemDark{\NexOption{B2}: The second condition to evaluate using \NexFunction{__nx_defined}.}
			\NexItemDark{\NexOption{B3}: Optional condition used for fallback adjustments via \NexFunction{__nx_else}.}
			\NexItemDark{\NexOption{B4}: Adjusted condition based on \NexOption{B3} if provided, otherwise unchanged.}
		\end{NexListDark}
	\end{NexMainBox}
\end{NexMainBox}

\begin{NexCodeBox}{bash}{title={Basic XOR: Compare B1 and B2}}
	B1 = 1
	B2 = 0
	B3 = ""
	B4 = ""
	result = __nx_xor(B1, B2, B3, B4)
	# result is true, as B1 is true (1) and B2 is false (0), making XOR true
\end{NexCodeBox}

\begin{NexCodeBox}{bash}{title={Complex XOR with fallback adjustment}}
	B1 = 0
	B2 = 0
	B3 = ""
	B4 = "_"
	result = __nx_xor(B1, B2, B3, B4)
	# result is true, as B2 satisfies the fallback condition (B4),
	# while B1 is false, fulfilling XOR
\end{NexCodeBox}

\begin{NexCodeBox}{bash}{title={XOR with both conditions as true}}
	B1 = 10
	B2 = 20
	B3 = 1
	B4 = 1
	result = __nx_xor(B1, B2, B3, B4)
	# result is false, as B1 and B2 both satisfy their respective truth and length conditions, 
	# violating XOR
\end{NexCodeBox}

\begin{NexCodeBox}{bash}{title={String XOR with adjusted truth conditions}}
	B1 = ""
	B2 = "def"
	B3 = 1
	B4 = "a"
	result = __nx_xor(B1, B2, B3, B4)
	# result is true, as B1 ("") fails the truth check required by B3, 
	# while B2 ("def") satisfies the length requirement via B4, fulfilling XOR
\end{NexCodeBox}

\newpage
\subsection{__nx_compare}
\label{__nx_compare}
\begin{NexCodeBox}{bash}{title={\NexFunction{__nx_compare(\NexOption{B1}, \NexOption{B2}, \NexOption{B3}, \NexOption{B4})}}}
function __nx_compare(B1, B2, B3, B4) {
	if (! B3) {
		if (length(B3)) {
			B1 = length(B1)
			B2 = length(B2)
			B3 = 1
		} else if (__nx_is_digit(B1, 1) && __nx_is_digit(B2, 1)) {
			B1 = +B1
			B2 = +B2
			B3 = 1
		} else {
			B1 = "a" B1
			B2 = "a" B2
			B3 = 1
		}
	}

	if (B4) {
		return __nx_if(__nx_is_digit(B4), B1 > B2, B1 < B2) || __nx_if(__nx_else(B4 == 1, tolower(B4) == "i"), B1 == B2, 0)
	} else if (length(B4)) {
		return B1 ~ B2
	} else {
		return B1 == B2
	}
}
\end{NexCodeBox}

\begin{NexMainBox}[light, hdrA, sdwA, crnA, grwB, ssecB]
	\begin{NexMainBox}[dark, crnA]
		Dynamically compares two inputs based on their type, value, and specified behavior. The function leverages \NexFunction{awk}'s dynamic capabilities, adjusting input values and logic based on context. Its flexibility allows for comparisons of numeric values, strings, lengths, or patterns.
	\end{NexMainBox}
	\begin{NexMainBox}[dark, crnA]
		\begin{NexListDark}
			\NexItemDark{\NexOption{B1}: The first input to compare.}
			\NexItemDark{\NexOption{B2}: The second input to compare.}
			\NexItemDark{\NexOption{B3}: Determines how inputs are normalized for comparison:
				\begin{NexListLight}
					\NexItemLight{\NexOption{1}: Inputs are treated as numeric values.}
					\NexItemLight{\NexOption{""}: Inputs are compared as strings.}
					\NexItemLight{\NexOption{0}: Inputs engage regex-based comparison.}
				\end{NexListLight}
			}
			\NexItemDark{\NexOption{B4}: Specifies the comparison rule. When \NexOption{B4} is:
				\begin{NexListLight}
					\NexItemLight{\NexOption{"i"}: Performs \texttt{>=} (greater than or equal to).}
					\NexItemLight{\NexOption{"a"}: Performs \texttt{>} (greater than) only, as it fails the second logic check.}
					\NexItemLight{\NexOption{"1"}: If numeric, performs \texttt{<=} (less than or equal to).}
					\NexItemLight{\NexOption{numeric but not "1"}: Performs \texttt{<} (less than).}
					\NexItemLight{\NexOption{""}: Engages strict equality comparison (\texttt{==}).}
					\NexItemLight{\NexOption{"0"}: Activates pattern matching (\texttt{\textasciitilde}).}
				\end{NexListLight}
			}
		\end{NexListDark}
	\end{NexMainBox}
\end{NexMainBox}

\begin{NexCodeBox}{bash}{title={Compare lengths of B1 and B2}}
	B1 = "hello"
	B2 = "world!"
	B3 = 1
	result = __nx_compare(B1, B2, B3)
	# result is false, as length("hello") < length("world!")
\end{NexCodeBox}

\begin{NexCodeBox}{bash}{title={Compare numeric values of B1 and B2}}
	B1 = "42"
	B2 = "24"
	B3 = 1
	result = __nx_compare(B1, B2, B3)
	# result is true, as 42 > 24 (numeric comparison)
\end{NexCodeBox}

\begin{NexCodeBox}{bash}{title={String comparison of B1 and B2}}
	B1 = "abc"
	B2 = "def"
	B3 = ""
	result = __nx_compare(B1, B2, B3)
	# result is false, as "abc" != "def"
\end{NexCodeBox}

\begin{NexCodeBox}{bash}{title={Pattern matching B1 against B2}}
	B1 = "abc123"
	B2 = "[a-z]+[0-9]+"
	B3 = 0
	result = __nx_compare(B1, B2, B3)
	# result is true, as "abc123" matches the regex "[a-z]+[0-9]+"
\end{NexCodeBox}

\begin{NexCodeBox}{bash}{title={Relational comparison using B4}}
	B1 = 10
	B2 = 20
	B3 = 1
	B4 = "1"
	result = __nx_compare(B1, B2, B3, B4)
	# result is true, as 10 <= 20 (numeric comparison with B4 = 1)
\end{NexCodeBox}

\begin{NexCodeBox}{bash}{title={Case-insensitive equality comparison}}
	B1 = "Hello"
	B2 = "hello"
	B3 = ""
	B4 = "i"
	result = __nx_compare(B1, B2, B3, B4)
	# result is true, as "Hello" == "hello" (case-insensitive)
\end{NexCodeBox}

\newpage
\subsection{__nx_equality}
\label{__nx_equality}
\begin{NexCodeBox}{bash}{title={\NexFunction{__nx_equality(\NexOption{B1}, \NexOption{B2}, \NexOption{B3})}}}
function __nx_equality(B1, B2, B3,	b, e, g) {
	b = substr(B2, 1, 1)
	if (b == ">") {
		e = 2
		g = 1
	} else if (b == "<") {
		e = "a"
		g = "i"
	} else if (b == "=") {
		e = ""
	} else if (b == "~") {
		e = 0
	} else {
		b = ""
	}
	if (b) {
		if (__nx_compare(substr(B2, 2, 1), "=", 1)) {
			b = g
		} else {
			b = e
		}
		e = substr(B2, length(B2), 1)
		if (__nx_compare(e, "a", 1))
			return __nx_compare(B1, B3, "", b)
		else if (__nx_compare(e, "_", 1))
			return __nx_compare(B1, B3, 0, b)
		else
			return __nx_compare(B1, B3, 1, b)
	}
	return __nx_compare(B1, B2)
}
\end{NexCodeBox}

\begin{NexMainBox}[light, hdrA, sdwA, crnA, grwB, ssecB]
	\begin{NexMainBox}[dark, crnA]
		Dynamically evaluates equality or relational conditions between inputs. \NexOption{B2} specifies the operator (\texttt{>}, \texttt{<}, \texttt{=}, or \texttt{\textasciitilde}) and controls the type of comparison. The function uses \NexFunction{__nx_compare} for nuanced behavior based on \NexOption{awk} capabilities.
	\end{NexMainBox}
	\begin{NexMainBox}[dark, crnA]
		\begin{NexListDark}
			\NexItemDark{\NexOption{B1}: The first input to compare.}
			\NexItemDark{\NexOption{B2}: Specifies the operator and additional flags for comparison logic. When \NexOption{B2} starts with:
				\begin{NexListLight}
					\NexItemLight{\texttt{">"}: Relational comparison (greater than).}
					\NexItemLight{\texttt{"<"}: Relational comparison (less than).}
					\NexItemLight{\texttt{"="}: Strict equality check.}
					\NexItemLight{\texttt{"~"}: Pattern matching (\texttt{\textasciitilde}).}
				\end{NexListLight}
			}
			\NexItemDark{\NexOption{B3}: The second input to compare against \NexOption{B1}.}
		\end{NexListDark}
	\end{NexMainBox}
\end{NexMainBox}

\begin{NexCodeBox}{bash}{title={Compare B1 > B3}}
	B1 = 5
	B2 = ">"
	B3 = 3
	result = __nx_equality(B1, B2, B3)
	# result is true, as 5 > 3
\end{NexCodeBox}

\begin{NexCodeBox}{bash}{title={Compare B1 == B3}}
	B1 = "hello"
	B2 = "="
	B3 = "hello"
	result = __nx_equality(B1, B2, B3)
	# result is true, as "hello" == "hello"
\end{NexCodeBox}

\begin{NexCodeBox}{bash}{title={Check if B1 matches regex B3}}
	B1 = "abc123"
	B2 = "~"
	B3 = "[a-z]+[0-9]+"
	result = __nx_equality(B1, B2, B3)
	# result is true, as "abc123" matches the regex "[a-z]+[0-9]+"
\end{NexCodeBox}

\begin{NexCodeBox}{bash}{title={Compare B1 <= B3}}
	B1 = 7
	B2 = "<="
	B3 = 10
	result = __nx_equality(B1, B2, B3)
	# result is true, as 7 <= 10
\end{NexCodeBox}

\newpage
\subsection{__nx_swap}
\label{__nx_swap}
\begin{NexCodeBox}{bash}{title={\NexFunction{__nx_swap(\NexOption{V}, \NexOption{D1}, \NexOption{D2})}}}
function __nx_swap(V, D1, D2, t) {
	t = V[D1]
	V[D1] = V[D2]
	V[D2] = t
}
\end{NexCodeBox}

\begin{NexMainBox}[light, hdrA, sdwA, crnA, grwB, ssecB]
	\begin{NexMainBox}[dark, crnA]
		Swaps the values of two indices within an array or associative array. This ensures flexibility in dynamically rearranging or reordering data structures. A temporary variable (`t`) protects against loss during the exchange.
	\end{NexMainBox}
	\begin{NexMainBox}[dark, crnA]
		\begin{NexListDark}
			\NexItemDark{\NexOption{V}: The array or associative array containing values to be swapped.}
			\NexItemDark{\NexOption{D1}: The first index (or key) whose value will be swapped.}
			\NexItemDark{\NexOption{D2}: The second index (or key) whose value will be swapped.}
		\end{NexListDark}
	\end{NexMainBox}
\end{NexMainBox}

\begin{NexCodeBox}{bash}{title={Basic swap of numeric values}}
	V = [10, 20, 30, 40]
	D1 = 1
	D2 = 3
	__nx_swap(V, D1, D2)
	# Result: V = [10, 40, 30, 20], as the values at indices 1 and 3 are swapped
\end{NexCodeBox}

\begin{NexCodeBox}{bash}{title={Swap in a string array}}
	V = ["apple", "banana", "cherry"]
	D1 = 0
	D2 = 2
	__nx_swap(V, D1, D2)
	# Result: V = ["cherry", "banana", "apple"], as the values at indices 0 and 2 are swapped
\end{NexCodeBox}

\begin{NexCodeBox}{bash}{title={Swapping the same index (no-op)}}
	V = [1, 2, 3]
	D1 = 1
	D2 = 1
	__nx_swap(V, D1, D2)
	# Result: V = [1, 2, 3], as swapping the same index has no effect
\end{NexCodeBox}

\begin{NexCodeBox}{bash}{title={Swapping in an associative array}}
	V["a"] = "x"
	V["b"] = "y"
	D1 = "a"
	D2 = "b"
	__nx_swap(V, D1, D2)
	# Result: V = {"a": "y", "b": "x"}, as the values at keys "a" and "b" are swapped
\end{NexCodeBox}

\begin{NexCodeBox}{bash}{title={Nested array swap}}
	V = [[1, 2], [3, 4], [5, 6]]
	D1 = 0
	D2 = 2
	__nx_swap(V, D1, D2)
	# Result: V = [[5, 6], [3, 4], [1, 2]], as the nested arrays at indices 0 and 2 are swapped
\end{NexCodeBox}

