\section{Null Law}
\label{Null Law}
\begin{NexMainBox}[light, hdrA, sdwA, crnA, grwB, secA]
	\begin{NexMainBox}[dark, crnA]
		The following functions provide utilities for logical and comparative operations, enabling versatile Boolean checks across various conditions.
	\end{NexMainBox}
	\begin{NexMainBox}[dark, crnA]
		\begin{NexListDark}
			\NexItemDark{\NexFunction{NOT__(D)}: Returns the logical NOT of \texttt{D}.}
		\end{NexListDark}
	\end{NexMainBox}
\end{NexMainBox}
\begin{comment}
\begin{NexMainBox}[light, hdrA, sdwA, crnA, grwB, secB]
	\begin{NexMainBox}[dark, crnA]
		\begin{NexListDark}
			\NexItemDark{\NexFunction{INEQ__(B1, B2, B3)}: Logical NOT of \NexFunction{IEQ__}.}
			\NexItemDark{\NexFunction{GT__(B1, B2, B3)}: Returns \texttt{true} if \texttt{B1} is greater than \texttt{B2}.}
			\NexItemDark{\NexFunction{LT__(B1, B2, B3)}: Returns \texttt{true} if \texttt{B1} is less than \texttt{B2}.}
			\NexItemDark{\NexFunction{LE__(B1, B2, B3)}: Returns \texttt{true} if \texttt{B1} is less than or equal to \texttt{B2}.}
			\NexItemDark{\NexFunction{GE__(B1, B2, B3)}: Returns \texttt{true} if \texttt{B1} is greater than or equal to \texttt{B2}.}
			\NexItemDark{\NexFunction{IN__(V, D, B)}: Determines if \texttt{D} is an element of array \texttt{V} and satisfies \NexFunction{TRUE__}.}
			\NexItemDark{\NexFunction{ORFT__(B1, B2, B3)}: Returns true if \texttt{B1} is false or \texttt{B2} is true, based on \texttt{B3}.}
		\end{NexListDark}
    \end{NexMainBox}
\end{NexMainBox}

		\bftext{Expression:}\smallskip\\

		$ A + 1 = 1 $\\
		 $ A \cdot 0 = 0 $\\

		\bftext{Explanation:} For the Null Law of OR (addition), if $A$ is 1 and you add 1, the result is 1, not 0.
		This is because in Boolean algebra, the OR operation is defined such that any variable OR-ed with 1 results in 1.
		It’s like saying if either condition is true, the outcome is true regardless of the other condition.\smallskip
	\end{Items}
\end{ColorThemedBox}
\smallskip\\
\end{comment}
