\begin{NxSBox}[][Introduction]
	\begin{NxCodeBox}{lua}{dark}
		print("Hello, Lua!")
	\end{NxCodeBox}
	\begin{NxIDBox}
		Lua is a lightweight, high-level, multi-paradigm programming language designed primarily for embedded use in applications. It was created in 1993 by a team at the Pontifical Catholic University of Rio de Janeiro in Brazil. The name "Lua" means "Moon" in Portuguese.\\[5mm]
		Lua is known for its speed, portability, and ease of use, making it popular in game development, web applications, and software engineering. It has been used in games like World of Warcraft and Angry Birds, as well as in Adobe Photoshop Lightroom.
	\end{NxIDBox}
	\begin{NxIDBoxL}[title=Key Features]
		\nxCheckDark{\nxIDTopic{Lightweight \& Fast} Lua is designed to be small and efficient, making it ideal for embedded systems and game engines.}
		\nxCheckDark{\nxIDTopic{Ease of Integration} It can be easily embedded into other applications and works well with C and C++.}
		\nxCheckDark{\nxIDTopic{Garbage Collection} Automatic memory management helps developers focus on coding without worrying about manual cleanup.}
		\nxCheckDark{\nxIDTopic{Extensibility} Lua allows developers to easily add functionality through modules and libraries.}
		\nxCheckDark{\nxIDTopic{Multi-paradigm Support} It supports procedural, object-oriented, and functional programming styles.}
		\nxCheckDark{\nxIDTopic{Portable \& Cross-Platform} Runs on various operating systems, including Windows, macOS, Linux, and even embedded devices.}
		\nxCheckDark{\nxIDTopic{Simple Syntax} Easy-to-learn syntax that makes coding straightforward and accessible.}
	\end{NxIDBoxL}
\end{NxSBox}

