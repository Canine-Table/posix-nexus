\begin{NxSBox}[][Introduction]
	\begin{NxIDBox}
		Lua is a lightweight, high-level, multi-paradigm programming language designed primarily for embedded use in applications. It was created in 1993 by a team at the Pontifical Catholic University of Rio de Janeiro in Brazil. The name "Lua" means "Moon" in Portuguese.\\[5mm]
		Lua is known for its speed, portability, and ease of use, making it popular in game development, web applications, and software engineering. It has been used in games like World of Warcraft and Angry Birds, as well as in Adobe Photoshop Lightroom.
	\end{NxIDBox}
	\begin{NxIDBoxL}[title=Key Features]
		\nxCheckDark{\nxIDTopic{Lightweight \nxAnd Fast} Lua is designed to be small and efficient, making it ideal for embedded systems and game engines.}
		\nxCheckDark{\nxIDTopic{Ease of Integration} It can be easily embedded into other applications and works well with C and C++.}
		\nxCheckDark{\nxIDTopic{Garbage Collection} Automatic memory management helps developers focus on coding without worrying about manual cleanup.}
		\nxCheckDark{\nxIDTopic{Extensibility} Lua allows developers to easily add functionality through modules and libraries.}
		\nxCheckDark{\nxIDTopic{Multi-paradigm Support} It supports procedural, object-oriented, and functional programming styles.}
		\nxCheckDark{\nxIDTopic{Portable \nxAnd Cross-Platform} Runs on various operating systems, including Windows, macOS, Linux, and even embedded devices.}
		\nxCheckDark{\nxIDTopic{Simple Syntax} Easy-to-learn syntax that makes coding straightforward and accessible.}
	\end{NxIDBoxL}
	\begin{NxIDBox}
		Lua is especially popular in game development, scripting for applications, and embedded systems.
	\end{NxIDBox}
\end{NxSBox}

\begin{NxSSBox}[][Basics of Tables]
	\begin{NxCodeBox}{lua}{dark, title=Tables are created using \nxLBracket\nxRBracket syntax}
		myTable = {} -- Empty table
		myTable["key"] = "value" -- Associative array (key-value pairs)
		myTable[1] = "First element" -- Acts like an array
	\end{NxCodeBox}
	\begin{NxIDBox}
		Unlike arrays in other languages, Lua tables start indexing at 1 instead of 0.
	\end{NxIDBox}
\end{NxSSBox}

\begin{NxSSBox}[][Key-Value Pairs]
	\begin{NxCodeBox}{lua}{dark, title={Tables can store both numerical indices (like arrays) and string keys (like dictionaries) and will this wrap}}
		person = {
		    name = "Alice",
		    age = 25,
		    profession = "Engineer"
		}
		print(person.name) -- Output: Alice
	\end{NxCodeBox}
	\begin{NxIDBox}
		Keys can be numbers, strings, or even tables!
	\end{NxIDBox}
\end{NxSSBox}

\begin{NxSSBox}[][Iterating Over a Table]
	\begin{NxCodeBox}{lua}{dark, title={Use pairs() for general iteration or ipairs() for numeric keys}}
		for key, value in pairs(person) do
		    print(key, value)
		end
	\end{NxCodeBox}
\end{NxSSBox}

\begin{NxSSBox}
	\begin{NxCodeBox}{lua}{dark, title=This prints}
			name    Alice
			age     25
			profession   Engineer
	\end{NxCodeBox}
\end{NxSSBox}

\begin{NxSSBox}[][Metatables and Custom Behavior]
	\begin{NxCodeBox}{lua}{dark, title={Metatables allow custom operations on tables, such as arithmetic or method overloading.}}
		meta = {
		    __index = function(t, k)
			return "Default value"
		    end
		}
		setmetatable(myTable, meta)
		print(myTable["missingKey"]) -- Output: Default value
	\end{NxCodeBox}
\end{NxSSBox}

