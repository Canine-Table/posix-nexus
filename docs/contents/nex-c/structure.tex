\begin{comment}
4. Data Structures in C: Evolution Through the Years
4.1 Early Data Representations (Pre-1970s)
4.1.1 Manual Data Management in Assembly

4.1.1.1 Memory Layout in Early Computers

4.1.1.2 Using Registers for Primitive Storage

4.1.1.3 Direct Memory Manipulation vs Abstracted Structures

4.1.2 BCPL, B, and Structuring Data

4.1.2.1 Introduction to Fixed-Type Data Management

4.1.2.2 First Approaches to Aggregating Data in Early Languages

4.2 Data Structures in K&R C (1978 - 1989)
4.2.1 Arrays and Basic Containers

4.2.1.1 Defining and Accessing Arrays

4.2.1.2 Relationship Between Arrays and Pointers

4.2.1.3 Multi-Dimensional Arrays in C

4.2.2 Strings and Character Arrays

4.2.2.1 String Handling with Null-Terminated Arrays

4.2.2.2 Limitations of Early String Processing

4.2.2.3 Introduction of <string.h>

4.2.3 Structs and Unions: Aggregating Data

4.2.3.1 Defining and Using struct

4.2.3.2 Memory Layout and Alignment of Structs

4.2.3.3 Using Unions for Efficient Memory Sharing

4.3 ANSI C (1989 - 1999) – Refinements and Safety
4.3.1 Introduction of typedef for Flexible Type Naming

4.3.1.1 Enhancing Readability in Data Structures

4.3.1.2 Common Use Cases in System Programming

4.3.2 Dynamic Data Structures with Pointers

4.3.2.1 Linked Lists (struct node)

4.3.2.2 Stacks and Queues Using Pointers

4.3.2.3 Hash Tables and Pointer-Based Efficiency

4.4 C99 (1999 - 2011) – Expanding Data Handling
4.4.1 Introduction of Flexible Arrays

4.4.1.1 VLA (Variable Length Arrays)

4.4.1.2 Use Cases and Compiler Behavior

4.4.2 Advancements in Memory Safety

4.4.2.1 Struct Alignment with #pragma pack

4.4.2.2 Enhanced String Security Functions

4.5 C11 (2011 - 2018) – Multi-threaded and Atomic Data Handling
4.5.1 Atomic Operations on Data Structures

4.5.1.1 Using _Atomic for Safe Data Modification

4.5.1.2 Lock-Free Programming for Performance

4.5.2 Threads and Data Synchronization

4.5.2.1 <threads.h> and Struct-Based Thread Management

4.5.2.2 Avoiding Race Conditions in Data Structures

4.6 C17 and Beyond – Refinements and Future Directions
4.6.1 Standardization of Data Structures in New Architectures

4.6.1.1 SIMD-Friendly Data Alignment

4.6.1.2 Cache-Aware Struct Design

4.6.2 Alternative Data Representations in Modern Computing

4.6.2.1 Rust’s Borrow-Checked Data Structures

4.6.2.2 The Future of Self-Optimizing Structures in C
\end{comment}

\begin{NxSBox}[][Structures in C]
    \begin{NxIDBox}
        Structures allow grouping related variables into a single data type, making it easier to manage complex data.
    \end{NxIDBox}
    \begin{NxIDBoxL}
        \nxTopicD{ssec:Defining Structures}{Defining Structures} How to declare and initialize `struct` types for better data organization.
        \nxTopicD{ssec:Memory Layout}{Memory Layout} How structures are stored in memory and how padding/alignment affects performance.
        \nxTopicD{ssec:Structs vs Classes}{Structs vs Classes} Comparing C-style structures to object-oriented class-based models in other languages.
    \end{NxIDBoxL}
\end{NxSBox}

