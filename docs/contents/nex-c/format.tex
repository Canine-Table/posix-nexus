\begin{NxSBox}[breakable][File Types and Formats]
	\begin{NxIDBox}
		Computer files exist in a variety of formats, each optimized for different use cases—executables, object files, libraries, and debugging symbols. This section explores how these formats function, interact, and play essential roles in software development.
	\end{NxIDBox}
	\begin{NxIDBoxL}
		%\nxTopicD{Application Binary Interface (ABI)}
		\nxTopicD{Binary Formats} Overview of how compiled executables are structured (e.g., ELF, PE, Mach-O) and their role in system execution.
		%\nxTopicD{Linkers and Linking} How linkers combine object files and libraries, resolving symbols and dependencies to create functional executables.
		%\nxTopicD{Object Files} The structure of compiled `.o` files, relocations, and how they fit into linking and execution.
		%\nxTopicD{Toolchains and File Handling} How compilers, assemblers, and binary analysis tools interact with these formats for software development and debugging.
	\end{NxIDBoxL}
\end{NxSBox}

%\begin{NxSSBox}[breakable][Application Binary Interface (ABI)]
	\begin{NxIDBox}
		The **Application Binary Interface (ABI)** defines low-level interactions between software and hardware, including calling conventions, system linkage, and executable format specifications.
	\end{NxIDBox}
	\begin{NxIDBoxL}
		\nxTopicD{Function Calling Conventions} Defines how function arguments are passed between registers and stack.
		%\nxTopicD{Register Usage Across ABIs} Specifies which registers are preserved, volatile, and used for return values.
		%\nxTopicD{Stack Frame and Memory Layout} Defines how functions allocate stack space and handle local variables.
		%\nxTopicD{Dynamic Linking and ABI Compatibility} Ensures shared libraries (`.so`) maintain consistent binary interfaces.
		%\nxTopicD{System V vs. Windows ABI Differences} Highlights how calling conventions differ between platforms.
		%\nxTopicD{Handling ABI Breakage and Compatibility} Discusses how changes in an ABI affect binary compatibility.
	\end{NxIDBoxL}
\end{NxSSBox}

\begin{NxSSSBox}[breakable][Function Calling Conventions]
	\begin{NxIDBox}
		Function calling conventions define how arguments are passed between registers and stack, how return values are handled, and which registers are preserved across function calls. A standardized convention ensures binaries remain ABI-compatible across different architectures.
	\end{NxIDBox}
	\begin{NxIDBox}
		In the System V AMD64 ABI (used in Linux ELF binaries), the first six function arguments are passed using registers: `RDI`, `RSI`, `RDX`, `RCX`, `R8`, and `R9`. Any additional arguments are pushed onto the stack.
	\end{NxIDBox}
	\begin{NxIDBox}
		When a function is called, the \textbf{caller must ensure volatile registers (`RAX`, `RDI`, `RSI`, etc.)} are preserved if needed**. The **callee is responsible for saving non-volatile registers (`RBX`, `RBP`, `R12-R15`) if modified**.
	\end{NxIDBox}
\end{NxSSSBox}

\begin{NxIDBoxT}{l|l}[title={System V AMD64 ABI Calling Convention (x86-64)}]
    Register & Purpose in Function Calls \\\hline
    RDI & First function argument \\\hline
    RSI & Second function argument \\\hline
    RDX & Third function argument \\\hline
    RCX & Fourth function argument \\\hline
    R8	& Fifth function argument \\\hline
    R9	& Sixth function argument \\\hline
    RAX & Holds function return value \\\hline
    RSP & Stack pointer (points to function arguments beyond the sixth) \\\hline
    RBP & Base pointer (used for stack frames) \\
\end{NxIDBoxT}

\begin{NxCodeBox}{c}{dark, sdwA, frmE, title={When calling a function like}}
	int add(int a, int b) {
		return a + b;
	}
\end{NxCodeBox}

\bigskip

\begin{NxCodeBox}{asm}{dark, sdwA, frmE, title={The assembly-level execution for add(2, 3) would look like:}}
	mov rdi, 2    ; First argument (a)
	mov rsi, 3    ; Second argument (b)
	call add      ; Function call
	mov rax, rdi  ; Store return value in RAX
\end{NxCodeBox}

\bigskip

\begin{NxCodeBox}{c}{dark, sdwA, frmE, title={When a function has more than six arguments, excess arguments are pushed onto the stack instead of registers. Example}}
	int complex_function(int a, int b, int c, int d, int e, int f, int g);
\end{NxCodeBox}

\begin{NxIDBoxT}{l|l}[title={System V AMD64 ABI Argument Passing}]
    Argument & Passed via \\\hline
    a & RDI (register) \\\hline
    b & RSI (register) \\\hline
    c & RDX (register) \\\hline
    d & RCX (register) \\\hline
    e & R8 (register) \\\hline
    f & R9 (register) \\\hline
    g & Stack (RSP) \\
\end{NxIDBoxT}

\begin{NxIDBoxT}{l|l|l}[title={Register Preservation Rules}]
    Register & Preserved? & Usage \\\hline
    RBX & Yes & Used as a preserved general-purpose register \\\hline
    RSP & Yes & Stack pointer (must remain unchanged) \\\hline
    RBP & Yes & Used for stack frame management \\\hline
    R12-R15 & Yes & General-purpose registers \\\hline
    RAX & No & Used for return values, can be overwritten \\\hline
    RDI-R9 & No & Used for argument passing, not preserved \\
\end{NxIDBoxT}

\begin{NxSSSSBox}[breakable][Stack Frame Layout]
    \begin{NxIDBox}
	    Every function call creates a \textbf{stack frame}, allocating space for local variables, register preservation, and return addresses. The frame follows a predictable structure that assists debugging and ensures smooth execution.
    \end{NxIDBox}
    \begin{NxIDBox}
	    In \textbf{System V AMD64 ABI}, the stack grows \textbf{downward}, meaning newer frames are allocated at lower addresses. The base pointer (`RBP`) helps track stack position.
    \end{NxIDBox}
\end{NxSSSSBox}

\begin{NxIDBoxT}{l|l|l}[title={Stack Frame Breakdown}]
    Stack Content & Purpose & Stored In \\\hline
    Return Address & Address to jump back after function exits & [RSP] \\\hline
    Saved Registers & Preserved values from caller function & [RBP-8], [RBP-16] \\\hline
    Local Variables & Temporary function storage & [RBP-24], [RBP-32] \\\hline
    Function Arguments & Passed values beyond 6th argument & [RBP+8], [RBP+16] \\
\end{NxIDBoxT}

\begin{NxSSSSBox}[breakable]
	\begin{NxIDBox}
		Stack grows downward, ensuring each call doesn’t overwrite another’s memory. Caller saves vital registers before calling the callee function. Return address ensures proper function return flow.
	\end{NxIDBox}
\end{NxSSSSBox}

\begin{NxSSSSBox}[breakable][Syscall Execution]
    \begin{NxIDBox}
	System calls allow ELF binaries to interact with the **Linux kernel** for process management, file handling, memory allocation, and device communication.
    \end{NxIDBox}
    \begin{NxIDBox}
	Each syscall follows a strict register convention, where the **syscall number is placed in `RAX`, arguments are passed in `RDI`, `RSI`, `RDX`, etc.**, and the kernel returns a result in `RAX`.
    \end{NxIDBox}
\end{NxSSSSBox}

\begin{NxIDBoxT}{l|l|l}[title={Syscall Execution Flow}]
    Step & Operation & Register Used \\\hline
    1 & Load syscall number (SYS\_write = 1) & RAX \\\hline
    2 & Set file descriptor (stdout = 1) & RDI \\\hline
    3 & Set buffer address & RSI \\\hline
    4 & Set byte count & RDX \\\hline
    5 & Call syscall instruction & syscall \\\hline
    6 & Kernel executes operation & — \\\hline
    7 & Return result to RAX & RAX (bytes written) \\
\end{NxIDBoxT}

\begin{NxSSSSBox}[breakable]
	\begin{NxIDBox}
		Syscall interface is register-based, avoiding function overhead. Direct interaction with kernel ensures efficient execution. Results are returned via RAX, reducing complexity.
	\end{NxIDBox}
\end{NxSSSSBox}

\begin{NxSSSSBox}[breakable][Windows vs. System V ABI]
    \begin{NxIDBox}
	Calling conventions differ between Windows and Linux. **Windows x64 ABI passes only four arguments via registers (`RCX`, `RDX`, `R8`, `R9`)**, whereas **System V uses six registers (`RDI`, `RSI`, `RDX`, `RCX`, `R8`, `R9`)**.
	\end{NxIDBox}
	\begin{NxIDBox}
	    Additionally, Windows \textbf{requires stack space reservation for all arguments}, while System V only uses stack if arguments exceed six.
    \end{NxIDBox}
\end{NxSSSSBox}

\begin{NxIDBoxT}{l|l|l}[title={ABI Feature Comparison: System V vs. Windows x64}]
    ABI Feature & System V (Linux) & Windows x64 \\\hline
    Register Arguments & RDI, RSI, RDX, RCX, R8, R9 & RCX, RDX, R8, R9 \\\hline
    Stack Usage & Only after six arguments & Always reserved \\\hline
    Caller-Saved Registers & RAX, RDI, RSI, RDX, RCX, R8, R9 & RAX, RCX, RDX, R8, R9 \\\hline
    Function Return Register & RAX & RAX \\
\end{NxIDBoxT}



\begin{NxSSBox}[breakable][Binary Formats]
        \begin{NxIDBox}
                Binary formats define the structure and behavior of executable files across different operating systems. Each platform has its own binary format tailored for system integration, memory management, and execution efficiency.
        \end{NxIDBox}
        \begin{NxIDBoxL}
                \nxTopicD{ELF (Executable and Linkable Format)} The standard binary format used in Unix-based systems (Linux, BSD), supporting dynamic linking and flexible section structures.
                %\nxTopicD{PE (Portable Executable)} The executable format for Windows, integrating headers, sections, and built-in security mechanisms.
                %\nxTopicD{Mach-O (Mach Object)} The binary format for macOS and iOS, optimized for Apple’s linker and dynamic loader.
        \end{NxIDBoxL}
\end{NxSSBox}

\begin{NxSSSBox}[breakable][ELF (Executable and Linkable Format)]
	\begin{NxIDBox}
		\nxGID{elf} is the standard binary format used in Unix-based operating systems such as Linux, BSD, and Solaris. It defines the structure of executable files, shared libraries, and object files, ensuring efficient linking and execution.
	\end{NxIDBox}
	\begin{NxIDBoxL}
		\nxTopicD{Linux x86-64 Syscall Numbers (System V ABI)} Linux system calls have specific numbers assigned to them, which are crucial when writing ELF64 assembly under the System V ABI.
	\begin{comment}
		\nxTopicD{ELF Headers and Structure} Breakdown of ELF headers, sections, and program headers.
		\nxTopicD{ELF Sections} Overview of `.text`, `.data`, `.bss`, `.rodata`, and other essential ELF segments.
		\nxTopicD{ELF Relocations and Linking} How ELF handles dynamic and static linking using relocation entries.
		\nxTopicD{ELF Symbol Tables} How ELF uses `.symtab` and `.dynsym` for symbol resolution.
		\nxTopicD{Executable vs. Shared Objects} Differences between ELF executables and dynamically linked libraries (`.so`).
		\nxTopicD{ELF Debugging and Analysis Tools} Inspecting ELF files using `readelf`, `objdump`, and `nm`.
		\nxTopicD{ELF in Different Architectures} Variations of ELF (`ELF32`, `ELF64`) across different CPU architectures.
		\nxTopicD{Security in ELF Binaries} Features like Address Space Layout Randomization (ASLR), Position Independent Executables (PIE), and Stack Smashing Protection (SSP).
		\nxTopicD{ELF Loader and Execution Flow} How the system loads and executes ELF binaries.
	\end{comment}
	\end{NxIDBoxL}
\end{NxSSSBox}

\begin{NxSSSSBox}[breakable][Linux x86-64 Syscall Numbers (System V ABI)]
	\begin{NxIDBox}
		Linux system calls are the fundamental interface between user-space programs and the operating system’s kernel. They allow programs to request services from the kernel, such as file manipulation, process control, memory management, and hardware interaction.
	\end{NxIDBox}
	\begin{NxIDBoxT}{l|l}[title={Common Categories of Linux System Calls}]
		Category & Example System Calls \\\hline
		Process Control & fork(), execve(), exit() \\\hline
		File Management & open(), read(), write(), close() \\\hline
		Memory Management & mmap(), brk() \\\hline
		Networking & socket(), bind(), send(), recv() \\\hline
		Signals \& IPC & kill(), sigaction(), msgget(), semop() \\\hline
		User Management & getuid(), setuid(), getgid(), setgid() \\
	\end{NxIDBoxT}
\end{NxSSSSBox}

\begin{NxSSSSBox}[breakable]
	\begin{NxIDBoxT}{l|l|l}[title={Complete File I/O \& Filesystem Syscalls}]
		Syscall & Number & Description \\\hline
		read & 0 & Read data from a file descriptor \\\hline
		write & 1 & Write data to a file descriptor \\\hline
		open & 2 & Open a file \\\hline
		close & 3 & Close a file \\\hline
		stat & 4 & Get file metadata \\\hline
		fstat & 5 & Get metadata from descriptor \\\hline
		lstat & 6 & Get metadata for a symbolic link \\\hline
		poll & 7 & Monitor multiple file descriptors \\\hline
		lseek & 8 & Move file offset \\\hline
		pread64 & 17 & Read from file descriptor with offset \\\hline
		pwrite64 & 18 & Write to file descriptor with offset \\\hline
		readv & 19 & Read multiple buffers at once \\\hline
		writev & 20 & Write multiple buffers at once \\\hline
		access & 21 & Check file access permissions \\\hline
		pipe & 22 & Create an inter-process pipe \\\hline
		select & 23 & Monitor file descriptors for readiness \\\hline
		fcntl & 72 & Modify file descriptor properties \\\hline
		flock & 73 & Apply file locks \\\hline
		fsync & 74 & Synchronize file contents with storage \\\hline
		fdatasync & 75 & Synchronize file metadata \\\hline
		truncate & 76 & Resize a file \\\hline
		ftruncate & 77 & Resize a file via descriptor \\\hline
		getdents64 & 217 & Read directory entries \\\hline
		getcwd & 79 & Get current working directory \\\hline
		chdir & 80 & Change working directory \\\hline
		fchdir & 81 & Change working directory via descriptor \\\hline
		rename & 82 & Rename a file \\\hline
		mkdir & 83 & Create a directory \\\hline
		rmdir & 84 & Remove a directory \\\hline
		creat & 85 & Create a new file \\\hline
		link & 86 & Create a hard link \\\hline
		unlink & 87 & Remove a file \\\hline
		symlink & 88 & Create a symbolic link \\\hline
		readlink & 89 & Read symbolic link contents \\\hline
		chmod & 90 & Change file permissions \\\hline
		fchmod & 91 & Change file permissions via descriptor \\\hline
	\end{NxIDBoxT}
\end{NxSSSSBox}

\begin{NxSSSSBox}[breakable]
	\begin{NxIDBoxT}{l|l|l}
		chown & 92 & Change file owner \\\hline
		fchown & 93 & Change file owner via descriptor \\\hline
		lchown & 94 & Change symbolic link ownership \\\hline
		umask & 95 & Set default file permissions \\\hline
		statfs & 137 & Get filesystem statistics \\\hline
		fstatfs & 138 & Get filesystem stats via descriptor \\\hline
		sync & 162 & Synchronize filesystems \\\hline
		mount & 165 & Mount a filesystem \\\hline
		umount2 & 166 & Unmount a filesystem \\\hline
		quotactl & 179 & Manage disk quotas \\\hline
		syncfs & 306 & Synchronize filesystem buffers \\\hline
		renameat2 & 316 & Rename a file atomically \\\hline
		linkat & 265 & Create a hard link at a specific path \\\hline
		symlinkat & 266 & Create a symbolic link at a specific path \\\hline
		unlinkat & 263 & Remove a file at a specific path \\\hline
		statx & 332 & Extended file metadata retrieval \\
	\end{NxIDBoxT}
\end{NxSSSSBox}

\begin{NxCodeBox}{asm}{dark, sdwA, frmE, title={A simple example that prints "Hello, world!\n" using the write syscall.}}
	section .text
	global _start

	_start:
		mov rax, 1        ; syscall: write
		mov rdi, 1        ; file descriptor: stdout (1)
		mov rsi, msg      ; pointer to message
		mov rdx, msg_len  ; message length
		syscall           ; invoke syscall

		mov rax, 60       ; syscall: exit
		xor rdi, rdi      ; exit code 0
		syscall           ; invoke syscall

	section .data
	msg db "Hello, world!", 0xA  ; Message with newline
	msg_len equ $-msg            ; Calculate message length
\end{NxCodeBox}


\begin{NxSSSSBox}[breakable]
	\begin{NxIDBoxT}{l|l|l}[title={Process Management}]
		Syscall & Number & Description \\\hline
		fork & 57 & Create a child process \\\hline
		vfork & 58 & Create process (different memory handling) \\\hline
		execve & 59 & Execute a binary file \\\hline
		exit & 60 & Terminate a process \\\hline
		wait4 & 61 & Wait for process termination \\\hline
		kill & 62 & Send a signal to a process \\\hline
		gettid & 186 & Get thread ID \\\hline
		clone & 56 & Create new process/thread \\\hline
		clone3 & 435 & Advanced version of clone with more control \\\hline
		set\_tid\_address & 218 & Define thread ID storage location \\\hline
		sched\_setscheduler & 144 & Set scheduler type \\\hline
		sched\_getscheduler & 145 & Get scheduler type \\\hline
		sched\_get\_priority\_max & 146 & Get max scheduling priority \\\hline
		sched\_get\_priority\_min & 147 & Get min scheduling priority \\
	\end{NxIDBoxT}
	\begin{NxIDBoxT}{l|l|l}[title={Complete Memory Management System Calls}]
		Syscall & Number & Description \\\hline
		mmap & 9 & Map memory pages into user space \\\hline
		mprotect & 10 & Change memory protection flags \\\hline
		munmap & 11 & Unmap memory pages \\\hline
		brk & 12 & Adjust heap memory allocation \\\hline
		mremap & 25 & Resize memory mappings \\\hline
		msync & 26 & Synchronize memory mappings with storage \\\hline
		mincore & 27 & Check residency of memory pages \\\hline
		madvise & 28 & Give hints to kernel about memory usage patterns \\\hline
		mlock & 149 & Lock memory pages to prevent swapping \\\hline
		munlock & 150 & Unlock memory pages \\\hline
		mlockall & 151 & Lock all memory pages for process \\\hline
		munlockall & 152 & Unlock all memory pages for process \\\hline
		remap\_file\_pages & 216 & Remap pages in a file-backed memory mapping \\\hline
		futex & 202 & Fast user-space locking mechanism \\\hline
		migrate\_pages & 256 & Move process pages to different nodes in NUMA \\\hline
		move\_pages & 279 & Manually move memory pages between NUMA nodes \\\hline
		process\_vm\_readv & 310 & Read memory from another process \\\hline
		process\_vm\_writev & 311 & Write memory to another process \\
	\end{NxIDBoxT}
\end{NxSSSSBox}

\begin{NxSSSSBox}[breakable]
	\begin{NxIDBoxT}{l|l|p{7cm}}[title={Time \& Timer Syscalls}]
		Syscall & Number & Description \\\hline
		gettimeofday & 96 & Get current system time (seconds + microseconds) \\\hline
		times & 100 & Get process execution time statistics \\\hline
		clock\_settime & 227 & Set system clock time \\\hline
		clock\_gettime & 228 & Retrieve system clock time \\\hline
		clock\_getres & 229 & Get resolution of a clock \\\hline
		clock\_nanosleep & 230 & Sleep for a precise time \\\hline
		timer\_create & 222 & Create a timer using a specified clock \\\hline
		timer\_settime & 223 & Set timer expiration time \\\hline
		timer\_gettime & 224 & Retrieve timer's current remaining time \\\hline
		timer\_getoverrun & 225 & Get timer expiration overrun count \\\hline
		timer\_delete & 226 & Remove a timer \\\hline
		nanosleep & 35 & Pause execution for nanoseconds \\\hline
		alarm & 37 & Set an alarm signal after a given time \\\hline
		settimeofday & 164 & Set system time (deprecated in favor of clock\_settime) \\\hline
		adjtimex & 159 & Fine-tune system clock adjustments \\
	\end{NxIDBoxT}
\end{NxSSSSBox}

\begin{NxSSSSBox}[breakable]
	\begin{NxIDBoxT}{l|l|p{7cm}}[title={Security \& Access Control (Full List)}]
		Syscall & Number & Description \\\hline
		getuid & 102 & Get user ID \\\hline
		setuid & 105 & Set user ID \\\hline
		getgid & 104 & Get group ID \\\hline
		setgid & 106 & Set group ID \\\hline
		geteuid & 107 & Get effective user ID \\\hline
		getegid & 108 & Get effective group ID \\\hline
		setpgid & 109 & Set process group ID \\\hline
		getppid & 110 & Get parent process ID \\\hline
		getpgrp & 111 & Get process group ID \\\hline
		setsid & 112 & Set session ID \\\hline
		setreuid & 113 & Set real and effective user ID \\\hline
		setregid & 114 & Set real and effective group ID \\\hline
		getgroups & 115 & Get list of supplementary group IDs \\\hline
		setgroups & 116 & Set list of supplementary group IDs \\\hline
		setresuid & 117 & Set real, effective, and saved user ID \\\hline
		getresuid & 118 & Get real, effective, and saved user ID \\\hline
		setresgid & 119 & Set real, effective, and saved group ID \\\hline
		getresgid & 120 & Get real, effective, and saved group ID \\\hline
		getpgid & 121 & Get process group ID \\\hline
		setfsuid & 122 & Set file-system user ID \\\hline
		setfsgid & 123 & Set file-system group ID \\\hline
		getsid & 124 & Get session ID \\\hline
		capget & 125 & Get capabilities (privileges) of a process \\\hline
		capset & 126 & Set capabilities (privileges) of a process \\\hline
		setns & 308 & Switch namespaces \\\hline
		seccomp & 317 & Apply syscall filtering (sandboxing) \\\hline
		landlock\_create\_ruleset & 444 & Create Landlock security ruleset \\\hline
		landlock\_add\_rule & 445 & Add a Landlock security rule \\\hline
		landlock\_restrict\_self & 446 & Restrict process permissions using Landlock \\\hline
		lsm\_get\_self\_attr & 459 & Get self security module attributes \\\hline
		lsm\_set\_self\_attr & 460 & Set security module attributes for self \\\hline
		lsm\_list\_modules & 461 & List loaded security modules \\
	\end{NxIDBoxT}
\end{NxSSSSBox}

\begin{NxSSSSBox}[breakable]
\begin{NxIDBoxT}{l|l|p{7cm}}[title={Complete Signals \& IPC System Calls}]
		Syscall & Number & Description \\\hline
		rt\_sigaction & 13 & Set up signal handlers \\\hline
		rt\_sigprocmask & 14 & Block/unblock signals \\\hline
		rt\_sigreturn & 15 & Return from signal handler \\\hline
		sigaltstack & 131 & Use an alternate signal stack \\\hline
		rt\_sigpending & 127 & Get pending signals \\\hline
		rt\_sigtimedwait & 128 & Wait for a signal with timeout \\\hline
		rt\_sigqueueinfo & 129 & Send real-time signal to process \\\hline
		rt\_sigsuspend & 130 & Suspend execution until signal arrives \\\hline
		kill & 62 & Send signal to a process \\\hline
		rt\_tgsigqueueinfo & 297 & Queue real-time signals to threads\\\hline
		tgkill & 234 & Send signal to a specific thread \\\hline
		tkill & 200 & Send signal to a thread (older version) \\\hline
		sigaction & 67 & Set up basic signal handler (legacy) \\\hline
		sgetmask & 68 & Get signal mask (legacy) \\\hline
		ssetmask & 69 & Set signal mask (legacy) \\\hline
		sigsuspend & 70 & Suspend process until signal arrives (legacy) \\\hline
		ipc & 117 & IPC system call multiplexer (legacy) \\\hline
		shmget & 29 & Allocate shared memory \\\hline
		shmat & 30 & Attach shared memory segment \\\hline
		shmctl & 31 & Control shared memory segment \\\hline
		shmdt & 67 & Detach shared memory segment \\\hline
		semget & 64 & Get semaphore \\\hline
		semop & 65 & Perform semaphore operation \\\hline
		semctl & 66 & Control semaphore \\\hline
		msgget & 68 & Get message queue \\\hline
		msgsnd & 69 & Send message to queue \\\hline
		msgrcv & 70 & Receive message from queue \\\hline
		msgctl & 71 & Control message queue \\\hline
		signalfd & 282 & Create file descriptor for handling signals \\\hline
		signalfd4 & 289 & signalfd with extra flags \\
	\end{NxIDBoxT}
\end{NxSSSSBox}

\begin{NxSSSSBox}[breakable]
	\begin{NxIDBoxT}{l|l|p{7cm}}[title={Complete Memory Management System Calls}]
		Syscall & Number & Description \\\hline
		mmap & 9 & Map memory pages into user space \\\hline
		mprotect & 10 & Change memory protection flags \\\hline
		munmap & 11 & Unmap memory pages \\\hline
		brk & 12 & Adjust heap memory allocation \\\hline
		mremap & 25 & Resize memory mappings \\\hline
		msync & 26 & Synchronize memory mappings with storage \\\hline
		mincore & 27 & Check residency of memory pages \\\hline
		madvise & 28 & Give hints to kernel about memory usage patterns \\\hline
		mlock & 149 & Lock memory pages to prevent swapping \\\hline
		munlock & 150 & Unlock memory pages \\\hline
		mlockall & 151 & Lock all memory pages for process \\\hline
		munlockall & 152 & Unlock all memory pages for process \\\hline
		remap\_file\_pages & 216 & Remap pages in a file-backed memory mapping \\\hline
		futex & 202 & Fast user-space locking mechanism \\\hline
		migrate\_pages & 256 & Move process pages to different nodes in NUMA \\\hline
		move\_pages & 279 & Manually move memory pages between NUMA nodes \\\hline
		process\_vm\_readv & 310 & Read memory from another process \\\hline
		process\_vm\_writev & 311 & Write memory to another process \\
	\end{NxIDBoxT}
\end{NxSSSSBox}

\begin{comment}

\end{NxSSSSBox}

\begin{NxSSSSBox}[breakable]
	
\end{NxSSSSBox}
\begin{NxIDBoxT}{l|l|l|p{7cm}}[title={ELF Identification (e\_ident)}]
	Offset	& 	Size & Field & Description \\\hline
	0x00		& 4 & e\_ident[0-3]		& Magic number (x7FELF) used to recognize ELF files. \\\hline
	0x04		& 1 & e\_ident[EI\_CLASS]	& Defines ELF Class (32-bit = 0x01, 64-bit = 0x02). \\\hline
	0x05		& 1 & e\_ident[EI\_DATA]	& Endianness (0x01 = Little Endian, 0x02 = Big Endian). \\\hline
	0x06		& 1 & e\_ident[EI\_VERSION]	& ELF version (0x01 = Current). \\\hline
	0x07		& 1 & e\_ident[EI\_OSABI]	& Defines OS ABI (0x00 = System V, 0x03 = Linux). \\\hline
	0x08		& 1 & e\_ident[EI\_ABIVERSION]	& ABI Version—determines execution compatibility. \\\hline
	0x09-0x0F	& 7 & e\_ident[EI\_PAD]		& Unused padding bytes. \\
\end{NxIDBoxT}

\begin{NxIDBoxT}{l|l|l|p{7cm}}[title={ELF Header Fields}]
	Offset  &   Size (ELF32/ELF64) & Field & Description \\\hline
	0x10		& 2 / 2 & e\_type		& Identifies file type (Executable, Shared Object, Relocatable). \\\hline
	0x12		& 2 / 2 & e\_machine	 & Specifies processor architecture (x86 = 0x03, ARM = 0x28). \\\hline
	0x14		& 4 / 4 & e\_version	 & ELF format version (0x01 = Current). \\\hline
	0x18		& 4 / 8 & e\_entry	   & Entry Point Address—where execution starts. \\\hline
	0x1C		& 4 / 8 & e\_phoff	   & Offset of Program Headers in the ELF file. \\\hline
	0x20		& 4 / 8 & e\_shoff	   & Offset of Section Headers for .text, .data, .bss. \\\hline
	0x24		& 4 / 4 & e\_flags	   & Architecture-specific execution flags. \\\hline
	0x28		& 2 / 2 & e\_ehsize	  & Size of ELF Header (ELF32: 52 bytes, ELF64: 64 bytes). \\\hline
	0x2A		& 2 / 2 & e\_phentsize   & Size of a single Program Header entry. \\\hline
	0x2C		& 2 / 2 & e\_phnum	   & Number of entries in the Program Header Table. \\\hline
	0x2E		& 2 / 2 & e\_shentsize   & Size of a single Section Header entry. \\\hline
	0x30		& 2 / 2 & e\_shnum	   & Number of entries in the Section Header Table. \\\hline
	0x32		& 2 / 2 & e\_shstrndx	& Index of the section header string table (names of .text, .data, etc.). \\
\end{NxIDBoxT}

\begin{NxIDBoxT}{l|p{5cm}|l|l}[title={ELF Program Header Types (Elf32\_Phdr / Elf64\_Phdr)}]
	Type (p\_type) & Purpose & Example Segment & Flags (p\_flags) \\\hline
	PT\_NULL & Unused entry & — & — \\\hline
	PT\_LOAD & Loadable segment into memory & .text, .data, .bss & R, W, X \\\hline
	PT\_DYNAMIC & Holds dynamic linking information & .dynamic & R \\\hline
	PT\_INTERP & Specifies the dynamic linker used & .interp & R \\\hline
	PT\_NOTE & Stores extra metadata & .note.ABI-tag & R \\\hline
	PT\_SHLIB & Reserved & — & — \\\hline
	PT\_PHDR & Self-reference to the program header table & .phdr & R \\
\end{NxIDBoxT}

\begin{NxIDBoxT}{l|p{5cm}|l}[title={ELF Relocation Entries (Elf32\_Rel / Elf64\_Rela)}]
	Relocation Type (r\_type) & Purpose & Used For \\\hline
	R\_386\_32 (x86) & Absolute address replacement & Static linking \\\hline
	R\_X86\_64\_PC32 (x86-64) & Relative addressing (PC-relative) & Position Independent Code \\\hline
	R\_ARM\_ABS32 (ARM) & Absolute memory address replacement & Global variables \\\hline
	R\_ARM\_CALL (ARM) & Function call relocation & Function pointers \\
\end{NxIDBoxT}

\begin{NxIDBoxT}{l|l|l}[title={ELF Debugging Symbols}]
	Symbol Type (st\_info) & Purpose & Example \\\hline
	STT\_FUNC & Function name \& address & main() \\\hline
	STT\_OBJECT & Global variable name \& address & int global\_var; \\\hline
	STT\_SECTION & Section symbol reference & .text, .data \\\hline
	STT\_FILE & File-level metadata & hello.c \\
\end{NxIDBoxT}

\begin{NxIDBoxT}{l|l|p{4.2cm}}[title={ELF Memory Layout of an Executable}]
	Memory Region & Purpose & Example Location \\\hline
	Text Segment & Contains executable code & 0x08048000 \\\hline
	Data Segment & Stores initialized global variables & 0x08049000 \\\hline
	BSS Segment & Holds uninitialized global variables & 0x0804A000 \\\hline
	Heap & Dynamically allocated memory (malloc()) & 0x08050000 \\\hline
	Stack & Stores local variables and function calls & 0xBFFF0000 (grows downward) \\
\end{NxIDBoxT}

\begin{NxIDBoxT}{l|l|p{1cm}}[title={ELF Symbol Types (STT\_*)}]
	Type (STT\_*) & Purpose & Example \\\hline
	STT\_FUNC & Represents a function in code & main() \\\hline
	STT\_OBJECT & Represents a global variable & int global\_var; \\\hline
	STT\_SECTION & Represents a section reference & .text \\\hline
	STT\_FILE & Represents a file symbol for debugging & "hello.c" \\
\end{NxIDBoxT}

\end{comment}

%\input{./contents/nex-c/format.d/binaryd/pe.tex}
%\input{./contents/nex-c/format.d/binary.d/macho.tex}

%\begin{NexColorBox}[color=1, style=2, title=Tools and Technologies]\relax
	\ShowSvgBox{0}{%
		../lib/svg/git.svg,%
		../lib/svg/docker.svg,%
		../lib/svg/podman.svg%
	}
	\begin{Items}
		\ItemArrow{\bftext{Neovim:} Skilled in configuring and customizing Neovim with Lua and Vim script to enhance productivity and streamline development workflows.}%
		\ItemArrow{\bftext{Git:} Experienced in using Git for source code management, branching, merging, and collaboration in a team environment.}%
		\ItemArrow{\bftext{Docker:} Proficient in creating, managing, and deploying Docker containers. Experienced in writing Dockerfiles and managing Docker Compose files.}%
		\ItemArrow{\bftext{Podman:} Skilled in using Podman for container lifecycle management, including building, running, and pushing containers.}%
		\ItemArrow{\bftext{runc:} Experienced in using runc as a container runtime for running and managing containers.}%
		\ItemArrow{\bftext{containerd:} Proficient in using containerd for container orchestration and management within Kubernetes clusters.}%
		\ItemArrow{\bftext{iproute2:} Skilled in using iproute2 tools for network configuration, routing, and traffic shaping.}%
		\ItemArrow{\bftext{nftables:} Experienced in using nftables for creating complex firewall rules and managing network security.}%
		\ItemArrow{\bftext{Extended Berkeley Packet Filter (eBPF):} Skilled in using eBPF for network performance analysis, monitoring, and enhancing security.}%
		\ItemArrow{\bftext{Buildah:} Proficient in using Buildah to create and manage container images, ensuring security and compliance.}%
		\ItemArrow{\bftext{crun:} Experienced in using crun as a container runtime to efficiently manage containerized.}%
	\end{Items}
\end{NexColorBox}
\smallskip


%\input{./contents/nex-c/format.d/linker.tex}
%\input{./contents/nex-c/object.d/linker.tex}

