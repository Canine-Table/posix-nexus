\begin{NxSSSBox}[breakable][Function Calling Conventions]
	\begin{NxIDBox}
		Function calling conventions define how arguments are passed between registers and stack, how return values are handled, and which registers are preserved across function calls. A standardized convention ensures binaries remain ABI-compatible across different architectures.
	\end{NxIDBox}
	\begin{NxIDBox}
		In the System V AMD64 ABI (used in Linux ELF binaries), the first six function arguments are passed using registers: `RDI`, `RSI`, `RDX`, `RCX`, `R8`, and `R9`. Any additional arguments are pushed onto the stack.
	\end{NxIDBox}
	\begin{NxIDBox}
		When a function is called, the \textbf{caller must ensure volatile registers (`RAX`, `RDI`, `RSI`, etc.)} are preserved if needed**. The **callee is responsible for saving non-volatile registers (`RBX`, `RBP`, `R12-R15`) if modified**.
	\end{NxIDBox}
\end{NxSSSBox}

\begin{NxIDBoxT}{l|l}[title={System V AMD64 ABI Calling Convention (x86-64)}]
    Register & Purpose in Function Calls \\\hline
    RDI & First function argument \\\hline
    RSI & Second function argument \\\hline
    RDX & Third function argument \\\hline
    RCX & Fourth function argument \\\hline
    R8	& Fifth function argument \\\hline
    R9	& Sixth function argument \\\hline
    RAX & Holds function return value \\\hline
    RSP & Stack pointer (points to function arguments beyond the sixth) \\\hline
    RBP & Base pointer (used for stack frames) \\
\end{NxIDBoxT}

\begin{NxCodeBox}{c}{dark, sdwA, frmE, title={When calling a function like}}
	int add(int a, int b) {
		return a + b;
	}
\end{NxCodeBox}

\bigskip

\begin{NxCodeBox}{asm}{dark, sdwA, frmE, title={The assembly-level execution for add(2, 3) would look like:}}
	mov rdi, 2    ; First argument (a)
	mov rsi, 3    ; Second argument (b)
	call add      ; Function call
	mov rax, rdi  ; Store return value in RAX
\end{NxCodeBox}

\bigskip

\begin{NxCodeBox}{c}{dark, sdwA, frmE, title={When a function has more than six arguments, excess arguments are pushed onto the stack instead of registers. Example}}
	int complex_function(int a, int b, int c, int d, int e, int f, int g);
\end{NxCodeBox}

\begin{NxIDBoxT}{l|l}[title={System V AMD64 ABI Argument Passing}]
    Argument & Passed via \\\hline
    a & RDI (register) \\\hline
    b & RSI (register) \\\hline
    c & RDX (register) \\\hline
    d & RCX (register) \\\hline
    e & R8 (register) \\\hline
    f & R9 (register) \\\hline
    g & Stack (RSP) \\
\end{NxIDBoxT}

\begin{NxIDBoxT}{l|l|l}[title={Register Preservation Rules}]
    Register & Preserved? & Usage \\\hline
    RBX & Yes & Used as a preserved general-purpose register \\\hline
    RSP & Yes & Stack pointer (must remain unchanged) \\\hline
    RBP & Yes & Used for stack frame management \\\hline
    R12-R15 & Yes & General-purpose registers \\\hline
    RAX & No & Used for return values, can be overwritten \\\hline
    RDI-R9 & No & Used for argument passing, not preserved \\
\end{NxIDBoxT}

\begin{NxSSSSBox}[breakable][Stack Frame Layout]
    \begin{NxIDBox}
	    Every function call creates a \textbf{stack frame}, allocating space for local variables, register preservation, and return addresses. The frame follows a predictable structure that assists debugging and ensures smooth execution.
    \end{NxIDBox}
    \begin{NxIDBox}
	    In \textbf{System V AMD64 ABI}, the stack grows \textbf{downward}, meaning newer frames are allocated at lower addresses. The base pointer (`RBP`) helps track stack position.
    \end{NxIDBox}
\end{NxSSSSBox}

\begin{NxIDBoxT}{l|l|l}[title={Stack Frame Breakdown}]
    Stack Content & Purpose & Stored In \\\hline
    Return Address & Address to jump back after function exits & [RSP] \\\hline
    Saved Registers & Preserved values from caller function & [RBP-8], [RBP-16] \\\hline
    Local Variables & Temporary function storage & [RBP-24], [RBP-32] \\\hline
    Function Arguments & Passed values beyond 6th argument & [RBP+8], [RBP+16] \\
\end{NxIDBoxT}

\begin{NxSSSSBox}[breakable]
	\begin{NxIDBox}
		Stack grows downward, ensuring each call doesn’t overwrite another’s memory. Caller saves vital registers before calling the callee function. Return address ensures proper function return flow.
	\end{NxIDBox}
\end{NxSSSSBox}

\begin{NxSSSSBox}[breakable][Syscall Execution]
    \begin{NxIDBox}
	System calls allow ELF binaries to interact with the **Linux kernel** for process management, file handling, memory allocation, and device communication.
    \end{NxIDBox}
    \begin{NxIDBox}
	Each syscall follows a strict register convention, where the **syscall number is placed in `RAX`, arguments are passed in `RDI`, `RSI`, `RDX`, etc.**, and the kernel returns a result in `RAX`.
    \end{NxIDBox}
\end{NxSSSSBox}

\begin{NxIDBoxT}{l|l|l}[title={Syscall Execution Flow}]
    Step & Operation & Register Used \\\hline
    1 & Load syscall number (SYS\_write = 1) & RAX \\\hline
    2 & Set file descriptor (stdout = 1) & RDI \\\hline
    3 & Set buffer address & RSI \\\hline
    4 & Set byte count & RDX \\\hline
    5 & Call syscall instruction & syscall \\\hline
    6 & Kernel executes operation & — \\\hline
    7 & Return result to RAX & RAX (bytes written) \\
\end{NxIDBoxT}

\begin{NxSSSSBox}[breakable]
	\begin{NxIDBox}
		Syscall interface is register-based, avoiding function overhead. Direct interaction with kernel ensures efficient execution. Results are returned via RAX, reducing complexity.
	\end{NxIDBox}
\end{NxSSSSBox}

\begin{NxSSSSBox}[breakable][Windows vs. System V ABI]
    \begin{NxIDBox}
	Calling conventions differ between Windows and Linux. **Windows x64 ABI passes only four arguments via registers (`RCX`, `RDX`, `R8`, `R9`)**, whereas **System V uses six registers (`RDI`, `RSI`, `RDX`, `RCX`, `R8`, `R9`)**.
	\end{NxIDBox}
	\begin{NxIDBox}
	    Additionally, Windows \textbf{requires stack space reservation for all arguments}, while System V only uses stack if arguments exceed six.
    \end{NxIDBox}
\end{NxSSSSBox}

\begin{NxIDBoxT}{l|l|l}[title={ABI Feature Comparison: System V vs. Windows x64}]
    ABI Feature & System V (Linux) & Windows x64 \\\hline
    Register Arguments & RDI, RSI, RDX, RCX, R8, R9 & RCX, RDX, R8, R9 \\\hline
    Stack Usage & Only after six arguments & Always reserved \\\hline
    Caller-Saved Registers & RAX, RDI, RSI, RDX, RCX, R8, R9 & RAX, RCX, RDX, R8, R9 \\\hline
    Function Return Register & RAX & RAX \\
\end{NxIDBoxT}

