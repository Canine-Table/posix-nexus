\begin{comment}
7. File Handling and I/O: Evolution Through the Years
7.1 Origins of File Systems and Early I/O (Pre-1970s)
7.1.1 The Birth of File Systems

7.1.1.1 Early Disk Storage Mechanisms

7.1.1.2 Primitive File Handling in Assembly

7.1.2 Raw I/O and Hardware-Based File Operations

7.1.2.1 Direct Disk Read/Write Instructions

7.1.2.2 Buffered vs Unbuffered Access

7.2 File Handling in K&R C (1978 - 1989)
7.2.1 Introduction of Standard File Operations

7.2.1.1 fopen, fclose: Opening and Closing Files

7.2.1.2 fread, fwrite: Binary vs Text File Reads

7.2.2 Early Buffering Concepts

7.2.2.1 setbuf, fflush for Controlling Output

7.2.2.2 Standard Streams (stdin, stdout, stderr)

7.3 ANSI C (1989 - 1999) – Standardization and Safety
7.3.1 Expanded File Handling Capabilities

7.3.1.1 fprintf, fscanf for Formatted Data Handling

7.3.1.2 File Seek and Positioning (ftell, fseek)

7.3.2 Error Handling and Robust File Processing

7.3.2.1 feof, ferror for File State Checking

7.3.2.2 Best Practices for Safe File Access

7.4 C99 (1999 - 2011) – Performance-Oriented Enhancements
7.4.1 Optimized File Handling with Direct Memory Access

7.4.1.1 Using mmap for Efficient File Reads

7.4.1.2 Performance Considerations of Buffered vs Unbuffered I/O

7.4.2 Extended Data Formatting with Wide Character Support

7.4.2.1 Unicode File Processing (fwprintf, fgetwc)

7.5 C11 (2011 - 2018) – Multi-threaded File Handling
7.5.1 Thread-Safe File Access

7.5.1.1 Handling Concurrent Reads/Writes

7.5.1.2 _Atomic for Data Integrity in File Operations

7.6 C17 and Beyond – Future of File Handling
7.6.1 Secure File Processing and Memory Safety

7.6.1.1 Avoiding Security Pitfalls in File Reads

7.6.1.2 Compiler-Based Detection for Unsafe File Operations

7.6.2 High-Speed Data Processing

7.6.2.1 Using SIMD (#include <immintrin.h>) for File Parsing

7.6.2.2 Optimizations for Large-Scale File I/O
\end{comment}

\begin{NxSBox}[][File Handling in C]
	\begin{NxIDBox}
		C provides powerful tools for working with files, allowing developers to read, write, and manipulate data stored on disk. This section explores file I/O operations and best practices for handling files efficiently.
	\end{NxIDBox}
	\begin{NxIDBoxL}
		\nxTopicD{Opening and Closing Files} Using `fopen()`, `fclose()`, and ensuring proper file handling.
		\nxTopicD{Reading and Writing Data} Managing file streams with `fread()`, `fwrite()`, `fprintf()`, and `fscanf()`.
		\nxTopicD{Error Handling} Detecting and managing file-related errors to prevent runtime failures.
	\end{NxIDBoxL}
\end{NxSBox}

