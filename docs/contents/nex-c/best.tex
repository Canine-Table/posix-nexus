\begin{comment}
10. Best Practices and Code Style: Evolution Through the Years
10.1 Early Coding Conventions and Manual Structuring (Pre-1970s)
10.1.1 Hardcoded Assembly-Like Structure in Early C

10.1.1.1 Minimal Use of Comments and Documentation

10.1.1.2 Direct Memory Manipulation vs Abstracted Code

10.1.1.3 Fixed Naming Conventions for Symbols

10.1.2 First Approaches to Readability

10.1.2.1 Single-Letter Variable Naming Practices

10.1.2.2 Lack of Modularization in Early Programs

10.2 K&R C and the Birth of Coding Standards (1978 - 1989)
10.2.1 Code Readability and Formatting

10.2.1.1 Indentation Styles: Tabs vs Spaces

10.2.1.2 Early Function Formatting

10.2.2 Modularization and Code Organization

10.2.2.1 Header Files (.h) and Modular Code

10.2.2.2 Using static for Internal Scope Management

10.3 ANSI C (1989 - 1999) – Standardization of Code Formatting
10.3.1 Structuring Programs for Maintainability

10.3.1.1 Clear Variable Naming Conventions

10.3.1.2 Function Naming and API Readability

10.3.2 Documentation and Commenting Standards

10.3.2.1 Using /* */ vs // for Comments

10.3.2.2 Writing Self-Documenting Code

10.4 C99 (1999 - 2011) – Modernization of Code Style
10.4.1 Function Organization and Consistency

10.4.1.1 Using Function Prototypes Consistently

10.4.1.2 Preventing Side Effects with const

10.4.2 Security and Safer Code Practices

10.4.2.1 Buffer Overflow Prevention with strncpy

10.4.2.2 Safer Memory Management with calloc

10.5 C11 (2011 - 2018) – Multithreading and Code Robustness
10.5.1 Writing Portable Multi-threaded Code

10.5.1.1 Using <threads.h> for Threading Safety

10.5.1.2 Avoiding Data Races with _Atomic

10.5.2 Using Compiler Features for Better Safety

10.5.2.1 Static Analysis and Code Linting

10.5.2.2 Bounds Checking for Safer Memory Usage

10.6 C17 and Beyond – Future Best Practices
10.6.1 Writing High-Performance and Maintainable Code

10.6.1.1 Compiler Optimization Flags (-O3, -march=native)

10.6.1.2 Using #pragma for Code Tuning

10.6.2 The Future of C Code Style

10.6.2.1 Influence of Rust on Safer Code Practices

10.6.2.2 AI-Based Code Formatting and Optimization
\end{comment}

\begin{NxSBox}[][Best Practices in C]
	\begin{NxIDBox}
		Writing clean, maintainable, and efficient C code requires following best practices. This section provides guidelines to ensure readability, safety, and performance in your programs.
	\end{NxIDBox}
	\begin{NxIDBoxL}
		\nxTopicD{Code Readability} Writing clear, well-documented code using proper indentation and naming conventions.
		\nxTopicD{Memory Safety} Avoiding memory leaks and buffer overflows by managing pointers responsibly.
		\nxTopicD{Efficient Algorithms} Choosing appropriate data structures and algorithms for optimal performance.
	\end{NxIDBoxL}
\end{NxSBox}

