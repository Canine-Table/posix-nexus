\begin{comment}
2.7 SIMD and CPU Architecture Integration
2.7.1 Introduction to SIMD (Single Instruction, Multiple Data)
2.7.1.1 What is SIMD and Why It Matters

2.7.1.2 How C Interfaces with SIMD Instructions

2.7.1.3 Differences Between Vectorized and Scalar Operations

2.7.2 MMX and SSE Evolution
2.7.2.1 MMX – The First Consumer SIMD Extension (1996)

2.7.2.1.1 Legacy 64-bit MMX Registers

2.7.2.1.2 MMX and Its Limitations

2.7.2.2 SSE (Streaming SIMD Extensions) Series

2.7.2.2.1 SSE1 (#include <xmmintrin.h>)

2.7.2.2.2 SSE2 (#include <emmintrin.h>) and 128-bit Registers

2.7.2.2.3 SSE3, SSSE3, and SSE4.x Enhancements

2.7.3 AVX and AVX2 – The 256-bit Revolution
2.7.3.1 Introduction to AVX

2.7.3.2 Key AVX Instructions (#include <immintrin.h>)

2.7.3.3 How AVX2 Improved Integer Performance

2.7.4 Detecting CPU Features and Using Them Safely
2.7.4.1 Checking CPU Capabilities (cpuid)

2.7.4.2 Writing Portable SIMD Code

2.7.4.3 Compiler Support for SIMD Optimizations

2.7.5 Future of SIMD – AVX-512 and Beyond
2.7.5.1 AVX-512 Registers and Expansions

2.7.5.2 Practical Use Cases of AVX-512
\end{comment}

\begin{NxSBox}[][Architecture in C]
	\begin{NxIDBox}
		C plays a critical role in system architecture, serving as the backbone of operating systems, embedded devices, and performance-critical applications.
	\end{NxIDBox}
	\begin{NxIDBoxL}
		\nxTopicD{ssec:System Programming}{System Programming} How C is used for writing operating systems, compilers, and device drivers.
		\nxTopicD{ssec:Embedded Systems}{Embedded Systems} C’s efficiency makes it a preferred choice for embedded development, from microcontrollers to IoT devices.
		\nxTopicD{ssec:Performance Considerations}{Performance Considerations} Architectural decisions that impact efficiency, such as cache usage, memory layout, and optimization techniques.
	\end{NxIDBoxL}
\end{NxSBox}

