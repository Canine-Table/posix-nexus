\begin{NxSSSBox}[breakable][ARMv7 (32-bit)]
	\begin{NxIDBox}
		ARMv7 is a 32-bit processor architecture designed for power efficiency and performance, making it ideal for mobile devices, embedded systems, and IoT applications.
	\end{NxIDBox}
\end{NxSSSBox}

\begin{NxIDBoxT}{l|l}[title={CPSR/SPSR Bit Meaning}]
	Bit(s) & Meaning \\\hline
	31 (N) \rightarrow 0 & Negative Flag (Not Set) \\\hline
	30 (Z) \rightarrow 0 & Zero Flag (Not Set) \\\hline
	29 (C) \rightarrow 1 & Carry Flag (Set) \\\hline
	28 (V) \rightarrow 1 & Overflow Flag (Set) \\\hline
	7-0 \rightarrow 1101 0011 & Processor Mode (0xD3) \\
\end{NxIDBoxT}

\begin{NxIDBoxT}{p{2cm}|p{2cm}|p{5cm}|p{5cm}}[title={ARMv7 Instruction Set \& Bit Encoding}]
	Instruction & Opcode (Machine Code) & Encoding Format & Description \\\hline
	MOV R0, \#5 & E3A00005 & Data Processing (Immediate) & Moves immediate value 5 into R0. \\\hline
	ADD R1, R2, R3 & E0811003 & Data Processing (Register) & Adds R2 + R3, storing result in R1. \\\hline
	SUB R4, R5, \#10 & E2454010 & Data Processing (Immediate) & Subtracts immediate 10 from R5, storing result in R4. \\\hline
	LDR R6, [R7] & E5976000 & Load/Store (Base Register) & Loads word from memory at R7 into R6. \\\hline
	STR R8, [R9, \#4] & E5898004 & Load/Store (Immediate Offset) & Stores R8 into memory at R9 + 4. \\\hline
	CMP R10, R11 & E15B0000 & Data Processing (Comparison) & Compares R10 with R11, setting status flags. \\\hline
	B label & EA000005 & Branch (PC-relative) & Unconditional branch to label, offset calculated from PC. \\\hline
	BL function & EB000008 & Branch \& Link & Calls function, storing return address in LR. \\\hline
	BX LR & E12FFF1E & Branch Exchange & Returns from a function (jump to LR). \\\hline
	AND R12, R0, R1 & E000C001 & Bitwise Logic (Register) & Bitwise AND between R0 and R1, result in R12. \\\hline
	ORR R14, R15, \#0xFF & E39F0EFF & Bitwise Logic (Immediate) & Bitwise OR between R15 and 0xFF, result in R14. \\\hline
	EOR R1, R2, R3 & E0221003 & Bitwise Logic (Exclusive OR) & Computes XOR (R2 ⊕ R3), stores result in R1. \\\hline
	LSL R2, R3, \#2 & E1A02083 & Shift (Logical Left) & Shifts R3 left by 2 bits, storing result in R2. \\\hline
	LSR R4, R5, \#3 & E1A041A5 & Shift (Logical Right) & Shifts R5 right by 3 bits, storing result in R4. \\\hline
	ASR R6, R7, \#1 & E1A070C1 & Shift (Arithmetic Right) & Shifts R7 right by 1 bit, preserving sign in R6. \\\hline
	ROR R8, R9, \#4 & E1A08919 & Rotate Right & Rotates R9 right by 4 bits, storing result in R8. \\\hline
	MUL R0, R1, R2 & E0010291 & Multiply (Register) & Multiplies R1 * R2, storing result in R0. \\\hline
	MLA R3, R4, R5, R6 & E0234596 & Multiply-Accumulate (Register) & Computes (R4 * R5) + R6, storing result in R3. \\\hline
	SWI \#0 & EF000000 & Software Interrupt & Triggers a system call (svc 0 in modern syntax). \\
\end{NxIDBoxT}

\begin{NxIDBoxT}{l|l|p{6cm}}[title={ARMv7 Privilege Modes}]
	Mode Name & Hex Value (Mode Bits) & Purpose \\\hline
	User Mode & 0x10 (10000) & Normal execution (least privileged) \\\hline
	FIQ Mode & 0x11 (10001) & Handles fast interrupts (higher priority) \\\hline
	IRQ Mode & 0x12 (10010) & Handles normal interrupts \\\hline
	Supervisor Mode (SVC) & 0x13 (10011) & Privileged OS mode (used for system calls) \\\hline
	Monitor Mode & 0x16 (10110) & Security mode for TrustZone \\\hline
	Abort Mode & 0x17 (10111) & Handles memory access faults \\\hline
	Hypervisor Mode & 0x1A (11010) & Virtualization mode (ARMv7-A with extensions) \\\hline
	Undefined Mode & 0x1B (11011) & Handles illegal instructions \\\hline
	System Mode & 0x1F (11111) & Like Supervisor mode, but allows full access to all registers \\
\end{NxIDBoxT}

\begin{NxIDBoxT}{l|l|l}[title={CPSR/SPSR Status \& Mode Bits}]
	Bit(s) & Value & Meaning \\\hline
	31-28 (NZCV) & 0001 & Negative = 0, Zero = 0, Carry = 1, Overflow = 1 \\\hline
	7-0 (Mode) & 1101 0011 & Mode = 0xD3 (Supervisor Mode with FIQ disabled) \\
\end{NxIDBoxT}

\begin{NxIDBoxT}{l|p{6cm}|p{6cm}}[title={ARMv7 Registers \& Their Usage}]
	Register & Name & Usage \& Description \\\hline
	R0 - R3 & Function Argument Registers & Used for passing arguments to functions; R0 also holds return values. \\\hline
	R4 - R11 & Callee-Saved Registers & Preserved across function calls, often used for storing local variables. \\\hline
	R12 & Scratch Register (Intra-Procedure Call) & Temporary storage used within functions but not preserved between calls. \\\hline
	R13 (SP) & Stack Pointer & Points to the top of the stack, controlling function calls and local variable storage. \\\hline
	R14 (LR) & Link Register & Holds the return address for function calls; used in BL and BX LR instructions. \\\hline
	R15 (PC) & Program Counter & Determines which instruction is executed next; modified by branch instructions (B, BL). \\\hline
	CPSR & Current Program Status Register & Stores condition flags (N, Z, C, V), interrupt settings, and execution mode. \\\hline
	SPSR & Saved Program Status Register & Used in exception handling to store the prior CPSR value for returning. \\\hline
	FPSCR & Floating-Point Status \& Control Register & Manages floating-point arithmetic behavior, rounding modes, and exceptions. \\
\end{NxIDBoxT}

\begin{NxIDBoxT}{l|l}[title={Multiplication \& Division Instructions}]
	Instruction & Operation \\\hline
	MUL R0, R1, R2 & Multiply R1 * R2, store result in R0 \\\hline
	MLA R3, R4, R5, R6 & Multiply R4 * R5 and add R6 \\\hline
	MLS R7, R8, R9, R10 & Multiply R8 * R9 then subtract R10 \\\hline
	SDIV R0, R1, R2 & Signed integer division (R1 / R2, result in R0) \\\hline
	UDIV R3, R4, R5 & Unsigned integer division (R4 / R5, result in R3) \\
\end{NxIDBoxT}

\begin{NxSSSSBox}[breakable][Syscalls in ARMv7]
	\begin{NxIDBox}
		System calls use the R7 register to store syscall numbers before invoking svc 0.
	\end{NxIDBox}
\end{NxSSSSBox}

\begin{NxSSSSBox}[breakable][Branching (ARMv7)]
	\begin{NxIDBox}
		Uses BL (Branch \& Link) for function calls, and BX LR for returning.
	\end{NxIDBox}
\end{NxSSSSBox}

\begin{NxIDBoxT}{l|l}[title={Common Conditional Instructions}]
	Instruction & Description \\\hline
	BEQ label & Branch if Equal (Z flag is set) \\\hline
	BNE label & Branch if Not Equal (Z flag is clear) \\\hline
	BLT label & Branch if Less Than (N ≠ V) \\\hline
	BGT label & Branch if Greater Than (Z = 0 and N = V) \\\hline
	BMI label & Branch if Negative (N = 1) \\\hline
	BPL label & Branch if Positive (N = 0) \\
\end{NxIDBoxT}

\bigskip

\begin{NxCodeBox}{asm}{dark, sdwA, frmE, title={Branching Examples}}
.global _start
_start:
	MOV R0,#12
	MOV R1,#2
	SUBS R0,R0,R1
	BL nx_flag_start
	B nx_svc_end
nx_flag_start:
	MRS R2, CPSR     	; Read CPSR into R2
	MOV R3, #0
nx_flag_n:
	TST R2, #0x80000000   	; Check N flag (Bit 31)
	BNE nx_negative       	; Branch if N is set
nx_flag_z:
	TST R2, #0x40000000   	; Check Z flag (Bit 30)
	BEQ nx_zero           	; Branch if Z is set
nx_flag_c:
	TST R2, #0x20000000   	; Check C flag (Bit 29)
	BNE nx_carry   	 	; Branch if C is set
nx_flag_v:
	TST R2, #0x10000000  	; Check V flag (Bit 28)
	BNE nx_overflow 	; Branch if V is set
nx_flag_t:
	TST R2, #0x00000020  	; Check T flag (Bit 5)
	BNE nx_thumb
nx_flag_f:
	TST R2, #0x00000040  	; Check F flag (Bit 6)
	BNE nx_fiq
nx_flag_i:
	TST R2, #0x00000080  	; Check I flag (Bit 7)
	BNE nx_irq
nx_flag_m:
	TST R2, #0x0000001F  	; Check processor mode bits (Bits 0-4)
	BNE nx_mode          	; Branch if any of Bits 0-4 are set
nx_flag_end:
	BX LR
nx_negative:
	ADD R3, R3, #67
	B nx_flag_z
nx_zero:
	ADD R3, R3, #31
	B nx_flag_c
nx_carry:
	ADD R3, R3, #37
	B nx_flag_v
nx_overflow:
	ADD R3, R3, #41
	B nx_flag_t
nx_irq:
	ADD R3, R3, #43
	B nx_flag_m
nx_fiq:
	ADD R3, R3, #47
	B nx_flag_i
nx_thumb:
	ADD R3, R3, #53
	B nx_flag_f
nx_mode:
	ADD R3, R3, #59
	B nx_flag_end
nx_svc_end:
	ADD R3, R3, #61
	MOV R7, #1   	; syscall: exit
	MOV R0, #0    	; exit code (0 = success)
	SVC 0         	; invoke syscall
\end{NxCodeBox}

\bigskip

\begin{NxSSSSBox}[breakable][Memory Model (ARMv7)]
	\begin{NxIDBox}
		Uses 32-bit memory addressing, limiting access to 4GB of RAM.
	\end{NxIDBox}
\end{NxSSSSBox}

\begin{NxSSSSBox}[breakable][SIMD in ARMv7]
	\begin{NxIDBox}
		Supports VFP (Vector Floating Point) and NEON for parallel processing.
	\end{NxIDBox}
\end{NxSSSSBox}

