\begin{NxSSSBox}[breakable][8-bit CISC Architectures]
	\begin{NxIDBox}
		Early 8-bit CPUs, such as the \nxGID{mos6502} and \nxGID{z80}, revolutionized personal computing by enabling affordable, accessible computing devices that laid the foundation for modern systems.
	\end{NxIDBox}
	\begin{NxIDBoxL}
		\nxTopicD{Registers in 8-bit CPUs} 8-bit CPUs had small general-purpose registers that stored temporary data, handled arithmetic operations, and facilitated efficient instruction execution.
		%\nxTopicD{Memory Management and Addressing} These processors typically used direct, indirect, and indexed addressing modes to interact with RAM, often constrained by small addressable memory spaces.
		%\nxTopicD{Instruction Sets amd Assembly Code} Each manufacturer designed unique instruction sets with varying efficiency, leading to differences in assembly programming across CPUs like the 8080, Z80, and 6502.
		%\nxTopicD{Arithmetic and Logic Operations} Basic math and logic operations, such as addition, subtraction, bitwise shifts, and comparisons, were performed using dedicated ALU circuits in an 8-bit framework.
		%\nxTopicD{Control Flow and Program Execution} Branching, loops, and conditional jumps allowed these CPUs to execute structured programs, relying on flags and status registers for decision-making.
		%\nxTopicD{I/O and System Interaction} Peripheral communication was handled via memory-mapped I/O or dedicated ports, enabling interaction with devices like keyboards, displays, and storage systems.
		%\nxTopicD{Evolution to 16-bit CPUs} Moving to 16-bit architectures improved processing power, increased addressable memory, and introduced new instruction capabilities, leading to more advanced applications
	\end{NxIDBoxL}
\end{NxSSSBox}


\begin{NxSSSSBox}[breakable][Registers in 8-bit CPUs]
	\begin{NxIDBox}
		Registers in 8-bit processors are small storage units inside the CPU that hold temporary data for calculations and instructions. Since these processors could only handle 8-bit values at a time, their registers were designed for efficiency in a limited space.
	\end{NxIDBox}
\end{NxSSSSBox}

\begin{NxIDBoxT}{l|p{6cm}|l}[title={General-Purpose Registers}]
    Register & Purpose & Examples (CPU) \\\hline
    A (Accumulator) & Used for arithmetic and logic operations & Intel 8080, MOS 6502 \\\hline
    B, C, D, E & Extra storage for temporary calculations & Intel 8080 \\\hline
    X, Y & Index registers for memory addressing & MOS 6502, Motorola 6800 \\\hline
    HL & Stores memory addresses & Zilog Z80 \\\hline
    SP (Stack Pointer) & Tracks top of the stack for function calls & Most 8-bit CPUs \\
\end{NxIDBoxT}

\begin{NxSSSSBox}[breakable]
	\begin{NxIDBox}
		Early 8-bit CPUs had \textbf{general-purpose registers}, which could store data for arithmetic, memory operations, and program execution.
	\end{NxIDBox}
\end{NxSSSSBox}

\begin{NxIDBoxT}{l|p{6cm}|l}[title={General-Purpose Registers}]
    Register & Purpose & Examples (CPU) \\\hline
    A (Accumulator) & Used for arithmetic and logic operations & Intel 8080, MOS 6502 \\\hline
    B, C, D, E & Extra storage for temporary calculations & Intel 8080 \\\hline
    X, Y & Index registers for memory addressing & MOS 6502, Motorola 6800 \\\hline
    HL & Stores memory addresses & Zilog Z80 \\\hline
    SP (Stack Pointer) & Tracks top of the stack for function calls & Most 8-bit CPUs \\
\end{NxIDBoxT}

\begin{NxSSSSBox}[breakable]
	\begin{NxIDBox}
		These registers helped manage \textbf{program execution}, memory access, and system control.
	\end{NxIDBox}
\end{NxSSSSBox}

\begin{comment}
\begin{NxSSSSBox}[breakable]
	\begin{NxIDBox}
	\end{NxIDBox}
\end{NxSSSSBox}
\end{comment}


