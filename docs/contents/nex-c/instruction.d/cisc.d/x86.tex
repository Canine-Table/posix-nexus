\begin{NxSSSBox}[breakable][x86]
	\begin{NxIDBox}
		x86 is a widely used processor architecture developed by Intel, known for its instruction set compatibility, scalability, and dominance in personal computers and enterprise systems.
	\end{NxIDBox}
	\begin{NxIDBoxL}
		\nxTopicD{Introduction to x86} x86 processors feature general-purpose registers (like EAX, EBX, ECX, and EDX) and special-purpose registers that help manage execution flow, data storage, and system interactions.
		%\nxTopicD{Registers in x86} x86 processors feature general-purpose registers (like EAX, EBX, ECX, and EDX) and special-purpose registers that help manage execution flow, data storage, and system interactions.
		%\nxTopicD{Memory Management in x86} In x86 systems, RAM interacts with registers via the memory hierarchy, leveraging paging, segmentation, and caching to optimize data access and execution speeds.
		%\nxTopicD{Instruction Set and Assembly Code Basics} x86 assembly consists of fundamental instructions such as MOV, ADD, SUB, and CMP, which allow direct manipulation of registers and memory for low-level programming.
		%\nxTopicD{Function Calling Conventions in x86} 32-bit x86 function calling conventions like cdecl, stdcall, and fastcall define how arguments are passed (often via the stack) and how registers are preserved across function calls.
		%\nxTopicD{Stack Operations and Control Flow} The stack in x86 plays a key role in function execution, storing return addresses, local variables, and facilitating structured control flow through CALL, RET, PUSH, and POP instructions.
	\end{NxIDBoxL}
\end{NxSSSBox}

\begin{NxSSSSBox}[breakable][Introduction to x86]
	\begin{NxIDBox}
		x86 is the \textbf{32-bit architecture} that served as the foundation for modern processors. Developed by Intel, it evolved from earlier 16-bit designs and introduced \textbf{protected mode}, \textbf{paging}, and efficient \textbf{function calling conventions} that optimized memory access. Most operating systems, including Windows and Linux, were originally designed around x86 processors.
	\end{NxIDBox}
	\begin{NxIDBox}
		While x86 relies on \nxGID{cisc} principles, enabling \textbf{powerful multi-step instructions}, it also incorporates \textbf{register-based optimizations} to improve execution speed.
	\end{NxIDBox}
	\begin{NxIDBox}
		CISC architectures emphasize \textbf{complex, variable-length instructions}, allowing a single instruction to perform \textbf{multiple operations} (e.g., memory access + computation). They prioritize \textbf{flexible addressing modes}, enabling direct interactions between registers and memory.
	\end{NxIDBox}
\end{NxSSSSBox}

%\begin{NxSSSSBox}[breakable][Registers in x86]

%\end{NxSSSSBox}
