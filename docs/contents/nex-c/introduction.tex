\begin{comment}

1.2 History of C
1.2.1 Origins and Development

1.2.1.1 From Assembly to Structured Programming

1.2.1.2 Dennis Ritchie’s Role in C’s Birth

1.2.1.3 K&R C and the UNIX Revolution

1.2.1.4 Standardization: ANSI C, ISO C

1.2.1.5 C’s Role in System Programming and Embedded Systems

1.2.2 Evolution Through Standards

1.2.2.1 C89/C90: The First Standard

1.2.2.2 C99: New Features and Modernization

1.2.2.3 C11: Multithreading, Atomics, and Unicode

1.2.2.4 C17 and Minor Updates

1.2.2.5 Future Directions of C

1.3 Setting Up the Development Environment
1.3.1 Choosing a Compiler

1.3.1.1 GCC

1.3.1.2 Clang

1.3.1.3 MSVC

1.3.1.4 Intel Compiler

1.3.1.5 ARM Compiler for Embedded Systems

1.3.2 Development Tools

1.3.2.1 Debuggers: GDB, LLDB, WinDbg

1.3.2.2 Profilers and Performance Tools

1.3.2.3 Static Analysis and Code Checking

1.3.2.4 IDEs vs. Command Line Development

1.3.2.5 Build Systems: Make, CMake, Meson

1.4 Compilation and Execution Workflow
1.4.1 Understanding Compilation Phases

1.4.1.1 Preprocessing (#include, #define)

1.4.1.2 Compilation and Optimization

1.4.1.3 Linking and Executables

1.4.1.4 Assembly Generation

1.4.1.5 ELF and PE Formats

1.4.2 Executing a Program

1.4.2.1 Understanding Executables

1.4.2.2 Running C Programs on Linux

1.4.2.3 Running C Programs on Windows

1.4.2.4 Running C Programs on Embedded Devices

1.4.2.5 Debugging Execution Errors

\end{comment}

\begin{NxSBox}[][Introduction to C]
	\begin{NxIDBox}
		sections have no detail, on anything, only subsections do, and subsubsections have greater detail of the subject
	\end{NxIDBox}
	\begin{NxIDBoxL}
		\nxTopicD{ssec:History of C}{History of C} The history of C is deeply tied to the evolution of programming languages and system development. Developed in the early 1970s, C emerged as a language that balanced efficiency, portability, and structured programming, leading to its widespread adoption in modern computing.
		what to expect if you read this section
	\end{NxIDBoxL}
\end{NxSBox}

\begin{NxSBox}[][History of C]
    \begin{NxIDBox}
        The history of C is closely tied to the evolution of computing and software development. This section explores its origins, major milestones, and continued influence on modern programming.
    \end{NxIDBox}
    \begin{NxIDBoxL}
        \nxTopicD{ssec:Origins and Development}{Origins and Development} C was created as an improvement over existing languages, particularly assembly, \nxGID{algol}, and \nxGID{bcpl}, to provide better portability and structure.
    \end{NxIDBoxL}
    \begin{NxIDBoxL}
        \nxTopicD{ssec:Evolution Through Standards}{Evolution Through Standards} Over the years, C has evolved through official standards such as C89, C99, C11, and C17, each refining features and improving compatibility across platforms.
    \end{NxIDBoxL}
\end{NxSBox}

\begin{NxSSBox}[][Origins and Development]
    \begin{NxIDBox}
        The development of C was driven by the need for a flexible, efficient, and portable programming language that could be used for system programming and application development.
    \end{NxIDBox}
    \begin{NxIDBoxL}
        \nxTopicD{sssec:Assembly to Structured Programming}{From Assembly to Structured Programming} Before C, programmers relied on assembly, which was efficient but lacked portability and structure.
    \end{NxIDBoxL}
    \begin{NxIDBoxL}
        \nxTopicD{sssec:Dennis Ritchie’s Role}{Dennis Ritchie’s Role in C’s Birth} Dennis \nxGID{ritchie} developed C at \nxGID{bell_labs} to provide a balance between low-level control and structured programming.
    \end{NxIDBoxL}
\end{NxSSBox}

\begin{NxSSSBox}[][Assembly to Structured Programming]
    \begin{NxIDBox}[title={Limitations of Assembly}]
        Assembly language allowed direct hardware manipulation, but its complexity made programming tedious, with poor readability and lack of portability.
    \end{NxIDBox}
    \begin{NxIDBox}[title={Influence of ALGOL and BCPL}]
        Early high-level languages like \nxGID{algol} and \nxGID{bcpl} introduced structured programming, which influenced the development of C.
    \end{NxIDBox}
\end{NxSSSBox}

\begin{NxSSSSBox}[][Limitations of Assembly]
    \begin{NxIDBox}[title={Challenges in Portability}]
        Code written in assembly was tightly linked to specific hardware architectures, making it difficult to adapt software across different systems.
    \end{NxIDBox}
    \begin{NxIDBox}[title={Complexity in Debugging}]
        Assembly programs required detailed memory management, making debugging significantly harder compared to structured programming languages.
    \end{NxIDBox}
\end{NxSSSSBox}

\begin{NxSSSBox}[][Origins and Development]
	\begin{NxIDBox}[title=From Assembly to Structured Programming]
		Before \nxGID{c_language}, programming was done mainly in assembly language, which was powerful but difficult to maintain and lacked portability across different hardware architectures.
	\end{NxIDBox}
	\begin{NxIDBox}[title={Dennis Ritchie’s Role in C’s Birth}]
		Dennis \nxGID{ritchie} developed C at \nxGID{bell_labs} in the early 1970s to enhance the \nxGID{unix} operating system, incorporating features that made programming more accessible while retaining system-level control.
	\end{NxIDBox}
\end{NxSSSBox}





