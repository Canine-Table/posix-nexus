\begin{comment}

1.2 History of C
1.2.1 Origins and Development

1.2.1.1 From Assembly to Structured Programming

1.2.1.2 Dennis Ritchie’s Role in C’s Birth

1.2.1.3 K&R C and the UNIX Revolution

1.2.1.4 Standardization: ANSI C, ISO C

1.2.1.5 C’s Role in System Programming and Embedded Systems

1.2.2 Evolution Through Standards

1.2.2.1 C89/C90: The First Standard

1.2.2.2 C99: New Features and Modernization

1.2.2.3 C11: Multithreading, Atomics, and Unicode

1.2.2.4 C17 and Minor Updates

1.2.2.5 Future Directions of C

1.3 Setting Up the Development Environment
1.3.1 Choosing a Compiler

1.3.1.1 GCC

1.3.1.2 Clang

1.3.1.3 MSVC

1.3.1.4 Intel Compiler

1.3.1.5 ARM Compiler for Embedded Systems

1.3.2 Development Tools

1.3.2.1 Debuggers: GDB, LLDB, WinDbg

1.3.2.2 Profilers and Performance Tools

1.3.2.3 Static Analysis and Code Checking

1.3.2.4 IDEs vs. Command Line Development

1.3.2.5 Build Systems: Make, CMake, Meson

1.4 Compilation and Execution Workflow
1.4.1 Understanding Compilation Phases

1.4.1.1 Preprocessing (#include, #define)

1.4.1.2 Compilation and Optimization

1.4.1.3 Linking and Executables

1.4.1.4 Assembly Generation

1.4.1.5 ELF and PE Formats

1.4.2 Executing a Program

1.4.2.1 Understanding Executables

1.4.2.2 Running C Programs on Linux

1.4.2.3 Running C Programs on Windows

1.4.2.4 Running C Programs on Embedded Devices

1.4.2.5 Debugging Execution Errors

\end{comment}

\begin{NxSBox}[][Introduction to C]
	\begin{NxIDBox}
		sections have no detail, on anything, only subsections do, and subsubsections have greater detail of the subject
	\end{NxIDBox}
	\begin{NxIDBoxL}
		\nxTopicD{History of C} The evolution of C, from assembly and BCPL to its role in system programming and modern computing.
		\nxTopicD{Design Philosophy and How C Was Designed} Core design principles—simplicity, efficiency, portability—and why they make C unique.
		\nxTopicD{Use Cases of C} Where C is applied—from system programming and embedded development to game engines and performance-critical applications.
	\end{NxIDBoxL}
\end{NxSBox}

\begin{NxSBox}[][History of C]
    \begin{NxIDBox}
        The history of C is closely tied to the evolution of computing and software development. This section explores its origins, major milestones, and continued influence on modern programming.
    \end{NxIDBox}
    \begin{NxIDBoxL}
        \nxTopicD{ssec:Origins and Development}{Origins and Development} C was created as an improvement over existing languages, particularly assembly, \nxGID{algol}, and \nxGID{bcpl}, to provide better portability and structure.
    \end{NxIDBoxL}
    \begin{NxIDBoxL}
        \nxTopicD{ssec:Evolution Through Standards}{Evolution Through Standards} Over the years, C has evolved through official standards such as C89, C99, C11, and C17, each refining features and improving compatibility across platforms.
    \end{NxIDBoxL}
\end{NxSBox}

\begin{NxSSBox}[][Origins and Development]
    \begin{NxIDBox}
        The development of C was driven by the need for a flexible, efficient, and portable programming language that could be used for system programming and application development.
    \end{NxIDBox}
    \begin{NxIDBoxL}
        \nxTopicD{sssec:Assembly to Structured Programming}{From Assembly to Structured Programming} Before C, programmers relied on assembly, which was efficient but lacked portability and structure.
    \end{NxIDBoxL}
    \begin{NxIDBoxL}
        \nxTopicD{sssec:Dennis Ritchie’s Role}{Dennis Ritchie’s Role in C’s Birth} Dennis \nxGID{ritchie} developed C at \nxGID{bell_labs} to provide a balance between low-level control and structured programming.
    \end{NxIDBoxL}
\end{NxSSBox}

\begin{NxSSSBox}[][Assembly to Structured Programming]
    \begin{NxIDBox}[title={Limitations of Assembly}]
        Assembly language allowed direct hardware manipulation, but its complexity made programming tedious, with poor readability and lack of portability.
    \end{NxIDBox}
    \begin{NxIDBox}[title={Influence of ALGOL and BCPL}]
        Early high-level languages like \nxGID{algol} and \nxGID{bcpl} introduced structured programming, which influenced the development of C.
    \end{NxIDBox}
\end{NxSSSBox}

\begin{NxSSSSBox}[][Limitations of Assembly]
    \begin{NxIDBox}[title={Challenges in Portability}]
        Code written in assembly was tightly linked to specific hardware architectures, making it difficult to adapt software across different systems.
    \end{NxIDBox}
    \begin{NxIDBox}[title={Complexity in Debugging}]
        Assembly programs required detailed memory management, making debugging significantly harder compared to structured programming languages.
    \end{NxIDBox}
\end{NxSSSSBox}

\begin{NxSSSBox}[][Origins and Development]
	\begin{NxIDBox}[title=From Assembly to Structured Programming]
		Before \nxGID{c_language}, programming was done mainly in assembly language, which was powerful but difficult to maintain and lacked portability across different hardware architectures.
	\end{NxIDBox}
	\begin{NxIDBox}[title={Dennis Ritchie’s Role in C’s Birth}]
		Dennis \nxGID{ritchie} developed C at \nxGID{bell_labs} in the early 1970s to enhance the \nxGID{unix} operating system, incorporating features that made programming more accessible while retaining system-level control.
	\end{NxIDBox}
\end{NxSSSBox}




\begin{NxSSBox}[][Design Philosophy of C]
	\begin{NxIDBox}
		C was designed with efficiency, portability, and minimalism in mind. Unlike many modern languages that prioritize abstraction, C maintains a fine balance between human-readable syntax and low-level hardware control.
		\nxTopicD{sssec:Simplicity and Minimalism}{Simplicity and Minimalism} C avoids excessive abstraction, keeping the language simple, predictable, and lightweight.
		\nxTopicD{sssec:Portability and Flexibility}{Portability and Flexibility} Programs written in C can run across multiple hardware architectures with little modification.
		\nxTopicD{sssec:Efficiency and Direct Hardware Access}{Efficiency and Direct Hardware Access} C provides precise memory control, allowing developers to write highly optimized code.
	\end{NxIDBoxL}
\end{NxSSBox}


\begin{NxSSSBox}[][Simplicity and Minimalism]
	\begin{NxIDBox}
		Unlike languages that emphasize high-level features, C keeps things straightforward with a small set of essential keywords and features.
	\end{NxIDBox}
	\begin{NxIDBoxL}
		\nxTopicD{sssec:Small Language Core}{Small Language Core} C has a minimal standard library compared to modern high-level languages.
	\end{NxIDBoxL}
	\begin{NxIDBoxL}
		\nxTopicD{sssec:Predictable Behavior}{Predictable Behavior} No hidden operations—C keeps execution transparent and efficient.
	\end{NxIDBoxL}
\end{NxSSSBox}




\begin{NxSSBox}[][Use Cases of C]
	\begin{NxIDBox}
		C remains one of the most widely used programming languages due to its efficiency, reliability, and low-level control over hardware.
	\end{NxIDBox}
	\begin{NxIDBoxL}
		\nxTopicD{Systems Programming} How C powers operating systems, compilers, and low-level utilities.
		\nxTopicD{Embedded Systems and IoT}Why C dominates microcontroller and firmware development.
		\nxTopicD{Game Development and Performance Critical Applications} How C enables fast, optimized graphics engines and scientific computing.
	\end{NxIDBoxL}
\end{NxSSBox}

\begin{NxSSSBox}[breakable][Embedded Systems and IoT]
	\begin{NxIDBox}
		C is the dominant language in embedded systems and \nxGID{iot} due to its efficiency, low-level control, and direct hardware access. Its lightweight footprint makes it ideal for resource-constrained environments.
	\end{NxIDBox}
	\begin{NxIDBoxL}
		\nxTopicD{Microcontroller Programming} How C is used to write firmware for embedded processors.
		\nxTopicD{Real-Time Operating Systems (RTOS)} Why C facilitates deterministic execution in embedded environments.
		\nxTopicD{Hardware Interaction and Optimization} How C enables direct control over registers, memory, and peripherals.
	\end{NxIDBoxL}
\end{NxSSSBox}

\begin{NxSSSSBox}[breakable][Microcontroller Programming]
	\begin{NxIDBox}
		C is the primary language for microcontroller programming due to its ability to interact directly with hardware while maintaining efficiency in resource-constrained environments.
	\end{NxIDBox}
	\begin{NxIDBox}
		Microcontrollers such as \textbf{ARM Cortex-M, AVR, and ESP32} rely on C for firmware development. This allows developers to control registers, configure peripherals, and optimize power consumption.
	\end{NxIDBox}
	\begin{NxIDBox}
		Unlike higher-level languages, C provides \textbf{precise memory control}, enabling developers to manipulate hardware with minimal overhead. Direct memory access ensures predictable execution in embedded applications.
	\end{NxIDBox}
	\begin{NxIDBox}
		Through \textbf{interrupt-driven programming}, developers can ensure real-time responsiveness in microcontroller-based systems. Efficient handling of external signals is essential in automotive, industrial automation, and consumer electronics.
	\end{NxIDBox}
\end{NxSSSSBox}

\begin{NxSSSSBox}[][Real-Time Operating Systems (RTOS)]
	\begin{NxIDBox}
		C is widely used in real-time operating systems (RTOS) due to its ability to manage hardware resources efficiently while ensuring deterministic execution.
	\end{NxIDBox}
	\begin{NxIDBox}
		An RTOS enables precise \textbf{task scheduling}, ensuring that time-sensitive operations execute predictably. This is essential for applications like robotics, aerospace, and medical devices.
	\end{NxIDBox}
	\begin{NxIDBox}
		Popular RTOS implementations written in C include \textbf{FreeRTOS, VxWorks, and RTEMS}. These systems provide lightweight multitasking and prioritize execution within strict timing constraints.
	\end{NxIDBox}
	\begin{NxIDBox}
		By leveraging C’s low-level control, RTOS developers can fine-tune system performance, ensuring minimal latency and consistent operation across embedded platforms.
	\end{NxIDBox}
\end{NxSSSSBox}

\begin{NxSSSSBox}[][Hardware Interaction and Optimization]
	\begin{NxIDBox}
		C enables direct hardware interaction, allowing developers to write efficient code that manages registers, memory, and communication interfaces.
	\end{NxIDBox}
	\begin{NxIDBox}
		Embedded applications require \textbf{register-level programming}, where developers control individual hardware components using memory-mapped I/O. This ensures optimal hardware utilization.
	\end{NxIDBox}
	\begin{NxIDBox}
		Optimizing embedded software requires \textbf{low-level memory manipulation} to minimize execution overhead. C provides precise control over stack and heap usage, ensuring predictable performance.
	\end{NxIDBox}
	\begin{NxIDBox}
		Through \textbf{direct peripheral access}, C facilitates efficient interaction with hardware components such as sensors, actuators, and communication buses, making it indispensable in IoT and embedded development.
	\end{NxIDBox}
\end{NxSSSSBox}


\begin{NxSSSBox}[breakable][Systems Programming]
	\begin{NxIDBox}
		C is the backbone of systems programming due to its efficiency, low-level control, and direct interaction with operating system components. It enables developers to write highly optimized code for kernel development, compilers, and system utilities.
	\end{NxIDBox}
	\begin{NxIDBoxL}
		\nxTopicD{Operating System Kernels} How C powers major operating systems, including Linux, Windows, and macOS.
		\nxTopicD{Compiler Development} Why C is used to build compilers that translate high-level languages into executable code.
		\nxTopicD{System Utilities and Performance Tools} How C enables the creation of fast, reliable system-level applications.
	\end{NxIDBoxL}
\end{NxSSSBox}

\begin{NxSSSSBox}[breakable][Operating System Kernels]
	\begin{NxIDBox}
		C is the dominant language for operating system kernels due to its low-level efficiency and direct hardware access. It enables precise memory management and system resource control.
	\end{NxIDBox}
	\begin{NxIDBox}
		Major operating systems like \textbf{Linux, Windows, macOS, and Unix} have their kernels primarily implemented in C. This ensures portability across architectures and allows fine-tuned performance optimizations.
	\end{NxIDBox}
	\begin{NxIDBox}
		Kernel development in C involves \textbf{interrupt handling, process scheduling, and memory allocation}, ensuring that the OS operates with minimal overhead while maintaining stability.
	\end{NxIDBox}
	\begin{NxIDBox}
		The \textbf{\nxGID{posix} standard}, developed for Unix-like systems, defines system APIs in C, making it the universal choice for kernel-level programming across multiple platforms.
	\end{NxIDBox}
\end{NxSSSSBox}

\begin{NxSSSSBox}[breakable][Compiler Development]
	\begin{NxIDBox}
		C has been instrumental in compiler development due to its simplicity, efficiency, and ability to produce highly optimized machine code.
	\end{NxIDBox}
	\begin{NxIDBox}
		Many widely used compilers, including \textbf{GCC (GNU Compiler Collection), Clang, and MSVC}, are written in C or C++. These compilers ensure efficient translation of high-level code into executable binaries.
	\end{NxIDBox}
	\begin{NxIDBox}
		C’s structured syntax and deterministic behavior make it easier to implement \textbf{lexical analysis, parsing, optimization, and code generation}, which are critical components in modern compiler architectures.
	\end{NxIDBox}
	\begin{NxIDBox}
		The \textbf{\nxGID{llvm} project}, an open-source compiler infrastructure, utilizes C for its core components, demonstrating C’s continued relevance in compiler construction and performance engineering.
	\end{NxIDBox}
\end{NxSSSSBox}

\begin{NxSSSSBox}[breakable][System Utilities and Performance Tools]
	\begin{NxIDBox}
		C is widely used for creating system utilities and performance-critical software due to its ability to execute efficiently with minimal overhead.
	\end{NxIDBox}
	\begin{NxIDBox}
		Command-line tools such as \textbf{grep, sed, awk, and ls} in Unix-based systems are implemented in C, ensuring optimal execution speed and system compatibility.
	\end{NxIDBox}
	\begin{NxIDBox}
		Performance monitoring tools like \textbf{htop, strace, and gprof} leverage C to provide real-time system diagnostics, ensuring efficient resource utilization and debugging capabilities.
	\end{NxIDBox}
	\begin{NxIDBox}
		C is also used in high-speed networking utilities, allowing developers to write software that directly interacts with system APIs and network protocols with minimal latency.
	\end{NxIDBox}
\end{NxSSSSBox}


\begin{NxSSSBox}[breakable][Game Development and Performance Critical Applications]
	\begin{NxIDBox}
		C is widely used in game development and performance-critical applications due to its ability to manage memory efficiently, optimize execution speed, and provide direct control over hardware resources.
	\end{NxIDBox}
	\begin{NxIDBoxL}
		\nxTopicD{Graphics Engines and Optimization} How C powers high-performance rendering frameworks.
		\nxTopicD{Physics and Computational Efficiency} Why C excels at real-time simulations and numerical processing.
		\nxTopicD{Low-Level System Interaction} How C allows direct hardware access for gaming and scientific applications.
	\end{NxIDBoxL}
\end{NxSSSBox}

\begin{NxSSSSBox}[breakable][Graphics Engines and Optimization]
	\begin{NxIDBox}
		C is widely used in graphics engines due to its ability to manage memory efficiently and optimize execution speed. It provides direct control over hardware, ensuring high-performance rendering.
	\end{NxIDBox}
	\begin{NxIDBox}
		Many leading graphics engines, such as \textbf{Unreal Engine, Unity (low-level components), and id Tech}, rely on C or C++ for performance-critical rendering pipelines.
	\end{NxIDBox}
	\begin{NxIDBox}
		C enables efficient \textbf{vertex processing, texture management, and shader execution}, ensuring smooth frame rates in real-time applications.
	\end{NxIDBox}
	\begin{NxIDBox}
		Through low-level \nxGID{opengl}, \nxGID{vulkan}, and \nxGID{directx} bindings, C provides access to GPU acceleration for graphics-intensive applications.
	\end{NxIDBox}
\end{NxSSSSBox}

\begin{NxSSSSBox}[breakable][Physics and Computational Efficiency]
	\begin{NxIDBox}
		C is essential for physics engines and computational simulations due to its ability to handle large-scale calculations with minimal overhead.
	\end{NxIDBox}
	\begin{NxIDBox}
		Physics engines like \nxGID{havok}, \nxGID{physx}, and \nxGID{bullet} rely on C/C++ for real-time collision detection, rigid body dynamics, and fluid simulations.
	\end{NxIDBox}
	\begin{NxIDBox}
	Numerical computing libraries, such as \nxGID{blas}, \nxGID{lapack}, and \nxGID{fftw}, leverage C’s low-level efficiency to optimize mathematical computations for scientific applications.
	\end{NxIDBox}
	\begin{NxIDBox}
		C’s deterministic execution ensures predictable processing times, which is essential for simulations requiring precision timing, such as robotics and computational physics.
	\end{NxIDBox}
\end{NxSSSSBox}

\begin{NxSSSSBox}[breakable][Low-Level System Interaction]
	\begin{NxIDBox}
		C facilitates direct hardware access, making it a core language for performance-critical applications, including gaming, high-performance computing, and graphics rendering.
	\end{NxIDBox}
	\begin{NxIDBox}
		Many game engines utilize C for \textbf{low-latency input handling}, allowing direct interaction with hardware devices such as controllers, keyboards, and GPUs.
	\end{NxIDBox}
	\begin{NxIDBox}
		C’s ability to interface with \textbf{multithreading and concurrency models} ensures optimal utilization of CPU cores in computationally intensive applications.
	\end{NxIDBox}
	\begin{NxIDBox}
		High-performance computing applications leverage C to interact with \textbf{parallel processing architectures}, such as GPU-accelerated deep learning frameworks.
	\end{NxIDBox}
\end{NxSSSSBox}




