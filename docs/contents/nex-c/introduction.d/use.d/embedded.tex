\begin{NxSSSBox}[breakable][Embedded Systems and IoT]
	\begin{NxIDBox}
		C is the dominant language in embedded systems and \nxGID{iot} due to its efficiency, low-level control, and direct hardware access. Its lightweight footprint makes it ideal for resource-constrained environments.
	\end{NxIDBox}
	\begin{NxIDBoxL}
		\nxTopicD{Microcontroller Programming} How C is used to write firmware for embedded processors.
		\nxTopicD{Real-Time Operating Systems (RTOS)} Why C facilitates deterministic execution in embedded environments.
		\nxTopicD{Hardware Interaction and Optimization} How C enables direct control over registers, memory, and peripherals.
	\end{NxIDBoxL}
\end{NxSSSBox}

\begin{NxSSSSBox}[breakable][Microcontroller Programming]
	\begin{NxIDBox}
		C is the primary language for microcontroller programming due to its ability to interact directly with hardware while maintaining efficiency in resource-constrained environments.
	\end{NxIDBox}
	\begin{NxIDBox}
	Microcontrollers such as \nxGID{arm_cortex_m}, \nxGID{avr}, and \nxGID{esp32} rely on C for firmware development. This allows developers to control registers, configure peripherals, and optimize power consumption.
	\end{NxIDBox}
	\begin{NxIDBox}
		Unlike higher-level languages, C provides \textbf{precise memory control}, enabling developers to manipulate hardware with minimal overhead. Direct memory access ensures predictable execution in embedded applications.
	\end{NxIDBox}
	\begin{NxIDBox}
		Through \nxGID{interrupt_driven}, developers can ensure real-time responsiveness in microcontroller-based systems. Efficient handling of external signals is essential in automotive, industrial automation, and consumer electronics.
	\end{NxIDBox}
\end{NxSSSSBox}

\begin{NxSSSSBox}[breakable][Real-Time Operating Systems (RTOS)]
	\begin{NxIDBox}
		C is widely used in \nxGID{rtos} due to its ability to manage hardware resources efficiently while ensuring deterministic execution.
	\end{NxIDBox}
	\begin{NxIDBox}
		An RTOS enables precise \textbf{task scheduling}, ensuring that time-sensitive operations execute predictably. This is essential for applications like robotics, aerospace, and medical devices.
	\end{NxIDBox}
	\begin{NxIDBox}
Popular RTOS implementations written in C include \nxGID{freertos}, \nxGID{vxworks}, and \nxGID{rtems}. These systems provide lightweight multitasking and prioritize execution within strict timing constraints.
	\end{NxIDBox}
	\begin{NxIDBox}
		By leveraging C’s low-level control, RTOS developers can fine-tune system performance, ensuring minimal latency and consistent operation across embedded platforms.
	\end{NxIDBox}
\end{NxSSSSBox}

\begin{NxSSSSBox}[breakable][Hardware Interaction and Optimization]
	\begin{NxIDBox}
		C enables direct hardware interaction, allowing developers to write efficient code that manages registers, memory, and communication interfaces.
	\end{NxIDBox}
	\begin{NxIDBox}
		Embedded applications require \textbf{register-level programming}, where developers control individual hardware components using memory-mapped I/O. This ensures optimal hardware utilization.
	\end{NxIDBox}
	\begin{NxIDBox}
		Optimizing embedded software requires \textbf{low-level memory manipulation} to minimize execution overhead. C provides precise control over stack and heap usage, ensuring predictable performance.
	\end{NxIDBox}
	\begin{NxIDBox}
		Through \textbf{direct peripheral access}, C facilitates efficient interaction with hardware components such as sensors, actuators, and communication buses, making it indispensable in IoT and embedded development.
	\end{NxIDBox}
\end{NxSSSSBox}

