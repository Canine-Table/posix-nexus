\begin{NxSSSBox}[breakable][Systems Programming]
	\begin{NxIDBox}
		C is the backbone of systems programming due to its efficiency, low-level control, and direct interaction with operating system components. It enables developers to write highly optimized code for kernel development, compilers, and system utilities.
	\end{NxIDBox}
	\begin{NxIDBoxL}
		\nxTopicD{Operating System Kernels} How C powers major operating systems, including Linux, Windows, and macOS.
		\nxTopicD{Compiler Development} Why C is used to build compilers that translate high-level languages into executable code.
		\nxTopicD{System Utilities and Performance Tools} How C enables the creation of fast, reliable system-level applications.
	\end{NxIDBoxL}
\end{NxSSSBox}

\begin{NxSSSSBox}[breakable][Operating System Kernels]
	\begin{NxIDBox}
		C is the dominant language for operating system kernels due to its low-level efficiency and direct hardware access. It enables precise memory management and system resource control.
	\end{NxIDBox}
	\begin{NxIDBox}
		Major operating systems like \textbf{Linux, Windows, macOS, and Unix} have their kernels primarily implemented in C. This ensures portability across architectures and allows fine-tuned performance optimizations.
	\end{NxIDBox}
	\begin{NxIDBox}
		Kernel development in C involves \textbf{interrupt handling, process scheduling, and memory allocation}, ensuring that the OS operates with minimal overhead while maintaining stability.
	\end{NxIDBox}
	\begin{NxIDBox}
		The \textbf{\nxGID{posix} standard}, developed for Unix-like systems, defines system APIs in C, making it the universal choice for kernel-level programming across multiple platforms.
	\end{NxIDBox}
\end{NxSSSSBox}

\begin{NxSSSSBox}[breakable][Compiler Development]
	\begin{NxIDBox}
		C has been instrumental in compiler development due to its simplicity, efficiency, and ability to produce highly optimized machine code.
	\end{NxIDBox}
	\begin{NxIDBox}
		Many widely used compilers, including \textbf{GCC (GNU Compiler Collection), Clang, and MSVC}, are written in C or C++. These compilers ensure efficient translation of high-level code into executable binaries.
	\end{NxIDBox}
	\begin{NxIDBox}
		C’s structured syntax and deterministic behavior make it easier to implement \textbf{lexical analysis, parsing, optimization, and code generation}, which are critical components in modern compiler architectures.
	\end{NxIDBox}
	\begin{NxIDBox}
		The \textbf{\nxGID{llvm} project}, an open-source compiler infrastructure, utilizes C for its core components, demonstrating C’s continued relevance in compiler construction and performance engineering.
	\end{NxIDBox}
\end{NxSSSSBox}

\begin{NxSSSSBox}[breakable][System Utilities and Performance Tools]
	\begin{NxIDBox}
		C is widely used for creating system utilities and performance-critical software due to its ability to execute efficiently with minimal overhead.
	\end{NxIDBox}
	\begin{NxIDBox}
		Command-line tools such as \textbf{grep, sed, awk, and ls} in Unix-based systems are implemented in C, ensuring optimal execution speed and system compatibility.
	\end{NxIDBox}
	\begin{NxIDBox}
		Performance monitoring tools like \textbf{htop, strace, and gprof} leverage C to provide real-time system diagnostics, ensuring efficient resource utilization and debugging capabilities.
	\end{NxIDBox}
	\begin{NxIDBox}
		C is also used in high-speed networking utilities, allowing developers to write software that directly interacts with system \nxGID{apis} and network protocols with minimal latency.
	\end{NxIDBox}
\end{NxSSSSBox}

