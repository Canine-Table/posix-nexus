\begin{NxSSSBox}[][Origins and Development]
	\begin{NxIDBox}
		The development of \nxGID{c_language} was driven by the need for a flexible, efficient, and portable programming language that could be used for system programming and application development.
	\end{NxIDBox}
	\begin{NxIDBoxL}
		\nxTopicD{sssec:Assembly to Structured Programming}{From Assembly to Structured Programming} Before C, programmers relied on assembly, which was efficient but lacked portability and structure.
		\nxTopicD{sssec:Dennis Ritchie’s Role}{Dennis Ritchie’s Role in C’s Birth} Dennis \nxGID{ritchie} developed C at \nxGID{bell_labs} to provide a balance between low-level control and structured programming.
	\end{NxIDBoxL}
\end{NxSSSBox}

\begin{NxSSSSBox}[][Assembly to Structured Programming]
	\begin{NxIDBox}[title={Limitations of Assembly}]
		Assembly language allowed direct hardware manipulation, but its complexity made programming tedious, with poor readability and lack of portability.
	\end{NxIDBox}
	\begin{NxIDBox}[title={Influence of ALGOL and BCPL}]
		Early high-level languages like \nxGID{algol} and \nxGID{bcpl} introduced structured programming, which influenced the development of C.
	\end{NxIDBox}
\end{NxSSSSBox}

\begin{NxSSSSBox}[][Limitations of Assembly]
	\begin{NxIDBox}[title={Challenges in Portability}]
		Code written in assembly was tightly linked to specific hardware architectures, making it difficult to adapt software across different systems.
	\end{NxIDBox}
	\begin{NxIDBox}[title={Complexity in Debugging}]
		Assembly programs required detailed memory management, making debugging significantly harder compared to structured programming languages.
	\end{NxIDBox}
\end{NxSSSSBox}


