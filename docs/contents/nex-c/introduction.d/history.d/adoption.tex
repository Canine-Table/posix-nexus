\begin{NxSSSBox}[breakable][Adoption and Standardization]
	\begin{NxIDBox}
		As C gained popularity, it became the dominant language for system programming, influencing operating systems, compilers, and embedded systems. Its adoption across universities and technology companies solidified its role as a foundational programming language.
	\end{NxIDBox}
	\begin{NxIDBoxL}
		\nxTopicD{University and Industry Adoption} How C became widespread in research, education, and corporate software development.
		\nxTopicD{K and R C} The publication of "The C Programming Language" and its role in defining early conventions.
		\nxTopicD{ANSI and ISO Standardization} The formalization of C standards from C89 to modern versions like C11 and C18.
	\end{NxIDBoxL}
\end{NxSSSBox}

\begin{NxSSSSBox}[breakable][University and Industry Adoption]
	\begin{NxIDBox}
		The widespread adoption of C in both academic and industrial settings was pivotal to its growth. Universities integrated C into their curricula, recognizing its importance in system programming and software development. At institutions such as \textbf{MIT, Berkeley, and Bell Labs}, C became a central part of computer science education, giving students the ability to understand both high-level abstraction and low-level programming concepts.
	\end{NxIDBox}
	\begin{NxIDBox}
		Beyond academia, major technology companies saw the value in C’s efficiency and portability. \textbf{AT\&T, IBM, and Microsoft} leveraged C for operating systems, networking tools, and hardware-level software. Its ability to manipulate memory directly while offering structured programming made it an ideal choice for developing robust and scalable applications.
	\end{NxIDBox}
	\begin{NxIDBox}
		By the early 1980s, C had transitioned from an experimental systems language into a global standard. Its widespread use in research and commercial projects laid the groundwork for its continued evolution. The prevalence of C-trained engineers in universities ensured that businesses had access to skilled developers, further reinforcing its status as a dominant language in professional computing.
	\end{NxIDBox}
\end{NxSSSSBox}

\begin{NxSSSSBox}[breakable][K and R C]
	\begin{NxIDBox}
		The release of \textbf{"The C Programming Language"} by \textbf{Brian Kernighan and Dennis Ritchie} in 1978 marked a significant milestone in C’s history. This book, commonly referred to as \nxGID{kr} C, became the definitive reference for learning and implementing the language. It introduced structured programming principles and best practices that shaped C’s usage for decades.
	\end{NxIDBox}
	\begin{NxIDBox}
		K\&R C standardized key elements of the language, including function prototypes, loops, pointers, and manual memory management. Despite lacking formal standardization, its influence was so profound that nearly all early C implementations followed its conventions.
	\end{NxIDBox}
	\begin{NxIDBox}
		However, with no strict governing body ensuring uniformity, minor inconsistencies arose between different compiler implementations. As C’s popularity grew, the need for a \textbf{formalized standard} became apparent, paving the way for efforts to unify C under a universally accepted specification.
	\end{NxIDBox}
	\begin{NxIDBox}
		Even today, K\&R C remains one of the most influential programming texts ever published, providing timeless insights that continue to guide developers in mastering C’s foundational concepts.
	\end{NxIDBox}
\end{NxSSSSBox}

\begin{NxSSSSBox}[breakable][ANSI and ISO Standardization]
	\begin{NxIDBox}
		To resolve inconsistencies and improve portability, the \nxGID{ansi} introduced the first official standard for C in 1989, known as \textbf{ANSI C (\nxGID{c89})}. This version enforced stricter type checking, improved function prototypes, and established a unified standard library.
	\end{NxIDBox}
	\begin{NxIDBox}
		ANSI C ensured that C programs would behave consistently across different compilers and platforms, making it easier for developers to write reliable, portable code. With its adoption, C became the de facto standard for system programming and software development worldwide.
	\end{NxIDBox}
	\begin{NxIDBox}
		Building on ANSI C, the \nxGID{iso} introduced \textbf{ISO C standards}, beginning with \nxGID{c99}. This version introduced inline functions, variable-length arrays, and better floating-point precision for numerical computations.
	\end{NxIDBox}
	\begin{NxIDBox}
	Further refinements in \nxGID{c11} and \nxGID{c18} continued to modernize the language, while maintaining backward compatibility to ensure legacy systems could still function without major rewrites.
	\end{NxIDBox}
	\begin{NxIDBox}
		Today, C remains one of the \textbf{most standardized and widely adopted programming languages}. Its structured evolution through ANSI and ISO ensures long-term stability, making it a fundamental choice for operating systems, embedded systems, and performance-critical applications.
	\end{NxIDBox}
\end{NxSSSSBox}

