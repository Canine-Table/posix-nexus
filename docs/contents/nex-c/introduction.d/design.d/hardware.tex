\begin{NxSSSBox}[breakable][Direct Hardware Access]
	\begin{NxIDBox}
		C was designed to provide direct control over system resources, allowing developers to interact closely with memory, registers, and peripheral devices.
	\end{NxIDBox}
	\begin{NxIDBoxL}
		\nxTopicD{Memory and Pointer Manipulation} How C enables developers to work directly with memory addresses.
		\nxTopicD{Bitwise Operations and Register Access} Why C supports fine-grained hardware control via bitwise manipulation.
		\nxTopicD{System-Level Integration} How C interacts with assembly language and hardware interrupts.
	\end{NxIDBoxL}
\end{NxSSSBox}

\begin{NxSSSSBox}[breakable][Memory and Pointer Manipulation]
	\begin{NxIDBox}
		C was designed to provide direct control over memory, allowing developers to manipulate data at the byte level. Unlike higher-level languages that abstract memory management, C gives programmers fine-grained access to memory locations.
	\end{NxIDBox}
	\begin{NxIDBox}
		\textbf{Pointers} are a fundamental feature that enables direct memory manipulation. By storing addresses rather than values, pointers allow developers to dynamically allocate memory, traverse data structures, and optimize performance-critical applications.
	\end{NxIDBox}
\end{NxSSSSBox}
\begin{NxCodeBox}{c}{dark, sdwA, frmE, title={The \textbf{malloc()}, \textbf{realloc()}, \textbf{calloc()} and \textbf{free()} functions provide explicit control over memory allocation and deallocation. This design choice ensures that programmers can efficiently manage memory usage while avoiding unnecessary overhead.}}
	#include <stdio.h>
	#include <stdlib.h>

	int nx_mem_chk(void*);

	int main()
	{
		int num = 5;
		/* Allocate memory using malloc */
		int *p_num = (int*)malloc(num * sizeof(int));
		if (nx_mem_chk(p_num) == -1) /* Handle memory allocation failure here */
			return(-1);

		int i = 0; /* Initialize values */

		int *c_num = (int*)calloc(num, sizeof(int)); // Allocates and initializes to 0
		if ((i = nx_mem_chk(c_num)) == -1) /* Handle memory allocation failure here */
			goto malloc_cleanup;

		for (i = 0; i < num; i++) {
			p_num[i] = i + 1;
			c_num[i] = i + 1;
		}

		num = num * 2;

		int *t_num = (int*)realloc(p_num, num * sizeof(int));
		if (nx_mem_chk(t_num) == -1) /* Handle memory allocation failure here */
			goto calloc_cleanup;
		p_num = t_num;

		t_num = (int*)realloc(c_num, num * sizeof(int));
		if ((i = nx_mem_chk(t_num)) == -1) /* Handle memory allocation failure here */
			goto calloc_cleanup;
		c_num = t_num;

		for (i = num / 2; i < num; i++) { /* Initialize new values */
			p_num[i] = (i + 1) * 2;
			c_num[i] = (i + 1) * 2;
		}
		for (i = 1; i < num; i++)
			printf("%d * %d = %d\n", p_num[i], c_num[i], p_num[i] * c_num[i]); /* Print values */

		calloc_cleanup:
			free(c_num); /* Free allocated memory */
		malloc_cleanup:
			free(p_num); /* Free allocated memory */
		return(i);
	}

	int nx_mem_chk(void *p)
	{
		if (p == NULL) {
			fprintf(stderr, "Memory allocation failed!\n");
			return -1;  /* Return an error indicator */
		}
		return 0;  /* Memory allocation was successful */
	}
\end{NxCodeBox}

\bigskip

\begin{NxSSSSBox}[breakable]
	\begin{NxIDBox}
		Manual memory management introduces risks such as \textbf{buffer overflows and segmentation faults}, requiring careful handling of pointers to prevent unintended behavior. C’s unrestricted memory access makes it powerful but demands disciplined programming practices.
	\end{NxIDBox}
\end{NxSSSSBox}

\begin{NxSSSSBox}[breakable][Bitwise Operations and Register Access]
	\begin{NxIDBox}
		C includes built-in support for bitwise operations, allowing direct manipulation of binary data. This capability is critical for working with hardware registers, optimizing storage, and implementing efficient algorithms.
	\end{NxIDBox}
\end{NxSSSSBox}

\begin{NxCodeBox}{c}{dark, sdwA, frmE, title={The \textbf{bitwise AND, OR, XOR, and shift operators} provide precise control over individual bits within a variable. These operations enable efficient flag manipulation, data compression, and low-level protocol implementations.}}
	#include <stdio.h>
	int main() {
		unsigned int x = 0b1100;  /* Binary: 1100 (Decimal: 12) */
		unsigned int y = 0b1010;  /* Binary: 1010 (Decimal: 10) */
		unsigned int and_result = x & y;  /* Bitwise AND */
		unsigned int or_result  = x | y;  /* Bitwise OR */
		unsigned int xor_result = x ^ y;  /* Bitwise XOR */
		unsigned int left_shift = x << 2; /* Left shift by 2 */
		unsigned int right_shift = y >> 1; /* Right shift by 1 */
		printf("AND: %u\n", and_result);  /* 1000 (Decimal: 8) */
		printf("OR: %u\n", or_result);	/* 1110 (Decimal: 14) */
		printf("XOR: %u\n", xor_result);  /* 0110 (Decimal: 6) */
		printf("Left Shift: %u\n", left_shift);  /* 110000 (Decimal: 48) */
		printf("Right Shift: %u\n", right_shift); /* 0101 (Decimal: 5) */
		return(0);
	}
\end{NxCodeBox}

\bigskip

\begin{NxSSSSBox}[breakable]
	\begin{NxIDBox}
		C's ability to interface with \textbf{hardware registers} makes it ideal for embedded development. By modifying register values directly, developers can configure peripheral devices, manage interrupts, and optimize system performance.
	\end{NxIDBox}
	\begin{NxIDBox}
		Bitwise operations also enhance data encryption, checksum calculations, and resource-efficient encoding schemes, reinforcing C’s role in security-critical applications.
	\end{NxIDBox}
\end{NxSSSSBox}

\begin{NxSSSSBox}[breakable][System-Level Integration]
	\begin{NxIDBox}
		C was designed to bridge the gap between software and hardware, enabling developers to integrate system-level components efficiently. It provides mechanisms for direct communication with system APIs and low-level routines.
	\end{NxIDBox}
\end{NxSSSSBox}

\begin{NxCodeBox}{c}{dark, frmE, sdwA, title={C supports \textbf{inline assembly}, allowing developers to mix assembly instructions with C code for performance optimization. This ensures minimal instruction overhead and precise hardware control.}}
	#include<stdio.h>
	int main()
	{
		const char msg[] = "Hello, World!\n";
		__asm__ (
			"mov $1, %%rax\n"  /* syscall: write */
			"mov $1, %%rdi\n"  /* file descriptor: stdout */
			"mov %0, %%rsi\n"  /* pointer to message */
			"mov $14, %%rdx\n" /* message length */
			"syscall\n"		/* invoke syscall */
			:
			: "r"(msg)		 /* input operand */
			: "rax", "rdi", "rsi", "rdx"
		);
		return(0);
	}
\end{NxCodeBox}

\bigskip

\begin{NxSSSSBox}[breakable]
	\begin{NxIDBox}
		The ability to interact with \textbf{operating system calls} makes C the foundation for kernel development. Functions like \textbf{syscall()} provide direct access to system resources such as file management, process control, and networking.
	\end{NxIDBox}
	\begin{NxIDBox}
		C's compatibility with low-level APIs enables efficient \textbf{device driver development}, where performance and hardware interaction are critical.
	\end{NxIDBox}
\end{NxSSSSBox}

