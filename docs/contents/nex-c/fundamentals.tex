\begin{comment}
2. Fundamentals of C Through the Years
2.1 Pre-C Origins and Structured Programming (Pre-1970s)
2.1.1 Assembly & Early Higher-Level Languages

2.1.1.1 Assembly as the First System Language

2.1.1.2 ALGOL’s Influence on C’s Syntax

2.1.1.3 FORTRAN and Early Programming Trends

2.1.2 Birth of C (1972 - 1978, Dennis Ritchie & UNIX)

2.1.2.1 Evolution from BCPL & B

2.1.2.2 Key Design Philosophy Behind C

2.1.2.3 First UNIX Implementations in C

2.1.2.4 Early Compilers and C’s Expanding Adoption

2.2 K&R C (1978 - 1989)
2.2.1 The First Edition of "The C Programming Language"

2.2.1.1 What K&R Defined as C’s Core Principles

2.2.1.2 Early Syntax and Code Structure

2.2.2 Standard Library Formation

2.2.2.1 The First Standardized Functions (printf, scanf, etc.)

2.2.2.2 Early Memory Management (malloc, free)

2.2.2.3 String Manipulation (strcpy, strlen, etc.)

2.3 ANSI C (1989 - 1999)
2.3.1 Introduction of C89/C90 Standard

2.3.1.1 Formal Standardization by ANSI

2.3.1.2 Key Additions: Function Prototypes, Type Safety

2.3.1.3 Portability and Compiler Compatibility

2.3.2 Standard Library Expansion

2.3.2.1 Buffered I/O (fopen, fclose, fprintf)

2.3.2.2 Math Functions (sqrt, pow, abs)

2.4 C99 – Modernization Begins (1999 - 2011)
2.4.1 New Language Features

2.4.1.1 Inline Functions (static inline)

2.4.1.2 Variable-Length Arrays (VLA)

2.4.1.3 restrict Keyword for Optimization

2.4.2 Library Updates

2.4.2.1 Complex Numbers (<complex.h>)

2.4.2.2 Boolean Type (<stdbool.h>)

2.4.2.3 Additional Floating-Point Handling (<fenv.h>)

2.5 C11 – Concurrency and Performance (2011 - 2018)
2.5.1 Threading and Atomics (<threads.h>)

2.5.1.1 Introduction of Portable Threading APIs

2.5.1.2 Atomic Operations for Data Synchronization

2.5.2 Library Evolution

2.5.2.1 Unicode and Multibyte Character Support

2.5.2.2 Safer Memory Functions (strncpy, memset_s)

2.6 C17 and Beyond – Refinements (2018 - Present)
2.6.1 Cleanup and Minor Adjustments

2.6.1.1 Deprecated Features Removal

2.6.1.2 Improved Compiler Support
\end{comment}

\begin{NxSBox}[][Fundamentals of C]
	\begin{NxIDBox}
		Understanding the fundamental aspects of C is crucial for mastering the language. This section covers essential concepts that define C, including its syntax, memory management, and data structures.
	\end{NxIDBox}
	\begin{NxIDBoxL}
		\nxTopicD{Core Syntax} The fundamental building blocks of C—variables, loops, conditionals, and operators.
		\nxTopicD{Core Syntax and Structure} Variables, loops, conditionals, functions, and fundamental programming constructs in C.
		\nxTopicD{Memory Management} How C handles memory using pointers, dynamic allocation, and manual memory control.
	\end{NxIDBoxL}
\end{NxSBox}

