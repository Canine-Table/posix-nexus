\begin{comment}
11. Real-World Projects and Case Studies: Evolution Through the Years
11.1 The Early Days of C in System Development (1970s - 1980s)
11.1.1 UNIX Operating System – C’s First Major Success

11.1.1.1 How C Became the Foundation of UNIX

11.1.1.2 Portability and Kernel-Level C Development

11.1.1.3 Early Performance Challenges in UNIX Programming

11.1.2 Embedded Systems and Microcontroller Development

11.1.2.1 How C Was Used in Early Microprocessors

11.1.2.2 Writing System Firmware in the 1980s

11.1.2.3 Case Study: Motorola 68000 Series and C Integration

11.2 C in Networking and Web Infrastructure (1980s - 1990s)
11.2.1 Development of Network Protocols in C

11.2.1.1 Writing TCP/IP Stack Implementations in C

11.2.1.2 The Role of C in Early Internet Servers

11.2.2 Web Server Evolution – Apache and NGINX

11.2.2.1 Early Web Infrastructure Built in C

11.2.2.2 Performance Optimization for High-Throughput Requests

11.3 C in Hardware and Compiler Development (1990s - 2000s)
11.3.1 Development of GCC – The Versatile Compiler

11.3.1.1 How GCC Was Designed for Portability

11.3.1.2 Compiler Optimization Case Study

11.3.2 OS Kernel Development – Linux and Windows

11.3.2.1 Linux Kernel's Extensive Use of C

11.3.2.2 Writing Device Drivers and System Calls

11.4 C in High-Performance Computing and AI (2000s - 2010s)
11.4.1 Supercomputing with C

11.4.1.1 Case Study: Writing High-Performance Matrix Multiplication

11.4.1.2 SIMD Optimization in Scientific Computing

11.4.2 Machine Learning Libraries in C

11.4.2.1 TensorFlow’s Low-Level C API

11.4.2.2 Efficient Neural Network Implementation in C

11.5 C Today – Critical Systems and Emerging Technologies (2010s - Present)
11.5.1 Security and Cryptography in C

11.5.1.1 Writing Cryptographic Algorithms in C

11.5.1.2 Case Study: OpenSSL Vulnerability Patching

11.5.2 Embedded Systems and IoT Development

11.5.2.1 Writing Firmware for Modern Processors

11.5.2.2 Optimizing Low-Power Embedded Code

11.6 The Future of C in Software and Hardware Innovations
11.6.1 Rust vs C – The Shift Toward Safer Systems Programming

11.6.1.1 Understanding C’s Future in System-Level Development

11.6.1.2 Interoperability Between C and Modern Languages

11.6.2 AI-Based Optimization for C Applications

11.6.2.1 How AI Assists in Compiler Optimizations

11.6.2.2 Future Trends in C-Based High-Performance Computing
\end{comment}

\begin{NxSBox}[][Project Development in C]
	\begin{NxIDBox}
		Writing large-scale C applications requires careful structuring, debugging techniques, and adherence to best practices.
	\end{NxIDBox}
	\begin{NxIDBoxL}
		\nxTopicD{Project Structure} Organizing source files, headers, and dependencies in multi-file projects.
		\nxTopicD{Debugging Techniques} Using tools like `gdb`, static analysis, and logging for efficient debugging.
		\nxTopicD{Version Control and Collaboration} Leveraging Git for managing code, contributions, and workflow efficiency.
	\end{NxIDBoxL}
\end{NxSBox}

