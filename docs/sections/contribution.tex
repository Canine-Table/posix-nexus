\section{Contribution Guidelines}
We welcome contributions from the community to help improve the project. Here are some guidelines to help you get started:

\subsection{Deployment Strategy}

Deployment will be executed incrementally, targeting supported systems and eventually expanding to all systems. This approach ensures thorough testing and stability, akin to Debian's release cycle methodology.

\subsection{Testing and Validation}

Testing will be conducted methodically, addressing each component individually to guarantee seamless integration and performance.

\subsection{Contribution Guidelines}

We welcome contributions from the community to help improve the project. Here are some guidelines to help you get started:

\subsection{Reporting Issues}
If you encounter any issues or have suggestions for improvements, please open an issue on our GitHub Issues page. Provide a clear and descriptive title, detailed description, steps to reproduce the issue, and any relevant logs or screenshots.

\subsection{Tool Restrictions}
To maintain compatibility and portability, it is prohibited to use any tools not explicitly required by POSIX. Ensure your scripts and contributions adhere strictly to POSIX standards.

\subsection{Testing}
Before submitting your pull request, please ensure your changes do not break existing functionality. Include tests for your changes if possible.

\subsection{Documentation}
If your contribution includes new features or changes to existing functionality, please update the documentation accordingly. Documentation files are located in the \texttt{docs} directory.

\subsection{License}
By contributing to POSIX Nexus, you agree that your contributions will be licensed under the GNU General Public License Version 3, 29 June 2007.

\subsection{Submitting Pull Requests}
We welcome pull requests for bug fixes, new features, and documentation improvements. Follow these steps to submit a pull request:
\begin{enumerate}
    \item \textbf{Fork the repository}: Click the "Fork" button at the top of the repository page to create a copy of the repository in your GitHub account.
    \item \textbf{Clone your fork}: Clone your forked repository to your local machine using the following command:
    \begin{lstlisting}
    git clone https://github.com/Canine-Table/posix-nexus.git
    \end{lstlisting}
\end{enumerate}
\newpage