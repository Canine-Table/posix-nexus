\section{POSIX Nexus Overview}

The \texttt{run.sh} script initiates the POSIX Nexus daemon, which encompasses the following functionalities:

\begin{itemize}
    \item Database Management
    \item SSH Authentication
    \item Command Distribution across specified Nexuses
    \item Webserver functionalities akin to Cockpit, applicable to any system, not limited to \texttt{systemd}
    \item Comprehensive security measures ensuring information sharing among Nexuses and connected machines via multicast groups, authenticated through SSH keys
    \item Interactive menu for managing IP addresses of child machines, enabling secure connections through public keys
    \item Dual authentication mechanism ensuring both Nexus and group member validation, with an automatic ejection of any machine posing a security risk detected by multiple systems
\end{itemize}

The POSIX Nexus aims to create a robust network that prioritizes security without compromising usability. Routine tasks are handled via AWK, while advanced features are developed using the C90 standard library.

\section{Deployment Strategy}

Deployment will be executed incrementally, targeting supported systems and eventually expanding to all systems. This approach ensures thorough testing and stability, akin to Debian's release cycle methodology.

\section{Testing and Validation}

Testing will be conducted methodically, addressing each component individually to guarantee seamless integration and performance.
