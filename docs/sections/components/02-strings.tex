\subsection{Text Processing}
\label{sec:component:sub:strinsg}

\phantomsection{}
\label{toupper}
\addcontentsline{toc}{subsubsection}{toupper}
\begin{baseBoxThree}{\sL{toupper}}{dark}
    \bigskip
    \sE{Purpose}: Converts a string to uppercase using AWK.
    \bigskip
    \begin{baseBoxThree}{\fO}{dark}
        \smallskip
        \begin{posnexItemize}
            \item[\sA] \sV{1}: The input string that will be converted to uppercase
        \end{posnexItemize}
        \smallskip
    \end{baseBoxThree}
    \smallskip
    \begin{baseBoxThree}{\fS}{success}
        \begin{posnex}
toupper "hello world"
# Output: HELLO WORLD
        \end{posnex}
    \end{baseBoxThree}
    \smallskip
\end{baseBoxThree}
\bigskip

\phantomsection{}
\label{tolower}
\addcontentsline{toc}{subsubsection}{tolower}
\begin{baseBoxThree}{\sL{tolower}}{dark}
    \bigskip
    \sE{Purpose}: Converts a string to lowercase using AWK.
    \bigskip
    \begin{baseBoxThree}{\fO}{dark}
        \smallskip
        \begin{posnexItemize}
            \item[\sA] \sV{1}: The input string that will be converted to lowercase
        \end{posnexItemize}
        \smallskip
    \end{baseBoxThree}
    \smallskip
    \begin{baseBoxThree}{\fS}{success}
        \begin{posnex}
tolower "HELLO WORLD"
# hello world
        \end{posnex}
    \end{baseBoxThree}
    \smallskip
\end{baseBoxThree}
\bigskip

\phantomsection{}
\label{ref}
\addcontentsline{toc}{subsubsection}{ref}
\begin{baseBoxThree}{\sL{ref}}{dark}
    \bigskip
    \sE{Purpose}: This function evaluates a variable name passed as an argument and prints its value.
    \bigskip
    \begin{baseBoxThree}{\fO}{dark}
        \smallskip
        \begin{posnexItemize}
            \item[\sA] \sV{1}: The name of the variable to be evaluated and printed
        \end{posnexItemize}
        \smallskip
    \end{baseBoxThree}
    \smallskip
    \begin{baseBoxThree}{\fS}{success}
        \begin{posnex}
# Example 1
# Define a variable 
my_variable="Hello, World!"
ref my_variable
# Hello, World!
        \end{posnex}
    \end{baseBoxThree}
    \smallskip
    \begin{baseBoxThree}{\fX}{error}
        \begin{posnex}
# Example 1:
# Not passing any argument ref
ref # (no Output)

# Example 2:
# Passing an undefined variable
ref undefined_variable # (no Output)

# Example 3:
# Passing a string instead of a variable name
ref "Hello, World!" # (no Output)
        \end{posnex}
    \end{baseBoxThree}
    \smallskip
\end{baseBoxThree}
\bigskip

\phantomsection{}
\label{format}
\addcontentsline{toc}{subsubsection}{format}
\begin{baseBoxThree}{\sL{format}}{dark}
    \bigskip
    \sE{Purpose}: Function to format a message by replacing placeholders with provided values.
    \bigskip
    \begin{baseBoxThree}{\fO}{dark}
        \smallskip
        \begin{posnexItemize}
            \item[\sA] \sV{1}: The string containing placeholders
            \item[\sA] \sV{2}: The string with the message to be formatted
        \end{posnexItemize}
        \smallskip
    \end{baseBoxThree}
    \smallskip
    \begin{baseBoxThree}{\fS}{success}
        \begin{posnex}
# Example 1:
format "30:Alice" "Hello, my name is <2> and I am <1> years old."
# Hello, my name is Alice and I am 30 years old.

# Example 2:
format "Japan:Kyoto" "Welcome to <>, enjoy your stay at <>."
# Welcome to Japan, enjoy your stay at Kyoto.

# Example 3:
format "Alice:30" "Hello, <1>. You are <2> years old. <1>, did you know you are <2> years old?"
# Hello, Alice. You are 30 years old. Alice, did you know you are 30 years old?
        \end{posnex}
    \end{baseBoxThree}
    \smallskip
    \begin{baseBoxThree}{\fX}{error}
        \begin{posnex}
# Example 1:
format "Hello:World" "<1> <2> <>"
# Hello World <>

# Example 2:
format "Hello:World" "<> <> <2>"
# Hello <> World
    \end{posnex}
    \end{baseBoxThree}
    \smallskip
\end{baseBoxThree}
\bigskip

\phantomsection{}
\label{kwargs}
\addcontentsline{toc}{subsubsection}{kwargs}
\begin{baseBoxThree}{\sL{kwargs}}{dark}
    \bigskip
    \sE{Purpose}: This function parses and processes key-value arguments from a string, separating each pair by commas and equal signs, and returning the processed values or "NULL" if a value is empty.
    \bigskip
    \begin{baseBoxThree}{\fO}{dark}
        \smallskip
        \begin{posnexItemize}
            \item[\sA] \sV{*}: Takes one or more key-value pairs as arguments, separated by commas and equal signs.
        \end{posnexItemize}
        \smallskip
    \end{baseBoxThree}
    \smallskip
    \begin{baseBoxThree}{\fS}{success}
        \begin{posnex}
# Example 1: 
# Handling empty values
for I in $(kwargs "key1=,key2=value2"); do echo -e $I; done
# key1
# NULL
# key2
# value2

# Example 2:
# Handling spaces in values (\x20) hexadecimal for space
for I in $(kwargs 'key 1  =    , key 2 =  value 2'); do
    echo ${I}
done
# key\x201
# NULL
# key\x202
# value\x202

# Example 3:
# Parsing a single key-value pair
# use -e to expand the spaces
for I in $(kwargs "key 1 = value 1"); do
    echo -e ${I};
done
# key 1
# value 1
        \end{posnex}
    \end{baseBoxThree}
    \smallskip
\end{baseBoxThree}
\begin{baseBoxThree}{\fU{kwargs}}{dark}
    \begin{baseBoxThree}{\fS}{success}
\begin{posnex}
# Example 4: 
# spacing makes no difference
for I in $(kwargs "
key 1 =       value 1         ,
key 2            
    =             
        value 2
"); do
    echo -e ${I}; 
done
# key 1
# value 1
# key 2
# value 2
        \end{posnex}
    \end{baseBoxThree}
    \smallskip
    \begin{baseBoxThree}{\fX}{error}
        \begin{posnex}
# Example 1:
# Missing equal sign between key and value
for I in $(kwargs "key1value1, key2=value2"); do
    echo -e $I;
done
# key1value1, key2
# value2

# Example 2:
# Misformatted string with multiple equal signs in key-value pairs
for I in $(kwargs "key1=value1=extra, key2=value2"); do
    echo -e ${I}
done
# key1
# value1=extra
# key2
# value2

# Example 3:
# Missing comma between key-value pairs
for I in $(kwargs "key1=value1 key2=value2"); do 
    echo -e ${I}
done
# key1
# value1 key2=value2
        \end{posnex}
    \end{baseBoxThree}
    \smallskip
\end{baseBoxThree}
