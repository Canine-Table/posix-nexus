\section{Sandbox}

\begin{tikzpicture}[overlay, remember picture]
    % Define the number of columns and rows
    \def\cols{6}
    \def\rows{9}
    % Define the width and height of each node
    \pgfmathsetlengthmacro\nodewidth{\paperwidth/\cols}
    \pgfmathsetlengthmacro\nodeheight{\paperheight/\rows}
    \foreach \col in {1,...,14} {
        \foreach \row in {1,...,\rows} {
            \node[
                transform shape,
                xslant=0.7,
                rotate=-10, 
                anchor=north west,
                inner sep=0pt,
                outer sep=0pt,
                minimum width=\nodewidth, minimum height=\nodeheight, xshift=(\col-1)*\nodewidth, yshift=-(\row-1)*\nodeheight
                ] at (0,10) {
                \includegraphics[width=\nodewidth,height=\nodeheight]{setup/images/posix-nexus-image-\posnexModulus{\col+\row}{10}.jpeg}
            };
        }
    }
\end{tikzpicture}
    \begin{baseBoxOne}{}{dark}

    \begin{abstract}
        This manual provides comprehensive guidance on using POSIX-Nexus, a cutting-edge daemon that leverages the speed of awk and the parsing power of LaTeX's expl3 syntax to deliver exceptional performance.
        The manual outlines the goals and objectives of POSIX-Nexus, highlights its key features, and offers detailed instructions to help you harness its capabilities to achieve unparalleled speed and efficiency in your scripts and applications.
        \bigskip
    \end{abstract}
\end{baseBoxOne}