% \section{Sandbox}

% \begin{tikzpicture}[overlay, remember picture]
%     % Define the number of columns and rows
%     \def\cols{6}
%     \def\rows{9}
%     % Define the width and height of each node
%     \pgfmathsetlengthmacro\nodewidth{\paperwidth/\cols}
%     \pgfmathsetlengthmacro\nodeheight{\paperheight/\rows}
%     \foreach \col in {1,...,15} {
%         \foreach \row in {1,...,10} {
%             \node[
%                 transform shape,
%                 xslant=0.7,
%                 rotate=-10, 
%                 anchor=north west,
%                 inner sep=0pt,
%                 outer sep=0pt,
%                 minimum width=\nodewidth, minimum height=\nodeheight, xshift=(\col-1)*\nodewidth, yshift=-(\row-1)*\nodeheight
%                 ] at (-1,10) {
%                 \includegraphics[width=\nodewidth,height=\nodeheight]{setup/images/posix-nexus-image-\posnexModulus{\col+\row}{10}.jpeg}
%             };
%         }
%     }
% \end{tikzpicture}
    
% \begin{baseBoxOne}{}{dark}
%     \title{
%         \includegraphics[width=0.5\textwidth]{setup/images/posix-nexus-icon.png}\\[1cm]
%         \textbf{POSIX-Nexus}: First Edition
%     }

%     \author{Canine-Table}
%     \date{\today}
%     \maketitle

%     \begin{abstract}
%         This manual provides comprehensive guidance on using POSIX-Nexus, a cutting-edge daemon that leverages the speed of awk and the parsing power of LaTeX's expl3 syntax to deliver exceptional performance.
%         The manual outlines the goals and objectives of POSIX-Nexus, highlights its key features, and offers detailed instructions to help you harness its capabilities to achieve unparalleled speed and efficiency in your scripts and applications.
%         \bigskip
%     \end{abstract}
% \end{baseBoxOne}

% \newpage
% % \begin{tikzpicture}[decoration=Koch curve type 1]
% %         \draw decorate{ decorate{ decorate{ (0,-3) -- (3,-3) }}};
% % \end{tikzpicture}
% % \begin{tikzpicture}[decoration=Koch curve type 2]
% %     \draw decorate{ decorate{ decorate{ (0,-3) -- (3,-3) }}};
% % \end{tikzpicture}
% % \begin{tikzpicture}[decoration=Koch snowflake]
% %     \draw decorate{ decorate{ decorate{ decorate{ (0,-3) -- (3,-3) }}}};
% % \end{tikzpicture}

% % \begin{tikzpicture}[decoration=Koch snowflake,draw=blue,fill=blue!20,thick]
% %     \filldraw decorate{ decorate{ (0,-2.5) -- ++(60:1) -- ++(-60:1) -- cycle }};
% % \end{tikzpicture}

% % \begin{tikzpicture}[decoration=]
% %     \draw[decorate, decoration={zigzag, amplitude=5pt, segment length=10pt}] (0,0) -- (10,0); 
% % \end{tikzpicture}

% Some text.
% \begin{posnexItemize}
% \item Alpha
% \item Beta
% \item Gamma
% \end{posnexItemize}
% More text.
% \begin{tikzpicture}[overlay, remember picture]
%     % Define the number of columns and rows
%     \def\cols{6}
%     \def\rows{9}
%     % Define the width and height of each node
%     \pgfmathsetlengthmacro\nodewidth{\paperwidth/\cols}
%     \pgfmathsetlengthmacro\nodeheight{\paperheight/\rows}
%     \foreach \col in {1,...,15} {
%         \foreach \row in {1,...,10} {
%             \node[
%                 transform shape,
%                 xslant=0.7,
%                 rotate=-10, 
%                 anchor=north west,
%                 inner sep=0pt,
%                 outer sep=0pt,
%                 minimum width=\nodewidth, minimum height=\nodeheight, xshift=(\col-1)*\nodewidth, yshift=-(\row-1)*\nodeheight
%                 ] at (-1,10) {
%                 \includegraphics[width=\nodewidth,height=\nodeheight]{setup/images/posix-nexus-image-\posnexModulus{\col+\row}{10}.jpeg}
%             };
%         }
%     }
% \end{tikzpicture}
