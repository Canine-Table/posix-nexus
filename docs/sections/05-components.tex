\section{Components}
\label{sec:component}
The \sTE{run.sh} script initiates the POSIX Nexus daemon, which encompasses the following functionalities:
\bigskip
\begin{baseBoxOne}{}{dark}
    \begin{posnexItemize}
        \item[\sA] \sE{Database Management}: Efficiently handles data storage and retrieval to support various applications.
        \item[\sA] \sE{SSH Authentication}: Secures connections using SSH keys to ensure only authorized access.
        \item[\sA] \sE{Command Distribution}: Distributes commands across specified Nexuses, enabling coordinated execution.
        \item[\sA] \sE{Web Server Functionalities}: Provides web server capabilities akin to Cockpit, applicable to any system, not limited to systemd.
        \item[\sA] \sE{Comprehensive Security Measures}: Ensures secure information sharing among Nexuses and connected machines via multicast groups, authenticated through SSH keys.
        \item[\sA] \sE{Interactive Menu}: Manages IP addresses of child machines, enabling secure connections through public keys.
        \item[\sA] \sE{Dual Authentication Mechanism}: Ensures both Nexus and group member validation, with automatic ejection of any machine posing a security risk detected by multiple systems.
    \end{posnexItemize}
\end{baseBoxOne}
\bigskip
The POSIX Nexus aims to create a robust network that prioritizes security without compromising usability.
Routine tasks are handled via \sTE{awk}, while advanced features are developed using the C90 standard library.
This approach ensures efficient script execution and consistent performance across various systems.
\newpage
\subsection{Directory Structure}

The posix-nexus.pid stores the process ID of the running instance of posix-nexus daemon.
The posix-nexus-directory is a symbolic link to a directory that gets recreated each time the application starts.
The previous directories with old process IDs are deleted, ensuring only the current process directory exists in /var/tmp/posix-nexus/.
The run.sh executable serves as the primary script to start the posix-nexus daemon.

\begin{tikzpicture}[overlay, remember picture]
    \tikzset{
        directory/.style={
            text=white, % Text color
            fill=blue, % Fill color
            rounded corners, % Rounded edges
            text centered,
            font=\sffamily\bfseries
        },
        file/.style={
            text=white, % Text color
            fill=brown!60!black, % Fill color
            rounded corners, % Rounded edges
            text centered,
            font=\sffamily\bfseries
        },
        symlink/.style={
            text=white, % Text color
            fill=teal, % Fill color
            rounded corners, % Rounded edges
            text centered,
            font=\sffamily\bfseries
        }
    }

    \node at (current page.center) {
        \begin{tikzpicture}[
            sibling distance=2cm,
            level distance=1cm,
            anchor=west,
            grow=east]
            \node[directory] {/} child {[palette.dark.frame!70!white]
                child {node[directory]{var}
                    child {
                        child {node[directory]{log}
                            child {
                                child {node[directory]{posix-nexus}}
                            }
                        }
                        child {node[directory]{run}
                            child {
                                child {node[directory]{posix-nexus}
                                    child {
                                        child {node[file]{posix-nexus.pid}}
                                    }
                                    child[draw,dashed,thick] {
                                        child {node[symlink](b) at (0,0){posix-nexus-directory}}
                                    }
                                }
                            }
                        }
                        child {node[directory] at (-10mm,10mm){tmp}
                            child {
                                child {node[directory]{posix-nexus}
                                    child {
                                        child {node[directory,fill=palette.success.back!70!black](a) at (0,0){\$!}}
                                    }
                                }
                            }
                        }
                    }
                }
                child {node[directory] at (0mm,25mm){usr}
                    child {
                        child {node[directory]{local}
                            child {
                                child {node[directory]{posix-nexus}
                                    child {
                                        child {node[directory] at (10mm,10mm){main}}
                                        child {node[directory] at (10mm,1mm){docs}}
                                        child {node[file,fill=palette.success.frame!50!black] at (10mm,-8mm){run.sh*}}
                                    }
                                }
                            }
                        }
                    }
                }
            };
            \draw[palette.dark.frame!70!white,->] (b) to node[midway, centered, white] {\textbf{symlink}} (a);
        \end{tikzpicture}
    };
\end{tikzpicture}
\newpage

\subsection{Exceptions}
\label{sec:component:sub:exception}
The try function is a core part of POSIX-Nexus script's error handling and validation mechanism.
It ensures that commands execute successfully and handle failures gracefully.
By structuring checks and validations using try, we maintain robust, reliable scripts that can preemptively handle potential errors without crashing.

\subsubsection{R Flag (Regular Expressions)}
\label{sec:component:sub:exception:sub:R}
\sE{Purpose}: Handle regular expression-related exceptions.
