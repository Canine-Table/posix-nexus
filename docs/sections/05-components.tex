\section{Components}
\label{sec:component}
The \sTE{run.sh} script initiates the POSIX Nexus daemon, which encompasses the following functionalities:
\bigskip
\begin{baseBoxOne}{}{dark}
    \begin{posnexItemize}
        \item[\sA] \sE{Database Management}: Efficiently handles data storage and retrieval to support various applications.
        \item[\sA] \sE{SSH Authentication}: Secures connections using SSH keys to ensure only authorized access.
        \item[\sA] \sE{Command Distribution}: Distributes commands across specified Nexuses, enabling coordinated execution.
        \item[\sA] \sE{Web Server Functionalities}: Provides web server capabilities akin to Cockpit, applicable to any system, not limited to \sTE{systemd}.
        \item[\sA] \sE{Comprehensive Security Measures}: Ensures secure information sharing among Nexuses and connected machines via multicast groups, authenticated through SSH keys.
        \item[\sA] \sE{Interactive Menu}: Manages IP addresses of child machines, enabling secure connections through public keys.
        \item[\sA] \sE{Dual Authentication Mechanism}: Ensures both Nexus and group member validation, with automatic ejection of any machine posing a security risk detected by multiple systems.
    \end{posnexItemize}
\end{baseBoxOne}
\bigskip
The POSIX Nexus aims to create a robust network that prioritizes security without compromising usability.
Routine tasks are handled via \sTE{awk}, while advanced features are developed using the \sTE{C90} standard library.
This approach ensures efficient script execution and consistent performance across various systems.

\subsection{Directory Structure}

The posix-nexus.pid stores the process ID of the running instance of posix-nexus daemon.
The posix-nexus-directory is a symbolic link to a directory that gets recreated each time the application starts.
The previous directories with old process IDs are deleted, ensuring only the current process directory exists in /var/tmp/posix-nexus/.
The run.sh executable serves as the primary script to start the posix-nexus daemon.

\begin{tikzpicture}[overlay, remember picture]
    \tikzset{
        directory/.style={
            text=white, % Text color
            fill=blue, % Fill color
            rounded corners, % Rounded edges
            text centered,
            font=\sffamily\bfseries
        },
        file/.style={
            text=white, % Text color
            fill=brown!60!black, % Fill color
            rounded corners, % Rounded edges
            text centered,
            font=\sffamily\bfseries
        },
        symlink/.style={
            text=white, % Text color
            fill=teal, % Fill color
            rounded corners, % Rounded edges
            text centered,
            font=\sffamily\bfseries
        }
    }

    \node at (current page.center) {
        \begin{tikzpicture}[
            sibling distance=2cm,
            level distance=1cm,
            anchor=west,
            grow=east]
            \node[directory] {/} child {[palette.dark.frame!70!white]
                child {node[directory]{var}
                    child {
                        child {node[directory]{log}
                            child {
                                child {node[directory]{posix-nexus}}
                            }
                        }
                        child {node[directory]{run}
                            child {
                                child {node[directory]{posix-nexus}
                                    child {
                                        child {node[file]{posix-nexus.pid}}
                                    }
                                    child[draw,dashed,thick] {
                                        child {node[symlink](b) at (0,0){posix-nexus-directory}}
                                    }
                                }
                            }
                        }
                        child {node[directory] at (-10mm,10mm){tmp}
                            child {
                                child {node[directory]{posix-nexus}
                                    child {
                                        child {node[directory,fill=palette.success.back!70!black](a) at (0,0){\$!}}
                                    }
                                }
                            }
                        }
                    }
                }
                child {node[directory] at (0mm,25mm){usr}
                    child {
                        child {node[directory]{local}
                            child {
                                child {node[directory]{posix-nexus}
                                    child {
                                        child {node[directory] at (10mm,10mm){main}}
                                        child {node[directory] at (10mm,1mm){docs}}
                                        child {node[file,fill=palette.success.frame!50!black] at (10mm,-8mm){run.sh*}}
                                    }
                                }
                            }
                        }
                    }
                }
            };
            \draw[palette.dark.frame!70!white,->] (b) to node[midway, centered, white] {\textbf{symlink}} (a);
        \end{tikzpicture}
    };
\end{tikzpicture}
\newpage


\subsection{Exceptions}
\label{sec:component:sub:exception}
The try function is a core part of POSIX-Nexus script's error handling and validation mechanism.
It ensures that commands execute successfully and handle failures gracefully.
By structuring checks and validations using try, we maintain robust, reliable scripts that can preemptively handle potential errors without crashing.

\subsubsection{R Flag (Regular Expressions)}
\label{sec:component:sub:exception:sub:R}
\sE{Purpose}: Handle regular expression-related exceptions.


\subsection{Text Processing}
\label{sec:component:sub:strinsg}

\begin{baseBoxThree}{\sL{toupper}}{dark}
    \bigskip
    \sE{Purpose}: Converts a string to uppercase using AWK.
    \\

    \begin{baseBoxThree}{Correct}{success}
        \begin{posnex}
    toupper "hello world"
    # Output: HELLO WORLD
        \end{posnex}
    \end{baseBoxThree}
\end{baseBoxThree}
\bigskip

\begin{baseBoxThree}{\sL{tolower}}{dark}
    \bigskip
    \sE{Purpose}: Converts a string to lowercase using AWK.
    \\

    \begin{baseBoxThree}{Correct}{success}
        \begin{posnex}
    tolower "HELLO WORLD"
    # Output: hello world
        \end{posnex}
    \end{baseBoxThree}
\end{baseBoxThree}
\bigskip
\begin{baseBoxThree}{\sL{format}}{dark}
    \bigskip
    \sE{Purpose}:
    \\
    % \faIcon{file}
	\begin{baseBoxThree}{Parameters}{dark}
        \smallskip
		\sTE{Arguments}:

        \sA~\sV{\textdollar}: The string containing placeholders

        \sA~\sV{\textdollar}: The string with the message to be formatted\\

        \sTE{Output}:\\
        \sA~The formatted message with placeholders replaced by the provided values
	\end{baseBoxThree}

    \begin{baseBoxThree}{Correct}{success}
        \begin{posnex}
        \end{posnex}
    \end{baseBoxThree}

    \begin{baseBoxThree}{Incorrect}{error}
        \begin{posnex}
        \end{posnex}
    \end{baseBoxThree}
\end{baseBoxThree}
\bigskip


\subsection{Miscelanious Functions}
\label{sec:component:sub:misc}
\begin{baseBoxThree}{\sL{cmd}}{dark}
    \bigskip
    \sE{Purpose}: This function checks if any given commands exist in the system's \sTE{PATH}.
    It iterates through a list of commands and returns the path of the first command it finds.
    If no command is found, it returns (1) failure.
    \\

    \begin{baseBoxThree}{Correct}{success}
        \begin{posnex}
    # Example 1: Check for the presence of an awk variant
    AWK=$(cmd mawk nawk awk gawk) && echo "${AWK} is available" || echo "awk is not available";

    # Example 2: Locates the first command in the PATH and executes it with the provided parameters
    $(cmd curl wget) 'https://example.com'
    # Output: Executes the first available command (curl or wget) to retrieve 'https://example.com'

    # Example 3: Check for the presence of `sh`, `dash`, `ash`, or `bash` and execute a command
    cmd sh dash ash bash -c "echo ${0}" 
    # Output: The path to the first shell found or false
    \end{posnex}
    \end{baseBoxThree}

    \begin{baseBoxThree}{Incorrect}{error}
        \begin{posnex}
    # Incorrect: Not passing any command to check
    cmd && echo "One of the commands is available" || echo "None of the commands are available"

    # Incorrect: Using commands that are not commonly found in PATH
    cmd someRandomCommand anotherRandomCommand && echo "One of the commands is available" || echo "None of the commands are available"
        \end{posnex}
    \end{baseBoxThree}
\end{baseBoxThree}
\bigskip


\subsection{File I/O}
\label{sec:component:sub:io}


\subsection{Network Operations}
\label{sec:component:sub:network}


\subsection{Data Structires}
\label{sec:component:sub:structire}


\subsection{Security Implementations}
\label{sec:component:sub:security}



\subsection{System Operations}
\label{sec:component:sub:system}

