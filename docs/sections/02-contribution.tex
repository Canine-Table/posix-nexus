\subsection{Contribution Guidelines}
\label{sec:introduction:sub:guidelines}
We welcome contributions from the community to help improve the project. Here are some guidelines to help you get started:

\subsection{Deployment Strategy}
\label{sec:introduction:sub:deployment}
Deployment will be executed incrementally, targeting supported systems and eventually expanding to all systems. This approach ensures thorough testing and stability, akin to Debian's release cycle methodology.

\subsection{Testing and Validation}
\label{sec:introduction:sub:valid}
Testing will be conducted methodically, addressing each component individually to guarantee seamless integration and performance.

\subsection{Reporting Issues}
\label{sec:introduction:sub:report}
If you encounter any issues or have suggestions for improvements, please open an issue on our GitHub Issues page.
Provide a clear and descriptive title, detailed description, steps to reproduce the issue, and any relevant logs or screenshots.

\subsection{Tool Restrictions}
\label{sec:introduction:sub:restrict}
To maintain compatibility and portability, it is prohibited to use any tools not explicitly required by POSIX.
Ensure your scripts and contributions adhere strictly to POSIX standards.

\subsection{Testing}
\label{sec:introduction:sub:test}
Before submitting your pull request, please ensure your changes do not break existing functionality.
Include tests for your changes if possible.

\subsection{Documentation}
\label{sec:introduction:sub:document}
If your contribution includes new features or changes to existing functionality, please update the documentation accordingly.
Documentation files are located in the \sTE{docs} directory.

\subsection{Submitting Pull Requests}
\label{sec:introduction:sub:pull}
We welcome pull requests for bug fixes, new features, and documentation improvements.
Follow these steps to submit a pull request:
\bigskip
\begin{baseBoxOne}{}{dark}
    \begin{posnexItemize}
        \item[\sA] \sE{Fork the repository}: Click the "Fork" button at the top of the repository page to create a copy of the repository in your GitHub account.
        \item[\sA] \sE{Clone your fork}: Clone your forked repository to your local machine using the following command:
        \begin{posnex}
    git clone "https://github.com/Canine-Table/posix-nexus.git";
        \end{posnex}
    \end{posnexItemize}
\end{baseBoxOne}
\subsection{Contact Us}
\label{sec:introduction:sub:contact}
We are here to help and support the community. If you have any questions, concerns, or suggestions, please don't hesitate to reach out to us. Your feedback is invaluable in helping us improve and grow.
\bigskip
\begin{baseBoxOne}{}{dark}
    \begin{posnexItemize}
        \item[\sA] \sE{Discord}: Join our  community on Discord for real-time discussions, support, and collaboration. [\href{https://discord.gg/GB9twwpCNM}{Discord Invite Link}]
        \item[\sA] \sE{GitHub Issues}: Report issues, request features, and track progress on our GitHub Issues page. [\href{https://github.com/Canine-Table/posix-nexus/issues}{GitHub Issues Link}]
        \item[\sA] \sE{Email}: For more detailed inquiries, feel free to email us at posix-nexus@duck.com.
    \end{posnexItemize}
\end{baseBoxOne}
