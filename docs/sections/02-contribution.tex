\section{Getting Started}
\label{sec:start}
To get started with POSIX-Nexus, follow these steps:
\bigskip
\begin{baseBoxOne}{}{dark}
    \begin{posnexItemize}
        \item[\sA] \sE{Clone the Repository}: To obtain the latest version of POSIX-Nexus, clone the repository from GitHub:\smallskip
        \begin{baseBoxThree}{}{dark}
            \posnexInline{git clone "https://github.com/Canine-Table/posix-nexus.git"}
        \end{baseBoxThree}\smallskip
        \item[\sA] \sE{Navigate to the Project Directory}: Change into the newly created directory:\smallskip
        \begin{baseBoxThree}{}{dark}\smallskip
            \posnexInline{cd "posix-nexus"}
        \end{baseBoxThree}\smallskip
        \item[\sA] \sE{Start the Daemon}: Use the provided \sTE{run.sh} script to start the POSIX-Nexus daemon. The daemon runs under the shell name specified in your configuration:\smallskip
        \begin{baseBoxThree}{}{dark}
            \posnexInline{./run.sh}
        \end{baseBoxThree}\smallskip
        \item[\sA] \sE{Verify the Daemon}: The script will automatically verify if the daemon has started successfully. If there are any issues, it will provide detailed information about what went wrong and how to fix it.
        \item[\sA] \sE{Manage the Daemon}: Use the provided interactive menu to manage the IP addresses of child machines and ensure secure connections through public keys.
    \end{posnexItemize}
\end{baseBoxOne}

\subsection{Documentation}
\label{sec:start:sub:document}
If your contribution includes new features or changes to existing functionality, please update the documentation accordingly.
Documentation files are located in the \sTE{docs} directory.
\bigskip
\begin{baseBoxThree}{Note}{info}
    \sE{For Contributors Not Familiar with \LaTeX}: If you are not comfortable with \LaTeX~, you can still contribute by providing clear and detailed descriptions of your changes or new features in plain text or a Markdown file. This information can be added to a temporary file in the docs directory or included as part of your pull request. Our team will take care of integrating your documentation into the \LaTeX~ files.\smallskip
\end{baseBoxThree}

\subsection{Contribution Guidelines}
\label{sec:start:sub:guidelines}
We welcome contributions from the community to help improve the project. Here are some guidelines to help you get started:
\bigskip

\begin{baseBoxOne}{}{dark}
    \begin{posnexItemize}
        \item[\sA] \sE{Fork and Clone}: Start by forking the repository and cloning it to your local machine.
        \item[\sA] \sE{Create a Branch}: Create a new branch for your feature or bug fix.
        \item[\sA] \sE{Make Changes}: Make your changes in the new branch. Ensure that your code adheres to the project's coding standards.
        \item[\sA] \sE{Document Changes}: Provide clear and detailed descriptions of any new features or changes in plain text or Markdown if you are not familiar with \\LaTeX~.
        \item[\sA] \sE{Commit and Push}: Commit your changes with a descriptive commit message and push the branch to your fork.
        \item[\sA] \sE{Create a Pull Request}: Open a pull request to the main repository. Provide a detailed description of your changes and the motivation behind them.
        \item[\sA] \sE{Code Review}: Be prepared to engage in a code review process. Address any feedback or requested changes promptly.
        \item[\sA] \sE{Merge}: Once approved, your pull request will be merged into the main branch.
    \end{posnexItemize}
\end{baseBoxOne}

\subsection{Deployment Strategy}
\label{sec:start:sub:deployment}
Deployment will be executed incrementally, targeting supported systems and eventually expanding to all systems.
This approach ensures thorough testing and stability, akin to Debian's release cycle methodology.
\bigskip
\begin{baseBoxOne}{}{dark}
    \begin{posnexItemize}
        \item[\sA] \sE{Initial Deployment}: Begin with a subset of supported systems, such as popular UNIX-like environments like Alpine Linux, Cygwin, and iSH.
        \item[\sA] \sE{Containerization}: Deploy on container platforms like Docker to ensure consistency across different environments.
        \item[\sA] \sE{Incremental Rollout}: Gradually expand deployment to additional systems based on feedback and testing results.
        \item[\sA] \sE{Continuous Monitoring}: Monitor deployed instances for performance and stability issues. Address any issues promptly to ensure a smooth user experience.
        \item[\sA] \sE{Feedback Loop}: Encourage users to provide feedback on the deployment process and address any reported issues.
    \end{posnexItemize}
\end{baseBoxOne}

\subsection{Testing and Validation}
\label{sec:start:sub:valid}
Testing will be conducted methodically, addressing each component individually to guarantee seamless integration and performance.
\bigskip
\begin{baseBoxOne}{}{dark}
    \begin{posnexItemize}
        \item[\sA] \sE{Unit Testing}: Write and execute unit tests for individual components to ensure they function as expected.
        \item[\sA] \sE{Integration Testing}: Conduct integration tests to verify that different components work together seamlessly.
        \item[\sA] \sE{System Testing}: Perform system tests to validate the overall functionality and performance of the entire system.
        \item[\sA] \sE{Performance Testing}: Test the system under various load conditions to ensure it meets performance requirements.
        \item[\sA] \sE{Regression Testing}: Run regression tests to ensure that new changes do not introduce any new issues.
    \end{posnexItemize}
\end{baseBoxOne}

\subsection{Reporting Issues}
\label{sec:start:sub:report}
If you encounter any issues or have suggestions for improvements, please open an issue on our GitHub Issues page.
Provide a clear and descriptive title, detailed description, steps to reproduce the issue, and any relevant logs or screenshots.
\bigskip
\begin{baseBoxOne}{}{dark}
    \begin{posnexItemize}
        \item[\sA] \sE{Title}: Use a concise and descriptive title for the issue.
        \item[\sA] \sE{Description}: Provide a detailed description of the issue, including the expected and actual behavior.
        \item[\sA] \sE{Steps to Reproduce}: List the steps to reproduce the issue, including any relevant configuration or input data.
        \item[\sA] \sE{Logs and Screenshots}: Attach any relevant logs or screenshots that can help diagnose the issue.
        \item[\sA] \sE{Environment Details}: Include details about your environment, such as the operating system, software versions, and any specific configurations.
    \end{posnexItemize}
\end{baseBoxOne}
\bigskip
Your contributions and feedback are invaluable as we continue to refine and expand the capabilities of POSIX-Nexus.
Together, we can overcome the challenges of portability and build a tool that stands the test of time across any platform

\subsection{Tool Restrictions}
\label{sec:start:sub:restrict}
To maintain compatibility and portability, it is prohibited to use any tools not explicitly required by POSIX.
Ensure your scripts and contributions adhere strictly to POSIX standards.

\subsection{Contact Us}
\label{sec:start:sub:contact}
We are here to help and support the community.
If you have any questions, concerns, or suggestions, please don't hesitate to reach out to us.
Your feedback is invaluable in helping us improve and grow.
\bigskip
\begin{baseBoxOne}{}{dark}
    \begin{posnexItemize}
        \item[\sA] \sE{Discord}: Join our  community on Discord for real-time discussions, support, and collaboration. [\href{https://discord.gg/GB9twwpCNM}{Discord Invite Link}]
        \item[\sA] \sE{GitHub Issues}: Report issues, request features, and track progress on our GitHub Issues page. [\href{https://github.com/Canine-Table/posix-nexus/issues}{GitHub Issues Link}]
        \item[\sA] \sE{Email}: For more detailed inquiries, feel free to email us at posix-nexus@duck.com.
    \end{posnexItemize}
\end{baseBoxOne}
