\nxSections{Algebraic Identities}{2}

\begin{NxLightListBox}[title={Difference of Squares}]
	\nxEachLabel{ArrowDark}{Secondary}{{4}{5}}{%
		{Definition}/{Product of sum and difference equals difference of squares},%
		{General Formula}/{\(a^2 - b^2 = (a-b)(a+b)\)},%
		{Pattern}/{Always collapses into two linear factors},%
		{Example}/{\(x^2 - 9 = (x-3)(x+3)\)},%
		{Extension}/{Used in rationalizing denominators and factoring quadratics}%
	}
\end{NxLightListBox}

\begin{NxLightListBox}[title={Sum of Cubes}]
	\nxEachLabel{ArrowDark}{Secondary}{{4}{5}}{%
		{Definition}/{Sum of cubes factors into binomial and trinomial},%
		{General Formula}/{\(a^3 + b^3 = (a+b)(a^2 - ab + b^2)\)},%
		{Pattern}/{Binomial carries the sum, trinomial balances with alternating signs},%
		{Example}/{\(x^3 + 8 = (x+2)(x^2 - 2x + 4)\)},%
		{Extension}/{Pairs with difference of cubes for full cubic factorization}%
	}
\end{NxLightListBox}

\begin{NxLightListBox}[title={Difference of Cubes}]
	\nxEachLabel{ArrowDark}{Secondary}{{4}{5}}{%
		{Definition}/{Difference of cubes factors into binomial and trinomial},%
		{General Formula}/{\(a^3 - b^3 = (a-b)(a^2 + ab + b^2)\)},%
		{Pattern}/{Binomial carries the difference, trinomial balances with all positives},%
		{Example}/{\(x^3 - 27 = (x-3)(x^2 + 3x + 9)\)},%
		{Extension}/{Complements sum of cubes in cubic identities}%
	}
\end{NxLightListBox}

\begin{NxLightListBox}[title={Perfect Square Trinomial}]
	\nxEachLabel{ArrowDark}{Secondary}{{4}{5}}{%
		{Definition}/{Squaring a binomial produces a trinomial pattern},%
		{General Formula}/{\((a+b)^2 = a^2 + 2ab + b^2\)},%
		{Negative Case}/{\((a-b)^2 = a^2 - 2ab + b^2\)},%
		{Pattern}/{Always: square + double product + square},%
		{Example}/{\((x+3)^2 = x^2 + 6x + 9\)},%
		{Extension}/{Recognizing this pattern speeds up factoring and simplification}%
	}
\end{NxLightListBox}

\begin{NxLightListBox}[title={Exponent and Radical Identities}]
	\nxEachLabel{ArrowDark}{Secondary}{{4}{5}}{%
		{Power of a Power}/{\((a^m)^n = a^{mn}\)},%
		{Negative Exponent}/{\(a^{-n} = \frac{1}{a^n}\)},%
		{Fractional Exponent}/{\(\sqrt[n]{a^m} = a^{m/n}\)},%
		{Product Rule}/{\(a^m \cdot a^n = a^{m+n}\)},%
		{Quotient Rule}/{\(\frac{a^m}{a^n} = a^{m-n}\)},%
		{Extension}/{Links radicals, reciprocals, and powers into one unified system}%
	}
\end{NxLightListBox}

\begin{NxLightListBox}[title={Binomial Expansion}]
	\nxEachLabel{ArrowDark}{Secondary}{{4}{5}}{%
		{Definition}/{Expansion of \((a+b)^n\) into a sum of terms},%
		{General Formula}/{\((a+b)^n = \sum_{k=0}^{n} \binom{n}{k} a^{n-k} b^k\)},%
		{Binomial Coefficient}/{\(\binom{n}{k} = \frac{n!}{k!(n-k)!}\)},%
		{Pattern}/{Exponents of \(a\) decrease, exponents of \(b\) increase},%
		{Symmetry}/{Coefficients are symmetric: \(\binom{n}{k} = \binom{n}{n-k}\)},%
		{Connection}/{Coefficients form Pascal’s Triangle},%
		{Example}/{\((x+1)^4 = x^4 + 4x^3 + 6x^2 + 4x + 1\)}%
	}
\end{NxLightListBox}

