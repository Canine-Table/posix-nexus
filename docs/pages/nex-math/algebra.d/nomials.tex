\nxSections{Nomials}{2}

\begin{NxLightListBox}[title={Nomial Lineage}]
		\nxEachLabel{ArrowDark}{Secondary}{{4}{5}}{%
				{Monomial}/{One term only, e.g. \(7x^2\)},%
				{Binomial}/{Two terms, e.g. \(x+2\)},%
				{Trinomial}/{Three terms, e.g. \(x^2+3x+2\)},%
				{Polynomial}/{Many terms, general family},%
				{Technique}/{Prefix indicates number of terms},%
				{Outcome}/{Classification helps organize algebraic expressions}%
		}
\end{NxLightListBox}

\begin{empheq}[box=\nxSuccessMathBox]{align}
    \text{Monomial: } & ax^n \label{eq:monomial} \\
    \text{Binomial: } & ax^n + bx^m \label{eq:binomial} \\
    \text{Trinomial: } & ax^n + bx^m + cx^k \label{eq:trinomial}
\end{empheq}
\bigskip

\begin{NxLightListBox}[title={Polynomial Forms}]
		\nxEachLabel{ArrowDark}{Secondary}{{4}{5}}{%
				{Monomial}/{One term only, e.g. \(7x^2\)},%
				{Binomial}/{Two unlike terms, e.g. \(x+2\)},%
				{Trinomial}/{Three terms, e.g. \(x^2+3x+2\)},%
				{Polynomial}/{General family with many terms},%
				{Technique}/{Prefix indicates number of terms},%
				{Outcome}/{Classification helps organize algebraic expressions}%
		}
\end{NxLightListBox}

\begin{NxLightListBox}[title={Like Terms}]
		\nxEachLabel{ArrowDark}{Secondary}{{4}{5}}{%
				{Definition}/{Expressions with identical variable parts, e.g. \(3x^2\) and \(-5x^2\)},%
				{Variable Match}/{Same variables with same exponents},%
				{Coefficient}/{Numbers in front may differ},%
				{Technique}/{Combine by adding or subtracting coefficients},%
				{Outcome}/{Simplifies polynomials by reducing to fewer terms}%
		}
\end{NxLightListBox}

\begin{NxLightListBox}[title={Coefficient}]
		\nxEachLabel{ArrowDark}{Secondary}{{4}{5}}{%
				{Definition}/{The number in front of a variable, e.g. in \(7x\) the coefficient is 7},%
				{Variable Match}/{It scales the variable part without changing its type},%
				{Examples}/{\(3x^2\) has coefficient 3, \(-5y\) has coefficient -5},%
				{Constants}/{A constant term like 4 can be seen as coefficient 4 of \(x^0\)},%
				{Outcome}/{Coefficients tell how strongly each variable contributes to the polynomial}%
		}
\end{NxLightListBox}

\nxAlgebraInput{exponent}%
\nxAlgebraInput{divide}%
\nxAlgebraInput{multiply}%
\nxAlgebraInput{combine}%
\nxAlgebraInput{foil}%
\nxAlgebraInput{factor}%

