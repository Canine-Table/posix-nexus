\nxSections{Radicals}{2}

\begin{NxLightListBox}[title={Definition of Radical}]
    \nxEachLabel{ArrowDark}{Secondary}{{4}{5}}{%
        {Radical}/{An expression that uses the root symbol \(\sqrt{}\)},%
        {Square Root}/{The most common radical, \(\sqrt{x}\), meaning the number which squared gives \(x\)},%
        {Index}/{The small number above the radical, e.g. \(\sqrt[3]{x}\) is the cube root},%
        {Radicand}/{The number or expression inside the radical sign},%
        {Example}/{\(\sqrt{25} = 5\), \(\sqrt[3]{8} = 2\)},%
        {Use}/{Radicals are used to express roots, simplify algebraic expressions, and solve equations}%
    }
\end{NxLightListBox}

\begin{empheq}[box=\nxWarningMathBox]{align*}
    \text{Simplify } \sqrt{50} \\
    \sqrt{50} &= \sqrt{25 \cdot 2} \quad \nxArw{5} \\
              &= \sqrt{25} \cdot \sqrt{2} \quad \nxArw{5} \\
              &= 5\sqrt{2}
\end{empheq}

\begin{NxLightListBox}[title={Using Radicals}]
    \nxEachLabel{ArrowDark}{Secondary}{{4}{5}}{%
        {Step 1}/{Identify the radicand},%
        {Step 2}/{Factor the radicand into perfect powers and leftovers},%
        {Step 3}/{Simplify by taking the root of the perfect power},%
        {Step 4}/{Leave the leftover inside the radical},%
        {Outcome}/{\(\sqrt{50} = 5\sqrt{2}\)}%
    }
\end{NxLightListBox}

