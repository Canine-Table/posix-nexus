\nxSections{Radicals}{2}

\begin{NxLightListBox}[title={Identity of $\sqrt[n]{a^n}$}]
	\nxEachLabel{ArrowDark}{Secondary}{{4}{5}}{%
		{Expression}/{\(\sqrt[n]{a^n}\)},%
		{Meaning}/{The \(n\)-th root of \(a^n\)},%
		{Simplification}/{Cancels the root and exponent, yielding \(a\)},%
		{Condition}/{Valid when \(a \geq 0\) for real roots, or in complex domain otherwise},%
		{Example}/{\(\sqrt[3]{2^3} = \sqrt[3]{8} = 2\)},%
		{Extension}/{\(\sqrt[n]{a^m} = a^{m/n}\)}%
	}
\end{NxLightListBox}

\begin{empheq}[box=\nxWarningMathBox]{align*}
	\sqrt[n]{a^n} = a
\end{empheq}
\bigskip

\begin{NxLightListBox}[title={Definition of Radical}]
	\nxEachLabel{ArrowDark}{Secondary}{{4}{5}}{%
		{Radical}/{An expression that uses the root symbol \(\sqrt{}\)},%
		{Square Root}/{The most common radical, \(\sqrt{x}\), meaning the number which squared gives \(x\)},%
		{Index}/{The small number above the radical, e.g. \(\sqrt[3]{x}\) is the cube root},%
		{Radicand}/{The number or expression inside the radical sign},%
		{Example}/{\(\sqrt{25} = 5\), \(\sqrt[3]{8} = 2\)},%
		{Use}/{Radicals are used to express roots, simplify algebraic expressions, and solve equations}%
	}
\end{NxLightListBox}

\begin{empheq}[box=\nxWarningMathBox]{align*}
		\sqrt{50} \;\nxArw{5}\;
		\sqrt{25 \cdot 2} \;\nxArw{5}\;
		\sqrt{25} \cdot \sqrt{2} \;\nxArw{5}\;
	5\sqrt{2}
\end{empheq}

\begin{NxLightListBox}[title={Using Radicals}]
	\nxEachLabel{ArrowDark}{Secondary}{{4}{5}}{%
		{Step 1}/{Identify the radicand},%
		{Step 2}/{Factor the radicand into perfect powers and leftovers},%
		{Step 3}/{Simplify by taking the root of the perfect power},%
		{Step 4}/{Leave the leftover inside the radical},%
		{Outcome}/{\(\sqrt{50} = 5\sqrt{2}\)}%
	}
\end{NxLightListBox}

\begin{NxLightListBox}[title={Overview of \texttt{nx\_squares}}]
	\nxEachLabel{ArrowDark}{Secondary}{{4}{5}}{%
		{Purpose}/{Computes square roots using iterative scaling and subtraction},%
		{Input}/{A single value \(x\)},%
		{Initialization}/{Takes absolute value, sets scaling factor, prepares working variables},%
		{Scaling}/{Doubles a seed until it exceeds the radicand},%
		{Iteration}/{Subtracts scaled values, updates root approximation step by step},%
		{Condition}/{Handles zero and invalid cases gracefully, returns early if needed},%
		{Output}/{Final square root approximation, or truncated value if iteration fails}%
	}
\end{NxLightListBox}

\begin{NxCodeBox}{c}{title={Square Roots}, phantomlabel={code:SquareRoot}}
define nx_nr_sqrt(x) {
	auto y, p
	if (nx_xy_breach(x, 1) == -1)
		return -1
	y = x / 2
	p = 0
	while (y != p) {
		p = y
		y = (y + x / y) / 2
	}
	return y
}
\end{NxCodeBox}

\begin{NxCodeBox}{c}{title={Square Roots}, phantomlabel={code:SquareRoots}}
define nx_squares(x) {
	auto a, s, b, y
	a = nx_abs(x)
	if (a == 0 || scale == 0)
		return x
	s = 1
	while (s < a)
		s = s * 2
	b = nx_scale(1)
	x = 0
	y = 0
	while (s > b) {
		if (s <= a) {
			a = a - s
			x = y
			y = nx_nr_sqrt(s)
			if (y == -1)
				return x
			if (x != 0)
				print x, ","
		}
		s = s / 3
	}
	return y
}
\end{NxCodeBox}

\begin{NxLightListBox}[title={Definition of Vinculum}]
	\nxEachLabel{ArrowDark}{Secondary}{{4}{5}}{%
		{Vinculum}/{The horizontal bar drawn over the radicand in a radical},%
		{Scope}/{Shows exactly which terms are included under the radical},%
		{Fraction Use}/{Also used as the bar separating numerator and denominator in fractions},%
		{Repeating Decimal}/{Used to mark repeating digits, e.g. \(0.\overline{3}\)},%
		{Example}/{In \(\sqrt{a+b}\), the vinculum extends over \(a+b\)},%
		{Use}/{Ensures clarity of grouping inside radicals, fractions, and repeating decimals}%
	}
\end{NxLightListBox}

\begin{NxLightListBox}[title={Definition of Radicand}]
	\nxEachLabel{ArrowDark}{Secondary}{{4}{5}}{%
		{Radicand}/{The number or expression placed under the radical sign \(\sqrt{}\)},%
		{Role}/{It is the quantity from which a root is extracted},%
		{Example}/{In \(\sqrt{25}\), the radicand is 25},%
		{Extended}/{In \(\sqrt[3]{8}\), the radicand is 8, and the index is 3},%
		{Scope}/{The vinculum (bar) shows exactly which terms belong to the radicand},%
		{Outcome}/{Identifying the radicand clarifies what is being rooted in the expression}%
	}
\end{NxLightListBox}

\begin{NxLightListBox}[title={Square Root Expansion}]
	\nxEachLabel{ArrowDark}{Secondary}{{4}{5}}{%
		{Radical}/{\(\sqrt{72}\)},%
		{Factor}/{Break into perfect square and leftover: \(72 = 36 \cdot 2\)},%
		{Expand}/{\(\sqrt{72} = \sqrt{36 \cdot 2} = \sqrt{36}\cdot\sqrt{2}\)},%
		{Simplify}/{\(\sqrt{36} = 6\)},%
		{Outcome}/{\(\sqrt{72} = 6\sqrt{2}\)}%
	}
\end{NxLightListBox}

\begin{NxLightListBox}[title={Square Root Truncation}]
	\nxEachLabel{ArrowDark}{Secondary}{{4}{5}}{%
		{Radical}/{\(\sqrt{72}\)},%
		{Approximation}/{Leave as decimal: \(\sqrt{72} \approx 8.485\)},%
		{Truncation}/{Cut after two decimals: \(8.48\)},%
		{Outcome}/{Truncated radical value \(\approx 8.48\)}%
	}
\end{NxLightListBox}

\begin{NxLightListBox}[title={Types of Square Roots}]
	\nxEachLabel{ArrowDark}{Secondary}{{4}{5}}{%
		{Integer Square Root}/{The root of a whole number. It may simplify to a rational (e.g. \(\sqrt{25} = 5\)) or remain irrational (e.g. \(\sqrt{2}\))},%
		{Fraction Square Root}/{The root of a ratio \(\tfrac{p}{q}\), defined as \(\sqrt{\tfrac{p}{q}} = \tfrac{\sqrt{p}}{\sqrt{q}}\)},%
		{Rational Square Root}/{Occurs when both numerator and denominator are perfect squares, e.g. \(\sqrt{\tfrac{9}{16}} = \tfrac{3}{4}\)},%
		{Irrational Square Root}/{Occurs when either numerator or denominator is not a perfect square, e.g. \(\sqrt{\tfrac{2}{9}} = \tfrac{\sqrt{2}}{3}\)},%
		{Impurity Case}/{When the denominator equals 0, the expression is undefined, e.g. \(\sqrt{\tfrac{p}{0}}\)}%
	}
\end{NxLightListBox}

