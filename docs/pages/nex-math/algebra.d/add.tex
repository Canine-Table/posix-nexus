\nxSections{Addition}{2}

\nxSections{Sum and Addend}{3}

\begin{NxLightListBox}[title={Sum}]
	\nxEachLabel{ArrowDark}{Secondary}{{4}{5}}{%
		{Definition}/{The result of an addition operation},%
		{Formula}/{Sum = Addend₁ + Addend₂ (+ \ldots)},%
		{Example}/{\(7 + 3 = 10\), so 10 is the sum},%
		{Connection}/{Glyph of combination, outcome of addition},%
		{Extension}/{Used in arithmetic, algebra, and series notation}%
	}
\end{NxLightListBox}

\begin{NxLightListBox}[title={Addend}]
	\nxEachLabel{ArrowDark}{Secondary}{{4}{5}}{%
		{Definition}/{A number that is added to another},%
		{Position}/{Each term in an addition is an addend},%
		{Example}/{In \(7 + 3\), both 7 and 3 are addends},%
		{Connection}/{Glyph of contribution, building blocks of the sum},%
		{Extension}/{In algebra, variables can also be addends: \(x + y\)}%
	}
\end{NxLightListBox}

\begin{NxLightListBox}[title={Series and Summation}]
	\nxEachLabel{ArrowDark}{Secondary}{{4}{5}}{%
		{Notation}/{\(\sum_{i=1}^{n} a_i\) means addends \(a_1, a_2, \ldots, a_n\)},%
		{Meaning}/{Sum is the collective result of all addends},%
		{Example}/{\(\sum_{i=1}^{4} i = 1+2+3+4 = 10\)},%
		{Connection}/{Summation generalizes addition to many addends},%
		{Extension}/{Foundation of series in calculus and analysis}%
	}
\end{NxLightListBox}

