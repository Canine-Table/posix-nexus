\nxSections{Notation}{2}

\nxSections{Delta in Mathematics}{3}

\begin{NxLightListBox}[title={Finite Difference}]
	\nxEachLabel{ArrowDark}{Secondary}{{4}{5}}{%
		{Definition}/{Delta denotes change or difference between two values},%
		{Formula}/{\(\Delta x = x_{\text{final}} - x_{\text{initial}}\)},%
		{Usage}/{Discrete counterpart to derivative: \(\Delta y / \Delta x \approx dy/dx\)},%
		{Example}/{\(\Delta y = f(x_2) - f(x_1)\)},%
		{Extension}/{Forms the basis of finite difference methods in numerical analysis}%
	}
\end{NxLightListBox}

\begin{NxLightListBox}[title={Discriminant}]
	\nxEachLabel{ArrowDark}{Secondary}{{4}{5}}{%
		{Definition}/{In quadratic equations, \(\Delta\) denotes the discriminant},%
		{Formula}/{\(\Delta = b^2 - 4ac\)},%
		{Usage}/{Determines the nature of roots of \(ax^2+bx+c=0\)},%
		{Example}/{\(\Delta > 0\): two real roots; \(\Delta = 0\): one real root; \(\Delta < 0\): complex roots},%
		{Extension}/{Generalized discriminants exist for higher‑degree polynomials}%
	}
\end{NxLightListBox}

\begin{NxLightListBox}[title={Triangle Symbol}]
	\nxEachLabel{ArrowDark}{Secondary}{{4}{5}}{%
		{Definition}/{\(\Delta\) also denotes a triangle in geometry},%
		{Usage}/{\(\Delta ABC\) means triangle with vertices A, B, C},%
		{Connection}/{Links algebraic glyph with geometric figure lineage},%
		{Extension}/{Area of a triangle often denoted by \(\Delta\)}%
	}
\end{NxLightListBox}

\nxSections{Theta in Mathematics}{3}

\begin{NxLightListBox}[title={Angle Representation}]
	\nxEachLabel{ArrowDark}{Secondary}{{4}{5}}{%
		{Definition}/{Greek letter \(\theta\) used to denote an angle},%
		{Trigonometry}/{Appears in sine, cosine, tangent: e.g. \(\sin(\theta), \cos(\theta), \tan(\theta)\)},%
		{Geometry}/{Used in right triangles to relate sides and angles},%
		{Polar Coordinates}/{Point \((r,\theta)\) defined by radius and angle},%
		{Example}/{In a right triangle, \(\cos(\theta) = \frac{\text{adjacent}}{\text{hypotenuse}}\)}%
	}
\end{NxLightListBox}

\begin{NxLightListBox}[title={Other Mathematical Uses}]
	\nxEachLabel{ArrowDark}{Secondary}{{4}{5}}{%
		{Statistics}/{\(\theta\) often denotes parameters in probability distributions},%
		{Complex Numbers}/{Angle of rotation in Euler’s formula: \(e^{i\theta} = \cos(\theta) + i\sin(\theta)\)},%
		{Calculus}/{Variable of integration in polar coordinates},%
		{Numerical Value}/{In Greek numerals, \(\theta\) has value 9},%
		{Connection}/{Symbol of rotation, periodicity, and parameterization across math}%
	}
\end{NxLightListBox}

\nxSections{Epsilon in Mathematics}{3}

\begin{NxLightListBox}[title={Infinitesimal Bound}]
	\nxEachLabel{ArrowDark}{Secondary}{{4}{5}}{%
		{Definition}/{Greek letter \(\varepsilon\) used to denote a very small positive quantity},%
		{Limit Definition}/{Appears in \(\varepsilon\,-\,\delta\) proofs: for every \(\varepsilon > 0\), there exists \(\delta > 0\)},%
		{Usage}/{Measures closeness of a function to a limit},%
		{Example}/{\(|f(x)-L| < \varepsilon\) whenever \(|x-c| < \delta\)},%
		{Connection}/{Core of rigorous calculus and real analysis}%
	}
\end{NxLightListBox}

\begin{NxLightListBox}[title={Error and Approximation}]
	\nxEachLabel{ArrowDark}{Secondary}{{4}{5}}{%
		{Definition}/{\(\varepsilon\) often denotes error tolerance or margin},%
		{Numerical Analysis}/{Used to bound approximation error},%
		{Example}/{If \(|x - x_0| < \varepsilon\), then \(x\) is within tolerance},%
		{Connection}/{Links exact mathematics with numerical computation}%
	}
\end{NxLightListBox}

\begin{NxLightListBox}[title={Other Uses}]
	\nxEachLabel{ArrowDark}{Secondary}{{4}{5}}{%
		{Set Theory}/{\(\varepsilon\) sometimes used as membership symbol, though ∈ is standard},%
		{Complexity}/{\(\varepsilon\) denotes arbitrarily small constants in algorithm analysis},%
		{Probability}/{\(\varepsilon\) used in inequalities like Chebyshev’s or \(\varepsilon\)‑nets},%
		{Connection}/{Universal glyph for \nxDQ{smallness} across math disciplines}%
	}
\end{NxLightListBox}

\nxSections{Delta in Mathematics}{3}

\begin{NxLightListBox}[title={Infinitesimal Change}]
	\nxEachLabel{ArrowDark}{Secondary}{{4}{5}}{%
		{Definition}/{Greek letter \(\delta\) denotes a very small positive quantity},%
		{Limit Proofs}/{Appears in \(\varepsilon\)-\(\delta\) definitions of limits},%
		{Formula}/{For every \(\varepsilon > 0\), there exists \(\delta > 0\)},%
		{Usage}/{Controls how close \(x\) must be to \(c\) for \(f(x)\) to be within \(\varepsilon\) of \(L\)},%
		{Example}/{If \(|x-c| < \delta\), then \(|f(x)-L| < \varepsilon\)},%
		{Connection}/{\(\delta\) measures input closeness, \(\varepsilon\) measures output closeness}%
	}
\end{NxLightListBox}

\begin{NxLightListBox}[title={Variation and Error}]
	\nxEachLabel{ArrowDark}{Secondary}{{4}{5}}{%
		{Definition}/{\(\delta\) often denotes small variation or tolerance},%
		{Numerical Analysis}/{Used to bound input error},%
		{Example}/{If \(|x-x_0| < \delta\), then \(x\) is within tolerance of \(x_0\)},%
		{Connection}/{Pairs with \(\varepsilon\) to formalize precision in analysis and computation}%
	}
\end{NxLightListBox}

\begin{NxLightListBox}[title={Other Mathematical Uses}]
	\nxEachLabel{ArrowDark}{Secondary}{{4}{5}}{%
		{Geometry}/{\(\delta\) sometimes used for small angles},%
		{Statistics}/{\(\delta\) may denote deviation or perturbation},%
		{Complexity}/{\(\delta\) used for small constants in algorithm analysis},%
		{Connection}/{Universal glyph for \nxDQ{small input change} across math disciplines}%
	}
\end{NxLightListBox}

\nxSections{Proportional Symbol}{3}

\begin{NxLightListBox}[title={Definition}]
	\nxEachLabel{ArrowDark}{Secondary}{{4}{5}}{%
		{Glyph}/{\(\propto\) resembles the left half of \(\infty\)},%
		{Meaning}/{Denotes proportionality between two quantities},%
		{Formula}/{\(y \propto x\) means \(y = kx\) for some constant \(k\)},%
		{Usage}/{Used in algebra, physics, and statistics to show direct proportionality},%
		{Example}/{Gravitational force: \(F \propto \frac{1}{r^2}\)}%
	}
\end{NxLightListBox}

\begin{NxLightListBox}[title={Key Properties}]
	\nxEachLabel{ArrowDark}{Secondary}{{4}{5}}{%
		{Constant of Proportionality}/{Always exists: \(y = kx\)},%
		{Direct Proportionality}/{If one doubles, the other doubles},%
		{Inverse Proportionality}/{Written as \(y \propto \frac{1}{x}\)},%
		{Scaling}/{Proportionality preserves ratios},%
		{Connection}/{Symbol bridges ratios, scaling laws, and functional dependence}%
	}
\end{NxLightListBox}

\nxSections{Infinity in Mathematics}{3}

\begin{NxLightListBox}[title={Concept of Infinity}]
	\nxEachLabel{ArrowDark}{Secondary}{{4}{5}}{%
		{Definition}/{\(\infty\) denotes an unbounded quantity, larger than any real number},%
		{Calculus}/{Appears in limits: \(\lim_{x \to \infty} f(x)\)},%
		{Set Theory}/{Represents cardinalities of infinite sets},%
		{Geometry}/{Used to mark points at infinity in projective geometry},%
		{Connection}/{Symbol of endlessness, beyond finite measurement}%
	}
\end{NxLightListBox}

\begin{NxLightListBox}[title={Infinity in Calculus}]
	\nxEachLabel{ArrowDark}{Secondary}{{4}{5}}{%
		{Improper Integrals}/{\(\int_1^\infty \frac{1}{x^2} dx\)},%
		{Limits}/{\(\lim_{x \to \infty} \frac{1}{x} = 0\)},%
		{Series}/{Infinite sums: \(\sum_{n=1}^\infty \frac{1}{n^2}\)},%
		{Connection}/{\(\infty\) marks the boundary of convergence and divergence},%
		{Example}/{Harmonic series diverges: \(\sum_{n=1}^\infty \frac{1}{n}\)}%
	}
\end{NxLightListBox}

\begin{NxLightListBox}[title={Infinity in Set Theory}]
	\nxEachLabel{ArrowDark}{Secondary}{{4}{5}}{%
		{Countable Infinity}/{Size of natural numbers, denoted \(\aleph_0\)},%
		{Uncountable Infinity}/{Size of real numbers, larger than \(\aleph_0\)},%
		{Comparison}/{Not all infinities are equal},%
		{Connection}/{\(\infty\) as a concept differs from cardinal numbers},%
		{Example}/{\(|\mathbb{N}| = \aleph_0\), but \(|\mathbb{R}| > \aleph_0\)}%
	}
\end{NxLightListBox}

\nxSections{Perpendicular Symbol}{3}

\begin{NxLightListBox}[title={Definition}]
	\nxEachLabel{ArrowDark}{Secondary}{{4}{5}}{%
		{Glyph}/{\(\perp\) is the mathematical symbol for perpendicularity},%
		{Meaning}/{Two lines, segments, or planes meet at a right angle \(90^\deg\)},%
		{Notation}/{\(AB \perp CD\) means line AB is perpendicular to line CD},%
		{Geometry}/{Used to denote orthogonality in Euclidean space},%
		{Example}/{In a square, adjacent sides are \(\perp\) to each other}%
	}
\end{NxLightListBox}

\begin{NxLightListBox}[title={Linear Algebra Connection}]
	\nxEachLabel{ArrowDark}{Secondary}{{4}{5}}{%
		{Orthogonality}/{\(\perp\) denotes vectors with dot product zero},%
		{Formula}/{\(\vec{u} \perp \vec{v} \iff \vec{u}\cdot\vec{v} = 0\)},%
		{Usage}/{Defines orthogonal bases and projections},%
		{Example}/{\((1,0)\) \(\perp\) \((0,1)\) in \(\mathbb{R}^2\)},%
		{Extension}/{Orthogonality generalizes perpendicularity to higher dimensions}%
	}
\end{NxLightListBox}

\begin{NxLightListBox}[title={Other Uses}]
	\nxEachLabel{ArrowDark}{Secondary}{{4}{5}}{%
		{Logic}/{\(\perp\) sometimes denotes contradiction or falsity},%
		{Probability}/{\(\perp\) used to denote independence in some texts},%
		{Connection}/{Symbol bridges geometry, algebra, and logic},%
		{Visual}/{Always evokes the right‑angle lineage}%
	}
\end{NxLightListBox}

\nxSections{Plus-Minus and Minus-Plus}{3}

\begin{NxLightListBox}[title={Plus-Minus (\(\pm\))}]
	\nxEachLabel{ArrowDark}{Secondary}{{4}{5}}{%
		{Definition}/{Symbol \(\pm\) means \nxDQ{plus or minus}},%
		{Usage}/{Represents two possible values: \(a+b\) or \(a-b\)},%
		{Example}/{Quadratic formula: \(x = \frac{-b \pm \sqrt{b^2-4ac}}{2a}\)},%
		{Connection}/{Encodes duality in solutions, symmetry in expansions},%
		{Extension}/{Used in error bounds and approximations: \(x \pm \varepsilon\)}%
	}
\end{NxLightListBox}

\begin{NxLightListBox}[title={Minus-Plus (\(\mp\))}]
	\nxEachLabel{ArrowDark}{Secondary}{{4}{5}}{%
		{Definition}/{Symbol \(\mp\) means \nxDQ{minus or plus,} paired with \(\pm\)},%
		{Usage}/{Ensures opposite choice when \(\pm\) is used earlier},%
		{Example}/{If first term is \(+\), second term takes \(−\); if first is \(−\), second takes \(+\)},%
		{Connection}/{Keeps expressions consistent in paired signs},%
		{Extension}/{Common in trigonometric identities and vector formulas}%
	}
\end{NxLightListBox}

\begin{NxLightListBox}[title={Combined Expression}]
	\nxEachLabel{ArrowDark}{Secondary}{{4}{5}}{%
		{Notation}/{\(a \pm b \mp c\)},%
		{Meaning}/{Two cases: \(a+b-c\) or \(a-b+c\)},%
		{Pattern}/{\(\pm\) and \(\mp\) always paired to flip signs consistently},%
		{Example}/{In trig: \(\sin(x \pm y) = \sin x \cos y \pm \cos x \sin y\)},%
		{Connection}/{Encodes dual solutions in compact symbolic form}%
	}
\end{NxLightListBox}


\nxSections{Parallel Symbol}{3}

\begin{NxLightListBox}[title={Definition}]
	\nxEachLabel{ArrowDark}{Secondary}{{4}{5}}{%
		{Glyph}/{\(\parallel\) is the mathematical symbol for parallelism},%
		{Meaning}/{Two lines, segments, or planes never intersect and remain equidistant},%
		{Notation}/{\(AB \parallel CD\) means line AB is parallel to line CD},%
		{Geometry}/{Used in Euclidean geometry to denote parallel lines and planes},%
		{Example}/{In a rectangle, opposite sides are \(\parallel\) to each other}%
	}
\end{NxLightListBox}

\begin{NxLightListBox}[title={Linear Algebra Connection}]
	\nxEachLabel{ArrowDark}{Secondary}{{4}{5}}{%
		{Vectors}/{\(\parallel\) denotes vectors that are scalar multiples of each other},%
		{Formula}/{\(\vec{u} \parallel \vec{v} \iff \vec{u} = k\vec{v}\)},%
		{Usage}/{Defines direction equivalence in vector spaces},%
		{Example}/{\((2,4)\) \(\parallel\) \((1,2)\) since \((2,4) = 2(1,2)\)},%
		{Extension}/{Parallelism generalizes beyond geometry into linear algebra and physics}%
	}
\end{NxLightListBox}

\begin{NxLightListBox}[title={Other Mathematical Uses}]
	\nxEachLabel{ArrowDark}{Secondary}{{4}{5}}{%
		{Analysis}/{\(\parallel\)x\(\parallel\) sometimes denotes norm of a vector},%
		{Logic}/{\(\parallel\) used in some texts for \nxDQ{parallel execution} or independence},%
		{Connection}/{Symbol bridges geometry, algebra, and analysis},%
		{Visual}/{Always evokes equidistant, non‑intersecting lineage}%
	}
\end{NxLightListBox}

\nxSections{Angle Symbol}{3}

\begin{NxLightListBox}[title={Definition}]
	\nxEachLabel{ArrowDark}{Secondary}{{4}{5}}{%
		{Glyph}/{\(\angle\) is the mathematical symbol for an angle},%
		{Meaning}/{Represents the measure of rotation between two intersecting lines or rays},%
		{Notation}/{\(\angle ABC\) means the angle formed at vertex B by rays BA and BC},%
		{Units}/{Measured in degrees (\(^\circ\)) or radians},%
		{Example}/{\(\angle ABC = 90^\circ\) denotes a right angle}%
	}
\end{NxLightListBox}

\begin{NxLightListBox}[title={Geometry Usage}]
	\nxEachLabel{ArrowDark}{Secondary}{{4}{5}}{%
		{Right Angle}/{\(\angle = 90^\circ\)},%
		{Acute Angle}/{\(\angle < 90^\circ\)},%
		{Obtuse Angle}/{\(90^\circ < \angle < 180^\circ\)},%
		{Straight Angle}/{\(\angle = 180^\circ\)},%
		{Reflex Angle}/{\(180^\circ < \angle < 360^\circ\)}%
	}
\end{NxLightListBox}

\begin{NxLightListBox}[title={Other Mathematical Uses}]
	\nxEachLabel{ArrowDark}{Secondary}{{4}{5}}{%
		{Trigonometry}/{\(\angle\) used inside sine, cosine, tangent functions},%
		{Polar Coordinates}/{Point \((r,\theta)\) defined by radius and angle},%
		{Complex Numbers}/{Angle defines argument of a complex number},%
		{Vector Analysis}/{Angle between vectors via dot product},%
		{Connection}/{Symbol bridges geometry, trigonometry, and analysis}%
	}
\end{NxLightListBox}

\nxSections{Similar vs Equivalent}{3}

\begin{NxLightListBox}[title={Similar Symbol (\(\sim\))}]
	\nxEachLabel{ArrowDark}{Secondary}{{4}{5}}{%
		{Glyph}/{\(\sim\) is the symbol for similarity},%
		{Geometry}/{\(\triangle ABC \sim \triangle DEF\) means triangles have equal angles and proportional sides},%
		{Algebra}/{Sometimes used to denote asymptotic equivalence: \(f(x) \sim g(x)\) as \(x \to \infty\)},%
		{Pattern}/{Similarity preserves shape but not necessarily size},%
		{Example}/{\(\triangle ABC \sim \triangle DEF\) if \(\angle A = \angle D, \angle B = \angle E, \angle C = \angle F\)},%
		{Connection}/{Symbol of proportionality in geometry and asymptotics in analysis}%
	}
\end{NxLightListBox}

\begin{NxLightListBox}[title={Equivalent Symbol (\(\equiv\))}]
	\nxEachLabel{ArrowDark}{Secondary}{{4}{5}}{%
		{Glyph}/{\(\equiv\) is the symbol for equivalence},%
		{Congruence}/{\(\triangle ABC \equiv \triangle DEF\) means triangles are identical in shape and size},%
		{Number Theory}/{Used for modular congruence: \(a \equiv b \pmod{n}\)},%
		{Logic}/{Denotes logical equivalence: \(P \equiv Q\)},%
		{Pattern}/{Equivalence preserves both shape and size, or exact relation},%
		{Example}/{\(17 \equiv 5 \pmod{12}\)},%
		{Connection}/{Symbol of exact sameness across geometry, algebra, and logic}%
	}
\end{NxLightListBox}

\nxSections{Approximate Symbol}{3}

\begin{NxLightListBox}[title={Definition}]
	\nxEachLabel{ArrowDark}{Secondary}{{4}{5}}{%
		{Glyph}/{\(\approx\) is the mathematical symbol for approximation},%
		{Meaning}/{Indicates two values are close but not exactly equal},%
		{Notation}/{\(a \approx b\) means \(a\) is approximately equal to \(b\)},%
		{Usage}/{Common in numerical analysis, applied math, and physics},%
		{Example}/{\(\pi \approx 3.1416\)}%
	}
\end{NxLightListBox}

\begin{NxLightListBox}[title={Contexts of Use}]
	\nxEachLabel{ArrowDark}{Secondary}{{4}{5}}{%
		{Numerical}/{Used when rounding or truncating decimals},%
		{Physics}/{Marks measured values close to theoretical ones},%
		{Statistics}/{Denotes approximate probabilities or estimates},%
		{Analysis}/{Signals asymptotic closeness in limits},%
		{Connection}/{Symbol bridges exact math with practical computation}%
	}
\end{NxLightListBox}

\begin{NxLightListBox}[title={Related Symbols}]
	\nxEachLabel{ArrowDark}{Secondary}{{4}{5}}{%
		{Equal Sign}/{\(=\) denotes exact equality},%
		{Tilde}/{\(\sim\) denotes similarity or asymptotic equivalence},%
		{Congruence}/{\(\equiv\) denotes exact equivalence or modular congruence},%
		{Approx}/{\(\approx\) specifically signals numerical closeness},%
		{Extension}/{Each glyph stages a different level of sameness}%
	}
\end{NxLightListBox}

\nxSections{Similar or Equal Symbol}{3}

\begin{NxLightListBox}[title={Definition}]
	\nxEachLabel{ArrowDark}{Secondary}{{4}{5}}{%
		{Glyph}/{\(\simeq\) is the symbol for \nxDQ{similar or equal}},%
		{Meaning}/{Indicates two quantities are nearly equal and share structural similarity},%
		{Usage}/{Common in analysis, approximation, and asymptotic notation},%
		{Example}/{\(f(x) \simeq g(x)\) means functions are close in value and form},%
		{Connection}/{Bridges similarity (\(\sim\)) and equality (=)}%
	}
\end{NxLightListBox}

\begin{NxLightListBox}[title={Contexts of Use}]
	\nxEachLabel{ArrowDark}{Secondary}{{4}{5}}{%
		{Approximation}/{Used when values are not exactly equal but very close},%
		{Asymptotics}/{Signals functions behave similarly as \(x \to \infty\)},%
		{Geometry}/{Sometimes used to denote figures nearly congruent},%
		{Physics}/{Marks quantities equal within experimental tolerance},%
		{Extension}/{\(\simeq\) is less strict than \(\equiv\) (equivalent) but stronger than \(\approx\) (approximate)}%
	}
\end{NxLightListBox}

\nxSections{Congruent Symbol}{3}

\begin{NxLightListBox}[title={Definition}]
	\nxEachLabel{ArrowDark}{Secondary}{{4}{5}}{%
		{Glyph}/{\(\cong\) is produced in LaTeX with \texttt{\textbackslash cong}},%
		{Meaning}/{Denotes congruence: \nxDQ{is congruent to}},%
		{Geometry}/{\(\triangle ABC \cong \triangle DEF\) means triangles are identical in shape and size},%
		{Pattern}/{Congruence preserves both angles and side lengths},%
		{Example}/{\(\triangle\) with sides 3,4,5 is \(\cong\) to another 3,4,5 triangle},%
		{Connection}/{Stronger than similarity (\(\sim\)), exact match without scaling}%
	}
\end{NxLightListBox}

\begin{NxLightListBox}[title={Other Mathematical Uses}]
	\nxEachLabel{ArrowDark}{Secondary}{{4}{5}}{%
		{Number Theory}/{Sometimes used interchangeably with \(\equiv\) for modular congruence},%
		{Analysis}/{Can denote \nxDQ{is approximately congruent} in some texts},%
		{Logic}/{Rarely used for structural equivalence},%
		{Extension}/{\(\cong\) bridges geometry congruence with algebraic congruence},%
		{Visual}/{Glyph resembles equality with a tilde, staging sameness plus shape relation}%
	}
\end{NxLightListBox}

\nxSections{Limit Symbol (lim)}{3}

\begin{NxLightListBox}[title={Definition}]
	\nxEachLabel{ArrowDark}{Secondary}{{4}{5}}{%
		{Glyph}/{\texttt{lim} denotes the limit of a function or sequence},%
		{Meaning}/{Describes the value a function approaches as the input approaches some point},%
		{Notation}/{\(\lim_{x \to c} f(x)\)},%
		{Example}/{\(\lim_{x \to 0} \frac{\sin x}{x} = 1\)},%
		{Connection}/{Core concept in calculus, analysis, and continuity}%
	}
\end{NxLightListBox}

\begin{NxLightListBox}[title={Types of Limits}]
	\nxEachLabel{ArrowDark}{Secondary}{{4}{5}}{%
		{Finite Limit}/{Function approaches a finite value as input approaches a point},%
		{Infinite Limit}/{Function grows without bound: \(\lim_{x \to 0^+} \frac{1}{x} = \infty\)},%
		{One-Sided Limit}/{\(\lim_{x \to c^-} f(x)\) or \(\lim_{x \to c^+} f(x)\)},%
		{At Infinity}/{\(\lim_{x \to \infty} f(x)\)},%
		{Sequence Limit}/{\(\lim_{n \to \infty} a_n\)}%
	}
\end{NxLightListBox}

\begin{NxLightListBox}[title={Formal Definition}]
	\nxEachLabel{ArrowDark}{Secondary}{{4}{5}}{%
		{\(\varepsilon\)-\(\delta\) Definition}/{For every \(\varepsilon > 0\), there exists \(\delta > 0\)},%
		{Condition}/{If \(|x-c| < \delta\), then \(|f(x)-L| < \varepsilon\)},%
		{Meaning}/{\(f(x)\) gets arbitrarily close to \(L\) as \(x\) approaches \(c\)},%
		{Connection}/{Defines continuity and rigor in calculus},%
		{Example}/{\(\lim_{x \to 2} (3x+1) = 7\)}%
	}
\end{NxLightListBox}


\nxSections{Meaning of \nxDQ{to}}{3}

\begin{NxLightListBox}[title={Limits}]
	\nxEachLabel{ArrowDark}{Secondary}{{4}{5}}{%
		{Usage}/{Appears in limit notation: \(\lim_{x \to c} f(x)\)},%
		{Meaning}/{\nxDQ{to} means the variable approaches a value},%
		{Example}/{\(\lim_{x \to 0} \frac{\sin x}{x} = 1\)},%
		{Connection}/{Glyph of approach, not exact arrival},%
		{Extension}/{Also used in one‑sided limits: \(x \to c^+\), \(x \to c^-\)}%
	}
\end{NxLightListBox}

\begin{NxLightListBox}[title={Mappings}]
	\nxEachLabel{ArrowDark}{Secondary}{{4}{5}}{%
		{Usage}/{Appears in function notation: \(f: A \to B\)},%
		{Meaning}/{\nxDQ{to} denotes mapping from domain to codomain},%
		{Example}/{\(f: \mathbb{R} \to \mathbb{R}, f(x)=x^2\)},%
		{Connection}/{Glyph of transformation, linking sets},%
		{Extension}/{Used in category theory: arrows \(A \to B\)}%
	}
\end{NxLightListBox}

\begin{NxLightListBox}[title={Ranges and Intervals}]
	\nxEachLabel{ArrowDark}{Secondary}{{4}{5}}{%
		{Usage}/{Appears in describing ranges: \nxDQ{from \(\ldots\) to \(\ldots\)}},%
		{Meaning}/{Marks boundaries of intervals or sums},%
		{Example}/{\(\sum_{i=1}^{n} a_i\) means i runs from 1 to n},%
		{Connection}/{Glyph of span, marking start and end},%
		{Extension}/{Used in integrals: \(\int_a^b f(x)\,dx\)}%
	}
\end{NxLightListBox}

\nxSections{Summation Symbol}{3}

\begin{NxLightListBox}[title={Definition}]
	\nxEachLabel{ArrowDark}{Secondary}{{4}{5}}{%
		{Glyph}/{\(\sum\) is produced in LaTeX with \texttt{\textbackslash sum}},%
		{Meaning}/{Denotes summation of a sequence of terms},%
		{Connection}/{Symbol of accumulation, discrete counterpart to integration},%
		{Visual}/{Large sigma glyph evokes totality}%
	}
\end{NxLightListBox}

\begin{NxLightListBox}[title={Basic Usage}]
	\nxEachLabel{ArrowDark}{Secondary}{{4}{5}}{%
		{Syntax}/{\texttt{\textbackslash sum\_\{i=1\}\^n a\_i}},%
		{Example}/{\(\sum_{i=1}^n a_i\)},%
		{Inline}/{Appears small in text mode},%
		{Display}/{Expands with limits above/below in display math},%
		{Connection}/{Bridges discrete series with continuous integrals}%
	}
\end{NxLightListBox}

\begin{NxLightListBox}[title={Properties}]
	\nxEachLabel{ArrowDark}{Secondary}{{4}{5}}{%
		{Linearity}/{\(\sum (a_i+b_i) = \sum a_i + \sum b_i\)},%
		{Index Shift}/{\(\sum_{i=0}^{n} a_i = \sum_{j=1}^{n+1} a_{j-1}\)},%
		{Empty Sum}/{Defined as 0},%
		{Connection}/{Properties mirror integral laws},%
		{Extension}/{Generalizes to multiple indices and infinite series}%
	}
\end{NxLightListBox}

\begin{NxLightListBox}[title={Examples}]
	\nxEachLabel{ArrowDark}{Secondary}{{4}{5}}{%
		{Arithmetic Series}/{\(\sum_{i=1}^n i = \frac{n(n+1)}{2}\)},%
		{Geometric Series}/{\(\sum_{i=0}^{n} r^i = \frac{1-r^{n+1}}{1-r}\)},%
		{Infinite Series}/{\(\sum_{n=1}^\infty \frac{1}{n^2} = \frac{\pi^2}{6}\)},%
		{Connection}/{Summation glyph stages discrete accumulation},%
		{Visual}/{Limits above/below emphasize range of accumulation}%
	}
\end{NxLightListBox}

\nxSections{Product Symbol}{3}

\begin{NxLightListBox}[title={Definition}]
	\nxEachLabel{ArrowDark}{Secondary}{{4}{5}}{%
		{Glyph}/{\(\prod\) is produced in LaTeX with \texttt{\textbackslash prod}},%
		{Meaning}/{Denotes the product of a sequence of terms},%
		{Connection}/{Discrete multiplicative counterpart to summation},%
		{Visual}/{Large Pi glyph evokes multiplication lineage}%
	}
\end{NxLightListBox}

\begin{NxLightListBox}[title={Basic Usage}]
	\nxEachLabel{ArrowDark}{Secondary}{{4}{5}}{%
		{Syntax}/{\texttt{\textbackslash prod\_\{i=1\}\^n a\_i}},%
		{Example}/{\(\prod_{i=1}^n a_i\)},%
		{Inline}/{Appears small in text mode},%
		{Display}/{Expands with limits above/below in display math},%
		{Connection}/{Bridges discrete multiplication with continuous exponentials}%
	}
\end{NxLightListBox}

\begin{NxLightListBox}[title={Properties}]
	\nxEachLabel{ArrowDark}{Secondary}{{4}{5}}{%
		{Empty Product}/{Defined as 1},%
		{Factorization}/{\(\prod a_i b_i = (\prod a_i)(\prod b_i)\)},%
		{Index Shift}/{\(\prod_{i=0}^{n} a_i = \prod_{j=1}^{n+1} a_{j-1}\)},%
		{Connection}/{Properties mirror exponential laws},%
		{Extension}/{Generalizes to multiple indices and infinite products}%
	}
\end{NxLightListBox}

\begin{NxLightListBox}[title={Examples}]
	\nxEachLabel{ArrowDark}{Secondary}{{4}{5}}{%
		{Factorial}/{\(\prod_{i=1}^n i = n!\)},%
		{Binomial}/{\(\prod_{i=1}^n (1+x_i)\)},%
		{Infinite Product}/{\(\prod_{n=1}^\infty \left(1-\frac{1}{p_n^2}\right)\) converges for primes \(p_n\)},%
		{Connection}/{Product glyph stages multiplicative accumulation},%
		{Visual}/{Limits above/below emphasize range of multiplication}%
	}
\end{NxLightListBox}

\nxSections{Coproduct Symbol}{3}

\begin{NxLightListBox}[title={Definition}]
	\nxEachLabel{ArrowDark}{Secondary}{{4}{5}}{%
		{Glyph}/{\(\coprod\) is produced in \LaTeX with \texttt{\textbackslash coprod}},%
		{Meaning}/{Denotes a categorical coproduct or disjoint union},%
		{Connection}/{Dual to the product symbol \(\prod\)},%
		{Visual}/{Large disjoint‑union glyph evokes branching lineage}%
	}
\end{NxLightListBox}

\begin{NxLightListBox}[title={Basic Usage}]
	\nxEachLabel{ArrowDark}{Secondary}{{4}{5}}{%
		{Syntax}/{\texttt{\textbackslash coprod\_\{i \in I\} A\_i}},%
		{Example}/{\(\coprod_{i \in I} A_i\)},%
		{Inline}/{Appears small in text mode},%
		{Display}/{Expands with limits above/below in display math},%
		{Connection}/{Stages categorical disjoint union across index sets}%
	}
\end{NxLightListBox}

\begin{NxLightListBox}[title={Properties}]
	\nxEachLabel{ArrowDark}{Secondary}{{4}{5}}{%
		{Duality}/{Coproduct is the categorical dual of product},%
		{Disjoint Union}/{In set theory, \(\coprod\) denotes union with tags},%
		{Direct Sum}/{In algebra, often used for direct sums of modules or vector spaces},%
		{Connection}/{Mirrors \(\prod\) but accumulates by union rather than multiplication},%
		{Extension}/{Generalizes to infinite coproducts in category theory}%
	}
\end{NxLightListBox}

\begin{NxLightListBox}[title={Examples}]
	\nxEachLabel{ArrowDark}{Secondary}{{4}{5}}{%
		{Sets}/{\(\coprod_{i=1}^n A_i\) is the disjoint union of sets \(A_i\)},%
		{Vector Spaces}/{\(\coprod_{i=1}^n V_i\) often written as \(\bigoplus_{i=1}^n V_i\)},%
		{Categories}/{Coproduct object satisfies universal property dual to product},%
		{Connection}/{Glyph stages branching accumulation across structures},%
		{Visual}/{Limits above/below emphasize index range of union}%
	}
\end{NxLightListBox}

\nxSections{Integral Symbol}{3}

\begin{NxLightListBox}[title={Definition}]
	\nxEachLabel{ArrowDark}{Secondary}{{4}{5}}{%
		{Glyph}/{\(\int\) is produced in LaTeX with \texttt{\textbackslash int}},%
		{Meaning}/{Denotes integration, the continuous counterpart to summation},%
		{Connection}/{Symbol of accumulation over a continuum},%
		{Visual}/{Elongated S glyph evokes summation lineage stretched across continuity}%
	}
\end{NxLightListBox}

\begin{NxLightListBox}[title={Basic Usage}]
	\nxEachLabel{ArrowDark}{Secondary}{{4}{5}}{%
		{Syntax}/{\texttt{\textbackslash int\_a\^b f(x)\,dx}},%
		{Example}/{\(\int_a^b f(x)\,dx\)},%
		{Indefinite}/{\(\int f(x)\,dx\) denotes antiderivative},%
		{Definite}/{\(\int_a^b f(x)\,dx\) accumulates area under curve},%
		{Connection}/{Bridges discrete sums with continuous accumulation}%
	}
\end{NxLightListBox}

\begin{NxLightListBox}[title={Properties}]
	\nxEachLabel{ArrowDark}{Secondary}{{4}{5}}{%
		{Linearity}/{\(\int (af(x)+bg(x))\,dx = a\int f(x)\,dx + b\int g(x)\,dx\)},%
		{Additivity}/{\(\int_a^c f(x)\,dx = \int_a^b f(x)\,dx + \int_b^c f(x)\,dx\)},%
		{Fundamental Theorem}/{\(\int_a^b f(x)\,dx = F(b)-F(a)\) if \(F'=f\)},%
		{Connection}/{Properties mirror summation laws in continuous domain},%
		{Extension}/{Generalizes to multiple integrals and contour integrals}%
	}
\end{NxLightListBox}

\begin{NxLightListBox}[title={Examples}]
	\nxEachLabel{ArrowDark}{Secondary}{{4}{5}}{%
		{Area}/{\(\int_0^1 x\,dx = \tfrac{1}{2}\)},%
		{Exponential}/{\(\int e^x\,dx = e^x + C\)},%
		{Trigonometric}/{\(\int_0^\pi \sin x\,dx = 2\)},%
		{Improper}/{\(\int_1^\infty \tfrac{1}{x^2}\,dx = 1\)},%
		{Connection}/{Integral glyph stages continuous accumulation across domains}%
	}
\end{NxLightListBox}

\nxSections{Contour Integral Symbol}{3}

\begin{NxLightListBox}[title={Definition}]
	\nxEachLabel{ArrowDark}{Secondary}{{4}{5}}{%
		{Glyph}/{\(\oint\) is produced in LaTeX with \texttt{\textbackslash oint}},%
		{Meaning}/{Denotes integration over a closed curve or contour},%
		{Connection}/{Specialized form of \(\int\) emphasizing closed paths},%
		{Visual}/{Integral sign with a circle evokes looping lineage}%
	}
\end{NxLightListBox}

\begin{NxLightListBox}[title={Basic Usage}]
	\nxEachLabel{ArrowDark}{Secondary}{{4}{5}}{%
		{Syntax}/{\texttt{\textbackslash oint\_C f(z)\,dz}},%
		{Example}/{\(\oint_C f(z)\,dz\)},%
		{Inline}/{Appears small in text mode},%
		{Display}/{Expands with limits above/below in display math},%
		{Connection}/{Stages accumulation along closed paths in complex plane}%
	}
\end{NxLightListBox}

\begin{NxLightListBox}[title={Properties}]
	\nxEachLabel{ArrowDark}{Secondary}{{4}{5}}{%
		{Cauchy Integral}/{\(\oint_C \frac{f(z)}{z-a}\,dz = 2\pi i f(a)\)},%
		{Residue Theorem}/{\(\oint_C f(z)\,dz = 2\pi i \sum \text{Residues}\)},%
		{Path Independence}/{Zero if integrand analytic inside contour},%
		{Connection}/{Properties mirror fundamental theorems of complex analysis},%
		{Extension}/{Generalizes to multiple closed integrals and homology cycles}%
	}
\end{NxLightListBox}

\begin{NxLightListBox}[title={Examples}]
	\nxEachLabel{ArrowDark}{Secondary}{{4}{5}}{%
		{Unit Circle}/{\(\oint_{|z|=1} \frac{dz}{z} = 2\pi i\)},%
		{Residue}/{\(\oint_C \frac{1}{z^2+1}\,dz = 2\pi i \cdot \tfrac{1}{2}\)},%
		{Physics}/{Closed path integrals in electromagnetism (Ampère’s law)},%
		{Connection}/{Glyph stages looping accumulation around singularities},%
		{Visual}/{Circle on integral sign mythifies closure of path}%
	}
\end{NxLightListBox}

\nxSections{Double Integral Symbol}{3}

\begin{NxLightListBox}[title={Definition}]
	\nxEachLabel{ArrowDark}{Secondary}{{4}{5}}{%
		{Glyph}/{\(\iint\) is produced in LaTeX with \texttt{\textbackslash iint}},%
		{Meaning}/{Denotes integration over a two‑dimensional region},%
		{Connection}/{Generalizes \(\int\) from 1D line segments to 2D areas},%
		{Visual}/{Two integral signs evoke accumulation across surfaces}%
	}
\end{NxLightListBox}

\begin{NxLightListBox}[title={Basic Usage}]
	\nxEachLabel{ArrowDark}{Secondary}{{4}{5}}{%
		{Syntax}/{\texttt{\textbackslash iint\_D f(x,y)\,dx\,dy}},%
		{Example}/{\(\iint_D f(x,y)\,dx\,dy\)},%
		{Region}/{Domain \(D\) specifies the area of integration},%
		{Display}/{Expands with limits above/below in display math},%
		{Connection}/{Stages continuous accumulation across two variables}%
	}
\end{NxLightListBox}

\begin{NxLightListBox}[title={Properties}]
	\nxEachLabel{ArrowDark}{Secondary}{{4}{5}}{%
		{Iterated Integrals}/{\(\iint_D f(x,y)\,dx\,dy = \int_a^b \int_c^d f(x,y)\,dy\,dx\)},%
		{Fubini’s Theorem}/{Allows evaluation as successive 1D integrals},%
		{Linearity}/{\(\iint (af+bg) = a\iint f + b\iint g\)},%
		{Connection}/{Properties mirror single integrals but extend to surfaces},%
		{Extension}/{Generalizes to triple integrals (\texttt{\textbackslash iiint}) and higher dimensions}%
	}
\end{NxLightListBox}

\begin{NxLightListBox}[title={Examples}]
	\nxEachLabel{ArrowDark}{Secondary}{{4}{5}}{%
		{Area}/{\(\iint_D 1\,dx\,dy\) gives area of region \(D\)},%
		{Volume}/{\(\iint_D f(x,y)\,dx\,dy\) underlies volume via surface integration},%
		{Polar Coordinates}/{\(\iint_D f(r,\theta)\,r\,dr\,d\theta\)},%
		{Physics}/{Used in flux calculations across surfaces},%
		{Connection}/{Glyph stages accumulation across 2D domains, bridging geometry and analysis}%
	}
\end{NxLightListBox}

\nxSections{Aleph-Null}{3}

\begin{NxLightListBox}[title={Definition}]
	\nxEachLabel{ArrowDark}{Secondary}{{4}{5}}{%
		{Glyph}/{\(\aleph_0\) is produced in LaTeX with \texttt{\textbackslash aleph\_0}},%
		{Meaning}/{Denotes the cardinality of the set of natural numbers},%
		{Connection}/{Smallest infinite cardinal, foundation of transfinite arithmetic},%
		{Visual}/{Hebrew letter Aleph with subscript zero evokes countable infinity}%
	}
\end{NxLightListBox}

\begin{NxLightListBox}[title={Properties}]
	\nxEachLabel{ArrowDark}{Secondary}{{4}{5}}{%
		{Countable Sets}/{Integers, rationals all have size \(\aleph_0\)},%
		{Closure}/{\(\aleph_0 + n = \aleph_0\), \(\aleph_0 \cdot n = \aleph_0\)},%
		{Comparison}/{Strictly smaller than continuum cardinality \(\mathfrak{c}\)},%
		{Connection}/{Stages hierarchy of infinities in Cantorian set theory},%
		{Extension}/{Generalizes to larger cardinals: \(\aleph_1, \aleph_2, \ldots\)}%
	}
\end{NxLightListBox}

\begin{NxLightListBox}[title={Examples}]
	\nxEachLabel{ArrowDark}{Secondary}{{4}{5}}{%
		{Naturals}/{\(|\mathbb{N}| = \aleph_0\)},%
		{Integers}/{\(|\mathbb{Z}| = \aleph_0\)},%
		{Rationals}/{\(|\mathbb{Q}| = \aleph_0\)},%
		{Connection}/{All countable infinite sets share cardinality \(\aleph_0\)},%
		{Visual}/{Glyph stages smallest rung of infinite ladder}%
	}
\end{NxLightListBox}


\nxSections{Omega Infinity}{3}

\begin{NxLightListBox}[title={Definition}]
	\nxEachLabel{ArrowDark}{Secondary}{{4}{5}}{%
		{Glyph}/{\(\Omega_\infty\) staged with \texttt{\textbackslash Omega\_\textbackslash infty}},%
		{Meaning}/{Used informally for absolute infinity or infinite asymptotic bound},%
		{Connection}/{Dual lineage: Cantorian philosophy vs algorithmic notation},%
		{Visual}/{Greek Omega with infinity evokes boundless closure}%
	}
\end{NxLightListBox}

\begin{NxLightListBox}[title={Cantorian Context}]
	\nxEachLabel{ArrowDark}{Secondary}{{4}{5}}{%
		{Absolute Infinite}/{Beyond all transfinite cardinals},%
		{Metaphysical}/{Cantor linked \(\Omega_\infty\) to divine infinity},%
		{Non-Cardinal}/{Not part of formal set hierarchy},%
		{Connection}/{Stages philosophical extension of infinity},%
		{Visual}/{Glyph evokes “beyond all” lineage}%
	}
\end{NxLightListBox}

\begin{NxLightListBox}[title={Algorithmic Context}]
	\nxEachLabel{ArrowDark}{Secondary}{{4}{5}}{%
		{Definition}/{\(\Omega^\infty\) means bound holds infinitely often},%
		{Contrast}/{Weaker than standard \(\Omega\) which holds eventually always},%
		{Usage}/{Appears in complexity analysis},%
		{Connection}/{Stages weaker asymptotic guarantee},%
		{Extension}/{Rare but useful in algorithmic lineage audits}%
	}
\end{NxLightListBox}

\nxSections{Intersection Symbol}{3}

\begin{NxLightListBox}[title={Definition}]
	\nxEachLabel{ArrowDark}{Secondary}{{4}{5}}{%
		{Glyph}/{\(\bigcap\) is produced in LaTeX with \texttt{\textbackslash bigcap}},%
		{Meaning}/{Denotes the intersection of a family of sets},%
		{Connection}/{Symbol of commonality, elements shared across sets},%
		{Visual}/{Large cap glyph evokes overlapping lineage}%
	}
\end{NxLightListBox}

\begin{NxLightListBox}[title={Basic Usage}]
	\nxEachLabel{ArrowDark}{Secondary}{{4}{5}}{%
		{Syntax}/{\texttt{\textbackslash bigcap\_\{i=1\}\^n A\_i}},%
		{Example}/{\(\bigcap_{i=1}^n A_i\)},%
		{Inline}/{Appears small in text mode},%
		{Display}/{Expands with limits above/below in display math},%
		{Connection}/{Stages accumulation of shared membership across sets}%
	}
\end{NxLightListBox}

\begin{NxLightListBox}[title={Properties}]
	\nxEachLabel{ArrowDark}{Secondary}{{4}{5}}{%
		{Commutative}/{Order of intersection doesn’t matter},%
		{Associative}/{Grouping doesn’t affect result},%
		{Idempotent}/{\(A \cap A = A\)},%
		{Connection}/{Properties mirror logical AND},%
		{Extension}/{Generalizes to infinite intersections}%
	}
\end{NxLightListBox}


\nxSections{Union Symbol}{3}

\begin{NxLightListBox}[title={Definition}]
	\nxEachLabel{ArrowDark}{Secondary}{{4}{5}}{%
		{Glyph}/{\(\bigcup\) is produced in LaTeX with \texttt{\textbackslash bigcup}},%
		{Meaning}/{Denotes the union of a family of sets},%
		{Connection}/{Symbol of totality, elements gathered across sets},%
		{Visual}/{Large cup glyph evokes merging lineage}%
	}
\end{NxLightListBox}

\begin{NxLightListBox}[title={Basic Usage}]
	\nxEachLabel{ArrowDark}{Secondary}{{4}{5}}{%
		{Syntax}/{\texttt{\textbackslash bigcup\_\{i=1\}\^n A\_i}},%
		{Example}/{\(\bigcup_{i=1}^n A_i\)},%
		{Inline}/{Appears small in text mode},%
		{Display}/{Expands with limits above/below in display math},%
		{Connection}/{Stages accumulation of all membership across sets}%
	}
\end{NxLightListBox}

\begin{NxLightListBox}[title={Properties}]
	\nxEachLabel{ArrowDark}{Secondary}{{4}{5}}{%
		{Commutative}/{Order of union doesn’t matter},%
		{Associative}/{Grouping doesn’t affect result},%
		{Idempotent}/{\(A \cup A = A\)},%
		{Connection}/{Properties mirror logical OR},%
		{Extension}/{Generalizes to infinite unions}%
	}
\end{NxLightListBox}

