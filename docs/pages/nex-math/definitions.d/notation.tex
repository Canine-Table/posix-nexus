\nxSections{Notation}{2}

\nxSections{Delta in Mathematics}{3}

\begin{NxLightListBox}[title={Finite Difference}]
	\nxEachLabel{ArrowDark}{Secondary}{{4}{5}}{%
		{Definition}/{Delta denotes change or difference between two values},%
		{Formula}/{\(\Delta x = x_{\text{final}} - x_{\text{initial}}\)},%
		{Usage}/{Discrete counterpart to derivative: \(\Delta y / \Delta x \approx dy/dx\)},%
		{Example}/{\(\Delta y = f(x_2) - f(x_1)\)},%
		{Extension}/{Forms the basis of finite difference methods in numerical analysis}%
	}
\end{NxLightListBox}

\begin{NxLightListBox}[title={Discriminant}]
	\nxEachLabel{ArrowDark}{Secondary}{{4}{5}}{%
		{Definition}/{In quadratic equations, \(\Delta\) denotes the discriminant},%
		{Formula}/{\(\Delta = b^2 - 4ac\)},%
		{Usage}/{Determines the nature of roots of \(ax^2+bx+c=0\)},%
		{Example}/{\(\Delta > 0\): two real roots; \(\Delta = 0\): one real root; \(\Delta < 0\): complex roots},%
		{Extension}/{Generalized discriminants exist for higher‑degree polynomials}%
	}
\end{NxLightListBox}

\begin{NxLightListBox}[title={Triangle Symbol}]
	\nxEachLabel{ArrowDark}{Secondary}{{4}{5}}{%
		{Definition}/{\(\Delta\) also denotes a triangle in geometry},%
		{Usage}/{\(\Delta ABC\) means triangle with vertices A, B, C},%
		{Connection}/{Links algebraic glyph with geometric figure lineage},%
		{Extension}/{Area of a triangle often denoted by \(\Delta\)}%
	}
\end{NxLightListBox}

\nxSections{Theta in Mathematics}{3}

\begin{NxLightListBox}[title={Angle Representation}]
	\nxEachLabel{ArrowDark}{Secondary}{{4}{5}}{%
		{Definition}/{Greek letter θ used to denote an angle},%
		{Trigonometry}/{Appears in sine, cosine, tangent: e.g. \(\sin(\theta), \cos(\theta), \tan(\theta)\)},%
		{Geometry}/{Used in right triangles to relate sides and angles},%
		{Polar Coordinates}/{Point \((r,\theta)\) defined by radius and angle},%
		{Example}/{In a right triangle, \(\cos(\theta) = \frac{\text{adjacent}}{\text{hypotenuse}}\)}%
	}
\end{NxLightListBox}

\begin{NxLightListBox}[title={Other Mathematical Uses}]
	\nxEachLabel{ArrowDark}{Secondary}{{4}{5}}{%
		{Statistics}/{\(\theta\) often denotes parameters in probability distributions},%
		{Complex Numbers}/{Angle of rotation in Euler’s formula: \(e^{i\theta} = \cos(\theta) + i\sin(\theta)\)},%
		{Calculus}/{Variable of integration in polar coordinates},%
		{Numerical Value}/{In Greek numerals, θ has value 9},%
		{Connection}/{Symbol of rotation, periodicity, and parameterization across math}%
	}
\end{NxLightListBox}

\nxSections{Epsilon in Mathematics}{3}

\begin{NxLightListBox}[title={Infinitesimal Bound}]
	\nxEachLabel{ArrowDark}{Secondary}{{4}{5}}{%
		{Definition}/{Greek letter ε used to denote a very small positive quantity},%
		{Limit Definition}/{Appears in \(\varepsilon\)-\(\delta\) proofs: for every \(\varepsilon > 0\), there exists \(\delta > 0\)},%
		{Usage}/{Measures closeness of a function to a limit},%
		{Example}/{\(|f(x)-L| < \varepsilon\) whenever \(|x-c| < \delta\)},%
		{Connection}/{Core of rigorous calculus and real analysis}%
	}
\end{NxLightListBox}

\begin{NxLightListBox}[title={Error and Approximation}]
	\nxEachLabel{ArrowDark}{Secondary}{{4}{5}}{%
		{Definition}/{ε often denotes error tolerance or margin},%
		{Numerical Analysis}/{Used to bound approximation error},%
		{Example}/{If \(|x - x_0| < \varepsilon\), then \(x\) is within tolerance},%
		{Connection}/{Links exact mathematics with numerical computation}%
	}
\end{NxLightListBox}

\begin{NxLightListBox}[title={Other Uses}]
	\nxEachLabel{ArrowDark}{Secondary}{{4}{5}}{%
		{Set Theory}/{ε sometimes used as membership symbol, though ∈ is standard},%
		{Complexity}/{ε denotes arbitrarily small constants in algorithm analysis},%
		{Probability}/{ε used in inequalities like Chebyshev’s or ε‑nets},%
		{Connection}/{Universal glyph for “smallness” across math disciplines}%
	}
\end{NxLightListBox}

\nxSections{Delta in Mathematics}{3}

\begin{NxLightListBox}[title={Infinitesimal Change}]
	\nxEachLabel{ArrowDark}{Secondary}{{4}{5}}{%
		{Definition}/{Greek letter δ denotes a very small positive quantity},%
		{Limit Proofs}/{Appears in \(\varepsilon\)-\(\delta\) definitions of limits},%
		{Formula}/{For every \(\varepsilon > 0\), there exists \(\delta > 0\)},%
		{Usage}/{Controls how close \(x\) must be to \(c\) for \(f(x)\) to be within \(\varepsilon\) of \(L\)},%
		{Example}/{If \(|x-c| < \delta\), then \(|f(x)-L| < \varepsilon\)},%
		{Connection}/{δ measures input closeness, ε measures output closeness}%
	}
\end{NxLightListBox}

\begin{NxLightListBox}[title={Variation and Error}]
	\nxEachLabel{ArrowDark}{Secondary}{{4}{5}}{%
		{Definition}/{δ often denotes small variation or tolerance},%
		{Numerical Analysis}/{Used to bound input error},%
		{Example}/{If \(|x-x_0| < \delta\), then \(x\) is within tolerance of \(x_0\)},%
		{Connection}/{Pairs with ε to formalize precision in analysis and computation}%
	}
\end{NxLightListBox}

\begin{NxLightListBox}[title={Other Mathematical Uses}]
	\nxEachLabel{ArrowDark}{Secondary}{{4}{5}}{%
		{Geometry}/{δ sometimes used for small angles},%
		{Statistics}/{δ may denote deviation or perturbation},%
		{Complexity}/{δ used for small constants in algorithm analysis},%
		{Connection}/{Universal glyph for “small input change” across math disciplines}%
	}
\end{NxLightListBox}


\nxSections{Proportional Symbol}{3}

\begin{NxLightListBox}[title={Definition}]
	\nxEachLabel{ArrowDark}{Secondary}{{4}{5}}{%
		{Glyph}/{∝ resembles the left half of ∞},%
		{Meaning}/{Denotes proportionality between two quantities},%
		{Formula}/{\(y \propto x\) means \(y = kx\) for some constant \(k\)},%
		{Usage}/{Used in algebra, physics, and statistics to show direct proportionality},%
		{Example}/{Gravitational force: \(F \propto \frac{1}{r^2}\)}%
	}
\end{NxLightListBox}

\begin{NxLightListBox}[title={Key Properties}]
	\nxEachLabel{ArrowDark}{Secondary}{{4}{5}}{%
		{Constant of Proportionality}/{Always exists: \(y = kx\)},%
		{Direct Proportionality}/{If one doubles, the other doubles},%
		{Inverse Proportionality}/{Written as \(y \propto \frac{1}{x}\)},%
		{Scaling}/{Proportionality preserves ratios},%
		{Connection}/{Symbol bridges ratios, scaling laws, and functional dependence}%
	}
\end{NxLightListBox}

\nxSections{Infinity in Mathematics}{3}

\begin{NxLightListBox}[title={Concept of Infinity}]
	\nxEachLabel{ArrowDark}{Secondary}{{4}{5}}{%
		{Definition}/{∞ denotes an unbounded quantity, larger than any real number},%
		{Calculus}/{Appears in limits: \(\lim_{x \to \infty} f(x)\)},%
		{Set Theory}/{Represents cardinalities of infinite sets},%
		{Geometry}/{Used to mark points at infinity in projective geometry},%
		{Connection}/{Symbol of endlessness, beyond finite measurement}%
	}
\end{NxLightListBox}

\begin{NxLightListBox}[title={Infinity in Calculus}]
	\nxEachLabel{ArrowDark}{Secondary}{{4}{5}}{%
		{Improper Integrals}/{\(\int_1^\infty \frac{1}{x^2} dx\)},%
		{Limits}/{\(\lim_{x \to \infty} \frac{1}{x} = 0\)},%
		{Series}/{Infinite sums: \(\sum_{n=1}^\infty \frac{1}{n^2}\)},%
		{Connection}/{∞ marks the boundary of convergence and divergence},%
		{Example}/{Harmonic series diverges: \(\sum_{n=1}^\infty \frac{1}{n}\)}%
	}
\end{NxLightListBox}

\begin{NxLightListBox}[title={Infinity in Set Theory}]
	\nxEachLabel{ArrowDark}{Secondary}{{4}{5}}{%
		{Countable Infinity}/{Size of natural numbers, denoted \(\aleph_0\)},%
		{Uncountable Infinity}/{Size of real numbers, larger than \(\aleph_0\)},%
		{Comparison}/{Not all infinities are equal},%
		{Connection}/{∞ as a concept differs from cardinal numbers},%
		{Example}/{\(|\mathbb{N}| = \aleph_0\), but \(|\mathbb{R}| > \aleph_0\)}%
	}
\end{NxLightListBox}

\nxSections{Perpendicular Symbol}{3}

\begin{NxLightListBox}[title={Definition}]
	\nxEachLabel{ArrowDark}{Secondary}{{4}{5}}{%
		{Glyph}/{⊥ is the mathematical symbol for perpendicularity},%
		{Meaning}/{Two lines, segments, or planes meet at a right angle (90°)},%
		{Notation}/{\(AB \perp CD\) means line AB is perpendicular to line CD},%
		{Geometry}/{Used to denote orthogonality in Euclidean space},%
		{Example}/{In a square, adjacent sides are ⊥ to each other}%
	}
\end{NxLightListBox}

\begin{NxLightListBox}[title={Linear Algebra Connection}]
	\nxEachLabel{ArrowDark}{Secondary}{{4}{5}}{%
		{Orthogonality}/{⊥ denotes vectors with dot product zero},%
		{Formula}/{\(\vec{u} \perp \vec{v} \iff \vec{u}\cdot\vec{v} = 0\)},%
		{Usage}/{Defines orthogonal bases and projections},%
		{Example}/{\((1,0)\) ⊥ \((0,1)\) in \(\mathbb{R}^2\)},%
		{Extension}/{Orthogonality generalizes perpendicularity to higher dimensions}%
	}
\end{NxLightListBox}

\begin{NxLightListBox}[title={Other Uses}]
	\nxEachLabel{ArrowDark}{Secondary}{{4}{5}}{%
		{Logic}/{⊥ sometimes denotes contradiction or falsity},%
		{Probability}/{⊥ used to denote independence in some texts},%
		{Connection}/{Symbol bridges geometry, algebra, and logic},%
		{Visual}/{Always evokes the right‑angle lineage}%
	}
\end{NxLightListBox}

\nxSections{Plus-Minus and Minus-Plus}{3}

\begin{NxLightListBox}[title={Plus-Minus (±)}]
	\nxEachLabel{ArrowDark}{Secondary}{{4}{5}}{%
		{Definition}/{Symbol ± means “plus or minus”},%
		{Usage}/{Represents two possible values: \(a+b\) or \(a-b\)},%
		{Example}/{Quadratic formula: \(x = \frac{-b \pm \sqrt{b^2-4ac}}{2a}\)},%
		{Connection}/{Encodes duality in solutions, symmetry in expansions},%
		{Extension}/{Used in error bounds and approximations: \(x \pm \varepsilon\)}%
	}
\end{NxLightListBox}

\begin{NxLightListBox}[title={Minus-Plus (∓)}]
	\nxEachLabel{ArrowDark}{Secondary}{{4}{5}}{%
		{Definition}/{Symbol ∓ means “minus or plus,” paired with ±},%
		{Usage}/{Ensures opposite choice when ± is used earlier},%
		{Example}/{If first term is \(+\), second term takes \(−\); if first is \(−\), second takes \(+\)},%
		{Connection}/{Keeps expressions consistent in paired signs},%
		{Extension}/{Common in trigonometric identities and vector formulas}%
	}
\end{NxLightListBox}

\begin{NxLightListBox}[title={Combined Expression}]
	\nxEachLabel{ArrowDark}{Secondary}{{4}{5}}{%
		{Notation}/{\(a \pm b \mp c\)},%
		{Meaning}/{Two cases: \(a+b-c\) or \(a-b+c\)},%
		{Pattern}/{± and ∓ always paired to flip signs consistently},%
		{Example}/{In trig: \(\sin(x \pm y) = \sin x \cos y \pm \cos x \sin y\)},%
		{Connection}/{Encodes dual solutions in compact symbolic form}%
	}
\end{NxLightListBox}


\nxSections{Parallel Symbol}{3}

\begin{NxLightListBox}[title={Definition}]
	\nxEachLabel{ArrowDark}{Secondary}{{4}{5}}{%
		{Glyph}/{∥ is the mathematical symbol for parallelism},%
		{Meaning}/{Two lines, segments, or planes never intersect and remain equidistant},%
		{Notation}/{\(AB \parallel CD\) means line AB is parallel to line CD},%
		{Geometry}/{Used in Euclidean geometry to denote parallel lines and planes},%
		{Example}/{In a rectangle, opposite sides are ∥ to each other}%
	}
\end{NxLightListBox}

\begin{NxLightListBox}[title={Linear Algebra Connection}]
	\nxEachLabel{ArrowDark}{Secondary}{{4}{5}}{%
		{Vectors}/{∥ denotes vectors that are scalar multiples of each other},%
		{Formula}/{\(\vec{u} \parallel \vec{v} \iff \vec{u} = k\vec{v}\)},%
		{Usage}/{Defines direction equivalence in vector spaces},%
		{Example}/{\((2,4)\) ∥ \((1,2)\) since \((2,4) = 2(1,2)\)},%
		{Extension}/{Parallelism generalizes beyond geometry into linear algebra and physics}%
	}
\end{NxLightListBox}

\begin{NxLightListBox}[title={Other Mathematical Uses}]
	\nxEachLabel{ArrowDark}{Secondary}{{4}{5}}{%
		{Analysis}/{∥x∥ sometimes denotes norm of a vector},%
		{Logic}/{∥ used in some texts for “parallel execution” or independence},%
		{Connection}/{Symbol bridges geometry, algebra, and analysis},%
		{Visual}/{Always evokes equidistant, non‑intersecting lineage}%
	}
\end{NxLightListBox}

\nxSections{Angle Symbol}{3}

\begin{NxLightListBox}[title={Definition}]
	\nxEachLabel{ArrowDark}{Secondary}{{4}{5}}{%
		{Glyph}/{∠ is the mathematical symbol for an angle},%
		{Meaning}/{Represents the measure of rotation between two intersecting lines or rays},%
		{Notation}/{\(\angle ABC\) means the angle formed at vertex B by rays BA and BC},%
		{Units}/{Measured in degrees (°) or radians},%
		{Example}/{\(\angle ABC = 90^\circ\) denotes a right angle}%
	}
\end{NxLightListBox}

\begin{NxLightListBox}[title={Geometry Usage}]
	\nxEachLabel{ArrowDark}{Secondary}{{4}{5}}{%
		{Right Angle}/{\(\angle = 90^\circ\)},%
		{Acute Angle}/{\(\angle < 90^\circ\)},%
		{Obtuse Angle}/{\(90^\circ < \angle < 180^\circ\)},%
		{Straight Angle}/{\(\angle = 180^\circ\)},%
		{Reflex Angle}/{\(180^\circ < \angle < 360^\circ\)}%
	}
\end{NxLightListBox}

\begin{NxLightListBox}[title={Other Mathematical Uses}]
	\nxEachLabel{ArrowDark}{Secondary}{{4}{5}}{%
		{Trigonometry}/{∠ used inside sine, cosine, tangent functions},%
		{Polar Coordinates}/{Point \((r,\theta)\) defined by radius and angle},%
		{Complex Numbers}/{Angle defines argument of a complex number},%
		{Vector Analysis}/{Angle between vectors via dot product},%
		{Connection}/{Symbol bridges geometry, trigonometry, and analysis}%
	}
\end{NxLightListBox}

\nxSections{Similar vs Equivalent}{3}

\begin{NxLightListBox}[title={Similar Symbol (\sim)}]
	\nxEachLabel{ArrowDark}{Secondary}{{4}{5}}{%
		{Glyph}/{\sim is the symbol for similarity},%
		{Geometry}/{\(\triangle ABC \sim \triangle DEF\) means triangles have equal angles and proportional sides},%
		{Algebra}/{Sometimes used to denote asymptotic equivalence: \(f(x) \sim g(x)\) as \(x \to \infty\)},%
		{Pattern}/{Similarity preserves shape but not necessarily size},%
		{Example}/{\(\triangle ABC \sim \triangle DEF\) if \(\angle A = \angle D, \angle B = \angle E, \angle C = \angle F\)},%
		{Connection}/{Symbol of proportionality in geometry and asymptotics in analysis}%
	}
\end{NxLightListBox}

\begin{NxLightListBox}[title={Equivalent Symbol (≡)}]
	\nxEachLabel{ArrowDark}{Secondary}{{4}{5}}{%
		{Glyph}/{≡ is the symbol for equivalence},%
		{Congruence}/{\(\triangle ABC \equiv \triangle DEF\) means triangles are identical in shape and size},%
		{Number Theory}/{Used for modular congruence: \(a \equiv b \pmod{n}\)},%
		{Logic}/{Denotes logical equivalence: \(P \equiv Q\)},%
		{Pattern}/{Equivalence preserves both shape and size, or exact relation},%
		{Example}/{\(17 \equiv 5 \pmod{12}\)},%
		{Connection}/{Symbol of exact sameness across geometry, algebra, and logic}%
	}
\end{NxLightListBox}

\nxSections{Approximate Symbol}{3}

\begin{NxLightListBox}[title={Definition}]
	\nxEachLabel{ArrowDark}{Secondary}{{4}{5}}{%
		{Glyph}/{≈ is the mathematical symbol for approximation},%
		{Meaning}/{Indicates two values are close but not exactly equal},%
		{Notation}/{\(a \approx b\) means \(a\) is approximately equal to \(b\)},%
		{Usage}/{Common in numerical analysis, applied math, and physics},%
		{Example}/{\(\pi \approx 3.1416\)}%
	}
\end{NxLightListBox}

\begin{NxLightListBox}[title={Contexts of Use}]
	\nxEachLabel{ArrowDark}{Secondary}{{4}{5}}{%
		{Numerical}/{Used when rounding or truncating decimals},%
		{Physics}/{Marks measured values close to theoretical ones},%
		{Statistics}/{Denotes approximate probabilities or estimates},%
		{Analysis}/{Signals asymptotic closeness in limits},%
		{Connection}/{Symbol bridges exact math with practical computation}%
	}
\end{NxLightListBox}

\begin{NxLightListBox}[title={Related Symbols}]
	\nxEachLabel{ArrowDark}{Secondary}{{4}{5}}{%
		{Equal Sign}/{\(=\) denotes exact equality},%
		{Tilde}/{\(\sim\) denotes similarity or asymptotic equivalence},%
		{Congruence}/{\(\equiv\) denotes exact equivalence or modular congruence},%
		{Approx}/{\approx specifically signals numerical closeness},%
		{Extension}/{Each glyph stages a different level of sameness}%
	}
\end{NxLightListBox}

\nxSections{Similar or Equal Symbol}{3}

\begin{NxLightListBox}[title={Definition}]
	\nxEachLabel{ArrowDark}{Secondary}{{4}{5}}{%
		{Glyph}/{\simeq is the symbol for “similar or equal”},%
		{Meaning}/{Indicates two quantities are nearly equal and share structural similarity},%
		{Usage}/{Common in analysis, approximation, and asymptotic notation},%
		{Example}/{\(f(x) \simeq g(x)\) means functions are close in value and form},%
		{Connection}/{Bridges similarity (\sim) and equality (=)}%
	}
\end{NxLightListBox}

\begin{NxLightListBox}[title={Contexts of Use}]
	\nxEachLabel{ArrowDark}{Secondary}{{4}{5}}{%
		{Approximation}/{Used when values are not exactly equal but very close},%
		{Asymptotics}/{Signals functions behave similarly as \(x \to \infty\)},%
		{Geometry}/{Sometimes used to denote figures nearly congruent},%
		{Physics}/{Marks quantities equal within experimental tolerance},%
		{Extension}/{\simeq is less strict than \equiv (equivalent) but stronger than \approx (approximate)}%
	}
\end{NxLightListBox}

\nxSections{Congruent Symbol}{3}

\begin{NxLightListBox}[title={Definition}]
	\nxEachLabel{ArrowDark}{Secondary}{{4}{5}}{%
		{Glyph}/{≅ is produced in LaTeX with \texttt{\textbackslash cong}},%
		{Meaning}/{Denotes congruence: “is congruent to”},%
		{Geometry}/{\(\triangle ABC \cong \triangle DEF\) means triangles are identical in shape and size},%
		{Pattern}/{Congruence preserves both angles and side lengths},%
		{Example}/{\(\triangle\) with sides 3,4,5 is \(\cong\) to another 3,4,5 triangle},%
		{Connection}/{Stronger than similarity (∼), exact match without scaling}%
	}
\end{NxLightListBox}

\begin{NxLightListBox}[title={Other Mathematical Uses}]
	\nxEachLabel{ArrowDark}{Secondary}{{4}{5}}{%
		{Number Theory}/{Sometimes used interchangeably with ≡ for modular congruence},%
		{Analysis}/{Can denote “is approximately congruent” in some texts},%
		{Logic}/{Rarely used for structural equivalence},%
		{Extension}/{≅ bridges geometry congruence with algebraic congruence},%
		{Visual}/{Glyph resembles equality with a tilde, staging sameness plus shape relation}%
	}
\end{NxLightListBox}

\nxSections{Limit Symbol (lim)}{3}

\begin{NxLightListBox}[title={Definition}]
	\nxEachLabel{ArrowDark}{Secondary}{{4}{5}}{%
		{Glyph}/{\texttt{lim} denotes the limit of a function or sequence},%
		{Meaning}/{Describes the value a function approaches as the input approaches some point},%
		{Notation}/{\(\lim_{x \to c} f(x)\)},%
		{Example}/{\(\lim_{x \to 0} \frac{\sin x}{x} = 1\)},%
		{Connection}/{Core concept in calculus, analysis, and continuity}%
	}
\end{NxLightListBox}

\begin{NxLightListBox}[title={Types of Limits}]
	\nxEachLabel{ArrowDark}{Secondary}{{4}{5}}{%
		{Finite Limit}/{Function approaches a finite value as input approaches a point},%
		{Infinite Limit}/{Function grows without bound: \(\lim_{x \to 0^+} \frac{1}{x} = \infty\)},%
		{One-Sided Limit}/{\(\lim_{x \to c^-} f(x)\) or \(\lim_{x \to c^+} f(x)\)},%
		{At Infinity}/{\(\lim_{x \to \infty} f(x)\)},%
		{Sequence Limit}/{\(\lim_{n \to \infty} a_n\)}%
	}
\end{NxLightListBox}

\begin{NxLightListBox}[title={Formal Definition}]
	\nxEachLabel{ArrowDark}{Secondary}{{4}{5}}{%
		{ε-δ Definition}/{For every \(\varepsilon > 0\), there exists \(\delta > 0\)},%
		{Condition}/{If \(|x-c| < \delta\), then \(|f(x)-L| < \varepsilon\)},%
		{Meaning}/{\(f(x)\) gets arbitrarily close to \(L\) as \(x\) approaches \(c\)},%
		{Connection}/{Defines continuity and rigor in calculus},%
		{Example}/{\(\lim_{x \to 2} (3x+1) = 7\)}%
	}
\end{NxLightListBox}


\nxSections{Meaning of "to"}{3}

\begin{NxLightListBox}[title={Limits}]
	\nxEachLabel{ArrowDark}{Secondary}{{4}{5}}{%
		{Usage}/{Appears in limit notation: \(\lim_{x \to c} f(x)\)},%
		{Meaning}/{“to” means the variable approaches a value},%
		{Example}/{\(\lim_{x \to 0} \frac{\sin x}{x} = 1\)},%
		{Connection}/{Glyph of approach, not exact arrival},%
		{Extension}/{Also used in one‑sided limits: \(x \to c^+\), \(x \to c^-\)}%
	}
\end{NxLightListBox}

\begin{NxLightListBox}[title={Mappings}]
	\nxEachLabel{ArrowDark}{Secondary}{{4}{5}}{%
		{Usage}/{Appears in function notation: \(f: A \to B\)},%
		{Meaning}/{“to” denotes mapping from domain to codomain},%
		{Example}/{\(f: \mathbb{R} \to \mathbb{R}, f(x)=x^2\)},%
		{Connection}/{Glyph of transformation, linking sets},%
		{Extension}/{Used in category theory: arrows \(A \to B\)}%
	}
\end{NxLightListBox}

\begin{NxLightListBox}[title={Ranges and Intervals}]
	\nxEachLabel{ArrowDark}{Secondary}{{4}{5}}{%
		{Usage}/{Appears in describing ranges: “from … to …”},%
		{Meaning}/{Marks boundaries of intervals or sums},%
		{Example}/{\(\sum_{i=1}^{n} a_i\) means i runs from 1 to n},%
		{Connection}/{Glyph of span, marking start and end},%
		{Extension}/{Used in integrals: \(\int_a^b f(x)\,dx\)}%
	}
\end{NxLightListBox}

