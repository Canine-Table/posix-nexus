\nxSections{Pairs}{2}

\begin{NxLightBox}[title={Factorial (n!)}]
    \begingroup
    \renewcommand{\arraystretch}{1.8}
    \begin{tabularx}{\linewidth}{|c|X|}
        \hline
        \textbf{Concept} & Factorial (\(n!\)) \\
        \hline
        \textbf{Definition} & The product of all positive integers up to \(n\). Defined as \(n! = n \times (n-1) \times \ldots \times 1\). \\
        \hline
        \textbf{Core Idea} & Factorial counts permutations and combinations — it grows extremely fast. \\
        \hline
        \textbf{Example} & \(5! = 120\). \\
        \hline
        \textbf{Applications} & Used in combinatorics, probability, and series expansions. \\
        \hline
        \textbf{Pair} & Inverse gamma function (not elementary). \\
        \hline
    \end{tabularx}
    \endgroup
\end{NxLightBox}

\begin{NxLightBox}[title={Logarithm Base 2 (\(\log_{2}\))}]
    \begingroup
    \renewcommand{\arraystretch}{1.8}
    \begin{tabularx}{\linewidth}{|c|X|}
        \hline
        \textbf{Concept} & Logarithm Base 2 (\(\log_{2}(x)\)) \\
        \hline
        \textbf{Definition} & The inverse of the power of 2. Defined as the exponent \(y\) such that \(2^y = x\). \\
        \hline
        \textbf{Core Idea} & \(\log_{2}(x)\) measures how many times you multiply 2 to reach \(x\). \\
        \hline
        \textbf{Example} & \(\log_{2}(8) = 3\). \\
        \hline
        \textbf{Applications} & Widely used in computer science, information theory, and binary systems. \\
        \hline
        \textbf{Pair} & Power of 2 function (\(2^x\)). \\
        \hline
    \end{tabularx}
    \endgroup
\end{NxLightBox}

\begin{NxLightBox}[title={Natural Logarithm (\ln)}]
    \begingroup
    \renewcommand{\arraystretch}{1.8}
    \begin{tabularx}{\linewidth}{|c|X|}
        \hline
        \textbf{Concept} & Natural Logarithm (\(\ln(x)\)) \\
        \hline
        \textbf{Definition} & The inverse of the exponential function. Defined as the power to which \(e\) must be raised to equal \(x\). \\
        \hline
        \textbf{Core Idea} & \(\ln(x)\) undoes exponentiation with base \(e\). \\
        \hline
        \textbf{Example} & \(\ln(e^3) = 3\). \\
        \hline
        \textbf{Applications} & Used in calculus, growth/decay models, and solving exponential equations. \\
        \hline
        \textbf{Pair} & Exponential function (\(e^x\)). \\
        \hline
    \end{tabularx}
    \endgroup
\end{NxLightBox}

\begin{NxLightBox}[title={Logarithm Base 2 (\log_{2})}]
    \begingroup
    \renewcommand{\arraystretch}{1.8}
    \begin{tabularx}{\linewidth}{|c|X|}
        \hline
        \textbf{Concept} & Logarithm Base 2 (\(\log_{2}(x)\)) \\
        \hline
        \textbf{Definition} & The inverse of the power of 2. Defined as the exponent \(y\) such that \(2^y = x\). \\
        \hline
        \textbf{Core Idea} & \(\log_{2}(x)\) measures how many times you multiply 2 to reach \(x\). \\
        \hline
        \textbf{Example} & \(\log_{2}(8) = 3\). \\
        \hline
        \textbf{Applications} & Widely used in computer science, information theory, and binary systems. \\
        \hline
        \textbf{Pair} & Power of 2 function (\(2^x\)). \\
        \hline
    \end{tabularx}
    \endgroup
\end{NxLightBox}

\begin{NxLightBox}[title={Natural Logarithm (\ln)}]
    \begingroup
    \renewcommand{\arraystretch}{1.8}
    \begin{tabularx}{\linewidth}{|c|X|}
        \hline
        \textbf{Concept} & Natural Logarithm (\(\ln(x)\)) \\
        \hline
        \textbf{Definition} & The inverse of the exponential function. Defined as the power to which \(e\) must be raised to equal \(x\). \\
        \hline
        \textbf{Core Idea} & \(\ln(x)\) undoes exponentiation with base \(e\). \\
        \hline
        \textbf{Example} & \(\ln(e^3) = 3\). \\
        \hline
        \textbf{Applications} & Used in calculus, growth/decay models, and solving exponential equations. \\
        \hline
        \textbf{Pair} & Exponential function (\(e^x\)). \\
        \hline
    \end{tabularx}
    \endgroup
\end{NxLightBox}

\begin{NxLightBox}[title={Tangent and Cotangent}]
    \begingroup
    \renewcommand{\arraystretch}{1.8}
    \begin{tabularx}{\linewidth}{|c|X|}
        \hline
        \textbf{Concept} & Tangent (\(\tan(x)\)) and Cotangent (\(\cot(x)\)) \\
        \hline
        \textbf{Definition} & \(\tan(x) = \frac{\sin(x)}{\cos(x)}\), while \(\cot(x) = \frac{\cos(x)}{\sin(x)}\). They are reciprocals: \(\cot(x) = \frac{1}{\tan(x)}\). \\
        \hline
        \textbf{Core Idea} & Tangent measures slope (rise/run). Cotangent flips that slope (run/rise). \\
        \hline
        \textbf{Example} & At \(45^\circ\), \(\tan(45^\circ) = 1\) and \(\cot(45^\circ) = 1\). \\
        \hline
        \textbf{Applications} & Used in trigonometry, calculus, and geometry — especially for slope and angle analysis. \\
        \hline
        \textbf{Pair} & Reciprocal functions: \(\tan(x) \leftrightarrow \cot(x)\). \\
        \hline
    \end{tabularx}
    \endgroup
\end{NxLightBox}

\begin{NxLightBox}[title={Factorial (n!)}]
    \begingroup
    \renewcommand{\arraystretch}{1.8}
    \begin{tabularx}{\linewidth}{|c|X|}
        \hline
        \textbf{Concept} & Factorial (\(n!\)) \\
        \hline
        \textbf{Definition} & The product of all positive integers up to \(n\). Defined as \(n! = n \times (n-1) \times \ldots \times 1\). \\
        \hline
        \textbf{Core Idea} & Factorial counts permutations and combinations — it grows extremely fast. \\
        \hline
        \textbf{Example} & \(5! = 120\). \\
        \hline
        \textbf{Applications} & Used in combinatorics, probability, and series expansions. \\
        \hline
        \textbf{Pair} & Inverse gamma function (not elementary). \\
        \hline
    \end{tabularx}
    \endgroup
\end{NxLightBox}

\begin{NxLightBox}[title={Factorial (n!)}]
    \begingroup
    \renewcommand{\arraystretch}{1.8}
    \begin{tabularx}{\linewidth}{|c|X|}
        \hline
        \textbf{Concept} & Factorial (\(n!\)) \\
        \hline
        \textbf{Definition} & The product of all positive integers up to \(n\). Defined as \(n! = n \times (n-1) \times \ldots \times 1\). \\
        \hline
        \textbf{Core Idea} & Factorial counts permutations and combinations — it grows extremely fast. \\
        \hline
        \textbf{Example} & \(5! = 120\). \\
        \hline
        \textbf{Applications} & Used in combinatorics, probability, and series expansions. \\
        \hline
        \textbf{Pair} & Inverse gamma function (not elementary). \\
        \hline
    \end{tabularx}
    \endgroup
\end{NxLightBox}


