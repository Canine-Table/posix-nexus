\nxSections{Keys}{2}
\begin{comment}
\nxSections{PRIMARY KEY}{3}

\begin{NxCodeBox}{sql}{title={PRIMARY KEY — Syntax Across Engines}}
-- PostgreSQL
CREATE TABLE users (
		id SERIAL PRIMARY KEY,
		username TEXT UNIQUE NOT NULL
);

-- MariaDB / MySQL
CREATE TABLE users (
		id INT AUTO_INCREMENT PRIMARY KEY,
		username VARCHAR(255) UNIQUE NOT NULL
);

-- SQLite3
CREATE TABLE users (
		id INTEGER PRIMARY KEY AUTOINCREMENT,
		username TEXT UNIQUE NOT NULL
);
\end{NxCodeBox}

\nxSections{Named Foreign Key Constraint}{3}
\begin{NxCodeBox}{sql}{title={CONSTRAINT fk\_name FOREIGN KEY — Syntax Across Engines}}
-- PostgreSQL
CREATE TABLE passports (
		user_id INT UNIQUE NOT NULL,
		CONSTRAINT fk_user FOREIGN KEY (user_id) REFERENCES users(id)
);

-- MariaDB / MySQL
CREATE TABLE passports (
		user_id INT UNIQUE NOT NULL,
		CONSTRAINT fk_user FOREIGN KEY (user_id) REFERENCES users(id)
);

-- SQLite3
PRAGMA foreign_keys = ON;

CREATE TABLE passports (
		user_id INT UNIQUE NOT NULL,
		CONSTRAINT fk_user FOREIGN KEY (user_id) REFERENCES users(id)
);
\end{NxCodeBox}

\begin{NxLightListBox}[title={Named Foreign Key Constraint — The Ritual of Identification}]
		\nxEachLabel{ArrowDark}{Secondary}{{1}{43}}{%
				{Purpose}/{Assigns a name to the foreign key constraint for clarity and future reference},%
				{Input}/{\nxLnArcBox[]{CONSTRAINT name FOREIGN KEY (col) REFERENCES table(col)}},%
				{Fallback}/{Unnamed constraints are valid but harder to manage or drop},%
				{Rendering}/{Constraint name appears in error messages and schema inspection},%
				{Use Case}/{Schema debugging, migrations, constraint deletion, documentation clarity}
		}
\end{NxLightListBox}

\begin{NxLightBox}[title={Named Foreign Key Constraint — Syntax Glyph Map}]
		\begin{tabularx}{\textwidth}{|l|X|}
				\hline
				\textbf{Engine} & \textbf{Supports Named Constraints} \\
				\hline
				PostgreSQL & {\nxCheckSuccess} fully supported \\
				MariaDB / MySQL & {\nxCheckSuccess} fully supported \\
				SQLite3 & {\nxCheckSuccess} supported in table-level syntax \\
				\hline
				\textbf{Engine} & \textbf{Can Drop by Name} \\
				\hline
				PostgreSQL & {\nxCheckSuccess} \nxLnArcBox[]{ALTER TABLE ... DROP CONSTRAINT name} \\
				MariaDB / MySQL & {\nxCheckSuccess} \nxLnArcBox[]{ALTER TABLE ... DROP FOREIGN KEY name} \\
				SQLite3 & {\nxTimesDanger} must recreate table to drop constraints \\
				\hline
		\end{tabularx}
\end{NxLightBox}

\begin{NxLightListBox}[title={Named Foreign Key Constraint — Beginner Questions}]
		\nxEachLabel{ArrowDark}{Secondary}{{1}{43}}{%
				{Why name a foreign key constraint?}/{It helps with debugging, dropping, and documenting relationships.},%
				{Is naming required?}/{No — constraints can be anonymous, but naming is recommended.},%
				{Can I drop a constraint by name?}/{Yes — in PostgreSQL and MariaDB/MySQL. SQLite requires table recreation.},%
				{Can I use any name?}/{Yes — but it must be unique within the table.},%
				{Does the name affect behavior?}/{No — it’s purely for identification.},%
				{Can I name other constraints too?}/{Yes — \nxLnArcBox[]{UNIQUE}, \nxLnArcBox[]{CHECK}, and \nxLnArcBox[]{PRIMARY KEY} can also be named.}
		}
\end{NxLightListBox}


\end{comment}

