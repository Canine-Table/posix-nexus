\nxSections{Relationships and Keys}{1}

\begin{NxLightBox}[title={Alternative Terminology}]
		\begin{tabularx}{\textwidth}{|X|X|X|}
		\hline
		\rowcolor{blue!10}
		Table & Column & Row \\
		\hline
		\rowcolor{green!10}
		Relation & Attribute & Tuple \\
		\hline
		\rowcolor{yellow!10}
		File & Field & Record \\
		\hline
		\rowcolor{}
	\end{tabularx}
	\bigskip

	\caption{Equivalent Terminologies in Data Management Systems}
\end{NxLightBox}

\nxSections{Relation}{2}
\begin{NxLightListBox}[title={Characteristics of Relations}]
	\nxEachItem{ArrowDark}{
		{Rows contain data about an entity.},
		{Columns contain data about attributes of the entities.},
		{All entries in a column are of the same kind.},
		{Each column has a unique name.},
		{Cells of the table hold a single value.},
		{The order of the columns is unimportant.},
		{The order of the rows is unimportant.},
		{No two rows may be identical.}
	}
\end{NxLightListBox}

\begin{NxCodeBox}{sql}{nxCol=dark,title={create the example database}}
DELIMITER $$

CREATE PROCEDURE SwitchDatabase(IN dbName VARCHAR(64))
BEGIN
    -- Build the USE statement dynamically
    SET @sql = CONCAT('USE ', dbName);
    PREPARE stmt FROM @sql;
    EXECUTE stmt;
    DEALLOCATE PREPARE stmt;
END$$

DELIMITER ;

CREATE OR REPLACE DATABASE nexusDB
	CHARACTER SET utf8mb4
	COLLATE uca1400_as_cs
	COMMENT 'Where collations go to argue about accents and case, but everything is encrypted anyway.';

CREATE DATABASE IF NOT EXISTS posixDB
	CHARACTER SET utf8mb4
	COLLATE uca1400_ai_ci
	COMMENT 'Because even databases deserve a POSIX-compliant bedtime story.';

	SHOW DATABASES;
\end{NxCodeBox}

\begin{NxLightBox}[title={EXAMPLE OF AN EMPLOYEE RELATION}]
		\begin{tabularx}{\textwidth}{|c|X|X|X|X|l|}
				\hline
				EmployeeNumber & FirstName & LastName & Department & EmailAddress & Phone \\
				\hline
				100 & Jerry		& Johnson		& Accounting & JJ@somewhere.com & 518-834-1101 \\
				200 & Mary		& Abernathy & Finance		 & MA@somewhere.com & 518-834-2101 \\
				300 & Liz			& Smathers	& Finance		 & LS@somewhere.com & 518-834-3102 \\
				400 & Tom			& Caruthers & Accounting & TC@somewhere.com & 518-834-1102 \\
				500 & Ken			& Jackson		& Production & KJ@somewhere.com & 518-834-2102 \\
				600 & Eleanor & Caldera		& Legal			 & EC@somewhere.com & 518-834-3101 \\
				700 & Richard & Bandalone & Legal			 & RB@somewhere.com & 518-834-3102 \\
				\hline
		\end{tabularx}
\end{NxLightBox}

\begin{NxCodeBox}{sql}{nxCol=dark,title={mariadb code example}}
USE nexusDB
-- Create the Employee table
CREATE TABLE IF NOT EXISTS Employees (
		EmployeeNumber INT PRIMARY KEY,
		FirstName			 VARCHAR(50) NOT NULL,
		LastName			 VARCHAR(50) NOT NULL,
		Department		 VARCHAR(50) NOT NULL,
		EmailAddress	 VARCHAR(100) UNIQUE NOT NULL,
		Phone					 VARCHAR(20)
);

-- Insert employee records
INSERT INTO Employees (EmployeeNumber, FirstName, LastName, Department, EmailAddress, Phone) VALUES
(100, 'Jerry',	 'Johnson',		'Accounting', 'JJ@somewhere.com', '518-834-1101'),
(200, 'Mary',		 'Abernathy', 'Finance',		'MA@somewhere.com', '518-834-2101'),
(300, 'Liz',		 'Smathers',	'Finance',		'LS@somewhere.com', '518-834-3102'),
(400, 'Tom',		 'Caruthers', 'Accounting', 'TC@somewhere.com', '518-834-1102'),
(500, 'Ken',		 'Jackson',		'Production', 'KJ@somewhere.com', '518-834-2102'),
(600, 'Eleanor', 'Caldera',		'Legal',			'EC@somewhere.com', '518-834-3101'),
(700, 'Richard', 'Bandalone', 'Legal',			'RB@somewhere.com', '518-834-3102');
\end{NxCodeBox}

\begin{NxLightBox}[title={Employee Directory with Multiple Phone Entries}]
		\begin{tabularx}{\textwidth}{|c|X|X|X|X|X|}
				\hline
				EmployeeNumber & FirstName & LastName & Department & EmailAddress & Phone \\
				\hline
				100 & Jerry			& Johnson		& Accounting & JJ@somewhere.com & 518-834-1101 \\
				\hline
				200 & Mary			& Abernathy & Finance		 & MA@somewhere.com & 518-834-2101 \\
				\hline
				300 & Liz				& Smathers	& Finance		 & LS@somewhere.com & 518-834-2102 \\
				\hline
				400 & Tom				& Caruthers & Accounting & TC@somewhere.com & Fax: 518-834-9711 \\ 
						&						&						&						 &									& Home: 518-834-9915 \\
				\hline
				500 & Tom				& Jackson		& Production & TJ@somewhere.com & 518-834-3101 \\
				\hline
				600 & Eleanore	& Caldera		& Legal			 & EC@somewhere.com & Fax: 518-834-9711 \\
						&						&						&						 &									& Home: 518-834-9915 \\
				\hline
				700 & Richard		& Bandalone & Legal			 & RB@somewhere.com & 518-834-3102 \\
				\hline
		\end{tabularx}
\end{NxLightBox}

\begin{NxCodeBox}{sql}{nxCol=dark,title={mariadb code example}}
USE nexusDB
	-- Create Employees table if it does not exist
	CREATE TABLE IF NOT EXISTS Employees (
		EmployeeNumber INT PRIMARY KEY,
		FirstName			 VARCHAR(50) NOT NULL,
		LastName			 VARCHAR(50) NOT NULL,
		Department		 VARCHAR(50) NOT NULL,
		EmailAddress	 VARCHAR(100) UNIQUE NOT NULL
	);

	-- Create EmployeePhones table if it does not exist
	CREATE TABLE IF NOT EXISTS EmployeePhones (
		PhoneID				 INT AUTO_INCREMENT PRIMARY KEY,
		EmployeeNumber INT NOT NULL,
		PhoneType			 VARCHAR(20) NOT NULL,	 -- e.g. 'Work', 'Fax', 'Home'
		PhoneNumber		 VARCHAR(30) NOT NULL,
		FOREIGN KEY (EmployeeNumber) REFERENCES Employees(EmployeeNumber)
	);

	DELIMETER$$
	CREATE PROCEDURE AddEmployeeWithPhone(
		IN pEmployeeNumber INT,
		IN pFirstName			 VARCHAR(50),
		IN pLastName			 VARCHAR(50),
		IN pDepartment		 VARCHAR(50),
		IN pEmailAddress	 VARCHAR(100),
		IN pPhoneType			 VARCHAR(20),
		IN pPhoneNumber		 VARCHAR(30)
	) BEGIN
	-- If employee does not exist, insert them
	IF NOT EXISTS (
		SELECT 1 FROM Employees WHERE EmployeeNumber = pEmployeeNumber
	) THEN
		INSERT INTO Employees (EmployeeNumber, FirstName, LastName, Department, EmailAddress)
		VALUES (pEmployeeNumber, pFirstName, pLastName, pDepartment, pEmailAddress);
	END IF;

		-- Insert the phone entry (can be multiple per employee)
		INSERT INTO EmployeePhones (EmployeeNumber, PhoneType, PhoneNumber)
		VALUES (pEmployeeNumber, pPhoneType, pPhoneNumber);
	EN$$

	DELIMITER ;

	CALL AddEmployeeWithPhone(
			800, 'Alice', 'Walker', 'HR', 'AW@somewhere.com',
			'Work', '518-834-4101'
	);

\end{NxCodeBox}


\nxSections{Cardinality}{2}

%\begin{nxTertiaryListBox}

%\end{nxTertiaryListBox}

\nxSections{Three Types of Minimum Cardinality}{3}
\begin{tikzpicture}
	\node (nodeA) [process] {EMPLOYEE}
		node (nodeB) [decision, right of=nodeA, xshift=2cm] {1:1}
		node (labelA) [below of=nodeB] {Employee\_Identity}
		node (labelB) [below of=labelA] {(a) Mandatory-to-Mandatory (M-M) Relaitonship}
		node (nodeC) [process, right of=nodeB, xshift=2cm] {BADGE};
	\draw (nodeA) -- (nodeB) -- (nodeC);
\end{tikzpicture}
\begin{tikzpicture}
	\node (nodeA) [process] {EMPLOYEE}
		node (nodeB) [decision, right of=nodeA, xshift=2cm] {1:N}
		node (labelA) [below of=nodeB] {Computer\_Assignment}
		node (labelB) [below of=labelA] {(b) Optional-to-Optional (O-O) Relaitonship}
		node (nodeC) [process, right of=nodeB, xshift=2cm] {COMPUTER};
	\draw (nodeA) -- (nodeB) -- (nodeC);
\end{tikzpicture}
\bigskip

\begin{tikzpicture}
	\node (nodeA) [process] {EMPLOYEE}
		node (nodeB) [decision, right of=nodeA, xshift=2cm] {N:M}
		node (labelA) [below of=nodeB] {Qualification}
		node (labelB) [below of=labelA] {(c) Optional-to-Mandatory (O-M) Relaitonship}
		node (nodeC) [process, right of=nodeB, xshift=2cm] {SKILL};
	\draw (nodeA) -- (nodeB) -- (nodeC);
\end{tikzpicture}
\bigskip


