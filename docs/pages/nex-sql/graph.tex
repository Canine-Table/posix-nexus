
\nxSections{Modelling}{1}

\begin{NxLightListBox}[title={Scenario 0 (Book Store System)}]
	\nxEachItem{ArrowDark}{
        {A book store is designing a database system to manage its inventory, customers, authors, and purchase records. The following describes the structure of their data.},
        {Customers are tracked by their name, email, address, phone number, and the date of their purchases.},
        {Each customer has a unique CustomerID.},
        {Books are tracked by their title, genre, price, and publication year.},
        {Each book has a unique BookID.},
        {Authors are tracked by their name, email, and biography.},
        {Each author has a unique AuthorID.},
        {Customers can purchase multiple books, and each purchase records the quantity and date.},
        {Each purchase links a CustomerID and a BookID, forming a many-to-many relationship between customers and books.},
        {Books may be authored by multiple authors, and authors may write multiple books.},
        {The AuthoredBy relationship tracks which author wrote which book, along with the royalty percentage, year of publication, and a unique AuthorshipID.},
        {Books and authors are also directly linked through a many-to-many relationship, separate from AuthoredBy.},
        {The system uses relationship nodes to clarify cardinalities, such as mandatory or optional participation and one-to-many or many-to-many connections.}
    }
\end{NxLightListBox}

\begin{tikzpicture}[line width=1pt]
	\begin{scope}[%
		grow=left,%
		level distance=4cm%
		] \node[process] (customer) at (0,0) {Customer} child {
			node[startstop, xshift=3cm, yshift=-1cm] {CustomerId}
		} child {
			node[startstop] {Name}
		} child {
			node[startstop] {Email}
		} child {
			node[startstop] {Address}
		} child {
			node[startstop] {Phone}
		};
	\end{scope}
	\begin{scope}[%
		grow=right,%
		level distance=2cm,%
	] \node[process, right of=customer, xshift=6cm] (book) {Book} child {
			node[startstop] {BookID}
		} child {
			node[startstop] {Title}
		} child {
			node[startstop] {Gendre}
		} child {
			node[startstop, xshift=-1cm] {Price}
		} child {
			node[startstop, xshift=-4cm, yshift=-1cm] {PubYear}
		};
	\end{scope}
	\begin{scope}[%
	grow=left,%
	level distance=4cm,%
] \node[process, below of=customer, yshift=-6cm] (purchases) {Purchases} child {
		node[startstop] {PurchaseID}
	} child {
		node[startstop] {PurchDate}
	} child {
		node[startstop] {Quantity}
	} child {
		node[startstop] {CustomerID (FK)}
	} child {
		node[startstop, xshift=2cm] {BookID (FK)}
	};
	\end{scope}

	\begin{scope}[%
	grow=right,%
	level distance=2cm%
] \node[process, right of=purchases, xshift=6cm] (author) {Author} child {
		node[startstop] {AuthorID}
	} child {
		node[startstop] {Email}
	} child {
		node[startstop] {Name}
	} child {
		node[startstop] {Biography}
	};
	\end{scope}

	\begin{scope}[%
		level distance=2cm,%
		sibling distance=3cm%
	] \node[process, below of=purchases, yshift=-2cm, xshift=2cm] (authored) {AuthoredBy} child {
			node[startstop] {BookID (FK)}
		} child {
			node[startstop, yshift=-1cm] {AuthorID (FK)}
		} child {
			node[startstop] {Royalty\%}
		} child {
			node[startstop, yshift=-1cm] {YearPub}
		} child {
			node[startstop] {AuthorshipID}
		};
	\end{scope}

		\node[decision, below of=customer, yshift=-2cm] (nodeA) {Relationship};
		\node[decision, below of=book, xshift=-5mm, yshift=-2.7cm] (nodeB) {Relationship};
		\node[decision, above of=authored, xshift=3cm, yshift=5mm] (nodeC) {Relationship};
		\node[decision, below of=customer, yshift=1cm, xshift=4cm] (nodeD) {Relationship};
		\node[decision, left of=book, yshift=-4.6cm, xshift=-2.3cm] (nodeE) {Relationship};

		\draw[mandatoryOne] (nodeA) -- (customer);
		\draw[optionalMany] (nodeA) -- (purchases);

		\draw[mandatoryMany] (nodeB) -- (book);
		\draw[mandatoryMany] (nodeB) -- (author);

		\draw[mandatoryOne] (nodeC) -- (authored);
		\draw[mandatoryOne] (nodeC) -- (author);

		\draw[mandatoryOne] (nodeD) -- (book);
		\draw[optionalMany] (nodeD) -- (purchases);

		\draw[mandatoryMany] (nodeE) -- (authored);
		\draw[mandatoryMany] (nodeE) -- (book);
\end{tikzpicture}

\begin{NxLightListBox}[title={Scenario 1 (Veterinary Hospital)}]
	\nxEachItem{ArrowDark}{
			{A local veterinary hospital is looking for a replacement for their patient tracking system. When asked to describe their data, they responded as follows.},
			{Clients may have one or more pets.},
			{Clients are identified by their phone number.},
			{Clients are tracked by their name and address.},
			{A pet is owned by a single client only.},
			{A pet is identified by their owner's name and the pet name.},
			{Pets are tracked by their species, breed, sex, and neutering status.},
			{A pet can be treated by several different doctors and/or technicians (staff).},
			{Each doctor and technician has an employee number and a name.},
			{A pet may have multiple visits that are tracked by date and reason for visit.},
		}
\end{NxLightListBox}

\begin{NxLightListBox}[title={Scenario 2 (Software Company)}]
	\nxEachItem{ArrowDark}{
		{A software development company is modeling its internal structure and project assignments.},
		{The company has many employees.},
		{Employees are tracked by their name and a unique employee ID.},
		{The company has several departments, such as engineering, quality assurance, and tech support.},
		{Each department has a name and a manager.},
		{Managers are also employees and have unique employee IDs.},
		{Each department must have at least one employee assigned to it.},
		{Employees must be assigned to at least one department, but may belong to multiple departments.},
		{Projects are tracked by a unique project ID and a project name.},
		{Each project must have at least one employee assigned to it.},
		{An employee may be assigned to zero or more projects.},
		{Departments, projects, managers, and employees are all tracked by their names.},
	}
\end{NxLightListBox}
