\nxSections{Definitions}{1}

\begin{NxLightBox}[title={Function}]
    \begingroup
    \renewcommand{\arraystretch}{1.8}
    \begin{tabularx}{\linewidth}{|c|X|}
        \hline
        \textbf{Concept} & Function \\
        \hline
        \textbf{Definition} & A function is a rule that takes an input and gives back exactly one output. It describes how one quantity depends on another. \\
        \hline
        \textbf{Core Idea} & You give it a number, and it gives you a result — like a machine that transforms input into output. \\
        \hline
        \textbf{Example} & The function \( f(x) = x^2 \) takes any number \( x \) and returns \( x \times x \). So \( f(3) = 9 \). \\
        \hline
        \textbf{Applications} & Used to model relationships in math, science, engineering, economics, and everyday situations. \\
        \hline
    \end{tabularx}
    \endgroup
\end{NxLightBox}

\begin{NxLightBox}[title={Derivative}]
    \begingroup
    \renewcommand{\arraystretch}{1.8}
    \begin{tabularx}{\linewidth}{|c|X|}
        \hline
        \textbf{Concept} & Derivative \\
        \hline
        \textbf{Definition} & A derivative tells you how fast a function’s output is changing compared to its input. It measures the rate of change — like speed for a moving object. \\
        \hline
        \textbf{Core Idea} & If a function is a machine, the derivative tells you how sensitive that machine is — how much the output jumps when you nudge the input. \\
        \hline
        \textbf{Example} & For \( f(x) = x^2 \), the derivative is \( f'(x) = 2x \). At \( x = 3 \), the slope is \( 6 \), meaning the output is rising quickly. \\
        \hline
        \textbf{Applications} & Used to study motion, growth, optimization, and any situation where change matters. \\
        \hline
    \end{tabularx}
    \endgroup
\end{NxLightBox}

\begin{NxLightBox}[title={Slope}]
    \begingroup
    \renewcommand{\arraystretch}{1.8}
    \begin{tabularx}{\linewidth}{|c|X|}
        \hline
        \textbf{Concept} & Slope \\
        \hline
        \textbf{Definition} & Slope tells you how steep a line is — how much the output changes compared to the input. It’s the rate of change between two points. \\
        \hline
        \textbf{Core Idea} & If you move one step in the input direction, the slope tells you how many steps the output moves. \\
        \hline
        \textbf{Example} & A line with slope \( 2 \) means every time \( x \) increases by \( 1 \), \( y \) increases by \( 2 \). \\
        \hline
        \textbf{Applications} & Used in graphs, physics (speed), economics (cost change), and in derivatives to describe how functions behave. \\
        \hline
    \end{tabularx}
    \endgroup
\end{NxLightBox}

\begin{NxLightBox}[title={Tangent Line}]
    \begingroup
    \renewcommand{\arraystretch}{1.8}
    \begin{tabularx}{\linewidth}{|c|X|}
        \hline
        \textbf{Concept} & Tangent Line \\
        \hline
        \textbf{Definition} & A tangent line is a straight line that touches a curve at exactly one point and moves in the same direction as the curve at that point. It shows the curve’s slope or rate of change at that location. \\
        \hline
        \textbf{Core Idea} & It’s the best straight-line approximation of the curve near a point — like zooming in until the curve looks flat. \\
        \hline
        \textbf{Example} & For the curve \( f(x) = x^2 \), the tangent line at \( x = 2 \) has slope \( 4 \), so it rises steeply. \\
        \hline
        \textbf{Applications} & Used in physics (instantaneous velocity), optimization (finding peaks), and calculus (defining derivatives). \\
        \hline
    \end{tabularx}
    \endgroup
\end{NxLightBox}

\begin{NxLightBox}[title={Tangent Function}]
    \begingroup
    \renewcommand{\arraystretch}{1.8}
    \begin{tabularx}{\linewidth}{|c|X|}
        \hline
        \textbf{Concept} & Tangent (tan) \\
        \hline
        \textbf{Definition} & The tangent of an angle is the ratio of the opposite side to the adjacent side in a right triangle. It’s also defined as \( \tan(x) = \frac{\sin(x)}{\cos(x)} \). \\
        \hline
        \textbf{Core Idea} & Tangent compares how tall something is versus how far it runs — it’s a slope. \\
        \hline
        \textbf{Example} & For a 45° angle, \( \tan(45^\circ) = 1 \), meaning rise equals run. \\
        \hline
        \textbf{Applications} & Used in trigonometry, navigation, physics, and calculus — especially for modeling slopes and angles. \\
        \hline
    \end{tabularx}
    \endgroup
\end{NxLightBox}

