\nxSections{Notation}{1}

\begin{NxLightBox}[title={Summation Notation}]
    \begingroup
    \renewcommand{\arraystretch}{1.8}
    \begin{tabularx}{\linewidth}{|c|X|}
        \hline
        \textbf{Symbol} & \( \Sigma \) (uppercase Greek Sigma) \\
        \hline
        \textbf{Meaning} & Summation — add a sequence of terms \\
        \hline
        \textbf{Syntax} & \( \sum_{i=1}^{n} a_i \) means add all \( a_i \) from \( i = 1 \) to \( i = n \) \\
        \hline
        \textbf{Example} & \( \sum_{i=1}^{4} i = 1 + 2 + 3 + 4 = 10 \) \\
        \hline
    \end{tabularx}
    \endgroup
\end{NxLightBox}

\begin{NxLightBox}[title={Product Notation}]
    \begingroup
    \renewcommand{\arraystretch}{1.8}
    \begin{tabularx}{\linewidth}{|c|X|}
        \hline
        \textbf{Symbol} & \( \prod \) (uppercase Greek Pi) \\
        \hline
        \textbf{Meaning} & Product — multiply a sequence of terms \\
        \hline
        \textbf{Syntax} & \( \prod_{i=1}^{n} a_i \) means multiply all \( a_i \) from \( i = 1 \) to \( i = n \) \\
        \hline
        \textbf{Example} & \( \prod_{i=1}^{4} i = 1 \times 2 \times 3 \times 4 = 24 \) \\
        \hline
    \end{tabularx}
    \endgroup
\end{NxLightBox}

\begin{NxLightBox}[title={Factorial Notation}]
    \begingroup
    \renewcommand{\arraystretch}{1.8}
    \begin{tabularx}{\linewidth}{|c|X|}
        \hline
        \textbf{Symbol} & \( n! \) \\
        \hline
        \textbf{Meaning} & Multiply all positive integers from \( 1 \) to \( n \) \\
        \hline
        \textbf{Syntax} & \( n! = n \times (n-1) \times (n-2) \times \dots \times 1 \) \\
        \hline
        \textbf{Example} & \( 5! = 5 \times 4 \times 3 \times 2 \times 1 = 120 \) \\
        \hline
    \end{tabularx}
    \endgroup
\end{NxLightBox}

\begin{NxLightBox}[title={Integral Notation}]
    \begingroup
    \renewcommand{\arraystretch}{1.8}
    \begin{tabularx}{\linewidth}{|c|X|}
        \hline
        \textbf{Symbol} & \( \int \) (elongated, slanted "S") \\
        \hline
        \textbf{Meaning} & Integration — summing infinitely small quantities over an interval \\
        \hline
        \textbf{Syntax} & \( \int_a^b f(x)\,dx \) means integrate \( f(x) \) from \( x = a \) to \( x = b \) \\
        \hline
        \textbf{Example} & \( \int_0^2 x\,dx = 2 \) \\
        \hline
    \end{tabularx}
    \endgroup
\end{NxLightBox}

\begin{NxLightBox}[title={Limit Notation}]
    \begingroup
    \renewcommand{\arraystretch}{1.8}
    \begin{tabularx}{\linewidth}{|c|X|}
        \hline
        \textbf{Symbol} & \( \lim \) \\
        \hline
        \textbf{Meaning} & Limit — the value a function approaches as input nears a point \\
        \hline
        \textbf{Syntax} & \( \lim_{x \to a} f(x) \) means the value of \( f(x) \) as \( x \) approaches \( a \) \\
        \hline
        \textbf{Example} & \( \lim_{x \to 0} \frac{\sin(x)}{x} = 1 \) \\
        \hline
    \end{tabularx}
    \endgroup
\end{NxLightBox}

\begin{NxLightBox}[title={Derivative Notation}]
    \begingroup
    \renewcommand{\arraystretch}{1.8}
    \begin{tabularx}{\linewidth}{|c|X|}
        \hline
        \textbf{Symbol} & \( \frac{df}{dx} \) or \( f'(x) \) \\
        \hline
        \textbf{Meaning} & Derivative — rate of change of a function with respect to its input \\
        \hline
        \textbf{Syntax} & \( \frac{df}{dx} \) means how \( f(x) \) changes as \( x \) changes \\
        \hline
        \textbf{Example} & If \( f(x) = x^2 \), then \( f'(x) = 2x \) \\
        \hline
    \end{tabularx}
    \endgroup
\end{NxLightBox}

