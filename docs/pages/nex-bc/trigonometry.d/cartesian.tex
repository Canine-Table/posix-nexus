\nxSections{Cartesian (xy) Plane}{2}
\includegraphics[width=\linewidth]{img/.env/axis-2025-10-16_13-5620251016_135736.png}

% Cartesian Plane — The Rectangular Canvas
\begin{NxLightListBox}[title={Cartesian Plane — The Rectangular Canvas}]
\nxEachLabel{ArrowDark}{Secondary}{{4}{5}}{%
        {Axes}/{Two perpendicular axes \(x\) and \(y\) defining horizontal and vertical directions},%
        {Origin}/{The central point \((0,0)\) where axes intersect and measurements begin},%
        {Grid}/{An infinite lattice of parallel lines marking unit intervals},%
        {Quadrants}/{Four sign realms labeling positions of points relative to the origin}%
    }
\end{NxLightListBox}

% Coordinate Systems Comparison
\begin{NxLightBox}[title={Rectangular vs Polar Coordinate Systems}]
\begingroup
\renewcommand{\arraystretch}{1.6}
\begin{tabularx}{\linewidth}{|l|X|X|}
    \hline
    \textbf{Feature} & \textbf{Rectangular} & \textbf{Polar} \\
    \hline
    Parameters & \((x,y)\) along orthogonal axes & \((r,\theta)\) as radius and angle \\
    Basis & Linear displacement on two lines & Radial distance and angular measure \\
    Conversion & \(x = r\cos\theta,\;y = r\sin\theta\) & \(r = \sqrt{x^2 + y^2},\;\theta = \atan2(y,x)\) \\
    Use Cases & Cartesian graphs, analytic geometry & Circular motion, waves, radial fields \\
    \hline
\end{tabularx}
\endgroup
\end{NxLightBox}

% Conversion Formulae
\begin{empheq}[box=\nxWarningMathBox]{align*}
x &= r \cos\theta, & y &= r \sin\theta, \\
r &= \sqrt{x^2 + y^2}, & \theta &= \atan2(y, x)
\end{empheq}
\bigskip

% Conceptual Paragraph
\begin{NxLightP}
The Cartesian Plane, conceived by René Descartes in the 17th century, is the foundational rectangular framework where points are given by ordered pairs \((x,y)\).  
This plane underlies the rectangular coordinate system, mapping linear displacements along perpendicular axes.  
Polar coordinates overlay this same plane with a circular measure, locating points by distance \(r\) from the origin and angle \(\theta\) from the positive \(x\)-axis.  
Thus the Cartesian Plane serves as the common canvas for both rectangular and polar systems.
\end{NxLightP}

% Key Insights List
\begin{NxLightListBox}[title={Key Insights}]
    \nxEachItem{ArrowDark}{
        {The Cartesian Plane is the rectangular grid defined by \(x\) and \(y\) axes},%
        {Rectangular coordinates use \((x,y)\) directly on this grid},%
        {Polar coordinates use \((r,\theta)\) over the same grid by circular mapping},%
        {Conversion relies on sine, cosine, and the Pythagorean theorem},%
        {Both systems coexist on the Cartesian Plane, each suited to different problems}%
    }
\end{NxLightListBox}

% Final Font Invocation
\begin{NxLightP}
{\nxSFont{14pt} The Cartesian Plane is the shared canvas, rectangular at its core, circular in its polar overlay.}
\end{NxLightP}

% Standard Position — The Glyph of Zero
\begin{NxLightListBox}[title={Standard Position — The Glyph of Zero}]
    \nxEachLabel{ArrowDark}{Secondary}{{5}{6}}{%
        {Vertex}/{The common endpoint of the angle, fixed at \((0,0)\)},%
        {Initial Side}/{The ray on the positive \(x\)-axis from the origin},%
        {Terminal Side}/{The rotating ray that sweeps from the initial side},%
        {Rotation Direction}/{Counterclockwise for positive angles; clockwise for negative},%
        {Quadrantal Angles}/{Angles whose terminal side lies on an axis (0°, 90°, 180°, 270°)}%
    }
\end{NxLightListBox}

% Angle Anatomy — Standard Position
\begin{NxLightBox}[title={Angle Anatomy — Standard Position}]
\begingroup
\renewcommand{\arraystretch}{1.6}
\begin{tabularx}{\linewidth}{|l|X|}
    \hline
    \textbf{Component}     & \textbf{Description}                                                                     \\
    \hline
    Vertex                 & Located at the origin where the two sides meet                                           \\
    Initial Side           & Fixed along the positive \(x\)-axis as the starting ray                                  \\
    Terminal Side          & The ray rotated from the initial side to form the angle                                 \\
    Positive Rotation      & Measured counterclockwise from the initial side                                         \\
    Negative Rotation      & Measured clockwise from the initial side                                                \\
    \hline
\end{tabularx}
\endgroup
\end{NxLightBox}

% Conceptual Paragraph
\begin{NxLightP}
An angle is in standard position when its vertex is at the origin and its initial side lies along the positive \(x\)-axis.  
The terminal side rotates about the origin: counterclockwise for positive measures and clockwise for negative.  
This convention provides a unified framework for defining trigonometric functions and angle measures, from quadrantal angles to arbitrary sweeps.
\end{NxLightP}

% Key Insights List
\begin{NxLightListBox}[title={Key Insights on Standard Position}]
    \nxEachItem{ArrowDark}{%
        {Standard Position fixes the vertex at \((0,0)\) and the initial side on the positive \(x\)-axis},%
        {Positive angles rotate counterclockwise; negative rotate clockwise},%
        {Terminal sides on axes define quadrantal angles},%
        {Essential for defining sine, cosine, tangent and more},%
        {Forms the basis for rectangular and polar coordinate linkage}%
    }
\end{NxLightListBox}

\begin{NxLightP}
{\nxSFont{14pt} Standard Position is the cardinal glyph of angle measurement — the origin, the positive \(x\)-axis, and the sweep of rotation.}
\end{NxLightP}

% Endpoint — The Termination Glyph
\begin{NxLightListBox}[title={Endpoint — The Termination Glyph}]
    \nxEachLabel{ArrowDark}{Secondary}{{4}{5}}{%
        {Definition}/{A point marking where a geometric object ends: one end of a segment or the single start of a ray},%
        {Line Segment Context}/{Each segment has two endpoints determining its finite span and boundary},%
        {Ray Context}/{A ray has exactly one endpoint from which it extends infinitely, defining its origin and direction},%
        {Angle Context}/{In an angle, the shared endpoint of its two sides is the vertex, serving as the angle’s hinge},%
        {Notation}/{Labeled by a capital letter (e.g., \(A\)); segments use \(\overline{AB}\), rays use \(\overrightarrow{AB}\)},%
        {Applications}/{Determines segment length, constructs polygons, anchors vectors and network nodes}%
    }
\end{NxLightListBox}

% Standard Position — Cardinal Angle Glyph
\begin{NxLightListBox}[title={Standard Position — Cardinal Angle Glyph}]
    \nxEachLabel{ArrowDark}{Secondary}{{4}{5}}{%
        {Definition}/{An angle placed with its vertex at \((0,0)\) and its initial side along the positive \(x\)-axis},%
        {Vertex}/{The origin \((0,0)\), the common endpoint of both rays},%
        {Initial Side}/{The fixed ray on the positive \(x\)-axis serving as the zero‐angle reference},%
        {Terminal Side}/{The rotating ray that sweeps from the initial side to form the angle},%
        {Rotation Direction}/{Counterclockwise for positive measures; clockwise for negatives},%
        {Purpose}/{Establishes a uniform reference for measuring angles in rectangular and polar systems}%
    }
\end{NxLightListBox}

% Clarification — Initial Side Notation
\begin{NxLightListBox}[title={Initial Side — The Positive Ray Notation}]
    \nxEachLabel{ArrowDark}{Secondary}{{9}{1}}{%
        {Definition}/{The set of points \((x,0)\) with \(x \ge 0\), i.e.\ the ray along the positive \(x\)-axis},%
        {Common Notation}/{Often denoted \([0,\infty)\) on the real line but not a summation \(\sum\) or integral \(\int\)},%
        {Role}/{Acts as the zero‐angle reference in standard position}%
    }
\end{NxLightListBox}

% Clarification — Y-Axis Rigidity
\begin{NxLightListBox}[title={Y-Axis — The Fixed Perpendicular Glyph}]
    \nxEachLabel{ArrowDark}{Secondary}{{9}{2}}{%
        {Definition}/{The line of all points \((0,y)\) for real \(y\), perpendicular to the \(x\)-axis},%
        {Fixed Role}/{In the standard Cartesian frame it remains orthogonal to the initial side},%
        {Variants}/{If you rotate the \(y\)-axis independently, you leave the standard rectangular system and enter a rotated or skewed frame}%
    }
\end{NxLightListBox}
\begin{NxLightListBox}[title={Ordered Pair — Cartesian Glyph}]
    \nxEachLabel{ArrowDark}{Secondary}{{4}{5}}{%
        {Definition}/{An ordered pair \((x,y)\) designates a point with horizontal coordinate \(x\) and vertical coordinate \(y\)},%
        {Parentheses}/{Here the parentheses simply group the two coordinates—they are not interval delimiters and do not imply exclusion},%
        {Components}/{First component \(x\) is horizontal displacement; second component \(y\) is vertical displacement}%
    }
\end{NxLightListBox}

% Point on the X-Axis
\begin{NxLightListBox}[title={Point on the X-Axis}]
    \nxEachLabel{ArrowDark}{Secondary}{{4}{5}}{%
        {Definition}/{All points of the form \((x,0)\) lie exactly on the \(x\)-axis because \(y=0\)},%
        {Range}/{In pure Cartesian form \(x\) may be any real number (including 0)},%
        {Initial-Side Context}/{For the standard-position initial side we further require \(x \ge 0\)}%
    }
\end{NxLightListBox}

% Interval Notation — Inclusive vs Exclusive
\begin{NxLightListBox}[title={Interval Notation — Inclusive vs Exclusive}]
    \nxEachLabel{ArrowDark}{Secondary}{{4}{5}}{%
        {Definition}/{Square brackets \([\,]\) include endpoints; parentheses \((\,)\) exclude them in real-number intervals},%
        {Clarification}/{In \((x,0)\) the parentheses are not interval notation but part of point notation},%
        {Implication}/{To describe the ray you’d write \(\{(x,0)\mid x\ge0\}\) or \([0,\infty)\) for the \(x\)-values}%
    }
\end{NxLightListBox}

% Reference Angle — The Acute Measuring Glyph
\begin{NxLightListBox}[title={Reference Angle — The Acute Measuring Glyph}]
    \nxEachLabel{ArrowDark}{Secondary}{{4}{5}}{%
        {Definition}/{The acute angle between the terminal side of an angle in standard position and the \(x\)-axis},%
        {Computation}/{Subtract the angle from the nearest multiple of \(90^\circ\) or \(\tfrac{\pi}{2}\) to yield \(0^\circ\le\theta_{\text{ref}}\le90^\circ\)},%
        {Range}/{Always between \(0\) and \(90^\circ\) (or \(0\) and \(\tfrac{\pi}{2}\) radians) inclusive},%
        {Use}/{Allows evaluation of trigonometric functions by referencing an acute angle with the same function value},%
        {Notation}/{Denoted \(\theta_{\text{ref}}\)}%
    }
\end{NxLightListBox}

% Coterminal Angle — The Rotational Glyph
\begin{NxLightListBox}[title={Coterminal Angle — The Rotational Glyph}]
    \nxEachLabel{ArrowDark}{Secondary}{{4}{5}}{%
        {Definition}/{Angles that share the same initial and terminal sides when placed in standard position},%
        {Formula}/{Given \(\theta\), coterminal angles are \(\theta + 360^\circ k\) or \(\theta + 2\pi k\) for any integer \(k\)},%
        {Property}/{They have identical sine, cosine, and tangent values because they land on the same point of the unit circle},%
        {Notation}/{Written as \(\theta \pm 360^\circ n\) or \(\theta \pm 2\pi n\)},%
        {Application}/{Used to find equivalent angles within a chosen interval (e.g., \([0,360^\circ)\) or \([0,2\pi)\))}%
    }
\end{NxLightListBox}

% Reference Angle — The Acute Angle Glyph
\begin{NxLightListBox}[title={Reference Angle — The Acute Angle Glyph}]
    \nxEachLabel{ArrowDark}{Secondary}{{4}{5}}{%
        {Definition}/{The acute angle between an angle’s terminal side and the \(x\)-axis when in standard position},%
        {Calculation}/{Take the absolute difference between the angle and the nearest multiple of \(180^\circ\) or \(\pi\)},%
        {Range}/{Always satisfies \(0^\circ \le \theta_{\mathrm{ref}} \le 90^\circ\) (or \(0 \le \theta_{\mathrm{ref}} \le \tfrac{\pi}{2}\))},%
        {Purpose}/{Allows evaluation of trig functions by referencing a corresponding acute angle},%
        {Notation}/{Denoted \(\theta_{\mathrm{ref}}\)}%
    }
\end{NxLightListBox}

% Coterminal Angle — The Rotational Equivalence Glyph
\begin{NxLightListBox}[title={Coterminal Angle — The Rotational Equivalence Glyph}]
    \nxEachLabel{ArrowDark}{Secondary}{{4}{5}}{%
        {Definition}/{Angles sharing the same initial and terminal sides in standard position},%
        {Formula}/{All coterminal angles are \(\theta + 360^\circ k\) or \(\theta + 2\pi k\) for integer \(k\)},%
        {Property}/{They yield identical sine, cosine, and tangent values by landing on the same unit‐circle point},%
        {Use Case}/{Finds an equivalent angle within a principal interval (e.g., \([0,360^\circ)\) or \([0,2\pi)\))},%
        {Notation}/{Written as \(\theta \pm 360^\circ n\) or \(\theta \pm 2\pi n\)}%
    }
\end{NxLightListBox}

% Translation — The Shifting Glyph
\begin{NxLightListBox}[title={Translation — The Shifting Glyph}]
    \nxEachLabel{ArrowDark}{Secondary}{{4}{5}}{%
        {Definition}/{A Euclidean transformation that shifts every point of the triangle by the same vector},%
        {Operation}/{Subtract the coordinates of the chosen vertex from all vertices so that vertex maps to \((0,0)\)},%
        {Vector}/{If the vertex is \((x_0,y_0)\), translate by \((-x_0,-y_0)\)},%
        {Notation}/{Denoted \(T_{(-x_0,-y_0)}\)},%
        {Purpose}/{Places the triangle’s vertex at the origin for standard positioning}%
    }
\end{NxLightListBox}

% Rotation — The Spinning Glyph
\begin{NxLightListBox}[title={Rotation — The Spinning Glyph}]
    \nxEachLabel{ArrowDark}{Secondary}{{4}{5}}{%
        {Definition}/{A rigid transformation that rotates points about the origin by a fixed angle},%
        {Operation}/{Compute the angle \(\theta\) between the chosen base side and the positive \(x\)-axis and rotate by \(-\theta\)},%
        {Formula}/{\((x',y')=(x\cos(-\theta)-y\sin(-\theta),\,x\sin(-\theta)+y\cos(-\theta))\)},%
        {Notation}/{Denoted \(R_{-\theta}\)},%
        {Purpose}/{Aligns the chosen triangle side along the positive \(x\)-axis}%
    }
\end{NxLightListBox}

% Procedure — Positioning a Triangle in Standard Position
\begin{NxLightListBox}[title={Procedure — Positioning a Triangle in Standard Position}]
    \nxEachLabel{ArrowDark}{Secondary}{{4}{5}}{%
        {Step 1}/{Translate the triangle so the chosen vertex goes to \((0,0)\)},%
        {Step 2}/{Compute the angle between the chosen base side and the positive \(x\)-axis},%
        {Step 3}/{Rotate the triangle by the negative of that angle to align the side with \(x\ge0\)},%
        {Result}/{The triangle sits with one vertex at the origin and one side on the positive \(x\)-axis, ready for trigonometric analysis}%
    }
\end{NxLightListBox}

% Pythagorean Theorem — Right‐Triangle Side Calculation
\begin{NxLightListBox}[title={Pythagorean Theorem — Right‐Triangle Side Calculation}]
    \nxEachLabel{ArrowDark}{Secondary}{{4}{5}}{%
        {Statement}/{In a right triangle with legs \(a,b\) and hypotenuse \(c\), \(a^2 + b^2 = c^2\)},%
        {Solve for Hypotenuse}/{\(c = \sqrt{a^2 + b^2}\)},%
        {Solve for a Leg}/{\(a = \sqrt{c^2 - b^2}\) or \(b = \sqrt{c^2 - a^2}\)},%
        {Context}/{Applies only when one angle is exactly \(90^\circ\)}%
    }
\end{NxLightListBox}


% Trigonometric Ratios — Right‐Triangle Angles and Sides
\begin{NxLightListBox}[title={Trigonometric Ratios — Right‐Triangle Angles and Sides}]
    \nxEachLabel{ArrowDark}{Secondary}{{4}{5}}{%
        {Definitions}/{\(\sin\theta=\tfrac{\text{opp}}{\text{hyp}},\,\cos\theta=\tfrac{\text{adj}}{\text{hyp}},\,\tan\theta=\tfrac{\text{opp}}{\text{adj}}\)},%
        {Find an Angle}/{\(\theta=\arcsin\frac{\text{opp}}{\text{hyp}}\), or \(\theta=\arccos\frac{\text{adj}}{\text{hyp}}\), or \(\theta=\arctan\frac{\text{opp}}{\text{adj}}\)},%
        {Context}/{Requires one acute angle and one side known}%
    }
\end{NxLightListBox}

% Law of Cosines — General Triangle Side or Angle
\begin{NxLightListBox}[title={Law of Cosines — General Triangle Side or Angle}]
    \nxEachLabel{ArrowDark}{Secondary}{{4}{5}}{%
        {Side Formula}/{\(c^2 = a^2 + b^2 - 2ab\cos C\)},%
        {Angle Formula}/{\(\cos C = \frac{a^2 + b^2 - c^2}{2ab}\)},%
        {Solve for Side}/{\(c = \sqrt{a^2 + b^2 - 2ab\cos C}\)},%
        {Solve for Angle}/{\(C = \arccos\!\bigl(\frac{a^2 + b^2 - c^2}{2ab}\bigr)\)},%
        {Context}/{Works for any triangle when two sides and included angle are known (SAS) or three sides (SSS)}%
    }
\end{NxLightListBox}

% Law of Sines — General Triangle Angle or Side
\begin{NxLightListBox}[title={Law of Sines — General Triangle Angle or Side}]
    \nxEachLabel{ArrowDark}{Secondary}{{4}{5}}{%
        {Statement}/{\(\displaystyle\frac{a}{\sin A}=\frac{b}{\sin B}=\frac{c}{\sin C}\)},%
        {Find a Side}/{\(a = b\frac{\sin A}{\sin B}\) or \(c = b\frac{\sin C}{\sin B}\)},%
        {Find an Angle}/{\(A = \arcsin\!\bigl(\tfrac{a\sin B}{b}\bigr)\)},%
        {Context}/{Valid when any side–angle opposite pair is known plus another side or angle (ASA, AAS, SSA)}%
    }
\end{NxLightListBox}

% Law of Cosines Term — Interaction Glyph
\begin{NxLightListBox}[title={Law of Cosines Term — Interaction Glyph}]
    \nxEachLabel{ArrowDark}{Secondary}{{4}{5}}{%
        {Interaction Term}/{\(2ab\cos C\) is twice the product of sides \(a\) and \(b\) multiplied by the cosine of the included angle \(C\)},%
        {Role}/{It adjusts the sum \(a^2 + b^2\) to account for the tilt between those sides},%
        {Sign}/{The minus sign in \(a^2 + b^2 - 2ab\cos C\) reduces the sum when \(0^\circ<C<90^\circ\) and increases it when \(90^\circ<C<180^\circ\)},%
        {Full Formula}/{\(c^2 = a^2 + b^2 - 2ab\cos C\)},%
        {Special Case}/{When \(C=90^\circ\), \(\cos C=0\) and it collapses to \(c^2 = a^2 + b^2\) (the Pythagorean theorem)}%
    }
\end{NxLightListBox}

% Interaction Term — Angular Adjustment Glyph
\begin{NxLightListBox}[title={Interaction Term — Angular Adjustment Glyph}]
    \nxEachLabel{ArrowDark}{Secondary}{{4}{5}}{%
        {Geometric Origin}/{Derived from the dot product of two sides of lengths \(a\) and \(b\)},%
        {Adjustment Role}/{Modifies \(a^2 + b^2\) by the projection of one side onto the other},%
        {Cosine Function}/{\(\cos C\) measures alignment of sides, not quadrant or 360° normalization},%
        {Sign Impact}/{For \(0^\circ<C<90^\circ\), \(\cos C>0\) subtracts; for \(90^\circ<C<180^\circ\), \(\cos C<0\) subtracting a negative adds},%
        {Purpose}/{Ensures the correct length of the third side for both acute and obtuse interior angles}%
    }
\end{NxLightListBox}

% Alignment Context — The Vector Alignment Glyph
\begin{NxLightListBox}[title={Alignment Context — The Vector Alignment Glyph}]
    \nxEachLabel{ArrowDark}{Secondary}{{4}{5}}{%
        {Definition}/{Refers to the angle between the two side‐vectors of lengths \(a\) and \(b\) at their common vertex},%
        {Vertex as Origin}/{In the dot‐product derivation we translate that vertex to \((0,0)\) so both sides become vectors from the origin},%
        {Cosine Measure}/{\(\cos C\) quantifies how much one side “leans” toward the other at that vertex},%
        {Coordinate Independence}/{This alignment is intrinsic to the triangle and doesn’t depend on the global \(x\)–\(y\) axes unless you choose standard position},%
        {Purpose}/{Ensures the third side’s length reflects the exact tilt between sides \(a\) and \(b\)}%
    }
\end{NxLightListBox}

% Labeling Convention — Angle vs Side Glyph
\begin{NxLightListBox}[title={Labeling Convention — Angle vs Side Glyph}]
    \nxEachLabel{ArrowDark}{Secondary}{{4}{5}}{%
        {Convention}/{Angles are denoted by uppercase letters (e.g., \(A,B,C\))},%
        {Sides}/{Sides are denoted by lowercase letters (e.g., \(a,b,c\))},%
        {Opposition}/{Side \(a\) is opposite angle \(A\); side \(b\) opposite \(B\); side \(c\) opposite \(C\)},%
        {Triangle Context}/{In \(\triangle ABC\), vertices \(A,B,C\) label angles; sides \(BC\), \(CA\), \(AB\) label \(a,b,c\) respectively},%
        {Usage}/{This convention ensures formulas like the Law of Sines and Law of Cosines remain consistent}%
    }
\end{NxLightListBox}

Polar Axis
Polar Angle


coordinate system Rectangular and Polar

