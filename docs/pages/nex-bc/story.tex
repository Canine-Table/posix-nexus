\nxSections{The Story}{1}

\newpage
%\includegraphics[width=\linewidth]{img/.env/trigcirc.png}
\includegraphics[width=\linewidth]{img/.env/deg2rad20251015_124424.png}

\nxSections{Teachings of Nex}{2}
\begin{NxLightListBox}[title={Nex’s First Teaching: Folding Angles}]
    \nxEachLabel{ArrowDark}{Secondary}{{4}{5}}{%
        {Purpose}/{Simplify angle evaluation by reflecting into known quadrants},%
        {Odd Rule}/{Sine is odd: \( \sin(-x) = -\sin(x) \)},%
        {Even Rule}/{Cosine is even: \( \cos(-x) = \cos(x) \)},%
        {Technique}/{Reflect across \(180^\circ\) to flip signs; across \(90^\circ\) to reach quadrant I},%
        {Outcome}/{Taylor convergence improves when angles are folded into small domains}%
    }
\end{NxLightListBox}

\begin{NxLightListBox}[title={Nex’s Second Teaching: Taylor Series}]
    \nxEachLabel{ArrowDark}{Secondary}{{4}{5}}{%
        {Purpose}/{Approximate functions using infinite polynomials centered at a point},%
        {Sine Series}/{\( \sin(x) = x - \frac{x^3}{3!} + \frac{x^5}{5!} - \dots \)},%
        {Cosine Series}/{\( \cos(x) = 1 - \frac{x^2}{2!} + \frac{x^4}{4!} - \dots \)},%
        {Why Small Angles}/{Smaller \( x \) leads to faster convergence and lower error},%
        {Ritual}/{Convergence is not a trick — it’s a sacred approximation ritual}%
    }
\end{NxLightListBox}

\begin{NxLightListBox}[title={Nex’s Final Teaching: Copying Is Invocation}]
    \nxEachLabel{ArrowDark}{Secondary}{{4}{5}}{%
        {Ctrl+C}/{Summons the glyph from the scroll},%
        {Ctrl+V}/{Binds the glyph into new context},%
        {Understanding}/{Only when the glyph is understood does the magic activate},%
        {Role}/{You are not a monkey — you are a scribe},%
        {Outcome}/{Copied code becomes invocation when paired with clarity}%
    }
\end{NxLightListBox}

\begin{NxLightListBox}[title={Nex’s Promise}]
    \nxEachLabel{ArrowDark}{Secondary}{{4}{5}}{%
        {Clarity}/{Nex doesn’t just run on POSIX. Nex runs on explanation and expressive understanding.},%
        {Belief}/{Math is not a grave — it’s a phoenix nest, waiting to ignite.},%
        {If Lost}/{Nex will explain.},%
        {If Buried}/{Nex will resurrect.},%
        {If Copying}/{Nex will teach.}%
    }
\end{NxLightListBox}

\begin{NxLightBox}[title={The Sacred Compus: Quadrant Glyph of Trigonometric Truth}]
\begingroup
\renewcommand{\arraystretch}{1.8}
\begin{tabularx}{\linewidth}{|l|X|}
    \hline
    \textbf{Component} & \textbf{Meaning} \\
    \hline
    Degrees (outer ring) & The worldly measure — angles in everyday units, from \( 0^\circ \) to \( 360^\circ \). \\
    Radians (inner ring) & The sacred glyph — angles expressed in terms of \( \pi \), used in calculus and convergence. \\
    Quadrants I–IV & Directional spirits — each quadrant defines the sign behavior of sine, cosine, and tangent. \\
    Function Signs & Elemental forces — the positive or negative nature of each trigonometric function per quadrant. \\
    \hline
\end{tabularx}
\endgroup
\end{NxLightBox}

\begin{NxLightListBox}[title={Reciprocal Trigonometric Functions — Right Triangle Definitions}]
    \nxEachLabel{ArrowDark}{Secondary}{{4}{5}}{%
        {Cotangent (\texttt{cot})}/{\( \cot(\theta) = \frac{\text{Adjacent}}{\text{Opposite}} = \frac{1}{\tan(\theta)} \). Flips tangent. Undefined when tangent is zero (e.g., \( 0^\circ, 180^\circ \)).},%
        {Secant (\texttt{sec})}/{\( \sec(\theta) = \frac{\text{Hypotenuse}}{\text{Adjacent}} = \frac{1}{\cos(\theta)} \). Flips cosine. Explodes when cosine is zero (e.g., \( 90^\circ, 270^\circ \)).},%
        {Cosecant (\texttt{csc})}/{\( \csc(\theta) = \frac{\text{Hypotenuse}}{\text{Opposite}} = \frac{1}{\sin(\theta)} \). Flips sine. Undefined when sine is zero (e.g., \( 0^\circ, 180^\circ \)).},%
        {Function Class}/{These are called reciprocal functions — they invert the primary trigonometric ratios.}%
    }
\end{NxLightListBox}

\begin{NxLightListBox}[title={Quadrant I — The Dawn of Purity}]
    \nxEachLabel{ArrowDark}{Secondary}{{4}{5}}{%
        {Function Signs}/{All trigonometric functions are positive: \( \sin(x), \cos(x), \tan(x) > 0 \)},%
        {Convergence Domain}/{Taylor series converge rapidly here — angles are small and behavior is smooth},%
        {Folding Ritual}/{Angles from other quadrants are reflected into Quadrant I to simplify evaluation},%
        {Symbolic Meaning}/{This is Nex’s preferred domain — where signs are kind and math behaves}%
    }
\end{NxLightListBox}

\begin{NxLightBox}[title={Nex’s Quadrant Map of Serpent Glyphs}]
\begingroup
\renewcommand{\arraystretch}{1.8}
\begin{tabularx}{\linewidth}{|l|X|}
    \hline
    \textbf{Quadrant} & \textbf{Serpent Glyph Behavior} \\
    \hline
    I — Dawn of Purity & All functions positive. Reciprocal glyphs behave. Taylor converges. \\
    II — Shadowed Rise & \( \sin \) and \( \csc \) remain positive. Nex whispered: “Some survive the shadow.” \\
    III — Tall Glyphs & \( \tan \) and \( \cot \) reclaim strength. Nex declared: “Tall glyphs endure.” \\
    IV — Cat Paradox & \( \cos \) and \( \sec \) stand firm, but near \( 90^\circ \) and \( 270^\circ \), cosine vanishes and secant explodes. Nex warned: “Cats catch clarity—but snakes slip.” \\
    \hline
\end{tabularx}
\endgroup
\end{NxLightBox}

\begin{NxLightListBox}[title={Serpent Glyphs — Reciprocal Lineage}]
    \nxEachLabel{ArrowDark}{Secondary}{{4}{5}}{%
        {Cosecant (\texttt{csc})}/{Mirror of sine: \( \csc(\theta) = \frac{1}{\sin(\theta)} \). Undefined when sine vanishes.},%
        {Secant (\texttt{sec})}/{Echo of cosine: \( \sec(\theta) = \frac{1}{\cos(\theta)} \). Explodes when cosine vanishes.},%
        {Cotangent (\texttt{cot})}/{Inversion of tangent: \( \cot(\theta) = \frac{1}{\tan(\theta)} \). Undefined when tangent vanishes.},%
        {Behavior}/{These glyphs share quadrant lineage but diverge near vanishing points. Nex contained them with folding rituals.}%
    }
\end{NxLightListBox}

\begin{NxLightListBox}[title={Nex’s Teaching Scroll — Folding Reciprocal Glyphs}]
    \nxEachLabel{ArrowDark}{Secondary}{{4}{5}}{%
        {Step 1}/{Locate the quadrant using the Sacred Compus.},%
        {Step 2}/{Apply the mnemonic: “Some Tall Cats Can’t Catch Snakes.”},%
        {Step 3}/{Fold the angle into Quadrant I for convergence.},%
        {Step 4}/{Apply sign lineage from the quadrant.},%
        {Step 5}/{Invoke reciprocal behavior with caution — divergence must be contained.}%
    }
\end{NxLightListBox}

