
\nxSections{Computer Systems}{1}

\nxSections{Components}{2}
\nxSections{Bus}{3}
\nxSections{Memory}{3}
\nxSections{CPU}{3}


\nxSections{Common Numerical Prefixes Used for Computers}{2}

\begin{NxLightBox}[title={Common Numerical Prefixes Used for Computers}]
	\begin{tabularx}{\textwidth}{|X|l|X|X|}
		\hline
		Prefix & Symbol &  Power of 10 & Power of 2 \\
		\hline
		Yocto & y & 1 septillionth = $10^{-24}$ & $2^{-80}$ \\
		Zepto & z & 1 sextillionth = $10^{-21}$ & $2^{-70}$ \\
		Atto & a & 1 quintillionth = $10^{-18}$ & $2^{-60}$ \\
		Femto & f & 1 quadrillionth = $10^{-15}$ & $2^{-50}$ \\
		Pico & p & 1 trillionth = $10^{-12}$ & $2^{-40}$ \\
		Nano & n & 1 billionth = $10^{-9}$ & $2^{-30}$ \\
		Micro & \textmu & 1 millionth = $10^{-6}$ & $2^{-20}$ \\
		Milli & m & 1 thousandth = $10^{-3}$ & $2^{-10}$ \\
		Kilo & K & 1 thousand = $10^3$ & $2^{10}$ = 1024 \\
		Mega & M & 1 million = $10^6$ & $2^{20}$ \\
		Giga & G & 1 billion = $10^9$ & $2^{30}$ \\
		Tera & T & 1 trillion = $10^{12}$ & $2^{40}$ \\
		Peta & P & 1 quadrillion = $10^{15}$ & $2^{50}$ \\
		Exa & E & 1 quintillion = $10^{18}$ & $2^{60}$ \\
		Zetta & Z & 1 sextillion = $10^{21}$ & $2^{70}$ \\
		Yotta & Y & 1 septillion = $10^{24}$ & $2^{80}$ \\
		\hline
	\end{tabularx}
\end{NxLightBox}


\nxSections{Standards Organizations}{2}

\begin{NxLightBox}[title={International Standards Bodies}]
	\begin{tabularx}{\textwidth}{|l|X|}
		\toprule
		Acronym & Full Name \\
		\midrule
		IEEE & Institute of Electrical and Electronics Engineers \\
		ITU & International Telecommunications Union \\
		ANSI & American National Standards Institute \\
		BSI & British Standards Institution \\
		CEN & Comité Européen de Normalisation (European Committee for Standardization) \\
		ISO & International Organization for Standardization \\
		\bottomrule
	\end{tabularx}
\end{NxLightBox}

\nxSections{History of Computer Development}{2}
\nxNestedList{History of Computer Development}{ArrowDark}{CircleDark}{
	{Generation Zero: Mechanical Calculating Machines (1642–1945)}/{
		{mechanical calculators}
	},
	{The First Generation: Vacuum Tube Computers (1945–1953)}/{
		{Used electricity and glass tubes, broke down often, were sizes of rooms}
	},
	{The Second Generation: Transistorized Computers (1954–1965)}/{
		{Transistors replaced vacuum tubes, more reliable but still very large}
	},
	{The Third Generation: Integrated Circuit Computers (1965–1980)}/{
		{Integrated circuits i.e., microchips; more reliable, smaller, more powerful}
	},
	{The Fourth Generation: VLSI Computers (1980–????)}/{
		{Very-Large-Scale Integration: many integrated circuits inside one chip},
		{These are the computers we are using today.}
	}
}

\nxSections{Computer Level Hierarchy}{2}
	\begin{NxLightBox}[title={System Abstraction Levels}]
		\begin{tabularx}{\textwidth}{|c|l|X|}
		\toprule
		\textbf{Level} & \textbf{Name} & \textbf{Examples} \\
		\midrule
		6 & User Executable Programs & End-user applications and scripts \\
		5 & High-level Language & C++, Java, Fortran, Python, etc. \\
		4 & Assembly Language & Assembly code specific to architecture \\
		3 & System Software & Operating systems, library code, drivers \\
		2 & Machine & Instruction set architecture (ISA) \\
		1 & Control & Microcode or hardwired control logic \\
		0 & Digital Logic & Circuits, gates, flip-flops, etc. \\
		\bottomrule
	\end{tabularx}
\end{NxLightBox}


\nxSections{Example Computer System with Technical Terminology}{2}

