\ExplSyntaxOn

\NewDocumentCommand{\posnexBoolean}{s t- m m t- O{\c_true_bool} O{\c_false_bool}}{
    \bool_if:NTF{#1}{
        % Starred version, Set to AND (*)
        \bool_if:NTF{#2}{
            % NAND (*-)
            \bool_lazy_and:nnTF {#3} {#4} {#7} {#6}

        }{
            % AND (*)
            \bool_lazy_and:nnTF {#3} {#4} {#6} {#7}

        }
    }{
        % Non-starred version, Set to OR ()
        \bool_if:NTF{#2}{
            % NOR (-)
            \bool_if:NTF{#5}{
                % NXOR
                \bool_xor:nnTF {#3} {#4} {#7} {#6}

            }{
                % NOR
                \bool_lazy_or:nnTF {#3} {#4} {#7} {#6}

            }
        }{
            % OR
            \bool_if:NTF{#5}{
                % XOR (--)
                \bool_xor:nnTF {#3} {#4} {#6} {#7}

            }{
                % OR ()
                \bool_lazy_or:nnTF {#3} {#4} {#6} {#7}
            }
        }
    }
}

\NewDocumentCommand{\posnexTCList}{s t< m m}{
    
    % Check if starred or the token list is empty
    \posnexBoolean{#1}{!\tl_if_exist_p:c {g_posnex_#3_tl}}[% If true, clear the token list and assign the mandatory token
        \tl_gclear_new:c {g_posnex_#3_tl}
        \tl_gput_right:cn {g_posnex_#3_tl} {#4}

    ][% If false, check if token '<' is provided
        \bool_if:NTF{#2}{% token < is true, append to the left
            \tl_gput_left:cn {g_posnex_#3_tl} {#4,~}

        }{% Otherwise, append to the right
            \tl_gput_right:cn {g_posnex_#3_tl} {,~#4}

        }
    ]%
}

\NewDocumentCommand{\posnexTCListGet}{mm}{ % Define a new command with two mandatory arguments
    \regex_match:nnT {\d+} {#2}{ % Check if the second argument matches a digit
        \clist_item:cn {g_posnex_#1_tl}{#2} % If it is a digit, get the item from the clist with name g_posnex_<first argument>_tl at position <second argument>

    }{
        \str_case:nn {#2}{ % If it's not a digit, use string case to determine what to do based on the value of the second argument
            {T}{\tl_use:c {g_posnex_#1_tl}} % If the second argument is 'T', use the entire token list g_posnex_<first argument>_tl
            {C}{
                \tl_set:Nx \l_tmpa_tl  {\clist_count:c {g_posnex_#1_tl}}
                \int_set:Nn \l_tmpa_int { \int_eval:n { \tl_use:N \l_tmpa_tl } }
                \int_new:c {g_posnex_#1_int}
                \int_gset:cn {g_posnex_#1_int} \l_tmpa_int
            } % If the second argument is 'C', count the items in the clist
            {L}{{\clist_item:cn {g_posnex_#1_tl}{\clist_count:c {g_posnex_#1_tl}}}} % If the second argument is 'L', get the last item in the clist
            {R}{
                \int_if_exist:NT {g_posnex_#1_tl} {
                    \tl_gclear:N {g_posnex_#1_tl}
                }
                
                \int_if_exist:NT{g_posnex_#1_int} {
                    \int_gset:cn {g_posnex_#1_int} {0}
                }
            }
            {D}{
                \cs_undefine:c {g_posnex_#1_tl}
                \cs_undefine:c {g_posnex_#1_int}
            }
        }
    }
}

\NewDocumentCommand{\posnexModulus}{mm}{
    \int_eval:n { (#1) - \int_div_truncate:nn {#1}{#2}  * (#2) }
}

\NewDocumentCommand{\posnexCaesarCipher}{m}{

    % transform #1 to lower case and store in \l_tmpa_str
    \str_set:Nx \l_tmpa_str {\tl_lower_case:n {#1}}
    % clear \l_tmpb_str to store results
    \str_clear:N \l_tmpb_str
    % \str_map_inline:Nn traverses the string
    % and pass each character as first argument
    \str_map_inline:Nn \l_tmpa_str {
        % `##1 gives the ASCII code of ##1
        % 91 is the ASCII code of 'a'
        % this allows us to compute the offset of ##1
        \int_set:Nn \l_tmpa_int { \int_eval:n {`##1 - 97} }
        % suppose the shifting of our Ceaser cipher is 3
        \int_set:Nn \l_tmpb_int { \int_mod:nn {\l_tmpa_int + 3}{26} }
        % place new character in \l_tmpb_str
        \str_put_right:Nx \l_tmpb_str {
            % this function generates a character given
            % character code and category code
            % because we are dealing with English letters
            % the category code is 11
            \char_generate:nn {\l_tmpb_int + 97}{11}
        }
    }

    % outputs \l_tmpb_str
    \str_use:N \l_tmpb_str
}

\NewDocumentCommand{\posnexGrid}{s O{1} mmm}{

    % Initialize the list
    \posnexTCList{#4}{#5}
    
    % Get the count of items in the list
    \posnexTCListGet{#4}{C}
    
    % Ensure temporary integers exist
    \int_if_exist:NF {\l_tmpc_int}{\int_new:N \l_tmpc_int}
    \int_if_exist:NF {\l_tmpd_int}{\int_new:N \l_tmpd_int}
    \int_if_exist:NF {\l_tmpe_int}{\int_new:N \l_tmpe_int}
    
    % Initialize counters
    \int_set:Nn \l_tmpc_int {1}
    \int_set:Nn \l_tmpe_int {1}
    % Outer loop
    \int_do_while:nNnn {\l_tmpc_int} < {1} {
        % Initialize inner loop counter
        \int_set:Nn \l_tmpd_int {1}
        
        % Inner loop
        \int_do_while:nNnn {\l_tmpd_int} < {\int_eval:n {(\int_use:c {g_posnex_#4_int}) + 1}} {
            \begin{tikzpicture}
                \node[#3] at (\l_tmpc_int, -\l_tmpd_int) {\edef\x{\posnexTCListGet{#4}{\int_eval:n \l_tmpe_int}}\x};
            \end{tikzpicture}

            % Increment counters
            \int_incr:N \l_tmpd_int
            \int_incr:N \l_tmpe_int

                % Conditional check for line break 
                \int_compare:nNnT {\posnexModulus{\l_tmpd_int}{#2}} = 1 { \\ }
            }
        
        % Increment outer loop counter
        \int_incr:N \l_tmpc_int
    }

    % Delete the list if the command is starred
    \IfBooleanT{#1}{
        \posnexTCListGet{#4}{D}
    }
}

\NewDocumentCommand{\posnexRead}{mO{sections/examples/}}{

    \str_new:N \l_tmpc_str
    \ior_new:N \l_posnex_shell_ior
    \ior_open:Nn \l_posnex_shell_ior {#2#1}

    \group_begin:
        \ior_map_variable:NNn \l_posnex_shell_ior \l_tmpc_str {
            \par\str_use:N \l_tmpc_str
        }
    
    \group_end:
    
    \ior_close:N \l_posnex_shell_ior
}

\ExplSyntaxOff

\NewDocumentCommand{\posnexInline}{m}{
    \setminted{
        linenos,
        frame=lines, % (none | leftline | topline | bottomline | lines | single)
        numberblanklines=true,
        texcomments=true,
        mathescape=true,
        rulecolor=\color{palette.dark.frame!50!white},
        framesep=3mm,
        style=dracula,
        encoding=UTF-8,
        escapeinside=@@,
        tabsize=4,
        fontsize=\footnotesize,
        keywordcase=lower, % (lower | upper | capitalize | none)
        numbers=left, % (left | right | both | none)
        breaklines,
        breakautoindent=false,
        breaksymbolindentleft=0pt,
        breaksymbolsepleft=0pt,
        breaksymbolindentright=0pt,
        mathescape,
    }
    \mintinline{bash}|#1|
}
