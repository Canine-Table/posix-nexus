\usepackage{subfiles} % For including separate files as subfiles within the main document
\usepackage{graphicx} % For including graphics (images)
\usepackage{pagecolor} % For setting the background color of the page
\usepackage{comment} % For including multi-line comments
\usepackage[listings,breakable,listingsutf8,skins,hooks,external,fitting,magazine,raster]{tcolorbox} % For creating colored and framed text boxes (consider replacing with LaTeX3 if possible)
\usepackage{tikz} % For creating graphics programmatically
\usepackage{multicol} % For creating multi-column layouts
\usepackage{marvosym} % For providing icons for contact details
\usepackage{pgfplots} % For creating plots and graphs
\usepackage{paracol} % For typesetting columns of text in parallel
\usepackage{xcolor} % Adds color support for text, tables, and other elements in your document
\usepackage{hyperref} % Enables hyperlinks within the document, making references, URLs, and citations clickable
\usepackage{fancyhdr} % Allows customization of headers and footers for a more polished and professional look
\usepackage{tocloft} % Customizes the table of contents, list of figures, and list of tables
\usepackage{titlesec} % Customizes section headings to control their appearance and style
\usepackage{expl3} % A programming layer used for implementing higher-level LaTeX macros
%\usepackage{minted} % For enhanced source code highlighting using Pygments\usepackage{listings} % Enhances the formatting of source code listings, making them more readable and visually appealing
\usepackage{xcolor} % Adds color support for text, tables, and other elements in your document
\usepackage{makeidx} % For creating an index to help readers navigate keywords and pages in the document
\usepackage{glossaries} % For managing and creating glossaries, lists of abbreviations, and acronyms
\usepackage{fancyhdr} % Allows customization of headers and footers for a more polished and professional look
\usepackage{incgraph}
\usepackage{amsmath}