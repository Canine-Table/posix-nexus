\usepackage{xcolor, eso-pic} % Enables color for text and backgrounds; eso-pic allows background overlays
\usepackage[twoside]{fancyhdr} % Customizes headers and footers for a professional document layout

\usepackage[
	margin=2cm, % Sets uniform margins around the page
	top=3cm, % Adds additional space at the top
	bottom=3cm % Adds additional space at the bottom
]{geometry} % Controls page dimensions

\usepackage[
	listings, % Enables syntax highlighting for code snippets
	breakable, % Allows tcolorbox to split boxes across pages
	listingsutf8, % Ensures proper encoding of listings
	skins, % Allows different styles for tcolorbox
	hooks, % Adds customization options
	external, % Enables external compilation of boxes
	fitting, % Ensures proper resizing of boxes
	poster, % Optimizes for poster layouts
	raster % Enables raster graphics support within tcolorbox
]{tcolorbox} % Provides enhanced text box formatting

\usepackage{fontawesome5} % Adds FontAwesome icons for visual elements
\usepackage{graphicx} % Enables image inclusion and manipulation
\usepackage{comment} % Allows multi-line comments for readability

\usepackage{tikz} % Powerful vector graphics package for creating diagrams
\usepackage{pgfplots} % Generates plots and graphs directly within LaTeX
\usepackage{pgfkeys} % Facilitates key-value storage for configurations

\usepackage{hyperref} % Makes references and URLs clickable within the document
\usepackage{tocloft} % Customizes table of contents, list of figures, and list of tables
\usepackage{titlesec} % Modifies section headings for improved typography

\usepackage{minted} % Uses Pygments for syntax highlighting of code
\usepackage{listings} % Alternative package for displaying code snippets

\usepackage{makeidx} % Generates an index for quick reference of keywords
\usepackage{glossaries} % Manages glossaries, acronyms, and abbreviations

\usepackage{amsmath} % Provides advanced mathematical formatting
\usepackage{svg} % Supports inclusion of scalable vector graphics

\usepackage[T1]{fontenc} % Improves font encoding for better typography
\usepackage{libertine} % Sets the document font to Libertine
\usepackage{fourier-orns} % Provides ornamental symbols from the Fourier font

\usepackage{varwidth} % Allows variable-width text boxes for alignment flexibility
\usepackage{lipsum} % Provides placeholder text for document drafts
\usepackage{incgraph} % Enables incremental graphics drawing within the document
\usepackage{emoji} % Adds support for emojis in LaTeX

\RequirePackage{ifluatex} % Checks if the document is compiled with LuaLaTeX
\ifluatex%
	\usepackage{luacode} % Allows embedding Lua code directly in LaTeX
\else%
	\PackageError{luacode}{LuaTeX is required, but not detected}{}%
\fi%

\pgfplotsset{compat=1.18}

\long\def\nxItalic#1{\textit{#1}}
\long\def\nxBold#1{\textbf{#1}}
\def\nxLDash{\unskip\ --\nobreak\ \ignorespaces}
\def\nxRDash{\unskip\nobreak\ --\ \ignorespaces}
\def\nxTab{$\qquad$}
\long\def\nxTerm#1\par{\index{#1}}
\def\nxCs#1{\char`\\#1}

\def\nxLBracket{\char`\{}
\def\nxRBracket{\char`\}}
\newenvironment{NxBracket}{\nxLBracket}{\nxRBracket}

\def\nxLAngle{$\langle$}
\def\nxRAngle{$\rangle$}
\newenvironment{NxAngle}{\nxLAngle}{\nxRAngle}

\newenvironment{NxBold}{\bgroup\bf}{\egroup}
\newenvironment{NxItalics}{\bgroup\it}{\egroup}
\newenvironment{NxTeleType}{\bgroup\tt}{\egroup}
\newenvironment{NxRemove}{\bgroup\rm}{\egroup}
\newenvironment{NxSmall}{\bgroup\sc}{\egroup}
\newenvironment{NxQuote}{\begin{quotation}}{\end{quotation}}
\newenvironment{NxInfo}{\begin{description}}{\end{description}}
\newenvironment{NxRight}{\begin{flushright}}{\end{flushright}}

\def\nxHDotFill{%
	\noindent\leaders\hbox{%
		\vrule width 1pt height 1pt 
		\kern 4pt 
		\vrule width.5pt height 0.4pt
	}\hfill\hbox{}\par%
}

\def\nxHDLineFill{%
	\noindent\leaders\hbox{%
		\vrule width 5.5pt height 0.4pt
	}\hfill\hbox{}\par%
}

\makeatletter

\makeatother

\message{Posix-Nexus Lua preamble fully loaded!}

