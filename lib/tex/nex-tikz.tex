\makeatletter

\def\nx@optionalMany{pp}
\def\nx@optionalOne{oo}
\def\nx@mandatoryMany{mm}
\def\nx@mandatoryOne{nn}
\def\nx@true{1}
\def\nx@false{0}

\def\nx@crowsfootmm#1#2{%
	\ifx#1\nx@mandatoryMany%
		\ifx#2\nx@true%
			\draw[thick] (-0.2,0) -- (0.0,0.1)
				(-0.2,0) -- (0.0,-0.1)
				(-0.2,0.0) -- (0,0)
				(-0.2,0.1) -- (-0.2,-0.1);
		\else\ifx#2\nx@false%
			\draw[thick]
				(0, -0.1) -- (0.2, -0)
				(0, 0.1) -- (0.2, -0)
				(0.2, 0) -- (0, 0)
				(0.2, -0.1) -- (0.2, 0.1);
		\fi\fi%
	\fi%
}

\def\nx@crowsfootnn#1#2{%
	\ifx#1\nx@mandatoryOne%
		\ifx#2\nx@true%
			\draw[thick] (-0.15,0.1) -- (-0.15,-0.1)
				(-0.14, 0) -- (0, 0)
				(-0.1,0.1) -- (-0.1,-0.1);
		\else\ifx#2\nx@false%
			\draw[thick] (0.15,0.1) -- (0.15,-0.1)
				(0.14, 0) -- (0, 0)
				(0.1,0.1) -- (0.1,-0.1);
		\fi\fi%
	\fi%
}

\def\nx@crowsfootpp#1#2{%
	\ifx#1\nx@optionalMany%
		\ifx#2\nx@true%
			\draw[thick, fill=white]
				(0, 0.1) -- (-0.2, 0)
				(-0.3, 0) circle[radius=0.8mm]
				(0, -0.1) -- (-0.2, 0)
				(-0.23, 0) -- (0, 0);
		\else\ifx#2\nx@false%
			\draw[thick, fill=white]
				(0, -0.1) -- (0.2, 0)
				(0.3, 0) circle[radius=0.8mm]
				(0, 0.1) -- (0.2, 0)
				(0.23, 0) -- (0, 0);
		\fi\fi%
	\fi%
}

\def\nx@crowsfootoo#1#2{%
	\ifx#1\nx@optionalOne%
		\ifx#2\nx@true%
			\draw[thick, fill=white]
				(-0.16, 0) -- (0, 0)
				(-0.25,0) circle[radius=1mm] (-0.25,-0.2)
				(-0.1,0.1) -- (-0.1,-0.1);
		\else\ifx#2\nx@false%
			\draw[thick, fill=white]
				(0.16, 0) -- (0, 0)
				(0.25, 0) circle[radius=1mm] (0.25, -0.2)
				(0.1,0.1) -- (0.1, -0.1);
		\fi\fi%
	\fi%
}

\tikzset{%
	line/.style={%
		draw,%
		-{Latex}%
	},%
	startstop/.style={%
		ellipse,%
		draw,%
		fill=red!20,%
		text centered,%
		minimum height=2em%
	},%
	process/.style={%
		rectangle,%
		draw,%
		fill=blue!20,%
		text centered,%
		minimum height=2em%
	},%
	decision/.style={%
		diamond,%
		draw, fill=green!20,%
		text centered,%
		aspect=2%
	},%
	io/.style={%
		trapezium,%
		trapezium left angle=70,%
		trapezium right angle=110,%
		draw,%
		fill=orange!20,%
		text centered%
	},%
	terminator/.style={%
		rounded rectangle,%
		draw, fill=purple!20,%
		text centered,%
		minimum height=2em%
	},%
	labeltop/.style={%
		above,%
		font=\small\itshape%
	},%
	labelbottom/.style={%
		below,%
		font=\small\ttfamily%
	},%
	labelleft/.style={%
		left,%
		font=\scriptsize%
	},%
	labelright/.style={%
		right,%
		font=\scriptsize\bfseries%
	},%
	stealth/.style={
		thick,%
		->,%
		>=Stealth%
	},%
	highlight/.style={draw=black, thick, fill=yellow!30},%
	error/.style={draw=red, thick, fill=red!10},%
	compute/.style={process, fill=cyan!20},%
	ioin/.style={io, fill=orange!30},%
	iout/.style={io, fill=lime!30},%
	labelmid/.style={midway, font=\scriptsize\ttfamily},%
	box/.style={draw, thick, rounded corners, fill=gray!10, inner sep=5pt},%
	/nx/.cd,%
	dash/.style={%
		thick,%
		dashed,%
		dash pattern=on 2pt off 1pt%
	},%
	crowsfoot/.style n args={2}{%
			postaction={%
				decorate,%
				decoration={%
					markings,%
					mark=at position 0 with {\pgfextra{\expandafter\csname nx@crowsfoot#1\endcsname{#1}{\nx@false}}},%
					mark=at position 1 with {\pgfextra{\expandafter\csname nx@crowsfoot#2\endcsname{#2}{\nx@true}}}%
			}%
		}%
	},%
	/nx/lbl/.cd,%
	above/.style={%
		above left=-2mm of #1.north%
	},%
	/nx/er/.cd,%
	rend/.style={
			shorten <=#1%
	},%
	rend/.default=5mm,%
	lend/.style={
			shorten >=#1,%
	},%
	lend/.default=5mm,%
	bend/.style args={#1,#2}{
		/nx/er/lend=#1,%
		/nx/er/rend=#2%
	},%
	bend/.default={5mm,5mm},%
	/nx/logic gates/.cd,%
	and/.style={%
			and port,%
			logic gate inputs={#1},%
			fill=blue!10%
	},%
	nand/.style={%
		nand port,%
		logic gate inputs={#1},%
		fill=purple!10%
	},%
	or/.style={%
		or port,%
		logic gate inputs={#1},%
		fill=yellow!10%
	},%
	nor/.style={%
		nor port,%
		logic gate inputs={#1},%
		fill=green!10%
	},%
	xor/.style={%
		xor port,%
		logic gate inputs={#1},%
		fill=gray!10%
	},%
	xnor/.style={%
		xnor port,%
		inputs={#1},%
		logic gate inputs={#1},%
		fill=brown!10%
	},%
	buf/.style={%
		buffer port,%
		fill=green!10%
	},%
	not/.style={%
		not port,%
		fill=red!10%
	}%
}

\long\def\nxCoverPage[#1]{%
	\pgfkeys{%
		nx cover page,%
		title,%
		author,%
		ref,%
		abstract,%
		Abstract,%
		image,%
		#1%
	}%
	\def\nx@ref{\pgfkeysvalueof{/nx/cover page/ref}}
	\begin{tikzpicture}[remember picture, overlay]
		\shade[
			right color=nex.danger.back,%
			left color=nex.dark.back,%
			draw=nex.danger.frame,%
			ultra thick,%
		] (current page.north west) -| (current page.south east) -| (current page.north west);
		\node[%
			circle,%
			outer color=nex.dark.back,%
			inner color=nex.warning.back,%
			minimum width=0.7\pagewidth,%
			circular drop shadow={%
				shadow scale=1.05%
			},%
			draw=nex.warning.frame,%
			ultra thick,%
			postaction={%
				decorate,%
				decoration={%
					raise=2mm,
					text along path,%
					text align={align=center},%
					text color=nex.warning.text,%
					text={\today}
				}
			}
		] (logo) at (current page.center) {};
		\node at (current page.center) {\includegraphics[width=0.6\pagewidth]{\nx@image}};
		\node[above=of logo, text=nex.danger.text, yshift=1cm, align=center, text width=\linewidth] (title) {\nxSFont{36pt} \bf \nx@title};
		\node[below=of logo, text=nex.danger.text, yshift=-1cm, align=center, text width=\linewidth] (author) {\nxSFont{24pt} \bf \nx@author};
		\node[below=of author, text=nex.primary.text, align=center, text width=\linewidth] (link) {\nxSFont{24pt} \href{\nx@link}{\bf \nx@ref}};
	\end{tikzpicture}
	\newpage
	\expandafter\ifx\nx@abstract\relax\else
		\begin{tcolorbox}[nxCol=light, sdwA]
			\begin{tcolorbox}[nxCol=light, title={\bf \nx@Abstract}, hdrD, sdwA]
				\nx@abstract
			\end{tcolorbox}
		\end{tcolorbox}
		\bigskip
	\fi
}

\def\nxSnowFade{%
	\AddToShipoutPictureBG{%
		\AtPageLowerLeft{%
			\begin{tikzpicture}[remember picture, overlay]
				\expandafter\shade\expandafter[%
					top color={\ifodd\the\count0 nex.page.top\else nex.page.bottom\fi},
					middle color=nex.page.middle,%
					bottom color={\ifodd\the\count0 nex.page.bottom\else nex.page.top\fi},
				](current page.south west) rectangle (current page.north east);
			\end{tikzpicture}
		}
	}
}

\def\nxTikzAxis#1{%
	\colorlet{mcol}{green!50!black}
	\colorlet{ycol}{red!50!black}
	\colorlet{xcol}{blue!50!black}
	\draw[help lines,step=5mm] (-#1,-#1) grid (#1,#1);
	\begin{scope}[axes]
		\draw[->, thick, color=xcol] (-#1,0) -- (#1,0) node[right] {$x$} coordinate(x axis);
		\draw[->, thick, color=red] (0,-#1) -- (0,#1) node[above] {$y$} coordinate(y axis);
		\foreach \x/\xtext in {-#1, #1} \draw[xshift=\x cm, color=mcol, thick] (0pt, 5pt)
			-- (0pt,-5pt) node[below, color=xcol, ultra thick] {$\xtext$};
		\foreach \y/\ytext in {-#1, #1} \draw[yshift=\y cm, color=mcol, thick] (5pt, 0pt)
			-- (-5pt,0pt) node[left, color=ycol, ultra thick] {$\ytext$};
	\end{scope}
}

\begin{comment}
\ctikzset{%
	logic ports/scale=1.2,%
	logic ports/label/.style={font=\small\ttfamily},%
	logic ports/input anchors=1,2,%
	logic ports/output anchor=out%
}
\end{comment}

\def\polartransformation{%
	% \pgf@x will contain the radius
	% \pgf@y will contain the distance
	\pgfmathsincos@{\pgf@sys@tonumber\pgf@x}%
	% pgfmathresultx is now the cosine of radius and
	% pgfmathresulty is the sine of radius
	\pgf@x=\pgfmathresultx\pgf@y%
	\pgf@y=\pgfmathresulty\pgf@y%
}

\pgfdeclarearrow{
	name = nxMM,
	parameters = {
		\the\pgfarrowlength
	},
	setup code = {
		% The different end values:
		\pgfarrowssettipend{.25\pgfarrowlength}
		\pgfarrowssetlineend{-.25\pgfarrowlength}
		\pgfarrowssetvisualbackend{-.5\pgfarrowlength}
		\pgfarrowssetbackend{-.75\pgfarrowlength}
		% The hull
		\pgfarrowshullpoint{.25\pgfarrowlength}{0pt}
		\pgfarrowshullpoint{-.75\pgfarrowlength}{.5\pgfarrowlength}
		\pgfarrowshullpoint{-.75\pgfarrowlength}{-.5\pgfarrowlength}
		% Saves: Only the length:
		\pgfarrowssavethe\pgfarrowlength
	},
	drawing code = {
		\pgfpathmoveto{\pgfqpoint{.35\pgfarrowlength}{0pt}}
		\pgfpathlineto{\pgfqpoint{\pgfarrowlength}{.3\pgfarrowlength}}
		\pgfpathmoveto{\pgfqpoint{.35\pgfarrowlength}{0pt}}
		\pgfpathlineto{\pgfqpoint{\pgfarrowlength}{-.3\pgfarrowlength}}
		\pgfpathmoveto{\pgfqpoint{-.25\pgfarrowlength}{0pt}}
		\pgfpathlineto{\pgfqpoint{\pgfarrowlength}{0pt}}
		\pgfpathmoveto{\pgfqpoint{.25\pgfarrowlength}{.3\pgfarrowlength}}
		\pgfpathlineto{\pgfqpoint{.25\pgfarrowlength}{-.3\pgfarrowlength}}
		\pgfpathclose
		\pgfusepathqstroke
		%\pgfpathclose
		%\pgfusepathqfill
	},
	defaults = {
		length = 5mm
	}
}

\pgfdeclarearrow{
	name = nxMO,
	parameters = {
		\the\pgfarrowlength
	},
	setup code = {
		% The different end values:
		\pgfarrowssettipend{.25\pgfarrowlength}
		\pgfarrowssetlineend{-.25\pgfarrowlength}
		\pgfarrowssetvisualbackend{-.5\pgfarrowlength}
		\pgfarrowssetbackend{-.75\pgfarrowlength}
		% The hull
		\pgfarrowshullpoint{.25\pgfarrowlength}{0pt}
		\pgfarrowshullpoint{-.75\pgfarrowlength}{.5\pgfarrowlength}
		\pgfarrowshullpoint{-.75\pgfarrowlength}{-.5\pgfarrowlength}
		% Saves: Only the length:
		\pgfarrowssavethe\pgfarrowlength
	},
	drawing code = {
		\pgfpathmoveto{\pgfqpoint{-.25\pgfarrowlength}{0pt}}
		\pgfpathlineto{\pgfqpoint{\pgfarrowlength}{0pt}}
		\pgfpathmoveto{\pgfqpoint{.6\pgfarrowlength}{-.3\pgfarrowlength}}
		\pgfpathlineto{\pgfqpoint{.6\pgfarrowlength}{.3\pgfarrowlength}}
		\pgfpathmoveto{\pgfqpoint{.3\pgfarrowlength}{-.3\pgfarrowlength}}
		\pgfpathlineto{\pgfqpoint{.3\pgfarrowlength}{.3\pgfarrowlength}}
		\pgfpathclose
		\pgfusepathqstroke
	},
	defaults = {
		length = 5mm
	}
}

\pgfdeclarearrow{
	name = nxOO,
	parameters = {
		\the\pgfarrowlength
	},
	setup code = {
		% The different end values:
		\pgfarrowssettipend{.25\pgfarrowlength}
		\pgfarrowssetlineend{-.25\pgfarrowlength}
		\pgfarrowssetvisualbackend{-.5\pgfarrowlength}
		\pgfarrowssetbackend{-.75\pgfarrowlength}
		% The hull
		\pgfarrowshullpoint{.25\pgfarrowlength}{0pt}
		\pgfarrowshullpoint{-.75\pgfarrowlength}{.5\pgfarrowlength}
		\pgfarrowshullpoint{-.75\pgfarrowlength}{-.5\pgfarrowlength}
		% Saves: Only the length:
		\pgfarrowssavethe\pgfarrowlength
	},
	drawing code = {
		\pgfpathmoveto{\pgfqpoint{.6\pgfarrowlength}{.3\pgfarrowlength}}
		\pgfpathlineto{\pgfqpoint{.6\pgfarrowlength}{-.3\pgfarrowlength}}
		\pgfpathmoveto{\pgfqpoint{-.25\pgfarrowlength}{0pt}}
		\pgfpathlineto{\pgfqpoint{\pgfarrowlength}{0pt}}
		\pgfpathclose
		\pgfusepathqstroke
		\pgfpathcircle{\pgfpoint{.03\pgfarrowlength}{0pt}}{.25\pgfarrowlength}
		\pgfpathclose
		\pgfusepathqfill
	},
	defaults = {
		length = 5mm
	}
}

\pgfdeclarearrow{
	name = nxOM,
	parameters = {
		\the\pgfarrowlength
	},
	setup code = {
		% The different end values:
		\pgfarrowssettipend{.25\pgfarrowlength}
		\pgfarrowssetlineend{-.25\pgfarrowlength}
		\pgfarrowssetvisualbackend{-.5\pgfarrowlength}
		\pgfarrowssetbackend{-.75\pgfarrowlength}
		% The hull
		\pgfarrowshullpoint{.25\pgfarrowlength}{0pt}
		\pgfarrowshullpoint{-.75\pgfarrowlength}{.5\pgfarrowlength}
		\pgfarrowshullpoint{-.75\pgfarrowlength}{-.5\pgfarrowlength}
		% Saves: Only the length:
		\pgfarrowssavethe\pgfarrowlength
	},
	drawing code = {
		\pgfpathmoveto{\pgfqpoint{.35\pgfarrowlength}{0pt}}
		\pgfpathlineto{\pgfqpoint{\pgfarrowlength}{.3\pgfarrowlength}}
		\pgfpathmoveto{\pgfqpoint{.35\pgfarrowlength}{0pt}}
		\pgfpathlineto{\pgfqpoint{\pgfarrowlength}{-.3\pgfarrowlength}}
		\pgfpathmoveto{\pgfqpoint{-.25\pgfarrowlength}{0pt}}
		\pgfpathlineto{\pgfqpoint{\pgfarrowlength}{0pt}}
		\pgfpathclose
		\pgfusepathqstroke
		\pgfpathcircle{\pgfpoint{.03\pgfarrowlength}{0pt}}{.25\pgfarrowlength}
		\pgfpathclose
		\pgfusepathqfill
	},
	defaults = {
		length = 5mm
	}
}

\makeatother
